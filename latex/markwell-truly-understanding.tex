%% Generated by Sphinx.
\def\sphinxdocclass{report}
\IfFileExists{luatex85.sty}
 {\RequirePackage{luatex85}}
 {\ifdefined\luatexversion\ifnum\luatexversion>84\relax
  \PackageError{sphinx}
  {** With this LuaTeX (\the\luatexversion),Sphinx requires luatex85.sty **}
  {** Add the LaTeX package luatex85 to your TeX installation, and try again **}
  \endinput\fi\fi}
\documentclass[letterpaper,10pt,english]{sphinxmanual}
\ifdefined\pdfpxdimen
   \let\sphinxpxdimen\pdfpxdimen\else\newdimen\sphinxpxdimen
\fi \sphinxpxdimen=.75bp\relax
\ifdefined\pdfimageresolution
    \pdfimageresolution= \numexpr \dimexpr1in\relax/\sphinxpxdimen\relax
\fi
%% let collapsible pdf bookmarks panel have high depth per default
\PassOptionsToPackage{bookmarksdepth=5}{hyperref}
%% turn off hyperref patch of \index as sphinx.xdy xindy module takes care of
%% suitable \hyperpage mark-up, working around hyperref-xindy incompatibility
\PassOptionsToPackage{hyperindex=false}{hyperref}
%% memoir class requires extra handling
\makeatletter\@ifclassloaded{memoir}
{\ifdefined\memhyperindexfalse\memhyperindexfalse\fi}{}\makeatother

\PassOptionsToPackage{booktabs}{sphinx}
\PassOptionsToPackage{colorrows}{sphinx}

\PassOptionsToPackage{warn}{textcomp}

\catcode`^^^^00a0\active\protected\def^^^^00a0{\leavevmode\nobreak\ }
\usepackage{cmap}
\usepackage{fontspec}
\defaultfontfeatures[\rmfamily,\sffamily,\ttfamily]{}
\usepackage{amsmath,amssymb,amstext}
\usepackage{polyglossia}
\setmainlanguage{english}


\usepackage{termes-otf}\usepackage{heros-otf}


\usepackage[Bjarne]{fncychap}
\usepackage{sphinx}

\fvset{fontsize=auto}
\usepackage{geometry}


% Include hyperref last.
\usepackage{hyperref}
% Fix anchor placement for figures with captions.
\usepackage{hypcap}% it must be loaded after hyperref.
% Set up styles of URL: it should be placed after hyperref.
\urlstyle{same}


\usepackage{sphinxmessages}



        \usepackage{qrcode}
        \newcommand{\DUrolepdfpage}[1]{\marginpar{\textcolor{gray}{\scriptsize [#1]}}}
        \newcommand{\LOCALaudiolink}[1]{ {\hypersetup{hidelinks}\qrcode[height=6em]{#1}}\kern3em\url{#1}\par }
        \gappto\captionsenglish{\renewcommand\chaptername{Talk}}
    

\title{Truly Understanding the Teachings of the Buddha}
\date{Dec 15, 2024}
\release{bc5a7e9}
\author{Anthony Markwell}
\newcommand{\sphinxlogo}{\vbox{}}
\renewcommand{\releasename}{Release}
\makeindex
\begin{document}

\pagestyle{empty}
\sphinxmaketitle
\pagestyle{plain}
\sphinxtableofcontents
\pagestyle{normal}
\phantomsection\label{\detokenize{index::doc}}


\begin{sphinxadmonition}{note}{Editorial note}

\sphinxAtStartPar
This document is available as \sphinxhref{https://edhamma.github.io/markwell-truly/html/}{web (HTML)}, \sphinxhref{https://edhamma.github.io/markwell-truly/latex/markwell-truly-understanding.pdf}{print (PDF)} and \sphinxhref{https://edhamma.github.io/markwell-truly/epub/markwell-truly-understanding.epub}{e\sphinxhyphen{}book (ePub)}.

\sphinxAtStartPar
It was prepared from the \sphinxhref{https://6db146.a2cdn1.secureserver.net/wp-content/uploads/2019/04/TrulyUnderstandingtheTeachingsoftheBuddha.pdf}{original PDF} from 2019 (linked from \sphinxhref{https://anthonymarkwell.com}{anthonymarkwell.com}), following encouragement by the author (indicated page numbers refer to the 2019 PDF). There was no alteration to the content. Sectioning hierarchy was slightly modified for easier navigation.

\sphinxAtStartPar
Created \sphinxhref{https://github.com/edhamma/markwell-truly/}{from the source}, git version bc5a7e9, on Dec 15, 2024.
\end{sphinxadmonition}

\begin{sphinxadmonition}{note}{About the book}

\sphinxAtStartPar
This book contains the Dhamma talks given by Anthony Markwell, when he was resident teacher at Wat Kow Tahm Insight Meditation Center in Thailand during Vipassana meditation retreats offered by him.

\sphinxAtStartPar
Anthony is teaching the Dhamma in a unique way, comprehensive and interlinked, which allows the reader to get a profound understanding of the teachings of the Buddha.

\sphinxAtStartPar
This book is not only a comprehensive guide to understanding Vipassana meditation but it gives the full background and practical instructions and maps out the way leading to Nirvana – here and now in this very life.
\end{sphinxadmonition}

\sphinxstepscope


\chapter{Day 0 afternoon: Orientation}
\label{\detokenize{0-a:day-0-afternoon-orientation}}\label{\detokenize{0-a::doc}}
\LOCALaudiolink{https://www.mixcloud.com/anthonymarkwell/day-0-orientation-talk/}

\sphinxAtStartPar
\DUrole{pdfpage}{1}   Welcome to Wat Kow Tahm insight meditation center. This afternoon
we are going to have an orientation talk for our seven day silent meditation
retreat. First of all I always like to start the retreat by offering our thanks to
Mae Chee Ahmon Pun, our 92 year old nun, that lives here in the monastery and also to Steve and Rosemary Weissmann, who helped develop this
meditation center. Over 25 years they taught more than 8000 people here.
They  developed  this  whole  meditation  center.  Rosemary  and  Steve,  with
some of you may have meditated with, came here when Rosemary was 35.
She left a couple of years ago when she was 60. So we always like to thank
them for their hard work they’ve done to develop this place. The buildings,
the walkways, the gardens, the meditation areas, all the terraces, everything
they have carved out of the jungle. So we are really very much appreciative,
that we could walk into this place two years ago and everything is set up for
us to do meditation retreats here.

\sphinxAtStartPar
Essential to their activities, Mae Chee Ahmon Pun, our nun, has been
here for 50 years. We have Mae Chee Nanika, who has been cooking here on
every retreat, every month for the last 27 years. So we thank them for their
efforts in the work in developing the center and giving us the chance to come
and practice here.

\sphinxAtStartPar
Koh  Phangan  is  very  famous  for  its  full  moon  party  and  its  various
\DUrole{pdfpage}{2}   other parties, but there is another aspect to the island. There is a lot of yoga
and other healing types of workshops going on, all types of spiritual activities are going on on the island and many types of meditations from many
different traditions.

\sphinxAtStartPar
We  offer  our  retreat  here  in  the  hope  that  Koh  Phangan  will  lead  to
your spiritual evolution and allowing us to become more useful members of
society. Keep that in mind when you’re meditating this week.

\sphinxAtStartPar
This orientation talk is meant to give us some introduction and some
additional information as to the purpose of our meditation retreat here. We’ll
be talking about the word «meditation». We’ll be talking about a «retreat».
We will also go through a few of the guidelines that we like to follow here
and also we’ll be having a look at the silence.

\sphinxAtStartPar
I very much and warmly welcome you here. Let’s have a good retreat
together.

\sphinxAtStartPar
After I finish this orientation talk, you will have the opportunity to ask
questions, that you may have before we enter into the silence.


\section{Meditation}
\label{\detokenize{0-a:meditation}}
\sphinxAtStartPar
First  of  all,  let’s  have  a  look  at  this  word  «meditation».  What  is  the
meaning of meditation? We use it to refer to training of the mind. To the cultivation and development of particular mental faculties. The faculty of faith
and confidence, the faculty of energy, the faculty of mindfulness or awareness, the faculty of concentration and the faculty of wisdom. These five faculties are what we are referring to, when we are talking about meditation.
When we are developing meditation, we are developing particular qualities
of the mind, that come to increase and grow in strength, which leads us to
deep and profound understandings and insights.

\sphinxAtStartPar
So the Buddha’s teaching is often divided into samatha and Vipassana,
or calmness meditation and insight meditation. Here we will be practicing
satipatthana or the foundations of mindfulness, which leads to both samatha
and Vipassana practice. We will be joining them both together. Just like a
bird has two wings, we join them together, so we can make the mind stabilize and then we can see things as they really are. We will talk more about the
meditation  practice  later  this  evening.  Satipatthana  meditation  is  all  about
investigating  our  own  mind  and  body  process.  Over  the  next  seven  days,
we will be attempting to continuously activate our awareness in the present
\DUrole{pdfpage}{3}   moment. We will be trying to stay in the present moment and internalizing
this present moment awareness into our bodies. So we will be in the present, internalized, and we are trying to do this continuously. Try to remember
these three words: present moment, internal, continuously. That’s what we
are  trying  to  do  here  on  the  retreat.  We  are  observing  our  own  mind  and
body  process  looking  at  our  body,  its  physical  sensations  here  and  there,
looking at our mind, its mental states and its thoughts. We are looking at the
pleasantness and unpleasantness that can arise both from physical sensations
and mental sensations. We are observing our reactions to things, our liking
and  our  disliking.  We  will  be  most  interested  in  observing  our  identification and appropriation of mental and physical phenomena. Our meditation
practice here this week, it’s only about investigating this thing here. We want
to understand what this mind and body process is. This is what meditation
practice  is  all  about  in  the  Buddha’s  teaching.  We  are  investigating  it,  so
that we know it and see it clearly. This is the meaning of Vipassana, to see
things clearly. The only thing we need to understand and see clearly is our
own mind and body process. The rest of it works itself out. It’s fine just as
it is. We do have issues, however, in our mind and body process. So we will
be investigating them, primarily to see them as impermanent, to see them as
unsatisfactory, relatively, and to see them as non\sphinxhyphen{}self, to see them as impersonal. The mind and body process is impermanent, it is unsatisfactory and
it is impersonal. It doesn’t belong to anybody, it doesn’t belong to you. Let’s
investigate that all for ourselves.

\sphinxAtStartPar
The primary goal of the meditation practice is to remove the sense of
self,  which  is  arising  continuously  in  the  present  moment.  Consciously  or
unconsciously  we  are  appropriating  and  identifying  our  sense  experience.
Things that we see and smell and taste and hear and touch, things that we
think about – we are identifying with them. And this continuous process of
identification over many years has built up a continuous expression of me
and mine, turned it into an I. It has turned into a self. And so this momentary
identification with phenomena has crystallized, become more concrete and
become a self. It’s become a person, it’s become a personality. So we will be
investigating all those things in this retreat, trying to free ourselves from this
delusion or this very persistent illusion.

\sphinxAtStartPar
\DUrole{pdfpage}{4}   Having  said  that,  meditation  is  neither  easy  nor  difficult,  but  it  does
take  some  patience.  It  does  take  some  perseverance. You  will  need  to  put
forth a lot of effort. You will need to be determined and you’ll need to be
disciplined. I have spoken to everyone about following the schedule of the
retreat, which is designed to keep you continuously practicing. This is the
most important part of our meditation retreat. Don’t think that the meditation finishes when I’m ringing the bell and the sitting ends. There are formal
walking meditation periods and there are formal sitting meditation periods.
The real meditation goes on during the daily activities, when we try to keep
our mind in the present moment, keep it from running off into the past and
running  off  into  the future, keeping it present, keeping it local, keeping it
internalized.  If  we  can  do  that  during  the  breaks,  when  we  come  in  here
and  sit  and  do  our  sitting  practice,  or  out  on  the  terraces  doing  the  walking practice, then the mind is going to be able to come together much more
efficiently and much more effectively. So please try to be as continuous as
possible in your meditation practice.

\sphinxAtStartPar
The Buddha’s meditation practice is not just to make us feel calm and
peaceful. It’s much more than that. It’s not about developing psychic powers
like telepathy or clairvoyance or clair\sphinxhyphen{}audiancy. It’s not about the escaping
reality either. Satipatthana meditation is not just a relaxation technique. We
are not just trying to be calm and cool and happy. We are not just coming
here to bliss ourselves out. The Buddha’s meditation is all about the cessation of dukkha, the cessation of suffering. So that’s what we will be doing
here this week.

\sphinxAtStartPar
Our  meditation  practice  is  a  learning  process. We  are  learning  about
things. It’s an investigative process not a creative process. We are not creating  objects  so  that  we  can  watch  them. We  are  watching  nature. We  are
observing and investigating nature as it’s arising, when it’s arising in its natural  state.  We  don’t  want  to  be  using  external  objects.  There  is  no  need  to
create an image or object we are going to meditate on. We are using our own
mind  and  body  process. That  is  the  object  of  our  meditation.  Real  things,
that are occurring in the present moment. That will give us a clear view or a
clear seeing of the nature of reality. If we want to understand reality, we are
going to have to investigate just reality. Investigating concepts or stabilizing
\DUrole{pdfpage}{5}   on concepts doesn’t leave to us understanding the true nature of reality. We
need to work with the real objects in real time for us to see things clearly as
they are.

\sphinxAtStartPar
So in mindfulness meditation we are not trying to do anything. We are
not trying to make anything. We are just watching and waiting and watching
and  seeing  what  is  happening  as  it  unfolds. We  are  not  wanting  anything.
We are not expecting anything. We are observing. We’ve bought a ticket to
the cinema and we are watching the show. We are not getting involved with
the show. We are not jumping up on the stage with the actors in the theater.
We are sitting back, reclining in our seat and watching what’s going on. So
our meditation practice is hands off. It’s turning our experience of the mind
and body process into an objective field of objects that are arising and passing away. We will be using those objects as the foundation for mindfulness.
We  need  to  practice  simply  and  continuously.  We  are  not  trying  to  make
something happen and we are not resisting anything either. We are not trying
to push anything away. We are not trying to make anything disappear. We
are not trying to create something. We are being present. We are activating
our awareness in the present moment and observing what’s there. You’re not
going to an object in trying to be mindful of it. The object has already passed
away if we do that. We are activating our awareness, so that we are present,
and, we are seeing the object of consciousness. And we are doing that over
and  over  again.  Activating  our  awareness,  seeing  what  is  there,  stepping
back from it, disengaging from it, so that it can pass away.

\sphinxAtStartPar
So  we  are  being  in  the  present,  we  are  just  staying  in  the  present
moment,  being  awake  and  knowing  and  aware.  Don’t  forget  what  is  happening now. Don’t allow your mind to go wandering here and there. Don’t
allow it to go into the past or into the future. If your awareness is present
and  if  wisdom  is  also  present,  then  your  mind  will  be  free  in  the  present
moment. And that’s the state of awareness we are aiming for. We are aiming
for freedom in the present moment from our own defilements, from our own
afflictions.

\sphinxAtStartPar
We will need to check our attitude on this meditation retreat. Some of
you have come here a second time, a third time, a fourth time. We need to
check our attitude. Sometimes we have great expectations. Sometimes we
\DUrole{pdfpage}{6}   really want to get it. We really want to become a meditator, we really want to
get this. We can be super enthusiastic. We can be super wanting. We desire to
get meditation. We desire to become a meditator. Please check your attitude.
Throw this one out! This is not the type of attitude that you need for meditation. Meditation is exactly the opposite of desire. We are letting go of desire.
Even  of  good  things,  even  for  success  in  meditation.  You  have  all  made
enough effort to arrive here. That is as much desire as you will need to complete the meditation practice. Don’t expect any results. This will create a lot
of anxiety and stress in your meditation practice. Just be cool with whatever
is arising and passing away in the moment. If it’s unpleasant, it’s unpleasant.
If it’s pleasant, it’s pleasant. And there it is and it’s passing away. If we hold
on  to  it,  whatever  the  object  is,  whatever  the  emotional  state  is,  whatever
the repetitive thought is, then we dukkher ourselves. We create suffering for
ourselves by holding and attaching to mental and physical states. When we
see things as they really are, the mind naturally disengages from mental and
physical phenomena. It’s the key to a happy life. It’s the key to understanding  the  mind’s  reactionary  processes.  This  week  we  will  be  talking  a  lot
about the nature and structure of the mind, not only the contents of the mind.
You will be able to, if you follow the instructions carefully, observe this for
yourself. You will be able to see the structure of your mind and how cause
and effect act together. Continuously believing that they are somebody. It’s a
very strange situation, that this thing, that came out of our mothers, this mind
and body thing, this process has come to believe that it is someone. And this
causes a lot of difficulties. When we are not under this false view, when we
are not being attacked by this view, then we are free from the sense of self.
We have gone back to nature. We’ve returned to our true and a beautiful state
of love and compassion, a state of selflessness.

\sphinxAtStartPar
Check  the  quality  of  your  awareness,  if  you’re  observing  the  mind.
Each time you come to the hall here, check your attitude. When you come in
and sit down and you really want to get something, just stop that! Just come
in with the attitude of «well, let’s see what happens.» Every sitting is going
to be different. This week you will be doing more than 50 sittings. There are
going  to  be  some  really  tough  ones. There  are  going  to  be  some  fantastic
ones. And there’s going to be a whole lot in the middle. Some of them boring,
\DUrole{pdfpage}{7}   some of them frustrating, some of them kind of okay, some of them tired,
some of them just buzzing and blissful beyond belief, some of them will give
you great insights, some of them will give you great frustrations, some of
them will just be repeating, repetitive conversations with yourself and your
family members, some of them will be silent, some of them will be real meditation, some of them will be thinking. Lots of different sittings this week.
Be prepared for them. Meditation is not just completely being blissed out.
It’s a lot more work than that but the work is extremely rewarding. The work
will lead you to holding the key to releasing your own mind from dukkha.

\sphinxAtStartPar
So we check our attitude. Are you trying to get something? Are you
trying to become something? Why are you meditating? Are you coming to
the hall to get something? Stop that attitude. That’s the wrong attitude. That’s
the attitude of desire.

\sphinxAtStartPar
The present moment is always defined as mind and matter which are
arising independent of desire. When the mind and body process no longer
want something to be different than it actually is, then we are in the present,
right here. If we are always wanting something, if we are always wanting
our experience to be different, then we are always projecting into the future.
We are never quite here. We are not satisfied with what’s present. We want
it to be otherwise. We want it to be different than what it actually is. This is
desire to get something, to change something, to become something. That it
is not what we are looking for. We want to see it as it actually is. So you need
to drop any kind of desire. That wanting is the very mind state that blocks
your meditation. You’re blocking your access yourself by wanting things to
be different than they actually are.

\sphinxAtStartPar
We don’t want to control the experiencing in any way. Our responsibility is right attitude.


\section{Retreat}
\label{\detokenize{0-a:retreat}}
\sphinxAtStartPar
Secondly let’s have a look at the word retreat. Retreat means a period
of seclusion for prayer or study or meditation. Retreat also means a place of
security or shelter. All these definitions are given here at Kow Tahm. This is
a period of time of meditation, a place of security and shelter from the world.
Being isolated from the external world a few days and being with yourself in
silence, you really get a chance to have a look at your own mind. For most of
\DUrole{pdfpage}{8}   us we haven’t been silent for seven days since we came out of our mothers.
We’ve always been talking, continuously chattering away. But here is no way
to escape our own mind. We’re going to be watching our own mind, our mind
states, our reactions, what we like and dislike, our identification with things.
Here we only have a chance for listening. There’s no Internet and no books to
read and no place to go to. We won’t be able to distract ourself with anything
external. We will be paying attention to the present moment and seeing what
is in our mind. Meditation is mind training. We are going to put the mind on
a leash this week. We try to train it a little bit.

\sphinxAtStartPar
Without these distractions, it’s an excellent opportunity to do our internal work. Usually when we have a problem when we face difficulties, we
like to talk about it with others. Here we don’t do that. Here we examine it.
We observe it to deal with it wisely on our own. We don’t suppress things
and we don’t run away from our mind states. We are patient. There will be
frustrating or agitated moments.

\sphinxAtStartPar
There will be moments when you’re thinking, «what am I doing here.
Really, what am I doing here?». And you start to loop and think about all
the excuses you can find to try and escape. Have a look at those mind states!
Here you have to learn how to deal with your own mind. You won’t be able to
bounce off your mind state with your partner or friends. You can regard your
work as an investigation. There will be unpleasantness. There will be a lot
of pleasantness as well. All different mind states are going to be coming up.
You shouldn’t regard any of these mind states as a problem for your meditation. Nothing is a problem in your meditation practice! Meditation practice
is the removal of the problems! It’s the observing of whatever is there. Every
object can be an object of awareness and wisdom. Every object can be noted,
known and let go of. We are learning how to function in the world without
being  disturbed,  without  creating  suffering  for  ourselves.  That’s  what  the
heart of meditation practice is all about. Learning how to deal with issues.
And the big issue is internal. Really, there’s nothing wrong with the outside
world! Things are as they are. They will be as they will be. What we can do
though, is train our mind, train our reactions to the world.
\sphinxstyleemphasis{It’s our reactions}
\sphinxstyleemphasis{to the external world that creates happiness or unhappiness}. If you can see
that, you will be the master of your own happiness. Happiness comes from
\DUrole{pdfpage}{9}   non\sphinxhyphen{}reaction. Happiness comes from seeing things as they really are. Happiness comes from contentment with how things are in the moment. When
we’re not content with how things are in the moment, we’re not happy. So a
lot of our meditation practice, especially in the first two days, will be about
practicing patience and contentment. Being content to be here and to do the
practice.

\sphinxAtStartPar
Whilst on a retreat, we try not to think about our work or studies. We’re
not interested in our family and friends and what they’re doing this week. We
are focusing on our own work. Just be present, don’t think about the future,
not even next week. A meditation retreat is not a time for you to come and
think  about  your  future.  That  is  not  meditation.  It’s  just  thinking.  It’s  just
your own imagination. The future is uncertain and doesn’t even exist. The
future exists as a thought in your mind in the present moment. Anything to
do with the future, is just your mind right now thinking about the future. It
doesn’t have any objective reality. It’s just what your imagination is doing.
The past is the same. It only comes into existence as a thought arising in the
present moment. The past exists as a thought in the present. The future exists
as a thought in the present.
\sphinxstyleemphasis{Past and future have no actual reality beyond
your thinking.}

\sphinxAtStartPar
On the retreat here you don’t have to use your time organizing or worrying about anything. We will take care of all those things for you. Your job
is just to remain present, to note and know and to let go.

\sphinxAtStartPar
In a retreat setting like this, we learn to become aware of whatever’s
arising in the present moment. We don’t allow our mind to get scattered here
and there. Try to bring it back to the present. If we’ve noticed that we’ve
gone  into  the  past  or  the  future,  bring  it  back.  Nothing  wrong!  The  mind
normally wanders here and there. We don’t get upset, when leaves fall off the
tree. That’s what they do! We don’t get upset, when fish go swimming. That’s
what they do! The mind wanders. So don’t be upset if your mind doesn’t do
what you exactly want it to do. It’s just like a new puppy. You have to train it.
So a meditation retreat like this provides the ideal training facility to
practice  meditation. You  can  consider  this  meditation  hall  as  a  swimming
pool. You spend seven days learning how to swim, learning how to watch
your own mind. The real practice comes, however, when you leave this place
\DUrole{pdfpage}{10}   to see if you can swim in the ocean.

\sphinxAtStartPar
Check your attitude, make sure you don’t expect anything to happen.
Sit  back  and  see  what  does  happen! Anything  valuable  does  take  time  to
develop. So don’t worry if it’s not working on the first or second day. Don’t
cling  to  any  pleasurable  experiences  that  may  happen  quickly.  Sometimes
people get very fast results and then they spend the rest of the days trying to
get it back again. Let go of everything that arises whether it’s pleasurable or
unpleasurable. If it’s pleasant or painful, just let it go.

\sphinxAtStartPar
We are trying to keep our mind present and stabilize our awareness in
presentness. Stabilizing our awareness in the stillness of the present, watching whatever flows through the mind and whatever sensations occur in the
body.

\sphinxAtStartPar
We  have  to  adjust  to  a  fairly  vigorous  daily  schedule.  You’re  going
to  have  to  deal  with  bodily  discomfort. You’re  going  to  have  to  deal  with
habitual  thought  patterns.  Our  job  is  to  observe  what  is  arisen,  step  back
from it, disengage and allow it to be let go of. We are observing things to the
point that they are so clearly, that the mind can no longer use it as a base for
self. It’s discarded. It only has use when it can be a base for the sense of self.
If that object is being used as a sense of me, mine or I, or as a condition for
the arising of me, mine or I, then it is still useful for creating that. You’ll be
stuck in it. If you can disengage from that, stop identifying with the object,
then you’re able to free yourself from the object. The problem doesn’t come
from the object, the problem comes from your identification with the object.
Your  attachment  to  it! We  identify  with  our  own  bodies,  we  identify  with
many different things and social concepts, that have come to surround us.
We are identifying with them. We are creating a sense of self out of them
both from internal and external sources. We have built this fine sense of me.
The Buddha’s teaching is to allow us to see this illusion, to see the fact that
we are actually living in the matrix.

\sphinxAtStartPar
Fortunately  for  us  there  is  a  way  out. The  Buddha’s  teachings  is  the
way to escape from the reality that we are living in.

\sphinxAtStartPar
When  you  experience  various  discomforts,  you  might  have  trouble
understanding what’s going on. You’ll be in your own reactive mind state.
Unpleasantness – got to move, got to get out of here. Pleasantness – more,
want  to  stay.  That’s  how  we  normally  react.
\sphinxstyleemphasis{\DUrole{pdfpage}{11}   We  are  normally  completely}
\sphinxstyleemphasis{conditioned  by  feeling.  The  citta  sankhara,  the  mind  conditioner.}
Feeling
is  conditioning  everything.  Unpleasant  feelings  are  conditioning  thoughts
of disliking, which turn into aversion and then anger, pushing things away.
Pleasant  feelings  turning  into  liking,  wanting,  desiring,  craving,  pulling
things towards us. Pleasant feelings and unpleasant feelings are conditioning the mind, conditioning our experience.

\sphinxAtStartPar
So we will be examining those feelings states. The state of pleasantness or unpleasantness in the present moment, so that we can free our mind
from  these  states.  The  states  will  still  be  there.  There  will  still  be  pleasantness or unpleasantness, but you just won’t attach to them anymore. You
won’t  identify  with  them  anymore,  so  they  won’t  cause  you  any  distress,
worry or concern. When we stop identifying with them, they will pass away
extremely  rapidly.  If  we  identify  with  something,  we’re  holding  it  and  it
stays. When we see something clearly with Vipassana insight, we’ve seen it,
it’s let go of and it passes away. Attachment leads to suffering, clear seeing
leads  to  letting  go.  Our  meditation  practice  is  all  about  letting  go.  Please
adjust your attitude if you’ve come here thinking that meditation is to try and
get something. We haven’t come here to get anything. We’re not going to get
anything here. There’s nothing here to get and there’s nobody that gets anything. Meditation is the practice of letting go. So please adjust your worldly
stats. The way that we normally view the world is to achieve and get stuff.
Meditation is exactly the opposite. We are doing as much as we can to let go
of things. We are trying to release and relinquish things. I hope you haven’t
come here to get something, to get meditation. Here we’ve come to let go
of all things. The more you can let go of, the more your mind will stabilize,
the more you’ll see in the present moment. As we start to do the meditation
practice, we’ll talk about the various objects that need to be let go of in order
for us to experience the state of calmness in which real insight can arise.


\section{Guidelines}
\label{\detokenize{0-a:guidelines}}
\sphinxAtStartPar
Thirdly,  let’s  talk  a  little  bit  about  some  guidelines,  some  rules  and
regulations that make it comfortable for us to stay here. Don’t regard them as
some kind of control. Just try to regard them as conditions. We are putting
in place the suitable conditions for the arising of insight. These are agreements \DUrole{pdfpage}{12}   to benefit all of us so that we can live harmoniously and peacefully
together  doing  the  practice.  It’s  wonderful  when  50  people  are  in  silence
living together for a week. You will get to know each other in a certain way,
in silence, and it’s wonderful to see each other after the retreat as well.

\sphinxAtStartPar
All of us have a strong commitment to stay here for the full retreat. Be
determined to follow all the activities and follow the schedule completely.

\sphinxAtStartPar
Most of us take a lot of interest in our bodies. We like to put nice clothing on and paint it in various ways. Here, we don’t worry too much about
that. We don’t need to get dressed up. However, always dress properly, cover
your shoulders and your knees, cover your midriff as well. Please don’t lie
down anywhere except for your room. Please don’t point your feet towards
the speaker or the Buddha in front here.

\sphinxAtStartPar
We  ask  you  to  only  practice  the  meditation  technique  we  do  here.
Those who continuously choose to switch between different techniques, will
have created a lot of doubt in their mind. You won’t know what to do from
moment to moment. You will be all confused. So please don’t bring that confusion into your meditation practice.

\sphinxAtStartPar
Do  your  chore  with  happiness  and  joy  in  your  mind. You’re  giving!
You’re giving something to all of us.


\section{Silence}
\label{\detokenize{0-a:silence}}
\sphinxAtStartPar
Fourthly, I want to talk about silence. Here we are on a silent retreat.
Probably the single most important factor for a successful retreat is the maintenance of the silence. Silence is sometimes called the doorway to insight.
When we’re silent, we start to observe things more intently. We don’t need
to verbalize our thoughts all the time. Our thoughts, instead of being pushed
all the time out of the door of our mouth, they bounce around inside. They
bounce around so that we can see them more clearly. We’re going to be watching them. Of course, it’s normal to have the impulse to talk. We have been
talking all our lives. So it’s normal that we have the wish to talk. Especially
when something funny happens, we would like to make a comment about
that. Please avoid doing that. Please avoid talking or disturbing your roommate or others. It’s not the time to socialize. We regard all talking here as
disturbing. So if you start talking to somebody, you start to disturb them and
their meditation practice. A single sentence can sometimes bounce around
\DUrole{pdfpage}{13}   in the head of a person all day. So please be aware that silence is the real
secret to having a good meditation retreat. If you start talking you’re going
to ruin it for yourself and for others. It’s very important that we maintain the
silence – not just silence of the mind, either. Physical communications, even
communications with the eyes. We’re not here to flirt with anyone. You’re not
here to get a new boyfriend or girlfriend. Be aware that your behavior can be
disturbing to somebody who is trying to meditate. We have all come here to
be in silence and to meditate.

\sphinxAtStartPar
You should observe bodily silence as well. Please move around quietly,
especially in the meditation hall. And calmly, and slowly. There’s no need
for you to rush in and out of the hall. Slowly and mindfully come back to the
meditation hall after the walking meditation. There shouldn’t be any gaps in
mindfulness between your walking meditation and your sitting meditation.
There should be one continuous flow over an hour and a half, 45 minutes
each. So try to take the walking and sitting meditation as one unit. Don’t go
to the toilet or to fill your water bottle in the middle of the walking meditation. Your meditation will be more successful if you can be continuous.

\sphinxAtStartPar
This is one of the few places in the world where we can come and be in
silence, where we don’t have to talk. We can enjoy each others company, we
can smile at each other. We’re not saying that you have to be a zombie. Smile
and still be pleasant to each other but don’t disturb each other.

\sphinxAtStartPar
If  you  hear  other  people  continuously  talking,  tell  one  of  the  volunteers. Do not go to them and make «ssshhh». This will cause you a lot of
problems  for  a  few  days. You  will  feel  awful  about  what  you  have  done.
You start to think about it, «oh, why did I do that. I shouldn’t have done it.
What will they think about me.» All kind of thoughts. Every time you see
that person, you get an uncomfortable and unpleasant sensation in the body,
your mind will feel a little bit embarrassed, you turn away from them, feel
uncomfortable.  So  please  don’t  do  that.  It  will  not  be  beneficial  for  your
meditation practice.

\sphinxAtStartPar
Of course, if there is any emergencies, you need to talk. If there is a
snake going into someone’s room, you need to tell them.

\sphinxAtStartPar
So  it’s  important  that  we  keep  the  silence  because  we  understand  its
value  and  its  benefits.  Not  only  for  our  meditation  practice  and  our  spiritual \DUrole{pdfpage}{14}   development and our well\sphinxhyphen{}being. It’s not just about the rules. It’s about
conditions.  It’s  all  about  putting  the  right  conditions  in  place.  All  of  our
meditation  instructions  are  just  conditions.  We  are  just  telling  you  things
you need to do for your meditation to work. So please try to follow all those
instructions. If you follow all the conditions perfectly, then insight is going
to unfold naturally for you. In fact, you can’t stop it. When all the conditions
are there, it’s going to keep overflowing. Just like when the tap is on and the
pot is full, it just keeps overflowing. The condition is water going into the
pot,  the  result  is  overflowing.  If  all  the  conditions  are  there,  the  dhamma
unfolds naturally by itself. If the conditions are not there, the water is not
flowing, it’s impossible for that water to start overflowing the pot. So think
about  your  meditation  practice  in  this  way.  We  are  putting  conditions  in
place,  the  results  will  completely  take  care  of  themselves.  In  fact,  there’s
nothing you can do to stop it from unfolding if you put the conditions there.
Of course, after the 19th silence will be lifted very early in the morning. At 5.30 in the morning on the 19th.

\sphinxstepscope


\chapter{Day 0 evening: Meditation Instructions}
\label{\detokenize{0-b:day-0-evening-meditation-instructions}}\label{\detokenize{0-b::doc}}
\LOCALaudiolink{https://www.mixcloud.com/anthonymarkwell/02-8_11_-day-0-b_-first-evening-meditation-instructions/}

\sphinxAtStartPar
This is our first meditation instructions that we will be giving you this
evening. We’ve mentioned that our meditation technique is known as
\sphinxstyleemphasis{satipatthana bhavana}. Satipatthana is made up of two words: sati and patthana.
We’re going to have to say much about this word and these practices called
\sphinxstyleemphasis{sati}
over the coming days. Its regular translation is mindfulness but we use
the word awareness as well interchangeably. Mindfulness and awareness. Its
meaning is attentiveness directed to the present moment. When we’re mindful,  we  are  directing  our  attention  to  whatever  is  occurring  in  the  present
moment. So we’re directing our attention to the present. We are activating
present moment awareness. We are switching it on, we are turning on our
mindfulness.
\sphinxstyleemphasis{Patthana}
can have two meanings. It means either establishing
or arousing. In its passive sense, it means establishing our awareness in the
present moment or directing our attention in the present moment on to something, on to a foundation. There are four foundations that we will be talking
about. Physical sensations, feelings, emotional states and thoughts – the four
foundations on which mindfulness can be directed. Patthana can also have
an  active  meaning.  It  means  arousing. Arousing  awareness  in  the  present
moment, arousing our attention in the present moment. So satipatthana, that
is what our meditation practice is called.

\sphinxAtStartPar
We  will  be  observing  and  activating  our  awareness  in  the  present
\DUrole{pdfpage}{16}  moment on the very fast flowing range of objects. We will be starting with
the physical sensations in the body, because they’re a little bit slower, a little
bit grosser and easier to see. We will be transitioning in our meditation practice  so  that  we  can  watch  feeling,  pleasant  and  unpleasant  and  emotional
states and our thought patterns as well.
\sphinxstyleemphasis{Bhavana}
means mind development,
mind cultivation or mind training.

\sphinxAtStartPar
So establishing mindfulness in the present moment on one of the four
foundations of mindfulness – that’s what we are doing here this week. And
your ability to establish your awareness in the present moment on whatever
object is arising will lead you to see that object as it really is. We see things
as they really are. And this is what is meant by clear seeing or
\sphinxstyleemphasis{Vipassana}. \sphinxstyleemphasis{Vi\sphinxhyphen{}passana}. \sphinxstyleemphasis{Pass} means to see. \sphinxstyleemphasis{Ana} means it is a noun.
\sphinxstyleemphasis{Vi} is an intensifying
prefix. We are seeing things really clearly. We are seeing things so well, that
the mind disengages, stops identifying with it, stops taking it as me, being
for me or happening to me or as mine or as being I and let’s it go. We have
clearly seen that object.

\sphinxAtStartPar
So satipatthana is the cause, Vipassana is the result. We establish our
awareness in the present moment on one of the four foundations of mindfulness and that allows Vipassana to occur, clear seeing to occur.

\sphinxAtStartPar
And what is it that we’re having insight into? We’re observing not only
the contents of our mind, but we’re observing the structure of our mind. We
are seeing that all objects arise and pass away, but in particular all objects
also have certain particular characteristics. They’re all impermanent, they’re
all  relatively  unsatisfactory,  in  relation  to  the  unconditioned  element  and
they’re all non\sphinxhyphen{}self. They don’t belong to anybody. It’s all impersonal. It’s an
impersonal flow that is going on.

\sphinxAtStartPar
When we see the structure of our experience in this way, it allows us to
let go of things. The sense of self is let go surely and gradually. Thoughts of
me, mine and I become less and eventually this leads to the complete cessation of being. The complete cessation of suffering. And that’s an experience
the Buddha called nirvana – that cessation of dukkha.

\sphinxAtStartPar
Tonight, so that we can begin our practice, we have divided our instructions into three parts.

\sphinxAtStartPar
First of all we will be having a look at the practice of virtue, because
\DUrole{pdfpage}{17}  it’s  one  of  the  major  conditions  that  we  need  to  put  in  place.  We  will  be
having  a  look  at  our  sitting  meditation  and  also  our  walking  meditation.
These instructions are for the formal sitting and walking periods. The practice of virtue is all day everyday whilst we’re on the retreat. So be continuous in your mindfulness and practice throughout the day in order for Vipassana knowledge to unfold. As I said this afternoon, these Vipassana insights
that  arise  from  satipatthana  meditation  are  conditioned.  Conditioned  by
satipatthana. Conditioned by your ability to maintain your awareness in the
present moment, to maintain your awareness internally, and your ability to
continuously do this over and over again over a period of time. When these
conditions are in place, the results start to flow naturally. As I said, there is
nothing you can do to stop them from unfolding naturally in the mind.


\section{Sila}
\label{\detokenize{0-b:sila}}
\sphinxAtStartPar
The  practice  of  sila  or  virtue  begins  with  abstaining  from  unwholesome  verbal  and  physical  actions. The  most  gross  of  our  three  intentional
doorways we have through body, speech and mind and our virtue practice
will be to moderate our verbal actions and our physical actions. Purification
of virtue is said by the Buddha to be the base upon which all other meditative practices are based. So if you are sincere in your effort and you wish to
have the experience of insight, it is well recommended to you to maintain
virtue. Tonight I’m not going to give you a sermon on morality but since the
success of your meditation practice is conditioned by the practice of virtue,
it’s worth taking note of these things. Of the noble eightfold path, the third,
fourth and fifth factors are all concerned with virtue. So three out of eight
factors of the noble eightfold path are to do with sila or morality.

\sphinxAtStartPar
The first one is called
\sphinxstyleemphasis{samma vaca,}
meaning right speech. Basically we
are abstaining or stopping four types of unwholesome speech occurring. Of
course in the retreat you’re all in silence. So this one is pretty much covered.
For the sake of completeness we avoid telling lies to others. We avoid hard,
rude  speech  to  others.  We  avoid  back  stabbing  others,  that  means  talking
about somebody to someone else in the hope that your opinion of that person
will be conveyed to another person. The fourth one is gossiping.

\sphinxAtStartPar
The fourth factor of the noble eight fold path,
\sphinxstyleemphasis{samma kammanta}
means
\DUrole{pdfpage}{18}  right action. We abstain from three types of particular actions. We don’t kill
anything,  we  don’t  steal  anything  or  take  something  which  has  not  been
given  to  us  and  we  abstain  from  sexual  misconduct.  Sexual  misconduct
means relations with people that are inappropriate for us to have sexual relations with, whether they are people who are minors, or younger or children,
whether they are people who are already in a relationship. Or even the kind
of  activity  like  raping  somebody.  This  is  all  sexual  misconduct. Adultery
where taking someone’s husband or wife and having sexual relations with
them. This is something that is very harmful for the other partner unless they
have of course some agreement between themselves.

\sphinxAtStartPar
The  fifth  factor  is
\sphinxstyleemphasis{samma  ajiva}
or  right  livelihood.  It  means  earning
a living without harming anybody else. Having a right livelihood, having a
livelihood which is beneficial for other beings. Specifically we avoid trading
in things like guns or poisons or drugs or intoxicants. We try to make sure
we don’t sell flesh from animals. We don’t make it to our business to be killing beings or taking unfairly the resources from other beings using them for
ourselves or profiting by them for ourselves.

\sphinxAtStartPar
So  we  are  practicing  right  speech,  right  action  and  right  livelihood
whilst we’re on the retreat here. This is a very firm base on which we can
practice our meditation.

\sphinxAtStartPar
Whilst we’re on the retreat here, we will be following what is known as
the eight precepts. In the morning we will chant the eight precepts. Basically
they are no killing, no stealing, no lying, abstaining from sexual intercourse,
abstaining from drugs and alcohol and abstaining from high and luxurious
seats  and  beddings,  abstaining  from  eating  in  the  afternoon  and  not  using
any jewelry or flowers or trying to look fancy in any way, not listening to
music or watching any shows.

\sphinxAtStartPar
An additional rule is not to speak with contempt about somebody who
has already attained the noble eight fold path, somebody who has already
become enlightened. You should be careful about this. This can disturb your
meditation practice, maybe not in the beginning but later on, if ever you have
said something detrimental or something negative about somebody who has
become enlightened. You should make an apology – just in your own mind
and  then  forgive  yourself  for  making  that  mistake.  That  can  clear  certain
\DUrole{pdfpage}{19}  blockages away.


\section{Sitting meditation}
\label{\detokenize{0-b:sitting-meditation}}
\sphinxAtStartPar
How do we do the sitting meditation? First of all we’re going to start to
have a look at the position. When we’re sitting in meditation we need to try
to be as comfortable as possible. We want to be comfortable for at least 45
minutes. That can be quite difficult in the beginning stages. When you’ve sat
for a full session, you feel very much invigorated. You feel like you’ve been
successful and you will find that you can sit for 45 minutes again and again
and again once you have broken through that limit.

\sphinxAtStartPar
What I’m going to show you now is a few of the sitting postures.
The most simple of them is just the cross legged position. It’s called
sukhasana. Happy posture if you like. The problem with this posture is that
all your weight will be distributed on to very small areas around your ankles
and on to the sit bones. So after a while it becomes quite painful.

\sphinxAtStartPar
The second posture that we use as well is called the quarter lotus. That’s
done by putting your foot up on the calf of the other leg. Normally, we put
the left leg under the right leg. But other traditions have other styles. If your
legs are not properly stretched, you will find some tension points underneath
because there’s not a great deal of flatness.

\sphinxAtStartPar
Then we have what is know as the half lotus. Half lotus is the posture
I like to sit in personally. We take the sole of our foot and tuck it into our
groin. Nice and close and then you put the sole of your foot on the inside of
your thigh. Your leg is kind of turned over a little bit that your anklebone is
not even touching the floor. It’s kind of tucked under. All the weight is along
this shinbone. So it’s a nice and continuous flat surface that doesn’t have any
pressure points.

\sphinxAtStartPar
Another  very  comfortable  posture  is  the  posture  that  the  Burmese
monks and nuns like to take. It starts with the same thing, sole running on
the inside of this thigh and then instead of putting it up on the thigh, leave it
in front. That’s the most comfortable one. The most important thing is that
you are flat and comfortable.

\sphinxAtStartPar
Those who want to get the little fancy can try to do the full lotus posture.  Quite  uncomfortable,  quite  painful  for  most  people.  You  can  get  it
\DUrole{pdfpage}{20}  going  for  a  few  minutes  and  after  that  it  starts  to  become  uncomfortable.
You can try to sit in the shallow water on the beach whilst you’re here on the
island. Gravity in the water is a little bit different.

\sphinxAtStartPar
We mentioned this afternoon the numbness in the legs, which can quite
often happen if we are leaning over our legs. So make sure that your back
is up straight. If you start leaning forward, all the weight will go down on
your legs and you start to get numb. If you find that you’re getting a little bit
numb, you can try just to leaning back a little bit. For a minute or two you
just lean back and take away the weight of your legs. We don’t want to be
tense with our back. We are not trying to stretch the head and touch the roof,
but we’re not leaning over either. You just want to be with a straight spine up
in a row allowing the spine and skeleton to support the weight of the body
rather than trying to use your muscles. In that way the breath can flow in and
out quite comfortably.

\sphinxAtStartPar
Where do we keep our hands? I like to keep my right hand on top of the
left and allow the thumbs to touch a little bit sometimes. If that becomes a
little bit too warm for you, you can move your hands and put them half way.
You can even spread your fingers a little bit. So they just sit down there in the
lap in a comfortable position. Some people like to put their hands on their
knees. That’s fine as well. That can be nice and stable as well. Just be careful
when you start to put your hands on your knees, it doesn’t make you start
leaning forward. And then your legs might become numb and your breathing
won’t be flowing as smoothly as it could be. Hands on the knees is fine or
even half way down on the thighs as well is fine.

\sphinxAtStartPar
As far as your eyes are concerned, you can keep your eyes open or you
can keep them closed or you can do a bit of both. Most people will meditate
with their eyes closed. But also meditating with the eyes open is very, very
effective. There’s nothing to stop us to come into the present with the eyes
open. In fact, as our practice develops, you’ll be meditating with your eyes
open  more  and  more.  Especially  in  our  daily  life,  we  have  our  eyes  open
most of the time when we are awake. So you can start keeping your eyes
open, but eventually they will close down when you become more and more
concentrated. – Just lightly close your eyes. As beginners, we often want to
see things with our eyes in the meditation, however, meditation is not carried
\DUrole{pdfpage}{21}  out by the eyes. It’s carried out by the mind. We are trying to see with the
mind. – Try to let the eyes go. I know it’s a place where consciousness very
often arises. It arises at the eye door. It arises at the ear door. Continuously,
over  and  over  again.  Consciousness  is  used  to  being  in  these  areas  of  the
body process. – So in our practice, we try to resist the urge to use the eyes
all the time in our meditation.

\sphinxAtStartPar
That’s the preliminary instruction. You’re setting yourself up, the posture. We want to be able to maintain this sitting for at least 45 minutes. If
during  the  sitting,  it  becomes  extremely  painful,  of  course,  you  can  move
your legs a little bit. When I started to sit in meditation, it was extremely
painful. I started by lightening an incense stick which burns for 45 minutes.
I started  sitting for 15 minutes cross legged, and then I moved by putting
my leg on the side, and do that for another 15 minutes, and then when that
became too painful I put my legs to the other side. And that was the first few
months of my meditation practice.

\sphinxAtStartPar
If you find that it is really, really uncomfortable, just stand up. You can
stand up for a few minutes and then sit back down again. Just because you
are changing postures from sitting to standing to sitting, it doesn’t mean that
your meditation has to stop at all. Your meditation can continue in the present moment as you change your body.

\sphinxAtStartPar
We learn to meditate by sitting as still as possible because it’s easier.
Once  you  have  the  hang  on  it,  we  don’t  need  to  meditate  in  a  still  sitting
position. We can meditate throughout the day in anyone of the four postures,
walking, standing, sitting, lying down, and in fact, in any of the other postures we get the body in when we’re doing other activities.


\section{First stage: Internalizing awareness – letting go of past and future}
\label{\detokenize{0-b:first-stage-internalizing-awareness-letting-go-of-past-and-future}}
\sphinxAtStartPar
So let’s have a look at our sitting meditation now with the first step.
We have established our body, found a comfortable position to sit in. What’s
the first thing we do? We start paying attention to the present moment. This
is the beginning of our mind development. We are gathering the mind into
the present moment. It likes to wander into the past and future. We are just
bringing it back and internalizing our awareness. We do this continuously
\DUrole{pdfpage}{22}  the first few moments of our meditation sitting. The first minute, at least, this
should be your only object. Don’t come into the hall and start immediately
watching your breath. You need to establish your awareness in the present
moment  and  you  need  to  internalize  your  awareness.  So  our  first  step  is
continuously directing the mind to the now. And just hold it with that as the
object now. Just being aware of now. Present. Present. Just keep directing
the mind bringing it into the center. Again and again. If it wanders off, don’t
be upset. We just bring it back and start again. Be aware when thoughts arise
of the past – memory, or of the future – planning, and bring yourself back
into the present. If your mind goes into the past or future, stop it quickly.
Be aware and alert that you’ve just exited meditation and you’re just sitting
on the floor in a hall on a mountain, no longer meditating. You’re engaging
in your thoughts. You’re no longer present, you’re no longer here. So bring
yourself back to the present. Many will find that this is a major part of their
meditation practice for the first few days that they are here. – This is the first
stage in letting go. We are letting go of the past and future, we are coming
into the present.


\section{Second  stage:  Feeling  body  sensation  –  letting  go  of  the body}
\label{\detokenize{0-b:second-stage-feeling-body-sensation-letting-go-of-the-body}}
\sphinxAtStartPar
Once  you’ve  been  able  to  do  that,  then  we  direct  our  attention  internally. We direct this present moment awareness, we have established in the
first  minute  and  bring  it  inside.  We  climb  inside  the  body.  We  direct  our
awareness to become aware of the whole body as it is sitting on our mat. Do
you become aware of how it feels like? What does the sensation of a whole
body  feel  like?  Bring  your  mind  in,  be  actually  inside  your  body.  Climb
inside it. If you can’t catch the whole body posture of sitting at once, start at
the top and know what the top half of your body feels like and then become
aware of the bottom half of the body. Become aware if there is something
sitting on the mat. There is something sitting on the mat and it’s breathing!
So climb inside and feel what it’s like to actually have that sensation of a
whole body sitting there. Start from the top, work your way down. You can
even move through your body, if that helps. We’re directing our attention to
the sensation of the whole mass of the body as it is sitting there.

\sphinxAtStartPar
\DUrole{pdfpage}{23}  How do you know that you have a body when your eyes are closed?
How do you know it’s there? Just examining that. Can you feel that there is
something sitting on the mat? You can’t see it when your eyes are closed and
you can’t smell it and you’ve stopped thinking about it. Conceptually you can
create an image of your body in your mind, so you can perceive the body.
We can perceive the body through the eye door. We can perceive the body
through the mind door as a concept. We can even smell the body. We can
hear it, making sounds. So there’s many different ways to perceive the body.
But what we are interested in here, is perceiving the body through the body
door itself. Through the skin! Through the internal sensations of the whole
thing. We are not looking at the body, we are not hearing the body, smelling
or even licking the body. We want to experience the body through the sense
door of touch.

\sphinxAtStartPar
So this is the second object of our meditation practice. And we do this
for the second minute of each sitting. Some people will experience the sensation of the body as being all around them. Some people will have the experience  being  aware  of  the  body  from  some  viewpoint  up  on  the  shoulders
somewhere. We are not trying to picture the body, to create a mind image of
the body, of what you think the body looks like when you’re sitting on the
mat. That’s not the object we’re doing here. The object is just the sensation
of the whole thing as it’s sitting here. It’s a very subtle physical sensation.
You can move your mind in and out, up and down. You can move it, you can
feel it. This sensation is there. We are observing the whole body.

\sphinxAtStartPar
First, we let go of the past and the future and we stick with the present.
Secondly, we let go of the external world and we are focusing on the internal
world. This is the second stage of letting go. Our meditation practice is letting go only. How far, how much can you let go of in the present moment.
If you’re noticing and identifying that you are attaching with something, let
it go. Our practice is only about letting go.
\sphinxstyleemphasis{The Buddha’s teaching is only about letting go.}
Letting go of the idea that you have a personality, that you
are a self. In fact, enlightenment takes place, when you have completely let
go of the sense of self. That is what we are aiming for. First we’re going to
let go of the past and future, we are going to let go of the external world,
and then we’re going to start to let go of the physical body. And then we are
\DUrole{pdfpage}{24}  going to let go of the mental states as well. When we have let go of everything, when there’s nothing left to let go of, the mind experiences that which
is beyond conditioned realm. It experiences the unconditioned.


\section{Third stage: Using the breath to maintain awareness}
\label{\detokenize{0-b:third-stage-using-the-breath-to-maintain-awareness}}
\sphinxAtStartPar
The third stage of our meditation practice is to start paying attention to
the breath. We start to become aware that the breath is flowing in through the
nostrils, down through the body to the abdomen. The breath is flowing in and
the breath is flowing out. We are going to use the breath, that occurs many
times a minute, 12 times in and 12 times out, if you are a regular breather.
Sometimes it’s faster, sometimes it’s slower. We are going to be using this
flow of the breath as a concept to maintain awareness in the present moment.
We are using the breath! As a conceptual structure to maintain our awareness
in the present moment. What we are really looking to paying attention to,
though, is the physical sensations that are created by the breath that flows in
and out of the body. This practice is called
\sphinxstyleemphasis{chatu dhatu bhavana manasikara},
which  means  paying  attention  to  the  four  elements.  The  four  elements  is
what makes up this body. This physical mass that is sitting here, it’s a combination of six elements, actually. We are going to use four of them for our
meditation practice in the beginning days. We can cover the other elements
later. So what we are directing our attention to is the physical sensations created by four of those elements. We normally just refer to them in English as
earth, water, fire and air.


\section{The four elements}
\label{\detokenize{0-b:the-four-elements}}
\sphinxAtStartPar
The body is made up of these four physical elements. Don’t take those
words literally. Don’t think of a handful of earth, a glass of water, or a mouth
full of breath. That’s not what we are talking about here. This is very different from what those words mean. So don’t get lost in the translation.

\sphinxAtStartPar
All along the breath path, as the breath flows in and as the breath flows
out, various physical sensations will take place. Some of the sensations will
be  around  the  nose  area,  some  of  them  will  be  around  the  throat  area,  or
around the heart area, some of them will be down in the abdomen. It’s possible  to  feel  the  breath  right  up  through  the  head  and  right  away  through
\DUrole{pdfpage}{25}  the body down to the legs. The anatomists and physiologist tell us, that the
breath is coming from the nose into the lungs and it does come out again
without going any further than the lungs. That may well be true. What we’re
actually interested in, however, is the sensations created by the breath as it
flows in and out. And the physical sensations go further than the lungs. The
physical  sensations  flow  right  away  through  the  body.  In  fact,  when  your
lungs expand, your diaphragm gets pushed down and your stomach moves
out. So the oxygen exchange is happening in the lungs. The physical sensations  that  the  breath  creates  are  happening  throughout  the  body. And  this
is our main object of meditation in the first days of practice. The physical
sensations that are created by the breath.

\sphinxAtStartPar
These physical sensations are the characteristics of the four elements.
They are the manifestation of the four elements. Those four elements in Pali
are
\sphinxstyleemphasis{pruṭhavī,  āpa,  teja  and  vāyu}.
Earth,  water,  fire  and  air.  We  can  think
about them in various ways.
\begin{itemize}
\item {} 
\sphinxAtStartPar
The earth element is really the element of extension. It’s the element
of solidity. Anything that is solid within this mind and body process
sitting  here,  whether  the  bones  or  flesh  or  the  ligaments  or  the  skin,
they’re  all  solid  things. There  is  a  certain  solidity  to  our  body.  It’s  a
base. It’s something real. My one weighs about 90 kg at the moment.
It’s a solid mass of stuff. This is just a foundation for the other elements
to play in.

\item {} 
\sphinxAtStartPar
There is also the fire element. The fire element is just the element of
temperature. This body does have a temperature aspect to it. 37 or 38°.
Sometimes it feels very warm, sometimes it feels very cold which may
be just a temperature fluctuation of half a degree or so. But the fact that
this  thing  has  any  temperature  aspect  to  it  at  all,  means  that  the  fire
element is there.

\item {} 
\sphinxAtStartPar
There is also the air element which manifests in many different ways.
It  manifests  as  the  breath  and  it  manifests  in  other  various  ways  as
well. As  pressures  and  tensions,  tingling  and  vibrations. The  air  element is flowing throughout the body. The air element is not restricting
itself to the nose, the lungs and back again. It’s the energies that flow
throughout the body. Different traditions will have different names for
\DUrole{pdfpage}{26}
the air element. In the Chinese tradition it’s known as chi. In the Indian
tradition  it’s  known  as  prana.  In  this  Pali  tradition  it’s  referred  to  as
\sphinxstyleemphasis{vāyu}
. This is the element of forces. There are those various forces that
are happening in the body. It’s the air element that activates the body
and moves it from here and there. The solidity and the fire get pushed
around by the air element combining together.

\item {} 
\sphinxAtStartPar
These three elements would be completely separate from each other,
if there wasn’t for the water element. The water element is the element
that  combines  these  things  together.  The  solid  base,  the  temperature
and the energies and forces, that are moving throughout the body, are
all held together into a unit, compacted into a body that we call mine,
my  body.  So  the  water  element  is  cohesive.  Now  just  imagine  when
you  are  making  some  bread,  you  put  some  flour  which  is  a  powder,
some salt which is a powder, some yeast which is a powder in a metal
bowl  and  spin  them  around.  So  these  three  ingredients  are  there  but
they’re just powder in a bowl. As soon as you start to add some water
to it, it starts to come together to a dough, into a solid mass. Well, this
is what our body is. It’s a combination of the earth element, the fire element, the air element and the water element.

\item {} 
\sphinxAtStartPar
There’s also the space element,
\sphinxstyleemphasis{akasa}. It’s also there. It’s very subtle.
It’s a different type of meditation practice, but it’s there. We are concentrating on the four elements when we are investigating the body.

\end{itemize}

\sphinxAtStartPar
These four elements can manifest in different ways. They manifest as
different physical sensations. An element can either be strong or weak and
they are always fluctuating, always flowing.

\sphinxAtStartPar
If the fire element becomes stronger, then the body becomes warm. If
the fire element and gets weak, then the body becomes cool. There’s a sensation of coldness. The fire element has a weak aspect and a strong aspect. The
other elements are the same as well.

\sphinxAtStartPar
If the earth element manifests in a strong way, it will be manifesting
as hardness, roughness and heaviness. It’s like your teeth are hard and solid.
Your fingernails are hard and solid. There’s kind of a roughness if you run
your  hand  through  your  hair  if  your  hair  is  short.  The  earth  element  can
also manifest in a weak manner. It can manifest as softness, smoothness and
\DUrole{pdfpage}{27}  lightness.  Touch  your  lips  with  your  tongue.  There  is  softness  there.  The
earth element has two aspects to it. It can be either weak or strong. How its
characteristics are manifesting, will give you different physical sensations.
Hardness, roughness, heaviness, softness, smoothness, lightness. The same
thing,  they  are  all  the  earth  element  and  they  will  be  manifesting  in  your
body continuously. Sometimes the body feels hard, sometimes it feels soft,
sometimes  it  feels  very  heavy,  sometimes  it  feels  very  light.  It’s  the  earth
element playing around.

\sphinxAtStartPar
The  air  element,  which  we  will  be  focusing  on  a  lot  this  week,  the
breath  in  particular  as  it  flows  in  and  out  and  the  abdomen  as  it  rises  and
falls, displays many different physical sensations. If the breath is fast, then
the physical sensations will be pushing and pressure and resistance. If the
breath is long and smooth and calm, you will get those smooth and comfortable sensations. You’ll feel gliding, you’ll feel flowing. So depending on how
you are breathing, different physical sensations will start to manifest. When
the air element manifests at full strength, it will appear as forces rising up,
from near the buttocks’ area and forcing up your back. Pushing, supporting
your whole back. You feel the energy keeping the body erect. If it wasn’t for
that air element, the body would just collapse on the floor. It’s the air element
constantly flowing up through the back around and up. It’s consistently flowing. Our arms extending and bending, it’s all the air element.

\sphinxAtStartPar
We’ll be watching out for these physical sensations as our meditation
unfolds. In fact, this is the main object of our meditation practice. To become
aware,  to  fully  understand,  to  clearly  see  the  physical  body.  The  physical
body needs to be let go of. We’ve been attaching and identifying with it too
long already. It has caused a lot of troubles when we attach and identify with
the body. So this meditation is to allow us to see the body as it really is. We
see  the  nature  of  the  body  and  will  be  able  to  let  go  the  body.  Letting  go
means not identifying with the body.

\sphinxAtStartPar
Our meditation practice will transition. We will start to use the breath.
It flows in and it flows out. This is a conceptual framework that we follow.
What we are really interested in, however, are the physical sensations created by the breath. I want you to pay particular attention to three areas as
the  breath  is  flowing  in  and  flowing  out.  In  particular  I  want  you  to  pay
\DUrole{pdfpage}{28}  attention to the nostril area, the area around the upper lip. Can you feel the
breath flowing in? Is there a physical sensation of touching? Can you feel the
warmth as it comes out, the coolness as it goes in? Can you feel the vibration
that is occurring? Vibrating, it’s the air element. Warmth or coolness, that’s
the fire element. Touching sensation, that’s the earth element. Can you feel
it  flowing  down  through  the  body?  So  we  are  following  it  all  through  the
body. We are really only interested in the physical sensations created by the
breath. We are not interested in the images or little maps, that you create for
yourself of the body.

\sphinxAtStartPar
As  it  flows  down  to  the  chest,  you  feel  some  rubbing  or  some  gliding. And then I want you all to pay particular attention to the abdomen as it
rises and falls. You breathe in, the abdomen rises. There’s a certain physical
sensations that occurs there. As the abdomen falls, there’s a certain physical
sensation as well. In fact, there are many different physical sensations. Pushing,  twisting,  pressure,  vibration,  tingling,  throbbing. All  those  sensations
are there for you to pay attention to.

\sphinxAtStartPar
These physical sensations are not the four elements themselves. These
physical sensations are the manifestation of the four elements. We are going
to be bringing ourselves into the present moment, internalizing our awareness, then we start paying attention to the breath as it flows from the nose to
the abdomen, and from the abdomen to the nose. We are using this concept
of the breath to get in touch with the physical sensations. And we’re going to
do this continuously over and over again.

\sphinxAtStartPar
We can give you an example of the type of knowledge that arises when
we start to watch this way. First of all we are looking at a conceptual thing.
And  then  we  are  looking  at  physical  sensations. And  then  we  will  breakthrough  and  we  will  see  the  actual  elements  themselves.  But  this  doesn’t
happen automatically. We will give you an example of the car down there
in the carpark. If someone was to ask you, «so what is it that is sitting there
in the carpark?» Our first reaction is, «it’s a pick\sphinxhyphen{}up, or it’s a Toyota or it’s a
car.» That is a conceptual answer. Yes, there is a general thing there that we
call a car. If we look a little bit closer, however, we will start to see actually,
what we call car is just a conceptual idea. It’s actually made up of car parts.
As we try to describe what it is, «well, there is some wheels, there is some
\DUrole{pdfpage}{29}  doors, there’s a bonnet, there’s a tailgate, there’s an engine». It’s actually not
a car that is sitting there. It’s just an arrangement of car parts. Parts have been
arranged in a particular way and so it produces this image of car. That’s what
we  are  normally  used  to  dealing  with  –  concepts! We  don’t  normally  deal
with ultimate realities. If they keep asking you, «well, what’s really there?»
What is really in the carpark? When you really examine what is sitting there
in the carpark, you will find out it is just metal, just glass, rubber and plastic. You will see the car in that way. So our perceptions can change in the
way that we see things. We are used to see the body as a conceptual thing.
As being my body. We are going to start to have a look at it a little bit more
closely.  We  will  start  to  observe,  okay,  let’s  make  our  mind  present,  let’s
internalize our awareness, and let’s really have a good look and see and find
out what this body is. We come to an understanding of the four elements by
going through the physical sensations created by the breath. The breath sensations contain all the four elements. The elements change and the way that
we can perceive their change is through the physical sensations that make
up the body.
\sphinxstyleemphasis{In fact, the body is nothing other than these physical sensations}
\sphinxstyleemphasis{and our concepts that we paint on top of it.}
So that’s what is actually happening in real\sphinxhyphen{}time. There is the real things, your physical sensations created by
the four elements and then there is what you paint on top. Your conceptual
notions of yourself, what you believe you are. Whether you believe it’s a man
or a woman, or it believes it’s this nationality or that nationality. Whether it
believes, it’s a smart person or a stupid person. Whether it believes it’s rich
or poor. Funny or serious. It’s painted all these different things on top of it.
We’re going to strip off all those concepts that you painted on yourself and
we just want to have a look and see actually what the body is. And this is
what our meditation practice starts with.

\sphinxAtStartPar
In order for us to break through and really see these elements as they
really are, we will need to pay attention continuously to the present moment.
Being able to pay attention to the present moment, means we need an object
which is continuously there. The breath is such an object. It arises, stays for
a while and ceases. So when we begin practicing mindfulness of breathing,
we are aware of the breath as it comes in, and we are aware of the breath as
it goes out. We are aware at the beginning of the breath, we are aware of the
\DUrole{pdfpage}{30}  breath as it flows, we are aware of the breath as it finishes. Three things to
pay attention to. When it first starts, as it continues and when it finishes. And
then there’s a little gap. And then it starts, continues and finishes. And then
there’s a little gap. You can pay attention to all of those things.

\sphinxAtStartPar
There are many advantages of watching the abdomen rising and falling. Make an effort to follow the breath down, and follow the breath up. We
will do this for the first day of our meditation practice. Keeping ourselves in
the present, keeping ourselves internally, following this concept of a breath
coming in and out. If we can follow the breath in and out, make the mind
continuous,  then  it  starts  to  stabilize.  It  starts  to  concentrate. The  concentrated mind allows us to see beyond the physical sensations. We really see
the elements themselves. And that is something that you will breakthrough
in time. But you will need to be continuous.
\sphinxstyleemphasis{Continuity is the secret of success in meditation practice.}

\sphinxAtStartPar
In the old days, they used to make fire by rubbing two sticks together.
If you try to rub two sticks together, you can do it for 10 minutes and stop
for five minutes and do it again for 10 minutes. If you do that, you will get
very tired and there will be no smoke or even flame, even any kind of fire
produced. Because when you’re working and stop, it cools down again. You
work  and  it  cools  down  again.  Just  stopping  and  starting.  Our  meditation
practice is exactly the same. You will need to be continuous in your efforts
and then eventually you will make some fire. If you can be continues for half
an hour, you will be able to make fire. If you start for five minutes and stop
for five minutes, you can do that for 30 years and you still never get a fire!
The  stop  and  start  method  doesn’t  work,  it  doesn’t  produce  any  fire.  Even
after many, many years of practicing. But after just 30 minutes of continuous application, it starts! So continuity is the secret of meditation practice.
You will need to resist any urge that the mind has for going at delighting in
some kind of objects. Delighting in the past, in the future, in the present at
the sense doors. That is when the mind is going out, that is when the sticks
have stopped rubbing. To reactivate your sticks, bring your mind back into
the present moment and internalize your awareness. Notice every time your
mind wanders away from the breath and bring it back as quickly as you can.
Do not entertain any thoughts. Do not entertain, «oh, I should be meditating.
\DUrole{pdfpage}{31}  I just want to think this story out a little bit. Just let that think a little bit and
then I will do my meditation.» That is wasting your time here! And this is
precious time! Only seven days that we can practice with full intensity.

\sphinxAtStartPar
So please try to maintain your awareness of the breath continuously as
it is flowing in and out. We are paying attention right at the beginning.

\sphinxAtStartPar
In fact, you can follow it all the way through and at the end of the inbreath, you can just say, «now», if you want to. Bring yourself back to the
present. At the end of every out\sphinxhyphen{}breath you can bring yourself back to the
present. You can do this over and over again. Be aware it is starting, finishing, now. Exiting, finishing, now. You keep bringing yourself back into the
present  moment,  observing  what  is  exactly  there.  Not  what  you’re  imaging to be there, not what you like to be there but just seeing what is there
over and over again. We are training the mind. We are narrowing the place
where we allow it to run. We allow it to run just into certain objects, certain
places. We are following the breath as continuously as we can, noting when
it goes in, we know that it’s going in. When it’s coming out, we know that
it’s  coming  out.  If  we  notice  any  physical  sensations  as  we  are  following
the breath, we pay attention to them. We can make a little note of them. ’In,
out’, if you feel the expansion in your chest, make a little note to yourself
‘expanding’. If you feel the pushing, make a note to yourself ‘pushing’. This
is the real life experience that you are experiencing. This is the real thing.
There is actually a physical sensation taking place in the present moment.
This is what we want to come to understand. We want to put ourselves in a
position, where we can watch so clearly and so closely the present moment,
that whatever physical sensation arises, we know it. We know exactly what it
is and we see it clearly for what it is. In this way we won’t attach to the body.
We will be able to let go of the body.

\sphinxAtStartPar
Normally, the elements will be nicely balanced. 25, 25, 25, 25\%. Occasionally one of the elements will become unbalanced. The earth element will
become 40\%, the other three will be 20, 20, 20\%. At that time you will feel
that the body becomes hard. There will be some hardness there. So the whole
body is becoming hard. Maybe the fire element becomes unbalanced. You
will feel warmth. So the four elements are always adjusting in ratios and the
four elements are manifesting in the present moment as sensations in your
\DUrole{pdfpage}{32}  body according to those ratios. We want to come to understand those physical sensations.

\sphinxAtStartPar
We are going to follow the breath that comes in. When it finishes, we
know that it’s finished. This is very important. As soon as one breath finishes,
tick the box, done. And then follow the next breath. One by one. We don’t
want to get ahead of ourselves. We don’t want to be thinking, «I’m going to
meditate for two hours now. I’m going to watch the breath, every single one
or even for the next 45 minutes». Just try to do it for that breath, that’s occurring right now. Just follow one breath successful, and then follow another
breath. Successful of one, follow another breath. Until you start to follow
the breath more closely. In the beginning stages it’s all over the place. The
breath will be coming in and out, in and out, your mind will be catching it
and then wandering away. Try to come back again. Your awareness is not
going to be able to follow it, but eventually, you will start following it a little
bit. You come back again, lose it. And then, oh, you’ve latched on. You’ve
come into the breath a little bit. Your awareness is starting to become internalized with the breath. You’re getting closer and closer to it. You’re coming
inside the body. You are starting to experience the inside nature of the body.
So different breath, different physical sensations. Short breath – hard
punchiness, long breath – smooth and gliding. You can make a note of this
phenomena.  You  can  start  to  understand  that  the  time  to  breath  in,  short
breath or long breath, will cause different physical sensations. You can start
to  see  that  there  is  some  cause  and  effect  happening  here  in  the  present
moment. You can start to see that the breath is conditioning the body. There
is  some  conditioning  going  on.  It  is  not  so  important  to  pay  attention  to
this now, but if it starts to become clear to you, make a note of that. There
is cause and effect happening between the body and the breath. Cause and
effect  happening  between  the  breath  and  the  mind  as  well.  All  three  are
linked  together.  Physical  sensations,  breath  and  mind.  They’re  all  adjusting. One is the cause, one is effect. Changing and swapping positions who
is causing, and who is affecting who. The mind and body process continues
to go on day and night. The breath continues to flow in and out regardless of
what you think, regardless of who you are. The four elements are the same.
Your four elements are the same as my four elements. The same as the four
\DUrole{pdfpage}{33}  elements  outside.  The  internal  elements  are  the  same  as  the  external  four
elements. This body is just a physical manifestation of your mind state. We
will talk about that later as well.

\sphinxAtStartPar
So  become  aware  of  just  following  the  breath.  I  want  you  to  try  to
follow  the  breath  as  closely  as  you  can.  From  beginning  until  end.  From
beginning until end. We follow them from the nose tip, paying attention to
the touching sensation. We follow it down to the abdomen. I want you all
to follow and chase the breath in and out. If you find that one place is more
comfortable than another to watch, especially I want you to be aware of the
rising and falling of the abdomen, if you want to watch the breath just there
or if you want to watch the breath just at the nose, you can do that for a little
bit. See which one feels more comfortable. But try to follow it in and out the
whole way and keep your attention and awareness continuously in the present moment. And then the characteristics of those four elements will start to
become clearer and clearer. They will only become clear if you have been
able to watch the breath, let’s say for five or 10 minutes. We start to be able
to continuously watch the breath over and over again without missing and
these physical sensations come to life. They start to display themselves quite
clearly and then you’ll break through. Instead of seeing a car or car parts,
you  will    break  through  and  you  will  see  metal,  plastic,  glass  and  rubber.
You will see the actual reality behind the concept of car, behind the concept
of breathing, behind the concept of body. You will see actually what is there.
And it will surprise you. It will shock you. It will amaze you, that the mind
can actually know the body in such away.

\sphinxAtStartPar
In fact, it has been able to do it for a long time already. We just haven’t
trained  the  mind  to  do  that  in  our  societies.  We  don’t  train  the  mind  in
this  way.  We  are  not  taught  to  observe  to  look  inside  the  body.  Most  of
our  knowledge  in  our  sciences  comes  from  observing  things  through  the
eyeball. Whether it’s the external world or whether it’s the body itself. We
look through microscopes. Our knowledge arises through the eye door. We
see things and then we think about it, and then we develop some theories
and it becomes knowledge. It becomes truth. But all this information only
comes  into  through  the  eye\sphinxhyphen{}door.  We  will  be  experiencing  the  body  in  a
very different way that physiologists don’t experience, that anatomists, the
\DUrole{pdfpage}{34}  doctors  don’t  experience.  Doctors  have  been  trained  to  examine  the  body
through their eye\sphinxhyphen{}door. They have a vast knowledge of the body, that they
have  managed  to  accumulate  through  the  eye\sphinxhyphen{}door. The  eye\sphinxhyphen{}door  leads  to
certain kinds of knowledge and types of information but it doesn’t lead to
wisdom in that way. Of course, wisdom arises when we notice and know and
let go at the eye\sphinxhyphen{}door. But that is not what the scientists have done. So we are
going to observe the body in a radically different way than scientists observe
the body. We’re not interested in looking at protons, neutrons and electrons.
We’re  not  interested  in  looking  at  bone  marrow.  We  are  not  interested  in
looking at amino acids. All these things have been investigated for a long
time and nobody got enlightened from doing so. That’s because they’ve been
observing through the eye\sphinxhyphen{}door. We’re going to be using the mind door to
investigate the body. The first stage is to become aware of the whole body as
it is sitting in its posture.

\sphinxAtStartPar
So  when  we  come  into  the  room,  we  establish  our  awareness  in  the
present and then we climb inside. So we are deep inside the body and we are
watching the breath as it flows in and flows out. Gliding in and gliding out.
We are doing that over and over again. This will give us some knowledge
about the nature of the mind and body process.

\sphinxAtStartPar
So we have been talking about sitting meditation. This will be enough
for us to get started tomorrow morning when we hit the mat at 4:30.


\section{Walking meditation}
\label{\detokenize{0-b:walking-meditation}}
\sphinxAtStartPar
I also want to talk about our walking meditation. In fact, let’s all stand
up and as you are standing up there, just bring your awareness into the present moment and become aware of the whole body as it is standing there on
the  mat.  See  if  you  can  just  become  aware  of  the  physical  sensations  on
standing. Close your eyes. How do you know that you have a body? Don’t
think about it, experience it. Experience the body with the body. Know the
body. Put your mind into your left foot. Now put your mind into your right
foot. See if you can swap your mind from the left foot to the right foot. How
does it feel like? Is there a different physical sensation when the mind is in
the foot as when the mind is not in the foot? Can you experience your mind,
your awareness, your consciousness jumping from one foot to the other? Do
\DUrole{pdfpage}{35}  you feel the physical sensations of softness when you are standing on a mat?
There are two objects very present. There is the standing posture, that’s one
object. And then there is the touching sensation of your feet on the mat. This
is  standing.  Standing  and  touching.  Left  foot  touching,  standing  posture,
right foot touching, standing posture.

\sphinxAtStartPar
You can move back and forth. If you need to stand up during the meditation  practice,  you  can  stand  up  and  become  aware  of  the  hole  standing
posture. You can become aware of the touching sensation of the feet on the
mat. You can move your consciousness from your nose, down to your abdomen. Move your mind around inside your body.

\sphinxAtStartPar
We  always  start  our  walking  meditation  with  the  standing  posture.
That’s how we start. We become aware, bring our mind into the present, we
climb inside the body. We become aware of the whole thing that’s standing
there. There is something standing there. It’s breathing. It’s got holes in the
side of its head, it hears stuff. It’s got eyes in the front of its head, it sees
stuff. Smelling and tasting go on as well. There’s something there. It thinks
about itself a lot. In fact, that’s all it thinks about. Its favorite subject, me.
It’s been all its day and all its night thinking about its past and future. It’s
completely and totally obsessed with itself. But now it’s just standing. Just
standing. – Okay, you can sit down now.

\sphinxAtStartPar
During the retreat we are going to alternate between walking and sitting meditation. We do 45 minutes of walking meditation and do 45 minutes
of sitting meditation. As I said before, try to join these two sessions together.
Try to make them into an hour and a half. One continuous session if possible.
When the formal walking meditation takes place, we try to find a place
where we can walk from one place to another. We’re walking from A to B,
15 paces long. 10, 15, 20 as space permits. We’re walking from one place
to another place, turning around, and walking back again. We are all going
to choose a place where we are going to do this for the week. Some of you
like to walk in the meditation hall here. That’s possible. There is also places
to  walk  around  the  Buddha  hall. You  can  walk  around  the  outside  of  the
Buddha hall or you can walk inside the Buddha hall. You can also walk on
these sandy areas. You can also use the yoga hall for your walking meditation practice. Find a place where you won’t be disturbed. Many people tell
\DUrole{pdfpage}{36}  me  that  they  get  disturbed  by  other  people  walking  around  them.  Find  a
place  where  you  can  lower  your  eyes  and  pay  attention  to  your  walking
meditation. It should be between 10 and 20 steps long. More than this will
make your mind wander. Try to keep it to a back\sphinxhyphen{}and\sphinxhyphen{}forth process. Walking
around a circle sometimes just leads the mind wandering and then you end
up with a body just walking around and a mind a million miles away.

\sphinxAtStartPar
Just like in the sitting meditation, where the breath is used as concept
for  maintaining  our  awareness  in  the  present  moment,  we  are  going  to  be
using the steps of the body to lead us to the physical sensations in the body.
What I mean by that is, in the sitting meditation, we are using the breath as
a concept, in, out, in, out. If we can do that continuously, we start to become
aware of the physical sensations in the body. We are using a concept to get
back to real objects, the physical body sensations.

\sphinxAtStartPar
We  are  going  to  use  the  walking  meditation  to  do  exactly  the  same
thing. In the beginning stages, you can just use two steps. ‘Lifting, placing,
lifting, placing’. This is a concept. The foot lifts up and the foot gets placed
down. We are walking back and forth. We’re becoming aware of the beginning of the step and the end of the step. Beginning of the step and the end of
the step. If we can put our mind into our feet and can follow along with this,
the physical sensations in the body will start to become clearer to us.

\sphinxAtStartPar
You can start by doing the two stage walking meditation or we can start
by doing the three stage walking meditation. Up to you! We will explain all
of them in a few moments. We can also do the one stage walking meditation.
Most of the time we will do, what we call the four stage walking meditation
practice. ‘Lifting, raising, moving, placing’.

\sphinxAtStartPar
The physical sensations that are produced in the feet, are our object in
this meditation practice. This is the most important thing that we’re paying
attention to. We’re keeping our mind in the present moment. We’re keeping
our mind internally. We are using a conceptual framework to maintain our
awareness  in  the  present  moment  and  we  are  noting  and  knowing  exactly
what is going on. At this stage of practice, just the physical sensations produced by walking. They’re all there. All four elements will be manifesting
themselves in the feet.

\sphinxAtStartPar
First you become aware of the standing posture. You stand completely
\DUrole{pdfpage}{37}  still and bring yourself into the present moment and become aware that the
whole body is standing and that the feet are touching the floor. In the walking meditation we don’t need to pay attention to the breath. We are paying
attention to the present moment physical sensations that are occurring in the
feet while we’re moving the feet.

\sphinxAtStartPar
‘Lifting, raising, moving, placing’. You can use these words if they’re
helpful  to  you. The  most  important  thing  about  the  walking  meditation  is
keeping your mind in your feet. Just like in the sitting practice, we are keeping the mind following the breath. In the walking practice, we must keep our
mind inside our feet. Specifically, try to keep your awareness on the sole of
your foot that is moving. We want to keep our mind in the moving foot at
all times. We’re not just walking around with our minds traveling here and
there. We are keeping our mind just in the part of the foot that is moving.

\sphinxAtStartPar
I want you to pay particular attention to the beginning and end of each
of these movements. You start to lift and it’s finished. Moving – it’s finished.
Placing – finished. Each of these stages comes to an end. It starts and it finishes. There’s a present moment event taking place. It starts and it finishes.
And  then  the  new  event  takes  place.  It  starts  and  finishes. The  moment  is
arising and passing away. Nothing else continues on from that moment. It’s
finished. It’s finished. Things are passing away continuously. So pay attention to this, keep your mind in your feet during the walking meditation.

\sphinxAtStartPar
Where  do  we  keep  our  eyes?  Our  eyes  in  our  walking  meditation
should be lowered. We want to keep our eyes looking at about two meters
in front of us. We have our eyes closed a little bit. We don’t want to be in
complete darkness – you can try that as well, it’s very effective. But we don’t
want to have our eyes open so that we can look at other people. We are not
going to spend our time looking at other peoples’ bodies. We are not going to
walk around looking at the trees or looking at insects or creatures that we’ve
never seen before. There is lots of things to look at. But looking doesn’t lead
to meditation practice. We want to keep our awareness inside the body as
much as we possibly can. We’re not allowing the mind to run out through
our eyes. We are keeping all our energy packaged within the body. We are
not looking at anybody. If you hear some sound, don’t go following and try
to chase it. Don’t go out trying to identify what that sound is. Is that a car or
\DUrole{pdfpage}{38}  a motor bike? We are not interested in what the sound is. We just note that
some hearing occurred and it’s gone. That’s it. You hear a dog bark and then
all the stuff you paint on top of that can be minutes’ worth of nonsense. Just
from one noise, one sound – hearing occurring. Hearing is very natural. You
have holes in your head. There is sounds outside. Hearing occurs. It doesn’t
belong to anyone. There’s nobody there that’s hearing. It’s a natural process.
Seeing  is  the  same.  There’s  nobody  who  sees  anything.  But  the  sensitive
matter  in  your  eyeballs  can  detect  light  and  forms,  they  are  out  there,  it’s
functioning, they’re coming together, seeing is occurring. There’s no person
there that needs to be a seer. It’s all natural. An impersonal process. Hearing
takes  place.  Seeing  takes  place.  So  in  our  walking  meditation  this  can  be
very distracting.

\sphinxAtStartPar
So  we  try  to  keep  our  heads  lowered,  keep  our  eyes  focused  on  the
ground, we are not looking at other people, we hear some sounds, we just
disregard it – okay, hearing occurred, finished, no longer interested in.

\sphinxAtStartPar
A good position for our hands in the walking meditation is just in front.
Some people like to keep their hands behind. That’s fine as well. You can
keep your hands still at the side. So we want to be as still as possible except
for  the  moving  parts  that  we’re  paying  attention  to.  The  lifting,  moving,
placing. So the most important thing is to keep your mind inside your feet
while you’re doing it.

\sphinxAtStartPar
Some  people  become  frustrated  with  the  walking  meditation.  They
think it’s not working. When I ask them where their mind is, they say, «what
do you mean?» You can’t follow the breath unless your mind is following
the breath. You can’t do the walking meditation unless your mind is in your
feet. This is where we see the sensations taking place. There is all kinds of
sensations happening in the foot. As we lift the foot up, there is one type of
sensation. We push the foot forward, there’s another type of sensation. As we
place the foot down, there’s another type of sensation. In fact, there is five or
six different sensations in each one. Lightness, vibration, softness, pushing,
hardness – all there. You can feel them as you do the practice and it will start
to unfold.

\sphinxAtStartPar
To  give  you  a  demonstration: We  stand,  we  start,  bringing  ourselves
into  the  present  moment,  bringing  ourselves  internal,  establish  our  awareness \DUrole{pdfpage}{39}   in  the  feet,  left  side,  right  side,  left  side,  right  side,  lower  the  head,
open the eyes a little bit so that the light comes in, make a note ‘seeing’ is
occurring, and then when you lift your heel have your mind really right in
your foot, ‘lifting’, stops, ‘moving’, stops, ‘placing’, stops.
\sphinxstyleemphasis{It’s really, really important  that  you  have  gaps  between  these  three  stages.}
There  needs  to
be a definite stop. There needs to be a lifting. Stop. ‘Moving’. Stop. ‘Placing’. Stop. ‘Standing, standing’. Bring yourself into the full body again. And
then when you turn around, ‘lifting, turning, placing’. You’re coming into
still  again.  Standing  is  taking  place.  ‘Standing,  standing’. And  off  you  go
again. 10 steps or 15 steps, back\sphinxhyphen{}and\sphinxhyphen{}forth, keeping your mind in the present
moment, keeping your mind inside the body.

\sphinxAtStartPar
Looking  out  for  physical  sensations  as  they  are  arising  and  passing
away. The most important thing, keep your mind in the present, keep your
mind internal, keep your mind in your feet, when you are doing the walking
meditation practice. Make sure there are gaps between the three stages of the
walking practice. You need to be able to see the arising and passing. Arising
and passing. Arising and passing. Doing a step as you normally walk, there
is no breakup. There is no arising and passing away. The sensations will be
shifting  and  changing  too  rapidly  for  your  mind  to  be  able  to  catch  up  to
them. So we need to slow down with the walking meditation. We are walking relatively slowly. In particular, the last 20 or 30 minutes of the walking
meditation should be much slower than the first 10 or 15 minutes. The first
10 or 15 minutes you establish some rhythm. You should try to be continuous in the rhythm like a watch. ‘Lifting, moving, placing, lifting, moving,
placing, lifting, moving, placing, lifting, moving, placing, lifting, moving,
placing’. So the mind gets a bit into a rhythm and starts to become aware
and contained in itself.

\sphinxAtStartPar
In  the  beginning  stages  you  will  find  it  quite  boring.  You  won’t  be
making much sense of it. But once it starts to work, it really works well. For
many newcomers to meditation, walking meditation gives them the results
before the sitting meditation. So don’t think that the walking meditation is
somehow secondary. Many people had some wonderful, deep and profound
insights, life changing insights on the walking path in this monastery. Just
by walking and paying attention you can learn a lot about the mind and the
\DUrole{pdfpage}{40}  body process. You can learn a lot about cause and effect.

\sphinxAtStartPar
Try to perform the turning around efficiently. When you hear the bell
at the end of the walking meditation practice, walk slowly to the meditation
hall. It should be a slow and gradual transition from the walking path into
your seat. There shouldn’t be any gaps of unmindfulness to occur. We want
to be continuous with our awareness.

\sphinxAtStartPar
You  start  to  see  things  that  you  normally  don’t  see. You  start  to  see
weird sensations in the feet. If you are new to walking meditation, I suggest
that you walk on a hard surface for the first few days. Walk inside the hall
here. Walking on the sand may be a little bit tricky for the first stages of the
walking meditation. We also do walking meditation, when we move around
the  monastery.  When  you  are  moving  from  this  hall  to  the  dining  hall  or
when you are moving from the dining hall to your dorm. Try to be aware
of the walking posture as well. There’s a whole thing that’s walking. Try to
keep your awareness internalized. You can pay attention to the touching of
your feet on the floor if you like. No shoes will be helpful for this. This is
something I should mention. During the walking meditation we should not
use shoes, no flip\sphinxhyphen{}flops, no rubber sandals or anything else. We use just our
bare feet.

\sphinxAtStartPar
So we need to activate our awareness in the present moment, bring our
awareness inside the body and start paying attention to the physical sensations that are occurring, using the concepts of the in\sphinxhyphen{} and out\sphinxhyphen{}breath, using
the concepts of lifting, moving, placing, we can bring our mind into an experience of nature. Into an experience of reality.

\sphinxAtStartPar
I want you all to pay very much attention making your walking meditation smooth and continuous. Just like the clock. Tic\sphinxhyphen{}tic\sphinxhyphen{}tic\sphinxhyphen{}tic\sphinxhyphen{}tic\sphinxhyphen{}tic. This
happens very smooth and continuously. So should our walking meditation
be.  Lifting,  moving,  placing,  lifting,  moving,  placing.  Try  to  get  yourself
into a rhythm.

\sphinxAtStartPar
Of  course,  there  is  going  to  be  thoughts  coming  up.  There  are  two
types of thoughts. Two ways that we can deal with thoughts that happen on
the walking meditation path. Some thoughts are just going to be little ones,
that just kind of bubble up and don’t lead anywhere. You just have a thought
about  something  and  then  it’s  gone  again.  When  those  kind  of  thoughts
\DUrole{pdfpage}{41}  happen, you can just note it. Thinking is occurring. And then continue with
your walking practice. If you find that thoughts come up and turn into a bit of
a story, if you find that you are just walking around, lifting, moving, placing,
but your mind is thinking about other things then you need to stop. You need
to stop and stand still and become quiet for a moment. And then start again.
During the first few days this may happen very often.

\sphinxAtStartPar
For  a  warm  up  you  can  just  stand  on  one  foot.  Just  lift  one  foot  up,
keep your awareness in the front of the foot, in the middle of the foot, and
than  just  drag  it  back  keeping  your  mind  inside  your  foot. And  then  push
it forward. And pull and push. Keeping your mind inside. Can you feel the
physical sensations on the bottom of the foot? On the sole of the foot? There
is some vibrations going on there. Keep your mind in your foot while you’re
pushing it forward. It is really a push. We say moving, but it’s really pushing
the foot through. Feel the friction and the resistance that’s there if you push
and pull your foot backwards and forwards. Like you’re pushing it through
water.

\sphinxAtStartPar
Sometimes our walking meditation feels like we are on cross country
skis  or  we  are  ice  skating. We  are  really  pushing  the  foot  and  pulling  the
foot. Try to catch that physical sensation there on the bottom of the foot. It’s
the physical sensation which is the object of our meditation. That is what we
follow. That physical vibration on the bottom of the foot.

\sphinxAtStartPar
It may take some time to get used to it. You’re certainly used to walking. But you’re not used to putting your mind in your feet. The mind normally goes out to enjoy the sensual pleasures. Something delightful through
the  eye,  ear,  nose,  tongue  or  body  door.  Something  to  think  about.  We’re
keeping the mind inside the body and we’re paying attention to the physical
sensations. If you like, before you start the walking meditation practice, just
warm your feet in this way up for your walking meditation. Push, pull, push,
pull. About 10 times on each foot. Just kind of warm\sphinxhyphen{}up your awareness for
that  physical  sensation. And  then  once  you  have  warmed  it  up,  try  doing
your walking practice and you will find, that you will be able to follow along
a lot easier.

\sphinxAtStartPar
There’s a few other things. No shoes when we do the walking meditation. We don’t need to lift up the feet high either and we are not walking one
\DUrole{pdfpage}{42}  foot in front of each other either. We’re not on a catwalk in kind of some
balancing act. Just the normal way that you walk, the normal height that you
walk. In fact, if you keep your feet lower to the floor, you will have a better
balance.

\sphinxAtStartPar
This is how we practice the walking meditation. The walking meditation  instructions  will  be  updated  almost  daily  and  we  will  go  further  and
further and so will be the sitting meditation practice.

\sphinxstepscope


\chapter{Day 1, morning}
\label{\detokenize{1-a:day-1-morning}}\label{\detokenize{1-a::doc}}
\LOCALaudiolink{https://www.mixcloud.com/anthonymarkwell/03-8_12_-day-1-a_-morning/}


\section{General outline}
\label{\detokenize{1-a:general-outline}}
\sphinxAtStartPar
This  morning  we’re  going  to  talk  about  Vipassana  and  give  you  a
general  outline  about  what  we’re  doing  here  this  week. We’ve  introduced
the practice last night giving you first initial mediation instructions, putting
the mind into the present moment, internalizing your awareness and paying
attention to the breath as it comes in and out of the body. In particular, we
are paying attention to the physical sensations produced by the breath as the
breath flows in and out of the body

\sphinxAtStartPar
This morning we’re going to look at Vipassana. We said last night «vi»
is an intensifying prefix, «passana» means to see. So to see clearly. To see
things as they really are, that’s what Vipassana means.

\sphinxAtStartPar
The ground of Vipassana is satipatthana practice. Vipassana starts to
arise when we practice satipatthana or activating our awareness on the four
foundations of mindfulness. Those four foundations are the physical sensations in the body, feelings – either pleasant or unpleasant – emotional states
and the fourth one is reaction or thoughts or identification if you like.

\sphinxAtStartPar
All Vipassana systems take these four foundations of mindfulness as
their base. Sometimes the techniques can be slightly different but they’re all
establishing awareness in the present moment using our own mind and body
\DUrole{pdfpage}{44}  process as the objects of our observation. So the purpose of Vipassana is to
see things as they really are, to see the mind and body process in the present
moment as impermanent, unsatisfactory and non\sphinxhyphen{}self.

\sphinxAtStartPar
The duty of this practice is to destroy defilements at the six sense doors.
Once your awareness and wisdom can be activated continuously in the present moment, we move from just observing the body door or selected objects
and we start to broaden our awareness out to the six sense doors. The eye
door, the ear door, the nose door, the tongue door, the body door, that we’re
working with at the moment, and the mind door. These six sense doors. We’ll
be able to note, know and let go at these six sense doors.

\sphinxAtStartPar
The purpose is to first of all start to wiggle free of this sense of self that
we’ve developed over a period of time moment after moment, after moment
of being unaware of what’s actually going on in the present moment, allowing  craving  for  being  to  enter  into  the  present  moment,  allowing  craving
to infiltrate the field of awareness. What happens is that craving for being
subjectifies  that  moment.  It  becomes  a  moment  where  things  are  taken  as
me, mine and I. It becomes a moment where duality is created. There’s a me
and there is the external world and all other people going on around it. And
this happens very naturally and continuously. It happens without us thinking
about it. It’s been going on since we were borne. It’s happening to everyone
unless, of course, they’ve done the practice, seen it happening for them in
real time and they’ve been able to follow the advice, the strategy given by
the Buddha in uprooting this sense of self, removing this sense of self. We’ll
be going into this subjectivity and how it arises as the retreat goes on. But
just know that the practice or the duty of the practice is to destroy defilement, destroy the liking and disliking, destroy the attachment and aversion
that is arising at the six sense bases continuously. The ultimate result of our
Vipassana practice is the full removal of dukkha, the full understanding of
dukkha, which allows us to understand the nature of craving and see the cessation of craving and that the path is fulfilled.

\sphinxAtStartPar
So  we’re  going  to  be  using  some  various  tools  on  this  journey  for
understanding the sense of self and removing the sense of self. First of all
we’ll be collecting some information on this
\sphinxstyleemphasis{sutta\sphinxhyphen{}maya pañña}, the type of
knowledge which arises from information. Somebody gives you some information \DUrole{pdfpage}{45}  and you’ll know something about it, something about the object or
something about the subject. Secondly, we use also a bit of thinking, this is
\sphinxstyleemphasis{citta\sphinxhyphen{}maya pañña}, the type of wisdom that arises through thinking. Not the
full understanding of an object – there are various levels of understanding of
things. The third type of wisdom that arises is the wisdom that arises from
pañña, from bhavana,
\sphinxstyleemphasis{bhavana\sphinxhyphen{}maya pañña}. This is the wisdom that arises
intuitively. We’re going to use all three of these types of wisdom. They all
have different levels.

\sphinxAtStartPar
If I was going to tell you about my parents’ house, what it looks like,
then  you’ll  have  some  information. You’ll  understand  to  a  certain  degree
of  what  type  of  house  they  live  in.  It  won’t  be  very  clear  but  you’ll  have
some  understanding. That’s  the  first  layer,  the  first  level  of  understanding
something. Secondly, we normally think about things. If I tell you about my
parents’ house and then you start to think about it. You start to analyze it.
You may come to a bit more of a clearer picture what this thing is, what this
house is my parents are living in. Still, you wouldn’t fully imagine exactly
how  it  looks  like. You  don’t  know  exactly  how  my  parents’  house  is  like.
You have never been there. You have never really examined it. You can look
at it on the map and get all kinds of information. Various people can tell you
about what my parents’ house looks like. You can even have some photos of
how my parents’ house looks like. There are many different ways of collecting information and trying to discover what it is. But until you have actually
been to my parents’ house, you don’t actually know what it truly is.

\sphinxAtStartPar
And it’s the same with the meditation practice. We don’t fully understand the nature of the mind and the body process simply by reading about
it, reading about meditation. We won’t understand the mind and body process  through  thinking  about  it.  Thinking  is  actually  an  object  of  wisdom.
Thinking about something doesn’t lead us to wisdom. Thinking is an object
of wisdom. Finally we get through all the information and all the thinking,
we  can  put  that  all  aside  and  we  can  experience  our  own  mind  and  body
process through our own meditation practice. That’s when we come to a true
understanding about the nature of the mind and body process – this mind and
body process, that’s going on here. We are not so interested in other people’s
mind  and  body  processes.  We’re  interested  in  investigating  this  one  here.
We’re interested in investigating what we know as
\sphinxstyleemphasis{nama\sphinxhyphen{}rupa}.
\sphinxstyleemphasis{Nama}
\DUrole{pdfpage}{46}   means
the mind,
\sphinxstyleemphasis{rupa}
means the matter. Mind and matter. It’s a mind and matter
process that we’ve painted in our perception, in our thinking, so we get a
conceptual idea about what it actually is. However, our concepts don’t truly
cover the nature of our mind and body process. So this week, we’re going to
be investigating this mind and body process. Essentially the mind and body
process are in a vortex, dependent upon each other. When the mind arises,
the body also arises. When the body arises, the mind also arises. So these
things are locked together in a vortex dependently conditioning each other.
We’re going to be looking into this.

\sphinxAtStartPar
Last night we talked about the nature of the body, the four elements.
The  elements  of  extension,  temperature,  movement  and  cohesion.  These
four elements joining together making this physical body. And then there is a
mind as well. There is a mind which has feelings, it has various perceptions,
various thoughts about it – primarily those three things are joined together
in the mind. And then there is consciousness.

\sphinxAtStartPar
Consciousness  is  that  which  knows  the  mind  and  body  process. The
mind  and  body  process  are  spinning,  are  dependent  on  each  other.  Consciousness is like a vessel that allows these things to arise and pass away.
Consciousness is also arising and passing away. In fact, the body, the mind
and consciousness – the five aggregates are arising and passing away every
single moment. Moment after moment. They’re arising and passing away at
the six sense bases. At these six doors the five aggregates are manifesting.
There will be a physical component to our experience, there will be a feeling component to our experience, there will be perceptions and recognitions
about what is going on, there will be some thought, liking and disliking and
subjectification about the experience that’s going on in the present moment.
Our experience which is the experience of consciousness is just simply mind
and  matter.  It  doesn’t  belong  to  anybody.  It  doesn’t  have  an  owner.  It’s  a
mental and physical phenomena arising and passing away according to their
conditioning, arising and passing away according to the law of dependent
arising. And we’ll be talking more on this as the week goes on.

\sphinxAtStartPar
Essentially  what  we’re  doing  in  our  Vipassana  meditation  practice
is  observing  this  mind  and  body  process.  Consciousness  which  is  just  the
\DUrole{pdfpage}{47}  knowing,  it’s  just  knowing,  it  knows  things.  That’s  its  only  function.  It’s
not yours, it’s not you! You’re not the consciousness and the consciousness
does not belong to you. It’s function is to know things. When presented with
something,  it  knows  it.  Depending  on  what  factors  are  surrounding  consciousness it’ll know in a particular way. If our consciousness is surrounded
by  helpful  mental  factors,  then  we’ll  start  to  see  things  as  they  really  are.
When consciousness is surrounded by hindrances, mental states that cloud
the mind and don’t allow us to see things clearly, then that will be our experience. So our meditation practice is to develop certain mental factors, so that
we can surround the knowing with able bodied workers so that they can note
and know exactly what is going on in the present moment.

\sphinxAtStartPar
So we’ll start to dissect our sense experience. First we’ll do it by me
talking to you and then we’ll start to realize for ourselves in our practice.
We’ll go through a few different exercises.

\sphinxAtStartPar
I’ve  said  last  night  that  the  present  moment  is  defined  as  mind  and
matter  which  is  arising  independent  of  desire.  This  is  what  nama\sphinxhyphen{}rupa  is.
Nama\sphinxhyphen{}rupa arising independent of desire. Mind and matter, when it arises
in its natural state, when it is not infected by craving, when it is not infected
by  craving  for  being,  when  it’s  not  under  the  illusion  of  a  self  trying  to
become  something,  it’s  not  trying  to  be  anything,  it’s  not  interested  in  a
self or becoming an ego or developing more sense of I, when the mind and
body process drops this behavior, we start to experience some freedom. We
start to experience the results of Vipassana practice. We start to experience
the world which is beyond defilement, beyond craving arising in the present  moment.  We’ll  be  able  to  see  this  for  ourselves.  This  is  a  well  tested
meditation technique that’s been practiced for a very long time, many, many
centuries. In fact for 26 centuries people have been practicing this meditation technique realizing the benefits of this meditation technique. I hope you
all have a good opportunity to really practice this week and to realize the
benefits of this practice.

\sphinxAtStartPar
As  we’re  practicing  in  this  way,  not  only  do  we  want  to  understand
what is the content of our mind, not only do we want to understand the physical sensations in the body. That’s the
\sphinxstyleemphasis{individual characteristics}
of things. We
know when the mind is happy, we know when the mind is feeling greedy or
\DUrole{pdfpage}{48}  when it’s feeling some aversion. We know when the mind is starting to calm
down and become still. We know when the mind is becoming agitated. These
are the individual characteristics that arise and pass away. This is mental and
physical  phenomena  arising  and  passing  away  in  the  present  moment.  It’s
one thing to understand these. It’s one thing to understand all the different
content but we don’t want to get stuck in content. Our Vipassana practice
goes far beyond the contents of our body and mind process.

\sphinxAtStartPar
What we’re looking at is the structure and the nature of the mind and
body process. There are many thoughts that will arise and pass away. There
will be many different subjects, they’ll be arising and passing away on the
past, arising and passing into the future. Arising and passing away thinking,
imagining, reflecting – all kinds of mental activities will be going on. These
are  all  independent  events  but  they  are  all  joined  together.  They  all  have
one common characteristic, or should we say they have three common characteristics. When we look at them from a structural point of view, they are
all impermanent. It doesn’t matter what the content is, it arises and passes
away.  All  the  content,  all  the  mind  and  body  processes  are  also  dukkha.
We’ll be using this word this week, dukkha, normally translated as suffering
or unsatisfactoriness. It’s a word the Buddha used to describe the unsatisfactory nature of the mind and body process when it’s under the illusion of a
self. When the mind and body process has chosen to believe it is somebody,
when the mind and body process has become subjectified. And this is called
dukkha. The Buddha’s term for it was
\sphinxstyleemphasis{pancha uppadhana kandha vidukkha}.
The five aggregates affected by clinging or attachment is dukkha.

\sphinxAtStartPar
So we’ll be investigating this and seeing that these mind states, as different as they may be, all share
\sphinxstyleemphasis{three common characteristics: they’re impermanent,  they’re  relatively  unsatisfactory  and  they’re  also  non\sphinxhyphen{}self}.  These
mind processes that arise and pass away, the physical body that arises and
passes away, don’t belong to anybody. It’s all happening by itself. In fact,
there is an intricate law of nature that combines things together, that combines  the  mind  and  body  together,  causes  and  effects  together,  to  produce
this result, this ongoing flow, this ongoing manifestation of mind and body
process in the present moment. It’s completely out of control.

\sphinxAtStartPar
So, impermanence, dukkha and non\sphinxhyphen{}self of all conditioned phenomena
\DUrole{pdfpage}{49}  is what we want to get to in our Vipassana practice. At the moment we’ll
start to watch the body, come to terms with that, then we’ll start to have a
look at the mind, come to terms with that. We’ll start to see that sometimes
the  body  is  the  cause,  sometimes  it’s  an  effect.  Sometimes  the  mind  is  a
cause, sometimes the mind is an effect. All these things, causes and effect,
mind and body, all are subject to the same three characteristics. They are all
impermanent, they’re all subject to change, they’re unstable, they’re arising
and  passing  away.  They’re  being  used  as  an  object  in  which  craving  can
enter upon and develop a sense of me. They are the matrix, these mental and
physical phenomena. They are not you! They arise and pass away so rapidly
that you can’t possibly identify as being you once you’ve seen them. You’ll
know  that  that’s  not  you.  That’s  not  me.  I’m  not  that. And  definitely  not
that. You’ll start to see that there is nobody there. Nobody in the mind and
body process! There’s no experiencer! There’s lots of experiencing. Hearing
is occurring but there is no hearer. Seeing is occurring but there is no seer.
Thoughts  arise  and  pass  away  but  there  is  nobody  who  is  thinking  them.
There is no thinker there. It’s all a dependently arisen, conditioned matrix,
that’s arising and passing away. Unfortunately, it has been affected by craving to be. This very strong desire to become has infected the mind and body
process. Each experience is taken as an experience for me, it’s taken as being
mine. This is happening to me. This is where I was born, these are my parents.  The  whole  world  is  subjectified  with  me  at  the  center.  It’s  me,  and
everything else is going around. We’ve created a duality.

\sphinxAtStartPar
Luckily for us, there is a way out of this. We need to be able to see the
structure of our sense experience. We need to understand that it’s all impermanent. We need to see that this is an unsatisfactory state of affairs. This is
an unsatisfactory place, we got ourselves in. When we see the arising and
passing away of mental and physical phenomena, you’ll come to understand
that this is indeed quite an unsatisfactory situation. You’ll come to understand why the Buddha called this dukkha. You’ll understand that this mind
and body process has been arising and passing away for a very, very, very
long time. And it will continue to do so, unless something is done about it.
Unless it can be broken through and seen as it really is. And that’s the purpose of our Vipassana practice. To break through and see that this mind and
\DUrole{pdfpage}{50}  body process doesn’t actually belong to anyone. It’s fueled by its old karmic
intentions. Old karmic intentions producing their resultants. That’s what this
mind and body is. Effectively resultants coming from old karma manifesting
in the present moment believing that they are somebody. And then, a new
karmic resultant has the opportunity to arise and pass away. And then a new
one. And then a new one arises and passes away. It’s incessant! It’s continuous! There are very little gaps.
\sphinxstyleemphasis{Impermanence is hidden from us by what is known as santati or continuity.}
We can’t see the gaps, just like when we’re
watching a film, we don’t see the individual frames of the film. We see the
movie. Actually that movie is made of each discrete individual events and so
is our life made up of exactly this at the six sense doors arising and passing
away so rapidly that we actually never see the frames. We only see the movie
and the movie is my life. The movie is about me.

\sphinxAtStartPar
You’ll  notice  this  week  that  you  spend  a  lot  of  time  thinking  about
yourself. This is not because of the retreat. This is how it normally is. We’re
just noticing it for the first time perhaps. You’ll see that this thing continuously,  incessantly  thinks  about  itself.  It’s  completely  obsessed  with  itself.
It can’t think of anything else to think about. It loves to think about itself.
Reminisces about the past or the traumas of the past or the worries of the
past. Or it’s interested in what its plans are going to be, how it’s going to
become,  what  it’s  going  to  manifest  as.  It  loves  to  think  about  that  stuff.
Diligently planning every detail of where it’s going to be and what it’s going
to  become.  If  it’s  not  doing  those  things  then  it’s  engaging  and  indulging
in enjoyment and delight in sensual pleasures finding little things for it to
enjoy. It finds little nice things to look at, pleasant things to listen to, nice
things  to  smell  and  taste.  When  those  things  don’t  interest  it  anymore,  it
goes and finds some new things. This is what this mind and body is doing.
Constantly  reacting  to  pleasant  feeling  and  unpleasant  feeling.
\sphinxstyleemphasis{Dukkha  is}
\sphinxstyleemphasis{hidden from us by the body posture.}
By simply not paying attention to the
body, not paying attention to the posture, we don’t see the nature of dukkha.
Dukkha is arising in this mind and body process continuously. Just try holding your body still for an hour. Any posture, it doesn’t matter, sitting, standing, walking or lying down. Try to do all, try to do one for an hour and see
what  happens  to  the  body.  See  if  you  can  feel  any  unpleasant  sensations
\DUrole{pdfpage}{51}  arising in the body. It’s continuously arising. This thing is a manifestation
of unpleasant physical sensations. That’s why we have to keep moving it.
That’s  why  we  have  to  constantly  transition  between  these  four  different
postures.  Walking,  standing,  sitting  and  lying  down.  Because  the  dukkha,
the painfulness, just keeps coming. Hold it like this for while. It becomes
too much, you have to move your leg. You sit down, you have to stand up.
Even when you’ve been lying down, even that becomes uncomfortable. You
have to sit up, you have to stand up, you have to walk a little bit. If you have
been walking, you have to sit down again. Constantly transitioning between
these four postures. This is what hides the nature of dukkha from us. By not
paying  attention  to  our  posture. When  we  don’t  see  the  posture,  when  we
don’t internalize our awareness in the present moment, we don’t see what’s
happening. We just think we’re moving our shoulders, just make ourselves a
little more comfortable, lean a little bit, do this, lean that way, find a wall to
lean on and slumping our shoulders. Finding any kind of comfortable position, see how long you can keep it and then, dukkha manifesting. You have
to move again! Constantly moving.

\sphinxAtStartPar
\sphinxstyleemphasis{The third characteristic of non\sphinxhyphen{}self is hidden by compactness or ghana\sphinxhyphen{}}
\sphinxstyleemphasis{saññā, perception of compactness.}
It appears like this mind and body process
is all joined together. It’s all pacted. We can’t see beyond it. Just like the car
down their in the car park. We don’t see beyond «car». It’s compacted. We
just see one thing. «Car». That’s it. We don’t bother to investigate. It is actually different car parts. Actually, it’s only rubber, glass, metal and plastic. We
don’t investigate that. We don’t see that. We don’t see the conceptual objects
and don’t see it in our own mind and body process. The fact that things are
separated is hidden from us because of this swiftly, constantly flowing river.
There is no gaps in it. Compacted together.

\sphinxAtStartPar
So these three characteristics of our structural experience is what Vipassana  practice  is  all  about. You’ll  intuitively  know  impermanence,  dukkha
and non\sphinxhyphen{}self. You’ll know them as dependent arisen and conditioned. When
we don’t take any as self then we are starting to free the mind. That’s the real
purpose of our meditation practice. Once we’ve removed the concept of personality, of this distorted view of individuality, we’ll understand that this is
actually the seed of all our problems. This is where our problems come from.
\DUrole{pdfpage}{52}  You’ll start to notice that actually when I’m being really selfish and I have a
very strong sense of self, there is quite a lot of dukkha associated with that.
Lots of self, lots of dukkha. Reduce your sense of self, reduce your dukkha.
A little bit of self, a little bit of dukkha. No self, no dukkha. The cessation
of  dukkha  is  completed  when  the  sense  of  self  is  fully  removed.  Dukkha
is dependent on me, mine and I. Dependent on this sense of self arising in
the present moment. Our conceptions of «I am» are based upon this present
moment  experience  continuously  multiplying  moment  after  moment  after
moment. It becomes so continuous, it flows so uninterruptedly, so mechanically that we don’t actually see what’s going on. Luckily 26 centuries ago
the Buddha saw what was going on and he was able to teach what he came
to know for himself. Not only that people that he taught were also able to
understand  the  nature  of  reality  and  they  also  became  enlightened  beings.
And  so  this  tradition  has  passed  down  to  us. There  are  still  people  on  the
planet who are enlightened. They had followed the teaching and completed
the  training.  I  have  met  them,  enlightened  beings  following  the  teaching.
This  is  our  opportunity  to  practice  this  wonderful  teaching,  to  realize  the
true  nature  of  the  mind  and  body  process,  to  stop  identifying  with  it. The
stuff is still going to be arising. Your old karma is still going to manifest in
this live. Mind and body process will still keep bubbling away, but the more
you can remove your sense of self, the more pleasant you’ll find the whole
experience.  If  you  can  completely  remove  your  sense  of  self  from  nature,
from this mind and body process that is arising and passing away, then you’ll
have found a state of freedom. A state of peace. A state of comfort. A state of
great bliss. A state of loving kindness and compassion for other beings. And
this is where our Vipassana practice is heading.

\sphinxAtStartPar
You will need to obtain a level of concentration, you will need to be
mindful  continuously  from  moment  to  moment.  It’s  through  continuous
mindfulness, continuous attention to what is actually going on in the present moment that concentration arises. If we can actively note what’s going
on and we do that continuously, we are aware, we’re witnessing, observing,
we’re paying attention but we’re not interfering with what’s going on. We’re
just watching. We are the detached watcher. Just being attentive observing.
Witnessing is taking place. This is the power of mindfulness and wisdom.
\DUrole{pdfpage}{53}  We’ll be talking more about them as the week goes on.

\sphinxAtStartPar
So we are establishing our awareness in the present moment, we are
activating our awareness continuously. In the beginning stages of the practice you’ll need to do this very often. Your mind will wander here and there.
You go into the past and future thinking about various things. You will need
to  reactivate  your  awareness  and  bring  it  back  into  the  present  and  know
what is the object of consciousness. Keep bringing your awareness back into
the present and seeing what’s there. That’s what our practice is all about. If
you  can  do  this  continuously,  keep  bringing  yourself  back  to  the  present,
keep bringing yourself back, what happens is that the mind stabilizes in this
mode of awareness. In the beginning stages that’s difficult just to get into it
for a few seconds, just to get into this awareness, that you’ll be there and
that you’re present with what’s going on. And then your mind will run away
again. You’ll follow the breath a little bit, you can be there with it and then
your mind will run away again. This is very normal. Anyone who has trained
a dog will know. You have to train it. It will take some patience, it will take
some effort but eventually the dog understands that it has to go and piddle
outside. You bring it in, it starts to piddle, you put it outside again. It comes
back inside, it gets comfortable, it starts a little piddle, you take it outside
and it does it out there. Eventually it knows that it has to go outside to go to
the toilet. In our meditation practice as well, if we continuously keep averting our intention and the awareness to the present moment, our mind starts to
learn, starting to train our mind to come back into the present. At the beginning it’s difficult, but after a while, we get used to it.


\section{Noting, Knowing, Letting go}
\label{\detokenize{1-a:noting-knowing-letting-go}}
\sphinxAtStartPar
We activate our awareness. What’s there? We note it! When I say «we
note  it»  that  means,  we  just  make  a  recognition  in  our  mind. And  this  is
what’s there. Just that! We are just noting it.

\sphinxAtStartPar
The second part of our operation is, we know the object. We note what’s
there and then we know it. We know it in a very special way. We know it as
an object. We see it for what it actually is. We know it as an independent,
discrete  event  that’s  occurring.  It’s  either  a  mental  event  or  it’s  a  physical
event. It’s impermanent, we know it’s arisen, it’s there, we watch it and it
\DUrole{pdfpage}{54}  passes away. We understand the nature of it. This is a conditioned, dependently arisen object, it’s out of control. It has arisen when its conditions are
in place and it has passed away when those conditions are no longer there.
It’s an independent, dependently arisen, conditioned mental or physical phenomena. It has probably been subjectified. It has probably been identified as
being me and mine. And so dukkha arises in that moment. If our awareness
can be sharp enough, if we can note and know the object so that craving for
being doesn’t have a chance to enter into the moment, then that’s a moment
of freedom. That’s a moment of cessation of dukkha. We can free the mind
in that moment.

\sphinxAtStartPar
So what we do is we note whatever’s there, we know it – by knowing
I  mean  we  step  back  from  it,  disengage  from  it,  stop  identifying  with  it,
we see it just as it is, put a little fence around it, if you like, or build a wall
around it or put it up on a high shelf, there it is, it’s not me, it’s not mine, it’s
not I. It could be «hearing». It’s not «I am hearing». There is some hearing
occurring. When we look at our physical sensations, normally, we look at it
from a subjective point of view. «My painful knee». There is some physical
sensations that is there. That’s all it is. There is some unpleasantness that’s
there, that’s all it is. There is a physical object, there is a mental object and
there is a subjectification of that. My painful knee. Three things are going
on there. We can note them. We can note and know them so that our mind
doesn’t get trapped, that it doesn’t get engaged, that it doesn’t get immured
with the object and it passes away. This is how the letting go process takes
place in the present moment.

\sphinxAtStartPar
We activate our awareness, we note what’s there, we step back from it,
not engaging with it in any way, we’re not thinking about it or analyzing it,
we are just stepping back from it and then we watch it pass away. It arises,
we note it, we step back and it passes away. It arises, we note it, we step back
and it passes away. We do this over and over and over again. This is what
our  meditation  practice  is  all  about.  Freeing  the  mind,  freeing  consciousness from the mind. Freeing consciousness from the mind and body process.
Consciousness, which is like a bowl; mind and body are like the fruit that
is  sitting  in  the  bowl. Without  a  bowl  the  fruit  can’t  have  a  sit. You  can’t
have a bowl that is just empty. It’s got to have something in it. So they’re
\DUrole{pdfpage}{55}  in kind of a relationship, this bowl and this fruit. They have to go together.
Consciousness is the bowl that allows the fruit to be there. The fruit can only
be there if there’s a consciousness. So they are in this symbiotic relationship
causing and effecting each other. Mind and body sometimes is the cause of
consciousness.  Consciousness  is  sometimes  the  cause  of  mind  and  body.
They are arising and passing away together. We need to be able to see this
in real\sphinxhyphen{}time.

\sphinxAtStartPar
To do this we need to be able to note and know and let go. Note, know
and let go. Note, know and let go. Note, know and let go. And we need to
do it continuously. We need to do it rapidly. The more rapidly you can do
the practice, the quicker you’re letting go will take place. You will be able
to see things more and more clearly. In the beginning it takes a lot of effort
to  put  forth  this  noting,  doing  this  noting,  being  aware  of  exactly  what  is
the  content  of  your  consciousness  right  now. What’s  happening  now:  Is  it
hearing? Is it seeing? Is it smelling? Is it tasting? What is it now? What is
happening now? Is it hearing? Is it seeing? Smelling? Tasting? Touching?
The five sense doors are constantly stimulated by the environment. If there
are  sounds,  hearing  occurs.  If  there  are  visible  forms,  seeing  occurs.  The
body is touching this mat, touching is occurring! The body is feeling a little
bit warm, warmth is occurring there! Vibration is there. These physical and
mental phenomena are going on all over the place continuously arising and
passing away. If we are aware in the present moment, we can block craving
from entering the moment and there is a moment of freedom.

\sphinxAtStartPar
If we’re unaware of the moment, then craving will enter the moment
and  cause,  what  the  Buddha  called  dukkha,  this  unsatisfactory  state  of
affairs,  where  the  mind  and  body  process  have  become  subjectified. They
think they are somebody. Trapped! We are trapped in a sense of self. We are
trapped in the matrix. Luckily for us, if we can note and know and let go and
keep noting and keep noting and keep noting, moment after moment after
moment, moment after moment, what happens, is this mode of perception
stabilizes. The  mind  becomes  concentrated  and  starts  to  see  things  in  this
mode more frequently, more commonly, more often, continuously. The curtain opens and we start to see what is really going on in the present moment.
We see things as they really are. The letting go mechanism takes place and
\DUrole{pdfpage}{56}  we are free in that moment. All we have to do is extend that moment out,
keep extending our awareness in the present, keep extending our wisdom,
which de\sphinxhyphen{}identifies with things. Being aware in the present moment is just
not  enough. Awareness  will  take  us  to  the  present  and  we’ll  know  what’s
there but we need wisdom as well. Wisdom allows not to get engaged with
the object. We note it and we are right there with it. Wisdom is that which
steps back. So we are noting and we are knowing. The result of this is letting go. We are noting, knowing, letting go. Whatever physical phenomena
is  arising  in  the  body,  make  a  note  of  it.  Whatever  mental  phenomena  is
arising, make a note of it. Whether it’s an emotional state or a judging state
or complaining state or if it’s pleasantness or if it’s unpleasantness or if it’s
hot or if it’s tired, if it’s hungry, what ever is going on. Vipassana meditation
is paying attention to just what is. We are not trying to create anything special. We are trying to observe nature as it bubbles up. We want to see it as it
comes up completely undisturbed. We are not trying to manipulate the mind
and  body  process. We  are  trying  to  let  it  go. We  are  trying  to  completely
and totally detach from it, so it can just do its thing without the sense of me
being in there.

\sphinxAtStartPar
When  you  experience  that  for  yourself,  you  will  experience  a  very
large amount of bliss, you will be very happy, very pleased and very free.
This is where our meditation practice goes.

\sphinxAtStartPar
We are not having reactions to the objects, we are not trying to like or
dislike them. We are just allowing them to arise and pass away. We are not
judging things as being good or bad. We are not comparing things. The mind
often likes to compare. It likes to go into the past. To think about how it was
then and how it is now. It likes to compare with the future. How much better
it’s going to be in the future compared to what it is now.

\sphinxAtStartPar
So just be aware of whatever is occurring without any entanglements.
We are going to follow the breath and use that as a conceptual framework to
keep us in the present as we observe what is going on. Watch things as they
occur. As our mindfulness gradually becomes more constant, more continuous and more powerful, the mind starts to stabilize in various levels, in various stages. Depending on the stage of stability, on the level of concentration
you will see things as they really are up to that level. The more you can let
\DUrole{pdfpage}{57}  go of, the more stable your awareness will become. So it will become concentrated. The mind will enter a concentrated state of samadhi. It will be able
to see things as they really are.

\sphinxAtStartPar
\sphinxstyleemphasis{The most important thing to be aware of is whatever object is arising}
\sphinxstyleemphasis{in  the  moment.  That’s  the  most  important  thing  to  note.}
If  you  are  unsure
of  what  you  need  to  be  noting,  whatever  is  there  is  the  answer. Whatever
is  arising  that’s  what  we  pay  attention  to. We  are  not  trying  to  create  any
special object. Just whatever is there. Activate your awareness in the present.  What’s  there?  What  are  you  thinking  about?  What  is  your  emotional
state? Are you feeling happy? Is there pleasantness in the moment or is there
unpleasantness in the moment? Is there some physical sensations occurring
in the moment? Is there any emotional state occurring? Are you looping on
something? Are  you  thinking  about  something? Are  you  becoming  angry
about something? Are you disliking something? Have a look at that! Maybe
you’re becoming happy about something.

\sphinxAtStartPar
Our mindfulness meditation is the practice of non\sphinxhyphen{}reaction. We are not
reacting, we are not judging. Mindfulness is a process which is not judging.
You  are  not  calling  things  as  good  or  bad. That  is  a  reaction  of  the  mind.
Things  are  as  they  are!  They  are  perfectly  fine!  Of  course,  we  can  make
trouble by saying this is good or this is bad. Things are as they are. So our
practice is just to be aware of whatever is occurring.

\sphinxAtStartPar
We  can  listen  to  something.  We  can  hear  those  insects  in  the  background there. We can listen. That involves somebody who is listening. Hearing doesn’t involve anybody. Hearing is occurring. Hearing is a natural state.
There  is  holes  in  your  head,  there  is  sound  outside.  Ear  drum  and  sound
meeting together, ear consciousness arising. It doesn’t belong to anyone. It’s
not you. It’s not yours.

\sphinxAtStartPar
Why  are  you  identifying  with  it? Why  are  you  becoming  upset  with
it? «Oh, that sound is happening to me!» Hmm, you have appropriated that
sound! You are unable to note it in the present moment, ’hearing, hearing’,
that is what is actually happening right now. Hearing process is activated.
So  we  are  aware  that  hearing  is  occurring. When  we  understand  it  as  just
hearing, hearing is just hearing. It’s not «I am hearing something» or «I am
listening». When we can see it as just hearing, only that, then we have freed
\DUrole{pdfpage}{58}  the mind from the subjectivity that we have become involved with – if we
can note and know fast enough, if we can keep noting continuously to block
this sense of self from getting into the sense perception process.

\sphinxAtStartPar
It is constantly trying to enter. Craving is constantly there, swirling like
a cloud in our present moment experience. It wants to be. This is its very
nature.  Craving  really,  really  wants  to  be  somebody.  It  really  wants  to  be
something. Think about it! It is all you are thinking about. «I want to be this,
I want to study this, I want to study that, I want to be this person, I want to
be that, I want to have this, I want to have that job title, this car, this amount
of money.» It wants to be. It doesn’t matter, unfortunately for us, it doesn’t
matter what it wants to be. It attaches to both pleasant and unpleasant experiences. It gains the sense of self from the pleasant experiences, «I am happy,
this is great», and it gains a sense of self from the unpleasant experiences.
«This is awful, what am I doing here».

\sphinxAtStartPar
All of those thoughts, when they are unseen, when we are unmindful
of them, when we haven’t been able to note and know in the present moment
exactly what that mind state is, we get sucked into it! We get sucked into it
and start identifying with it. The pleasantness is no longer just pleasantness.
It is pleasantness for me. The unpleasantness, it’s just unpleasantness. Oh,
there is some unpleasantness arising. If we note it and know it, we can free
our mind from that unpleasantness. If we don’t see it fast enough, we will
get sucked into it. Absorbed into it. That unpleasantness will be happening
to me. It will be happening for me. And so we dukkher ourselves and bring
dukkha  into  the  moment  through  unawareness. When  we  have  awareness,
dukkha ceases.

\sphinxAtStartPar
When we have no awareness, dukkha is arising. It’s two sides of the
coin. You can choose how you want to live your life in the present moment.
If you want to be aware, mindful and fully aware of what your mental state
is, so that you can note it, know it and let it go, then you can live in freedom
from this onslaught of craving to be. If you don’t want to do that, if you just
want to let it come into every sense experience that you have got, create a
sense of self, prepare yourself for the ups and downs of life, the vicissitudes
of life. Lots of unpleasantness, lots of pleasantness. You’ll be swirling like in
a washing machine. Swirling around. Swirling around by being conditioned
\DUrole{pdfpage}{59}  by pleasantness and unpleasantness if you are unable to note it, know it and
let it go.

\sphinxAtStartPar
Our  meditation  technique  this  week  is  designed  to  understand  the
nature of our mind. To see how it is functioning, real data in real\sphinxhyphen{}time. And
you’re not just listening to me telling you about this, you’re going to practice  it. We  are  going  to  witness  it  for  ourselves. The  Buddha’s  teaching  is
not something you believe in. It is not a belief\sphinxhyphen{}system. The Buddha never
asked us to believe something. He asked us to investigate for ourselves and
to understand for ourselves. That is what the Buddha’s teaching is all about.
It’s about understanding. He never says, hey, come and believe me. He says,
have a look at that and understand it. He never asked us to believe anything.
He asked us just to do the practice and experience it for ourselves. You will
all be able to experience some level of dhamma on this retreat. To some level
or another, you will all experience some aspect of nature that is arising and
passing away in your mental and physical process.

\sphinxAtStartPar
If we don’t observe what is going on, when we see something or when
we hear something, then we are very quick to react. We normally react with
aversion  or  attraction,  attachment.  If  it  is  pleasant,  we’ll  like  it,  we  want
it, more. If it is unpleasant, we dislike it, we push it away, we get rid of it.
This is what we do. If we are unmindful, we just go through life, processing
things in this way. Unpleasant and push away. Pleasant, more. Unpleasant,
push away. Pleasant, more. If we are surrounded by unpleasantness, then we
will have to work out a way to try to escape by searching for something pleasurable. Trying to open the fridge door, get some food out, put the music on,
find some kind of pleasurable distraction to take us away from the present
moment. Up until now, our only way to be dealing with any dukkha\sphinxhyphen{}states
has been through sensuality. Trying to find something nice to eat, something
nice to look at, something nice to listen to, something nice to touch of clothing, touch of another person. We go searching for those things. Drugs, for
alcohol, for sex, for food, gambling. We search for things to pull us out from
our dukkha.

\sphinxAtStartPar
Dukkha is manifesting in the present moment continuously. And this is
the first noble truth. The Buddha said, there is dukkha. He’s doesn’t say, you
are dukkha. He doesn’t say, your life is dukkha. He doesn’t say, the world
\DUrole{pdfpage}{60}  is dukkha. All he says is, there is dukkha. There it is! Right there! Where?
Here! This thing. This is where dukkha is arising, and this is where dukkha
is ceasing. Only here! Dukkha is not out there. Suffering, the Buddha is talking about, is not outside. He is talking about this thing here. So let’s have a
look inside. See if you can see this dukkha arising in the present moment. If
we can see the dukkha that is arising in the present moment, we will also see
the craving that produces that dukkha and also see where that dukkha ceases
to exist. It’s called walking the noble eightfold path and we’ll be doing that
this week.

\sphinxAtStartPar
When we see things that we like, when we hear things that are pleasant,
we just take those things as being good things. We like them, we love them,
we want to get them, we want to hold them, we want to keep them, we want
to become the owner of them. That is mine, that is happening to me. If it is
not mine, it’s not good enough. Living in a fancy house, wow, it would be
great, if we owned that place. Why? What is great about owning it? Anyway
you get to live in it already, why do you need to own it? Why will owning
be any better? It’s just the story of your self. The story of me. It wants stuff,
it clings and tries to hold on to stuff. It tries to accumulate things. It’s constantly seeing what the mind, what craving is doing. It’s just trying to build
an  identity  for  itself.  It  is  very  effective,  it’s  very  skillful,  it  does  so  very
efficiently. It’s creating a personality, building an identity for itself and each
moment  is  used  as  a  little  backup,  a  little  prop,  a  little  crutch  to  keep  the
show rolling. To keep this mirage of a person rolling on. It keeps having to
identify to things in the present moment to keep backing it up. To keep backing up the story that this is somebody. That there is somebody here.

\sphinxAtStartPar
Vipassana  insight  blows  this  apart. When  we  see  things  clearly  with
insight,  we  understand  that  things  are  impermanent,  things  are  unsatisfactory and things are ultimately out\sphinxhyphen{}of\sphinxhyphen{}control and non\sphinxhyphen{}self. They don’t belong
to anybody. It’s not happening to anyone. It’s just happening. How it happens, we will talk about as the days go on. Why it’s happening, we’ll talk
about as the days go on.

\sphinxAtStartPar
For our purposes here today, it’s happening. All we have to do is observe
it. We have to watch it. We are watching the physical sensations in the body.
Really pay attention to the movement inside the body when the breath goes
\DUrole{pdfpage}{61}  in and out. Don’t pay so much attention to the breath. Pay a lot of attention
to the sensations created by the breath. What does it feel like to be inside the
body knowing those physical sensations? So just be mindful of whatever is
arising in the present moment. That’s what our practice is all about. We are
noting, we are knowing and we are letting go moment after moment after
moment. And this leads us to a state of concentration, a stabilized mode of
perception in which we start to see things as they really are. When we see
things as they really are, the mind can no longer build a sense of self from
that object in that present moment. If you see something clearly with Vipassana insight, it’s impossible for consciousness to build a story and identify
with that object. We have seen it clearly. Clear seeing has taken place. You
will stop identifying with that physical sensation. It’s not me, mine or I. You
will  stop  identifying  with  that  thought  process,  that  habitual  craving  that
you may have. You will stop identifying with some aversion state that often
arises. You will stop identifying with many different things if you can keep
your awareness in the present moment, noting, knowing and letting go. So
that is what our practice is all about this week.

\sphinxAtStartPar
We are going to do some walking meditation now. Just to remind you
about the walking meditation. We find a place outside, nice and calm, come
to attention, standing, become aware of the standing posture, feel the feet as
they are touching the floor, touching the sand or wherever you are standing,
come inside the body, lower your head, keep your mind inside the body, don’t
allow it to go wandering into the jungle, wandering on to somebody’s body.
Keep your mind inside and start gently making movements lifting, moving,
placing. Lifting, moving, placing. And try to walk from one end to the other
end. It can be 12 or 15 steps, lifting, moving, placing, lifting, moving, placing. Try to take each one as a single event. Lifting. Finished. Moving. Finished. Placing. Finished. Lifting. Finished. Moving. Finished. Placing. Finished. Dividing your experience up into discrete individual moments where
you can note and see what is happening in that present moment experience.
We are going to start to divide the stream of your life up and start just to
divide and conquer, if you like, the flow of mental and physical phenomena
arising and passing away extremely rapid up until now. Your life has very
\DUrole{pdfpage}{62}  successfully  fooled  you,  believing  that  you  are  somebody. This Vipassana
is going to remove this false view so that you can start to live in a lot more
freedom and a lot more peace and a lot less dukkha.

\sphinxstepscope


\chapter{Day 1, afternoon}
\label{\detokenize{1-b:day-1-afternoon}}\label{\detokenize{1-b::doc}}
\LOCALaudiolink{https://www.mixcloud.com/anthonymarkwell/day-1-afternoon-talk/}

\sphinxAtStartPar
This morning we had a look at some aspects of Vipassana meditation
practice. Last night we had a look at the formal instructions for the sitting
meditation practice and for the walking meditation practice. This afternoon
we are going to have a look at the third main section of our meditation practice here at the monastery, which is awareness of our daily activities.


\section{Awareness of daily activities}
\label{\detokenize{1-b:awareness-of-daily-activities}}
\sphinxAtStartPar
The  importance  of  continuing  our  mindfulness  practice  between  the
periods  of  formal  walking  and  sitting  meditation  practice  cannot  be  overemphasized  or  stressed  enough.  In  fact,  one  of  my  teachers  used  to  say,
awareness of the daily activities is the heart and soul of the meditator.
\sphinxstyleemphasis{Fail\sphinxhyphen{}}
\sphinxstyleemphasis{ing to observe an activity during the day, one ceases to be a meditator.}
Without constant awareness during these periods, there is little chance of developing  enough  stability  or  concentration,  that  will  allow  wisdom  to  unfold
naturally by itself. It is really the difference between those meditators who
see the dhamma, who see the truth for themselves inside their own mind and
body, and those who don’t.

\sphinxAtStartPar
Walking meditation practice and sitting meditation practice needs to be
backed up by awareness during our daily activities. During our meal times,
during our chore times, while we are having a rest, while we are taking a
\DUrole{pdfpage}{64}  bath, while we are walking to the viewpoint or we’re having hot chocolate,
all of those things. We need to keep some awareness about us. We need to
keep our awareness in the present moment so that when we do sit, when we
do walk, we can go quite quickly into the present moment. We go faster, we
go deeper. Our meditation becomes more consistent. Our practice is more
continuous.  So  I  urge  and  encourage  you,  during  the  periods  when  we’re
not walking and sitting, please try to maintain your awareness in the present moment, internalizing your awareness, bringing yourself back inside the
body as much and as often as you can. This will be very beneficial for your
sitting meditation practice. If you can maintain your awareness in your daily
activities and in the walking meditation, if you can do that really well, when
it  comes  to  your  sitting  meditation  your  mind  will  drop  in  quite  quickly.
More quickly than it would have before.

\sphinxAtStartPar
It is simply cause and effect.
\sphinxstyleemphasis{Vipassana insights are conditioned phe\sphinxhyphen{}}
\sphinxstyleemphasis{nomena.}
The causes and conditions need to be continuous. We need to keep
rubbing those sticks. Don’t let there be any breaks. If you’re letting there be
breaks  in  your  stick  rubbing,  then  the  sticks  cool  down. The  friction  isn’t
created. There’s not enough warmth to start a little fire by rubbing the sticks
together. Our meditation is the same. We need to be as continuous as possible. The  faculty  of  mindfulness  becomes  powerful  and  constant  through
uninterrupted noting of our daily activities. If we fail to do this, there will
be  wide  gaps  in  our  mindfulness. The  schedule  is  like  it  is  to  allow  us  to
practice  as  continuous  as  possible.  The  largest  gap  throughout  the  whole
day, apart from sleeping time, is two hours between sittings. We try to keep
you  sitting,  then  walking  or  daily  activities. Try  to  maintain  continuity  in
your meditation practice. There are many new things to discover when you
manage  to  do  this.  Doing  things  slowly,  will  help  us.  Slow  down  into  the
present moment. We perform our activities slowly. It helps us to observe.

\sphinxAtStartPar
The Buddha has given us seven different sections under the heading of
\sphinxstyleemphasis{sati\sphinxhyphen{}sampajana}
or  mindfulness  and  clear  comprehension.  These  are  given
in the satipatthana text under the section of body. That’s where we start our
meditation practice. We are becoming aware of the physical sensations created in the body when the breath flows in and out. The physical sensations
are  one  type  of  body  experience. We  are  also  paying  attention  to  the  postures \DUrole{pdfpage}{65}  of the body. The walking, standing, sitting and lying down. And we’re
paying attention to the breath that flows in and out. So we’re paying attention
to the body in various ways.

\sphinxAtStartPar
During  the  walking  meditation,  we  particularly  pay  attention  to  the
soles of the feet and our legs as we’re walking back\sphinxhyphen{}and\sphinxhyphen{}forth.

\sphinxAtStartPar
During the daily activities our main focus of concern is the whole body.
The whole posture of the body. We’re paying attention to the whole body and
understand, when it’s sitting, we know that it’s sitting. When it’s walking, we
know that it’s walking. When it’s standing, we know that it’s standing. And
when it’s lying down, we know that it’s lying down. We understand that the
body transitions between those four postures throughout the day and we are
observing that. Always noting, keeping our awareness inside, «Okay, now
it’s standing. Now it’s walking. Now it’s sitting.» Have a look! In our daily
activities we use this mindfulness and awareness to observe the postures that
are occurring in the present. Just be aware how your body is disposed at the
moment. We bring our attention to the body, inside, make sure our awareness doesn’t leak outside through the eyes and the ears. We keep bringing our
attention back to the body as we move around through the monastery. Try to
keep your awareness inside. Try to keep present.

\sphinxAtStartPar
If you notice that your mind is wandering into the stories of the past,
make  a  note  of  that,  ‘remembering,  remembering’.  Step  back  from  those
stories! See that the mind is wandering around in the past. There are good
memories and there are unfortunate memories. Sometimes we can get upset,
sometimes we get happy when these memories come up. Just make a note
of them and step back from them. Stop identifying with them. Allow them
to  arise,  we  don’t  get  upset  by  them,  but  just  make  a  note,  «this  is  what
this actually happening in the present moment». There is some remembering
going on. And that’s it!

\sphinxAtStartPar
Maybe your mind goes into the future a lot. Maybe you are a planer.
Planning is occurring. Just make a note of that. Be sure to step away from it.
If you find yourself getting caught in the stories of your mind, make
a note, you are getting caught. We go into the past all the time. We go into
the future all the time. It’s okay. The mind does this. This is its normal state.
But we just want to watch it and observe it. We need to keep up with it. We
\DUrole{pdfpage}{66}  need to be able to note and be aware what kind of mental states are arising
in the moment. And we step back from them. We note them, we know them
and we let them go.

\sphinxAtStartPar
We  note  when  the  body  is  standing.  There’s  many  opportunities  for
standing during the retreat. We stand and wait for food sometimes. We stand
and wait to come into the door sometimes. We stand on the walking meditation path. If our sitting becomes uncomfortable, we can stand up for a few
minutes  and  then  sit  down  and  again.  Try  to  catch  all  of  these  standings.
When you’re standing in the shower, at the bathroom. Be aware of that. Try
to maintain continuous awareness of what body posture you are standing in.


\section{Wise attention}
\label{\detokenize{1-b:wise-attention}}
\sphinxAtStartPar
Use the faculty known as
\sphinxstyleemphasis{yoniso manasikara}
or wise attention. It’s a
mental state. Wise attention is the ability to have a look at the reason. Why?
Why are we doing the things that we are doing? Why are we standing now?
Why are we sitting now? Why are we walking? Why are we lying down?
What is the purpose of this? We start to examine why we go into those four
postures and why we come out of those four postures. Why is it necessary
to  continuously  change  posture? Why  do  we  have  to  keep  moving  all  the
time? After  we’ve  been  walking,  we  have  to  sit  down. After  we’ve  been
sitting for a while, we have to stand up. When we have been standing, we
want to lie down. Been lying down for a while, we have to stand up again.
Try to examine your own body and see the reason, why we change postures
all the time. This mental state, this ability to see the reason why, to see our
own motivations, to see our own intentions is a mind state that leads us out
of samsara. It’s a very important mind state in the practice of the Buddha’s
teaching.  Yoniso  manasikara.
\sphinxstyleemphasis{Manasikara}
means  attention.
\sphinxstyleemphasis{Yoniso}
means
‘coming before’, attention as to ‘before states’. Attention to reasons. Reasons why we do the things that we do.

\sphinxAtStartPar
The Buddha teaches us,
\sphinxstyleemphasis{«there is suffering»}. It’s the first noble truth.
There it is! He says,
\sphinxstyleemphasis{«I cannot see any single dhamma, that leads more surely}
\sphinxstyleemphasis{to right view then this one.»}
If we want to attain
\sphinxstyleemphasis{right view}, the first factor
of the noble eightfold path, we want to see things as they really are, we’re
going to need to employ, we’re going to need to develop yoniso manasikara.
\DUrole{pdfpage}{67}  Sometimes we have a reason, sometimes we know why we do what we do.
It’s  not  a  completely  undeveloped  faculty.  It’s  very  under\sphinxhyphen{}used.  We  don’t
often challenge ourselves for the reasons why we’re doing what we’re doing.
When we start to have a look, we will start to understand what motivates us.
We start to see it’s the sense of being. This craving to be is manipulating the
mind and body process. So it gets what it wants. It does what it can always
trying to take some advantage.

\sphinxAtStartPar
The Buddha also says in the Sabasava sutta:
\sphinxstyleemphasis{«The destruction of the
taints is for one who knows and sees, not for one who does not know and}
\sphinxstyleemphasis{does not see.»}
Does not know and see what? – Yoniso manasikara. When one
attends unwisely, unarisen defilements arise and arisen defilements increase.
When one attends wisely, unarisen defilements don’t have a chance to arise
and arisen defilements cease. So combined with
\sphinxstyleemphasis{right effort}
– right effort is
the  effort  to  remove  defiled,  corrupted,  imperfected  mind  states  of  greed,
hatred  and  delusion  –  when  we’re  trying  to  remove  these  from  the  mind,
to note, know them and let them go, it’s very important that we use yoniso
manasikara, this wise attention. It will bring us to the reasons of why we are
trapped in the mind states that we are trapped in. We’ll start to understand
what motivates our suffering, while we see suffering, while we experience
suffering.  So  the  abandoning  of  sensual  desire,  the  abandoning  of  being,
the  abandoning  of  ignorance  –  these  three  taints  or  these  three
\sphinxstyleemphasis{asavas}
–
are removed through yoniso manasikara. Unwise attention is the root of the
rounds of samsara.
\sphinxstyleemphasis{Not paying attention to the present moment and to why}
\sphinxstyleemphasis{we’re  doing  what  we’re  doing  is  the  fuel  of  samsara.}
That’s  why  we  keep
manifesting as mind and body process in various places and various ways.
Wise  attention  is  at  the  root  of  liberation.  Liberation  from  the  rounds  of
samsara. And it leads to the development of the noble eightfold path. In fact,
the venerable Sariputta answers the venerable Mahakolita’s question about
right  view  by  saying:  «Friend,  there  are  two  conditions  for  the  arising  of
right view. The voice of another and wise attention. These are the conditions
for  the  arising  of  right  view.»  So  we  can  start  to  understand  things  from
listening,  like  we  do  today.  But  we  come  to  an  even  better  understanding
of things if we start to investigate for ourselves. When we start to see really
what is going on in this mind and body creation. The arising of right view,
\DUrole{pdfpage}{68}  the destruction of defilements is all conditioned by this activity called yoniso
manasikara.

\sphinxAtStartPar
So we use this yoniso manasikara in our daily activities, combining it
with the four postures of the body. We use it, we pay attention to it. What
is it? It’s mental advertency. Turning your mind! We are turning our focus
on to the reasons, our motivations. We’re not just blindly following on what
our mind wants to do. We start to have a look at it. Have a look at the reasons! Why? Are we guided by some selfish idea? Are we being guided by
an  unselfish  idea? Are  we  guided  by  something  wholesome  or  something
unwholesome? What is motivating us?

\sphinxAtStartPar
Wise attention or proper attention, adverting our attention, adverting
our mind so that we can see causes. We want to see causes! We are looking
at the effects in our meditation practice. The effects are obvious to us when
awareness and wisdom are present. That’s what we are noting. The effect of
things as they are arising and passing away. Yoniso manasikara goes a little
deeper. It sees the causes, the reasons and the motivations for why we do
what we do. Unwise attention can lead us to believe things are all permanent, things are all satisfactory and that all belongs to me, it’s all self. This is
\sphinxstyleemphasis{vipalassa sañña}
or a perversity of perception. Vipassana gives us exactly the
opposite experience. Things are impermanent, unsatisfactory and non\sphinxhyphen{}self.

\sphinxAtStartPar
We change the body. I would like you to really have a look at this on
the retreat. Have a look why we are changing the body all the time! We are
moving it around. Why do we need to adjust our shoulder sometimes? Why
do we need to wiggle a little bit? Why do we need to wipe the sweat off our
face? – It is to cure suffering. That’s the reason why we’re doing everything
in our life. To try and escape from dukkha. The Buddha points it out to us
very clearly. Here’s the problem. There is dukkha occurring. If we don’t see
that dukkha occurring, then our daily life is simply about curing our dukkha.
From morning through the afternoon into the evening. We’re spending the
day  trying  to  cure  ourselves  from  what  is  very  natural  –  dukkha  arising.
When we wake up, when we eat, when we wash our body. So have a look
at when suffering is felt. See it in your sitting meditation practice as well.
See  if  you  can  feel  it  arising  and  the  need  to  twist  or  change  or  move  or
drink water.
\sphinxstyleemphasis{It’s the reason why we do everything that we do}. When we are
\DUrole{pdfpage}{69}  sitting already for 20 minutes in meditation, we start to become a little bit
uncomfortable. We feel like we need to change. Why? Why do we need to
change?  Have  a  look  at  the  reason! Are  you  trying  to  cure  suffering? Are
you trying to remove the suffering? This is what we have been doing all our
lives. When we are unmindful, the suffering manifests, we move our body
to try and cure it in some way. And then the suffering disappears. Then it
arises again because it’s continuous. It’s continuously arising. That is what
having a body is. We have taken rebirth in the sense sphere world. We are
sense sphere beings endowed with this body. We have this thing, these four
elements  combined  together  and  consciousness  is  addicted  to  it. Addicted
to the pleasantness and unpleasantness that arises through the six doors. It
believes it is this body. It has come to a conclusion that it is somebody. – A
very interesting conclusion.

\sphinxAtStartPar
The correct practice when you start to feel something uncomfortable,
when you want to move in sitting meditation is to make a note before moving
as to why you want to move. Try to become aware of what actually is going
on. When we are unmindful, we are trying to push away, we are trying to
escape  what  is  unpleasant  and  we  are  trying  to  replace  it  with  something
that is pleasant. Sometimes it’s just as easy as stretching your leg. If I am
unmindful I just transition from one mind state of disliking to another mind
state of liking. I am pushing away the unpleasant and I am seeking the pleasant, the new posture. This is just reacting to what is there. This is not seeing
things clearly!

\sphinxAtStartPar
When  we  see  things  clearly,  when  we  are  meditating  and  the  body
starts to become uncomfortable, we make a note of it, we make a note of the
sensation that is occurring in the moment, we make a note of the unpleasantness of that sensation occurring in the moment – there is two things going on
there: one of them is physical, the body sensations are physical; the pleasantness or unpleasantness of the situation is mental – mind and body arising
together  in  the  present  moment,  consciousness  that  is  there  knowing  that,
witnessing that, observing that.

\sphinxAtStartPar
Become aware of the aversion that starts to arise in the mind before
you move. If you can note the unpleasant state that you’re sitting in before
you change your posture, if you can note and know it, you can free yourself
\DUrole{pdfpage}{70}  from  this  aversion  state. You  can  go  to  the  new  posture  but  you  have  not
reacted, you make the change. But don’t get excited about the new posture
thinking,  «oh,  I  am  comfortable  and  happy  now».  You’ll  make  a  note  of
that  as  well.  In  that  way  you’ll  be  noting  exactly  what  is  going  on.  You
are not just reacting to unpleasentness. That is normally how we function.
Something unpleasant is happening, we try to escape it! We can escape with
mindfulness and awareness and wise attention. That’s the skillful way. That’s
the appropriate strategy to use. If we don’t have this ability, then our only
option is sensuality. A little bit of suffering, find some sensual objects to satisfy us. Find something nice to eat, touch, listen, turn on the TV or computer,
watch  something,  touch  somebody,  to  go  to  new  places  and  get  new  nice
eye impressions, get fresh delight into the eyes, ear, tongue, nose and body
doors. That has been our only escape up until now. Unpleasantness occurring, do something nice. That’s what we do moment after moment.

\sphinxAtStartPar
When we start examining our daily activities, when we get up in the
morning after the bell rang, why do you wash your face? I want you to pay
attention to that early\sphinxhyphen{}morning activity. Why do we need to go to the bathroom  early  in  the  morning?  Make  a  note  of  that.  –  It  is  to  cure  suffering!
The  bladder  is  full.  It’s  starting  to  feel  uncomfortable  now.  When  having
a shower, why do you do that? Why is it that you need to wash your body
so regularly? – It’s to cure suffering. The body has become unbearable. It’s
sticky and all hot. The only way we can cure that is by having a shower. If
we don’t know how to be mindful and aware of what’s going on, we can get
caught up in the unpleasantness of the situation!

\sphinxAtStartPar
There is unpleasantness arising continuously. All we need to do is to be
aware of it. When we are aware of it, the unpleasantness falls away. There
is just what is. We don’t find anything unpleasant anymore. We are able to
use our mindfulness and awareness to note it, to see it really as it actually is,
to not get trapped and caught in whatever that physical sensation is. We can
move beyond it.

\sphinxAtStartPar
So when changing from posture to posture, do not dislike the old position  and  start  to  like  the  new  position.  You  don’t  want  those  mind  states
coming up. Unpleasantness, disliking. «I have to sit on a stool.» And then
liking. We are trying to watch our reaction process of liking and disliking.
\DUrole{pdfpage}{71}  What’s fueling it? What’s behind it? – Pleasantness and unpleasantness.

\sphinxAtStartPar
Try to maintain your awareness in the present, until another pain or discomfort comes along. And then see what happens. You will have to change
that as well.


\section{Labelling technique}
\label{\detokenize{1-b:labelling-technique}}
\sphinxAtStartPar
So we look at all the activities we go through whilst we are here. We
have a look at the most simple things. Our daily postures, our daily movements, our eating. In an aid to help us with this meditation, we can use the
technique of labelling. Using very soft, little words to keep us in the present
moment if necessary. When something is occurring, we can just put a little
label  there. To  make  us  aware  of  the  fact  that  hearing  is  taking  place,  we
can just make a note ‘hearing, hearing’. ‘Seeing, seeing’. ‘Smelling, smelling’. ‘Thinking, thinking’. ‘Remembering, remembering’. ‘Planning, planning’. All these little things. Just watch the mind and make a note of what’s
going on. We are stepping back and watching the mind spinning and turning
by  itself.  It  will  keep  thinking.  It’ll  keep  manifesting  various  thoughts  in
the  past  and  future.  Just  keep  watching  them.  Just  keep  observing  them.
Don’t get caught by them. We are not going into the story that keeps arising.
We are stepping back. Meditation is the practice of stepping back from the
events that are occurring in our lives. Stepping back and having an objective
view of them. When we step back like this, we disconnect from them. We
have stopped identifying with them. We don’t take the event or experience as
being me or mine so strongly anymore. The me and mine, the identification
with the events, the identification with the physical sensations in the body,
the  identification  with  the  mental  states,  the  emotional  states,  the  thought
patterns – we start to let go of them, stop identifying with them.

\sphinxAtStartPar
You will find the less you identify, the less you suffer. The more you
identify, the more you suffer. You can get yourself into a real state, spinning
around, worrying, freaking out, just by being in a mind state, just by owning
that mind state. Just by thinking that that mind state is somehow important.
– Dependently arisen, conditioned phenomena, that doesn’t have an owner,
that arises and passes away. Why do you give so much attention and care to
a mind state? It’s not you! It’s not yours! It doesn’t belong to you! So drop it!
\DUrole{pdfpage}{72}  You don’t have to hang on to things. You don’t have to be the owner of your
crazy thoughts, your unbalanced emotional states. They are not you! They
have never been you! They do not belong to you! But we have been identifying with them for a while now. And you know the mind states that cause
you the problems. Those mind states that take you into depression, anxiety,
stress, worry, fear, concern. All these mental states are fine when they’re just
noted, known and let go of. They will arise! It’s the nature of having a mind
and body! Mental states do come and go! Allow them just to arise and pass
away. Don’t get entangled in them. Don’t take every thought that comes into
your head as mine. Don’t take all of those things literally or personally. They
are not yours! They are very, very impermanent mind states that are arising
and passing away. They don’t belong to anybody. It’s conditioned phenomena. It’s sankharas.


\section{Karma}
\label{\detokenize{1-b:karma}}
\sphinxAtStartPar
\sphinxstyleemphasis{Sankharas}
have arisen because conditions are in place. That’s the only
way old karma can manifest. We have all done billions and billions of old
karmas in the past. Karma is done through body, speech and mind. Intentional activities.
\sphinxstyleemphasis{Intentional activities that give a resultant, that give a result}
\sphinxstyleemphasis{known as kamma\sphinxhyphen{}sati or karmic energy}. That karmic energy remains latent.
It remains unproductive until the conditions are in place for it to manifest.
Billions and billions of karmic intentions we have done in the past. Whatever  is  set  up  in  the  present  moment,  will  allow  that  karmic  intention  to
manifest as a fruit. If the conditions are not in place, the fruit can’t grow. All
our karmic intentions are like a mango seed which has not been planted yet.
We cut open a mango seed, you can have a look inside, there’s no mango tree
in there. And yet there exists in that mango seed the potentiality, the possibility, the latency of a mango tree arising from that seed. But there needs to be
the right conditions in place. There needs to be good earth, some water, some
good sunlight, maybe a gardener around. If all the conditions are in place,
then the latency that is in that seed, will have the potential to give its fruit.
It will start to grow a tree. It will start to give us some fruit. But only if the
conditions are in place. Our situation is very much like that. Karmic energy
is  manifesting  in  the  present  moment  bubbling  up  manifesting  as  mental
\DUrole{pdfpage}{73}  and physical phenomena, arising and passing away in the present moment.
This is an experience of samsara, this is dukkha. The Buddha said, there is
dukkha. Have a look at it! When this mind and body process is unable to be
aware of itself, then it starts to subjectify the experience. The mind and body
process is either aware and present of the moment, free from a state of duality of me, mine and I and the other. Or it is not present and unaware in the
moment, not free. We have a choice in each moment. We can be aware and
free or we can be unaware and trapped in the conditioned phenomena identifying with it. When we’re identifying with that phenomenon, that is called
dukkha. The first noble truth. There is dukkha. The Buddha’s teaching is all
about removing dukkha. It’s all about understanding dukkha. Understanding
craving, the cause of dukkha and relinquishing and letting it go.

\sphinxAtStartPar
In your daily activities, you can choose a few activities in the beginning and a few things you do regularly. For example drinking water. Try to
make each drink of water a mindful experience. Don’t just grab the bottle,
open the lid and throw it down. Try to have a look before you start to drink
the water. What’s the condition of the mind? Is there any suffering there? Is
there  thirstiness?  Make  a  note  of  thirstiness  if  it’s  there.  ‘Thirsty,  thirsty’.
And  you’re  going  to  reach  for  the  bottle,  just  be  fully  aware,  watch  your
hand, move it deliberately, touch the bottle, bring it back, open the lid and
intending to drink, drinking, tasting, feel the sensation in your mouth of the
liquid. What does it feel like? Is it hard? Is it soft? Is it heavy? Is it smooth?
Is it pleasant? Is it unpleasant? Does it make you want more? Does it cure the
suffering, the thirstiness, you were experiencing? Neither like it nor dislike
it! See it for what it is!

\sphinxAtStartPar
We spend much of our day going through curing suffering. Why do we
perform these activities? Why do we eat? What is it about eating that has got
us so fascinated? Are we eating really just to survive? Or is it some kind of
enjoyment process? Is it some kind of fun? Are we eating to cure suffering?
When we are mindful of the eating process, that is the conclusion that you
will come to. You will come to understand, «Oh, this is why we have to eat».
If we don’t eat, a great suffering is coming. People tend to get upset if they
haven’t been eating for a few days. They become unbalanced. Very uncomfortable. We start looking at that. How important is eating? Why do we do
\DUrole{pdfpage}{74}  it? Why do we normally do it? Normally it is some kind of sensual pleasure
activity. It’s become a social activity. We eat even when we’re not hungry.

\sphinxAtStartPar
Don’t rush through your activities. «Oh, I do this quickly and then I
want to go to do meditation.» The meditation is in the daily activities. The
more attention you can pay to them, the deeper your sitting meditation will
go  and  the  better  you  can  take  the  meditation  back  into  your  daily  life  as
well. We’re not only meditating by sitting still and walking very slowly. We
are meditating on our other activities.

\sphinxAtStartPar
When  you’re  going  to  turn  the  body,  intending  to  turn,  make  a  note
before you turn it. When you hear a sound and then instantly you want to
look finding out what it is, make a note, why you are doing that. Why are
you  turning  your  head? What  is  it  that  is  so  interesting  and  requires  your
attention right now?

\sphinxAtStartPar
When we go to sit down after walking, when we come back to the hall,
just stand a moment in front of your mat and slowly lower your body down.
Try  to  follow  the  whole  transition  from  the  walking  path  into  the  sitting
meditation. There really shouldn’t be any gaps. Sometimes people just rush
to the toilet after the walking meditation and think, «oh I should quickly go
back to the sitting meditation». This breaks the whole process of awareness.
Examine that for yourself.

\sphinxAtStartPar
Why do we go to the toilet? Why do we put on clothing? Why do we
eat? Why do we drink? Why do we turn our head? Do we enhance our sense
of identity, our personality when we go shopping for clothes? Or do we use
clothing wisely, to protect and cover our body. Why do we use clothing? – To
cure suffering! If not we would be really cold. That is why we have to use
clothing. Or we use appropriate clothing when it’s really hot. We are curing
suffering. That is what clothing is for.

\sphinxAtStartPar
Why  do  we  go  to  sleep  at  night?  Curing  suffering!  Why  do  we  eat
during the day? Curing suffering! Why do we drink during the day? Curing
suffering!  Why  do  we  change  postures  continuously  throughout  the  day?
Because we are curing suffering. We need to be able to see this for ourselves.
Just listening to it isn’t enough. When you see how the mind is maneuvering
and  manipulating  itself,  so  it  gets  what  it  wants,  it’s  always  seeking  some
kind of happiness. Unfortunately it doesn’t know the way to happiness. The
\DUrole{pdfpage}{75}  way to happiness is contentment. Contentment with whatever is occurring
in the present moment that leads to happiness. That is the condition for happiness. The proximate cause for happiness. Being content with whatever is
right now. That makes us happy. When the mind is happy it becomes concentrated. The concentrated mind sees things as they really are.

\sphinxAtStartPar
So make sure you’re becoming content whilst you’re here on the retreat.
Put in some effort to make your mind content. Don’t struggle with the frustrations. I know it’s hot. We know that you are tired. We know that you’re
exhausted. Frustrated. «What am I doing here, sitting on the floor, listening
to this guy? What are we doing?» Try to see the mind making excuses. Try
to witness the mind as it comes to whisper: «Don’t do this. You shouldn’t be
doing that. The beach is better for you.» Try not to listen to that. Don’t get
caught in it. Don’t believe the whispers. That is what crazy people do, listening to the whispers in their head. Don’t get caught by that stuff. It’ll make
you quite agitated. Whatever is arising, note it, know it and let it go. Here
it’s a radical change in viewing the body and mind process as we’re normally
doing. We don’t normally spend our lives looking at what we are doing in
this way. We are always seeking and searching for something pleasant out in
front looking for the new enjoyment. A new toy to buy, a new restaurant to
go to, a new piece of clothing to climb into. All those things. We are looking
for those continuously.

\sphinxAtStartPar
They  never  bring  satisfaction.  They  never  bring  happiness.  Because
searching is not contentment. When we are searching and seeking for something, trying to get more and more of something, it means we’re not content
with  how  things  are.  We’re  unable  to  just  rest  with  what  is.  The  mind  is
wanting. Have a look at that wanting mind. Have a look at the mind which
has a little bit of aversion towards the unsatisfactory present moment. When
we start reacting to it, there is disliking and we have to change our posture,
we have to have a drink, or we have to go to the bathroom, or we have to lie
down. All these things are the results of not seeing clearly what’s going on.
These activities are just what occurs when we are not being mindful.

\sphinxAtStartPar
Of course, you still have to sleep. You still have to drink. No\sphinxhyphen{}one is suggesting that you don’t. Just do it with some attention to the present moment
as to why you are doing it. That is a shift in your experience. If you feel any
\DUrole{pdfpage}{76}  itching sensations or feel uncomfortable pains or any kind of sensations you
find unpleasant, make a note of that. Don’t get caught up by them. «My poor
knee. Oh my knee.» Stop identifying with it! It’s not yours! It’s not you! It’s
just a physical sensation. It may be unpleasant, that is something else. If you
see it clearly, the unpleasantness disappears, the sensation will still be there,
whatever  it  is,  the  sensation  in  your  back,  in  your  knee,  the  sensation  of
hunger. Whatever sensation it could possibly be. If you just note it and know
it, you can step out of it. It’s still there but the unpleasantness associated with
it disappears. It turns into a neutral feeling.

\sphinxAtStartPar
This is how we begin to manage ourselves, we can manage our lives.
We are learning a technique here, the method and techniques of managing
our own mind when we find that our mind is getting into a state of unpleasantness, when we are getting into an aversion state or a craving state. We’ll
be able to note that. «Hm, danger is there. Suffering is occurring.» You can
step out of it. As soon as you see it, it’s no longer suffering. If you’re still in
it, there’s a lot of suffering in there. If you’re still identifying with the experience or the event, dukkha is your companion. If you’re noting and knowing,
letting the conditioned phenomena arise and pass away without identifying
with it, then you’re free in that moment!

\sphinxAtStartPar
Our mediation practice is to free the mind, moment after moment after
moment. So that we become skillful at it. Like learning to play the guitar. We
try and make some effort. At the beginning stages it’s difficult but after a few
days you start to play a few accords. Meditation is a practicing event. We
are practicing meditation! It doesn’t happen immediately. It takes practice.
Bringing our minds into the present moment when our mind is in a state of
stress. This is a good starting point. When you find yourself worrying, freaking out, maybe you don’t understand the meditation, what we’re doing here.
– If that’s the case come and talk to me, maybe you’re expecting something
to happen, you’ve heard people talking about meditation and all the wonderful results and you think you just need to sit down and it’s going to happen to
you. Maybe you think sitting on the floor cross legged with your eyes closed
is  somehow  miraculously  giving  you  a  great  deal  of  pleasantness.  If  you
are thinking in that way, you may be quite surprised after a couple of days.
There’s not a great deal of pleasantness of sitting there on the floor. Meditation \DUrole{pdfpage}{77}  is mind work. We train the mind. We’re training it to endure, to have
some skills in dealing with this dukkha that arises.

\sphinxAtStartPar
The Buddha’s first noble truth: There is dukkha. That’s all we need to
understand. If we understand dukkha, we have understood the world. Nothing becomes a problem for us anymore. Because when we see dukkha, we
also see the cessation of dukkha. We see them at the same time. We see the
craving that causes dukkha and we see the mindfulness and wisdom and the
other factors, that cause freedom from dukkha. The noble eightfold path and
the four noble truths manifest in the present moment together.

\sphinxAtStartPar
When  you  see  dukkha,  you  also  see  the  cause  of  it  occurring  right
there. And when you see dukkha and its cause, immediately the mind snaps
out of it and goes into cessation. It ceases! And we see the cause of that cessation  is  our  meditation  practice. The  noble  eightfold  path  is  the  cause  of
cessation.

\sphinxAtStartPar
We can make our choice in the present moment intelligently. We have
the ability to monitor each moment. So make the effort. Here we have a special training facility where you can keep your mind in the present moment as
continuously as possible so you can see these things happening in real\sphinxhyphen{}time.
Once you have seen it, once you have broken through this, it is a lifetime
lasting insight. It will affect and change your mind and perception. It will be
a shift. This is called evolution. This is the purpose of our species actually.
This is why we are here. This is what humans are meant for. We are meant to
evolve. We are meant to be coming out, evolving. Meditation is the practice
of evolution. The evolution of our species. Two and a half thousand years
ago this technique was discovered where beings can escape from samsara,
the  rounds  of  rebirth,  the  continuous  arising  and  passing  away  of  mental
and physical phenomena that believe they are somebody. It’s all manifesting right in front of us! Luckily the Buddha broke through that. He saw the
patterns. He saw the framework, the structure and he gave us the teaching,
pointing out where the problem is and he pointed out the techniques that we
need  to  use  so  that  we  can  overcome  the  problem. And  people  have  been
doing that very successfully for two and a half thousand years.

\sphinxAtStartPar
Unfortunately for us, it didn’t kind of make it to where we were born
and grew up. It’s taken a while to get over to the western side of the planet.
\DUrole{pdfpage}{78}  But this century and last century it’s starting to make inroads. The teaching
is starting to spread into the western hemisphere. So now it’s our opportunity to do the practice. Take advantage of this unique situation. This unique
opportunity to become deeply in touch, not only with a wonderful spiritual
teaching, but also with your own mind. We’re starting to work out what this
mind is and what its reactions are.

\sphinxAtStartPar
When  we  start  to  see  it,  we  start  to  release  our  identity  with  it.  Our
identity crisis will start to abate, start to be let go of this identity, this personality that we have created for ourselves. It’s being reinforced by the social
institutions which we live in. You’ll start to see it crumble before your very
eyes. You’ll start to see it crumble through your insight. The sense of self
will start do dissolve and when it starts to dissolve, a lot of freedom arises.

\sphinxAtStartPar
What  replaces  the  sense  of  self? When  the  sense  of  self  starts  to  go
down, it’s replaced by
\sphinxstyleemphasis{love and compassion, metta and karuna}. That’s what
replaces the sense of self. So don’t think that you’re going to be empty. Actually there’s a great deal of bliss and a great deal of love that arise in the heart
in the person whose sense of self is being diminished. When you’re very selfish, you don’t have much love and compassion. When your sense of selfishness starts to go down, your sense of love and compassion for others starts
to go up. You will start to see that. When we start to let go of our selfishness,
we start to broaden our interests.

\sphinxAtStartPar
I  really  would  like  you  to  become  aware  of  your  various  activities
throughout the day. When you’re having your meal, listen to the food reflection. Look into your bowl. See the food that is there. Hold the spoon. Pick
it up. Fill the food in your mouth. What are you doing? You’re putting food
into  your  head! What  for?  Is  it  for  your  enjoyment?  Pay  attention  to  that.
You’re not talking to anyone while we’re eating. We are not distracted by any
screens. Be aware of the arm as it’s bending putting the food into the mouth.
Feel the warmth, the temperature in your mouth. Feel the softness, the greasiness, oiliness, the crunchiness. Feel that! Feel what it’s like to swallow. How
far down can you follow the sensation of swallowing? Look at the pleasure
that arises. Is there some happiness? Is it pleasant? Is liking occurring? Try
to capture all the experience of eating! Try to see exactly what is going on.
We can also pay attention to our thoughts in our daily activities. Look
\DUrole{pdfpage}{79}  at our intentions, look at our ideas and imaginings. We’re not getting sucked
into them either. Look at the planning. Look at the remembering. If you’re
imagining  something,  make  a  note,  ‘imagining’.  If  you’re  thinking  about
something,  ’thinking’.  If  you’re  remembering,  ‘remembering’.  If  you’re
planning, ‘planning’. Try to keep on top of exactly what’s going on in the
present moment. Where is your mind at the moment?

\sphinxAtStartPar
We are trying to keep our awareness in the present and internalizing
our awareness so that we have a continuous awareness of the body at this
stage  in  our  meditation.  But  there  will  be  thoughts  coming  up  as  well.  If
they come up, we make a note of them and we return to the awareness of the
whole body in our daily activities.

\sphinxAtStartPar
It’s the same with our sitting meditation. We are watching the breath
as it flows in and out. That’s our main object. That’s what we should be pay
attention  to  90\%  of  the  time.  But  if  there  is  some  other  object,  sound  or
something comes to interfere, we make a note, ‘hearing’. That’s what’s happening. Hearing is occurring. This body is hearing some sound. That’s all.
Don’t get caught in the sound. Don’t get caught in the story of that sound.
Who is making that sound? When will they stop that sound? We don’t go into
stories when we hear a sound. We just make a note that hearing has occurred
and  come  back  to  our  main  object.  So  we  try  to  keep  our  awareness  and
attention on the main object.

\sphinxAtStartPar
We  will  be  talking  about  that  more  tomorrow  morning.  We  will  go
into some further instructions about the walking meditation and the sitting
meditation.

\sphinxAtStartPar
For  now  I  encourage  you  to  keep  your  mind  in  the  present  moment
and  internalized  during  the  daily  activities.  It  makes  up  the  third  part  of
our meditation retreat. We do sitting meditation, walking meditation and the
daily activities. They all will contribute to your success and your unfolding
of insight. May it happen swiftly for you!

\sphinxstepscope


\chapter{Day 2, morning}
\label{\detokenize{2-a:day-2-morning}}\label{\detokenize{2-a::doc}}
\LOCALaudiolink{https://www.mixcloud.com/anthonymarkwell/day-2-morning-talk-sitting-meditation/}

\sphinxAtStartPar
Some may be tired this morning. Many people are, don’t worry, you’re
not the only one. Just maintain some awareness of it. Don’t become involved
with  the  state  of  tiredness  otherwise  you  will  start  to  create  some  dukkha
for yourself. You can just note and know that there’s some tiredness there
without identifying with the tiredness and you can free your mind from that
particular mind state.

\sphinxAtStartPar
This morning we’re going to talk about the sitting meditation again. We
are going to refine our instructions for the sitting meditation.


\section{Refining sitting meditation instructions}
\label{\detokenize{2-a:refining-sitting-meditation-instructions}}
\sphinxAtStartPar
Up  until  now  we  have  been  simply  following  the  breath  as  it  enters
into the nostrils, glides down through the chest, comes to the abdomen, it
raises,  and  then  the  abdomen  falls  and  it  comes  back  out  again.  We  have
been following it, in and out, trying to be as continuous as we possibly can.
Trying to maintain our awareness in the present, internalizing that present
moment awareness and following the breath. This is the preliminary step of
just bringing ourselves into the present and trying to connect with the body
a little bit more. Some of you will have noticed, that the breath doesn’t only
go to the lungs, but travels right down into the abdomen. In fact, the breath
can seem to flow all the way from the head right down to the toes and back
\DUrole{pdfpage}{81}  up again. There is a subtle movement of energy inside the body. They call it
\sphinxstyleemphasis{vayo dathu}, the air element. It’s moving. The breath is part of this element.
There are actually quite a few different types of energies flowing through the
body. Those types of movements that the arms do, those types of movements
of  digestion  and  clearing  the  bowels  and  the  bladder. All  types  of  movements, the chest expanding and contracting, the abdomen rising and falling,
the supporting feeling that’s coming up from down below and coming up the
back into the head. These are all characteristics of the air element. Please try
to follow them for a little bit.

\sphinxAtStartPar
As we start to stabilize our awareness, just on the path of breath sensations, some of you will have noticed that it is a little bit clearer at some
points with regards to the physical sensations that are occurring. Some will
have noticed that the sensation is greater around the nostril area, around the
upper lip. They can connect there a little bit easier. For some people it is in
the chest or the heart area. For others, it’s in the abdomen area. Lots of sensations are going on here as the abdomen rises and falls.

\sphinxAtStartPar
Now  at  this  point  in  the  retreat,  I’d  like  you  to  choose  one  of  these
places and maintain your awareness just at that one particular place. So at
this point we drop the chasing and the following of the breath as it flows in
and out and we just choose a single place. Some people have already started
to practice meditation having their awareness in the nostril area. That’s fine.
You can maintain your awareness there, continue to practice in this spot as
long  as  you’re  paying  attention  to  the  touching  sensation  that  is  occurring
there. Other people will find, that they feel their abdomen rising and falling
more easily.

\sphinxAtStartPar
For those of you who are new to meditation, who haven’t yet established  a  practice,  all  those  of  you  who  have  tried  different  meditations,  I
recommend  that  you  try  to  practice  at  the  abdomen.  The  rise  and  fall  of
the abdomen. At the abdomen we’re paying attention almost exclusively to
the different sensations that are occurring there. The breath doesn’t actually
arrive at the abdomen. There is just movement going in and out. That movement is the air element. The air element will manifest itself in many different ways. Sometimes there will be a pushing and pulling. Sometimes there
will be a little bit of pressure. Sometimes there is some tension. Sometimes
\DUrole{pdfpage}{82}  twisting. At other times, it will be vibrating. There is lots of different physical sensations to be aware of at the abdomen. You’re not watching the breath
going in and out. For those of you who have ever cut a piece of wood with a
saw, we’re not looking at the saw as it goes in and out – that’s the breath. We
look at the piece of wood and the little slit, that starts to be made there as we
run the saw over the wood. That’s where we watch. You’re always watching
just  at  that  spot. We’re  much  more  interested  in  the  touching  of  the  wood
than we are in the metal of the saw. The saw will keep going back and forth.
The  breath  will  keep  flowing  in  and  flowing  out.  That’s  just  a  conceptual
framework, just a guide to keep us present. What we are interested in here,
is the touching sensation on the physical body.

\sphinxAtStartPar
The first stage of our satipatthana practice is all about body awareness.
It’s  called
\sphinxstyleemphasis{kaya\sphinxhyphen{}nupassana},  reflection  upon  the  body.  We’re  interested  in
establishing mindfulness on the body. In fact, all meditation systems begin
with mindfulness of the body. When mindfulness of the body is established,
we can go on and establish all kinds of different types of meditation.

\sphinxAtStartPar
For those of you who are not going to watch the touching point at the
nose, I advise that you watch the abdomen as it is rising and falling. Become
aware of the physical sensations that are there. You need to put your mind
down  into  the  abdomen. You  will  need  to  keep  it  down  there,  to  stabilize
it  down  there.  It  will  run  up.  It  comes  up  to  its  old  playground,  the  eyes
and the ears. It loves to come up and manifest here. That’s where it’s been
manifesting for most of our lives. Sights and sounds. But we want to try and
bring our awareness down to the abdomen and become aware right of the
very beginning, moving through and finishing. And then there’s a little gap.
And then the abdomen falls. So we are paying attention to the movement of
the abdomen and the sensations created by the movement of the abdomen.
As long as the object of our mindfulness is either the sensation of touch or
the sensation created by the movement in the body – our object is a body
object – our practice will lead us to see
\sphinxstyleemphasis{sabhava}, the ultimate reality of the
four elements.

\sphinxAtStartPar
A little note of warning for those of you who like to practice at the nose
tip. Sometimes what can happen is that we loose our awareness as our meditation starts to develop. We loose our awareness of the touching sensation
\DUrole{pdfpage}{83}  and the mind turns toward the breath exclusively. And the breath transforms
from an object of the body into a conceptual object. We start to watch the
breath as a concept, as an idea. We start to imagine it as a particular flow, or
a particular thing, or a tube of breath, or a tube of light or any type of other
kind of conceptualizations. These manifest as various images, various pictures, lights, all kinds of shapes. This shouldn’t be a concern to you. If you
start to see things in your meditation practice, just make a note of what is
exactly going on. You’re seeing. Seeing is occurring. You’re not seeing with
your physical eye but you’re seeing with your mind. Your mind is starting
to generate some kind of images. Geometric patterns, colors or lights. Just
make  a  gentle  note,  ‘seeing,  seeing’.  If  you  become  excited  if  something
starts happening like that, make a note, ‘excitement, excitement’. We want
to stay on top of exactly what is happening. We don’t want to get stuck. We
don’t want to get involved in the object. We don’t want to start identifying
with the object. We don’t want to go into the object. We want to keep stepping back, keep detaching, this is not a meditation where we go in and start
identifying with the body. This is a meditation practice where we step out
and stop identifying with what is actually going on in the present moment.

\sphinxAtStartPar
Just a word of warning to you. For those of you who experience lights
and sounds and colors, when you practice at the nose tip, just make a note.
‘Seeing’.  Establish  your  awareness  and  wisdom  at  the  same  level  as  your
concentration.  In  that  way  you  will  be  able  to  balance  the  faculties. And
Vipassana insight loves the balanced faculties. When the five faculties, faith,
energy,  mindfulness,  concentration  and  wisdom  are  balanced,  Vipassana
insight starts to flow. We’ll be talking about balancing the faculties as the
week goes on.

\sphinxAtStartPar
So for those of you who wish to continue at the nose tip, you can do so.
But for those of you who are new to meditation, I suggest establishing your
awareness at the movement of the abdomen.

\sphinxAtStartPar
I strongly suggest that you work it out this morning four yourself. Don’t
become someone who switches between the two places sometimes practicing at the abdomen, sometimes practicing at the nose always creating doubt
about  where  you  should  follow  the  breath  or  what  you  should  be  doing.
This is not the type of state that we need to be developing. We don’t want to
\DUrole{pdfpage}{84}  develop uncertainty, doubt within ourselves. We want to be clear about what
we’re doing. So you need to choose one of these two places and stick with it
for the rest of the retreat and that will be for your benefit.

\sphinxAtStartPar
The work of unremitting mindfulness and awareness on our breath as it
arises and passes away, as the physical sensations arise and pass away, will
lead to certain states of calmness. Quietness occasionally. When we start to
connect  with  the  breath,  don’t  get  carried  away  by  these  beginning  stages
of meditation. It’s also interesting to experience that, but that’s not the end
goal of our meditation. Whilst we are practicing in this way, rising, falling,
rising, falling, I also want to incorporate a third object into our meditation
practice. So we are changing quite a lot this morning. We are changing from
chasing the breath as it’s moving in and out, to establishing the breath at a
single place.

\sphinxAtStartPar
At the beginning stages of our meditation practice, we come into the
hall, we sit down, arrange our legs, our hands, our eyes, establish our awareness in the present moment, just the present moment is our object, and then
after  a  minute  or  so,  we  start  to  internalize  –  you  can  start  to  internalize
before  the  minute  is  up  if  you  like  –  just  start  to  internalize  that  awareness keeping your mind deep inside the body. And when it runs up through
another door, through a sound or through a thought, just gently bring it back.
Don’t scold yourself but just bring it back. You have to be quite firm in the
training  of  your  mind.  You  can’t  allow  it  to  be  slack  anytime  during  the
week. You will have to firmly decide, this is what you’re going to be doing
here and work towards that goal. It’s no good working a little bit and then
resting a little bit. That doesn’t really work. You can do that for many years
and still the practice will not be successful. So I encourage you to work consistently and persistently.

\sphinxAtStartPar
When we’re watching the rise and fall of the abdomen, as it rises, we
make a note from the beginning until the end of the physical sensation. So
we are aware when it starts, we follow the sensation through till it finishes.
And then the abdomen will start to fall. We’re aware of the beginning of the
sensation,  following  it  through  and  then  it  finishes. At  the  end  of  the  outbreath, in, out, there will be a little gap there. In that gap, I want you to turn
your attention, or focus your awareness, to the sensation of the whole body
\DUrole{pdfpage}{85}  as it is sitting there. Just for a snap second. Just for that little gap between the
out\sphinxhyphen{}breath and the in\sphinxhyphen{}breath. So it goes something like this: ‘rising, falling,
sitting’. We become aware of the whole sitting posture. And then: ‘rising,
falling,  sitting.  Rising,  falling,  sitting’.  So  we  are  moving  our  awareness
from the place here in the center of the body, where the sensations are strong
– the vibrations are strong, the heat is strong, the movement is strong, the
sensations are manifesting very well to observe at this location, as it rises
and falls, we’re keeping our attention right here. As soon as the out\sphinxhyphen{}breath
is finished, I want you to broaden your awareness, and become aware of the
whole  body.  Become  aware  of  the  head  and  shoulders,  down  through  the
legs, just the whole thing. Try to capture the single sensation that is existing
between  the  end  of  the  out\sphinxhyphen{}breath  and  the  beginning  of  the  in\sphinxhyphen{}breath.  So:
‘rising,  falling,  sitting’.  What’s  there?  Try  to  activate  your  awareness  and
catch the whole body. This will continue to keep us in the present moment
and  it’ll  continue  to  internalize  our  awareness.  In  fact,  the  rising  and  the
falling  keeps  us  present  and  the  internalization  on  the  physical  sensation
keeps us internalized. So we are trying to make it continuous by doing this
over and over again ‘rising, falling, sitting, rising, falling, sitting’. This is
not a new object. We’ve been doing it at the start of the meditation sessions
already. The  second  minute  of  the  sittings  have  been  present  moment  and
then internalizing. Try to take a snapshot of the sensation in the whole body.
We’re not trying to create an image of the body. We’re not trying to imagine
what we look like sitting on the mat. We’re just trying to experience: what is
it now that you can feel?

\sphinxAtStartPar
What  can  you  experience?  Hardness,  softness,  warmth,  coolness,
vibrations, tingling…capture the physical sensation that is occurring in the
present moment right at the point between the out\sphinxhyphen{}breath and the in\sphinxhyphen{}breath.
Become aware of the whole body as it is sitting there just for a second at the
end of the out\sphinxhyphen{}breath. ‘Rising, falling, sitting, rising, falling, sitting, rising,
falling,  sitting,  rising,  falling,  sitting’.  So  this  becomes  your  new  main
object. Do this 80\sphinxhyphen{}90\% of the time! Going around in this sequence trying to
make it continuous. Trying to follow from beginning to end.

\sphinxAtStartPar
The snapshot has to be quick. In the beginning stages you may not be
able  to  capture  the  sensation  of  the  whole  body.  Don’t  let  that  worry  you.
\DUrole{pdfpage}{86}  Avert your awareness to the whole sitting posture and then come back to the
breath. The more time you do it, the clearer it will become. Don’t spend your
time trying to search for it at the beginning, this will disturb your breath! It’s
very quick. It’s very short, as fast as your iPhone can take a photo. And then
back to the breathing again. ‘Rising, falling, sitting, rising, falling, sitting,
rising, falling, sitting’. We keep going around and around in this way.

\sphinxAtStartPar
For  those  of  you  who  have  been  here  before,  you  can  even  add  the
fourth object. Be aware that sometimes it can be a little bit busy for beginners doing the three stage sitting meditation before you’re going to the fourth.
You can also become aware of the touching of the body on the mat. This is
a  fourth  object.  ‘Rising,  falling,  sitting,  touching’.  Touching  is  the  sensation of the body sitting on the mat. It’s quite different from sitting. Sitting is
the sensation of the whole body as a posture. Touching is a particular part,
whether it’s your butt\sphinxhyphen{}cheeks or your knees or your ankles. I propose that
you  go  from  butt\sphinxhyphen{}cheek  to  butt\sphinxhyphen{}cheek.  ‘Rising,  falling,  sitting,  touching’.
Feel that cheek sitting on the mat. There is a touching sensation there. And
then: ‘Rising, falling, sitting, touching’. On the left side. ‘Rising, falling, sitting, touching’. On the right side. {[}or: ‘rising, falling, sitting, touching left,
touching right’{]}.

\sphinxAtStartPar
This  will  do  many  things  for  your  meditation  practice.  Most  importantly it will increase the amount of energy that we are using. Normally we
breath about 12 times every minute. That is 12 times in and 12 times out.
If we’re only noting the breath, we’ll spend a whole minute to just note 24
times. It’s not really fast enough. It’s not really continuous enough for our
insight  to  break  through. We  add  a  few  extra  objects  so  that  we  can  keep
the mind busy and engaged with being in the present moment. If we only
have two, we will find ourselves getting nice and comfortable and pleasant
and  then  a  kind  of  pleasant  dullness  comes  upon  us.  It’s  kind  of  nice  but
nothing  really  happens.  Our  meditation  comes  to  a  kind  of  plateau.  Many
people experience this. If we add an extra object, not only are we doing it
24 times a minute, now we have added 12 extra objects. We’re doing it 36
times a minute. So we need to use a lot more energy and effort to activate
our awareness in the present moment. We are increasing our workflow by
50\%.  It’s  very  effective  in  maintaining  our  awareness  continuously. You’ll
\DUrole{pdfpage}{87}  will find yourself becoming closer and closer, further and further, deeper and
deeper into the body. You will be able to stay inside. For those of you whose
mind is wandering in the past and future, you will also find, if you’re paying
attention to more objects, then your mind will have less time to go wandering. You won’t get distracted. It takes a lot of effort to be continuously aware
in  this  way.  ‘Rising,  falling,  sitting,  touching’. About  that  pace.  We  don’t
interrupt the breath. So we keep going around and around that way. You may
have to put your hand on the abdomen or you may have to take a couple of
deep  breaths  to  be  able  to  find  that  sensation  of  movement. Take  three  or
four deep breaths and then allow the breath to come to its natural breathing.
Do not neglect the first and second stages of the practice. Don’t come to
the hall and try to immediately start to watch your breath. You should establish your awareness in the present and spend a minute doing that. Establish
your awareness internally. Spend a minute doing that. It’s important that you
do this, so that the mind has time to calm down before we start watching the
breath. We allow the breath just a couple of minutes to calm down and get to
its natural state. We want it to be as natural as possible. We don’t want to be
manipulating the breath in any way. Sometimes it will feel it is impossible
to not manipulate the breath. And that’s ok. Just try not to do it very much.
The breath has an intentional structure. So it is a little bit under the control
of consciousness but most of the time we want it to be completely natural. So
try to allow the breath to establish its natural rhythm at the beginning before
we start to observe it.

\sphinxAtStartPar
So in this way, we start to use this as our main object. ‘Rising, falling,
sitting,  touching.  Rising,  falling,  sitting,  touching’.  Over  and  over  again.
Keep your mind in the present. Internalizing keep doing this. Keep circling
around and around. If your mind starts to wander and you think of the past,
make a note, ‘thinking’. Or ‘remembering’. And then quickly come back to
the ‘rising, falling, sitting, touching’.
\sphinxstyleemphasis{Our main object needs to be followed
as  much  as  we  can.  We  only  note  the  secondary  objects,  such  as  hearing,}
\sphinxstyleemphasis{when  they’re  actually  disturbing  the  practice},  when  they  actually  come  in
into our cycling of the rising, falling, sitting, touching.

\sphinxAtStartPar
Some of you want to use the fourth object. You can use hearing. I find
this also very useful to bring myself back to the present moment. Hearing is
\DUrole{pdfpage}{88}  an object which is always available. It’s always there. The ears are always
hearing. We can’t turn them off. It’s a continuous object. It’s there. Of course,
we are not always paying attention to it. But in our meditation practice we
can use it, we can note hearing whenever it interferes our practice. Or we can
incorporate hearing into our Vipassana. In that way you can do it like this:
‘rising, falling, sitting, hearing. Rising, falling, sitting, hearing. Rising, falling, sitting, hearing’. You keep your awareness at the center of the body. In
and out. Then you broaden your awareness. And then you bring it to the ears.
And then you bring it back down again. So you’re moving your awareness
around inside your body. You’re following where consciousness, the bowl,
is arising and you’re noting what’s in the bowl. Mostly we’re noting physical sensations. When we’re doing ‘rising, falling, sitting, touching’, we’re
noting all the physical sensations. This is all body objects. This is all rupa.
It’s all the four elements.

\sphinxAtStartPar
So try to let go of the conceptual idea of a body. Try to let go of the
conceptual idea of breath. Tune your mind, tune your awareness to just be
interested in physical sensations. Turn off the picture that you have of your
body or your ideas that you have about the breath. Try to be aware of just
what is exactly there without you painting your ideas on top of it. We want
to get back to nature. We want to strip away the conceptual world and come
back and visit what is actually going on in the body. This is the main part of
our practicing in the beginning stages. We are paying attention to the body.
We’re paying attention in particular to the postures. Walking, standing, sitting and lying down. During our daily activities we are trying to keep ourselves aware of whatever posture we’re in from moment to moment. We’re
continuously  aware  and  knowing  that  now  it’s  sitting.  Continuing  to  sit.
Now it’s standing up. Now it’s walking. Now it’s standing. Just try to keep
your awareness in the present. Keep it inside and maintain it during the daily
activities.

\sphinxAtStartPar
During  the  sitting  and  the  walking  meditation  we  use  the  posture  as
well but we’re also paying attention to the physical sensations. It’s really two
kinds of objects that we use for becoming aware of the body.

\sphinxAtStartPar
Different people will break through in slightly different ways. We pay
attention to the postures and the body sensations.
\begin{itemize}
\item {} 
\sphinxAtStartPar
\DUrole{pdfpage}{89} Those of us who are more inclined to watching the breath as it arises
and passes, arises and passes, we are turning into the general characteristic of
\sphinxstyleemphasis{anicca}
or impermanence. Seeing things arise and pass away,
arise and pass away.

\item {} 
\sphinxAtStartPar
Those connecting with the bodily postures (the walking, sitting, lying
down) you will start to become aware of the general characteristics of
\sphinxstyleemphasis{dukkha}
. You start to see that things are not so wonderful as you thought
they were. Just a relatively speaking. There is some suffering going on.

\item {} 
\sphinxAtStartPar
Those who manage to penetrate through the characteristics of the four
elements, will start to become aware of the characteristic of
\sphinxstyleemphasis{anata}
or
non\sphinxhyphen{}self. There will be no self that’s arising. You will see that this thing
sitting  here,  is  just  a  bag  of  physical  sensations,  without  a  bag.  It  is
a  field  of  physical  sensations.  We  painted  it,  that  it  is  a  body,  arms,
legs, head, as me, as mine – these are just concepts! They don’t exist
in the ultimate reality. They are man made inventions, which we have
decided to call a body, an arm, a leg. And in your language you may
have another word for it. This is just conceptual, this is not reality. Our
Vipassana will go further than the conceptual and penetrate deeply to
see the ultimate realities. We want to understand the nature of earth,
water, fire and air.

\end{itemize}

\sphinxAtStartPar
We  want  to  see  the  nature  of  rupa,  matter.  When  we  see  matter,  by
default  we  will  also  see  the  nature  of  the  mind.  It’s  immaterial,  the  mind
has no physicality to it. We can’t know the mind by seeing it. We know it’s
knowing and we know it’s power. It is energy. We can develop many different mind states, so that we can know rupa. Rupa does not know anything. It’s
kind of dumb. Earth, water, fire and air – it’s a lump of stuff sitting there on
the mat. It’s breathing in and out. Physical sensations are occurring. The four
elements are balancing or unbalancing with each other as the day goes on.
Sometimes it’s warm, sometimes it becomes hard, sometimes it’s vibrating.
Physical sensations manifesting throughout the body.

\sphinxAtStartPar
The sitting posture will reveal its own set of characteristics. When we
avert  to  the  sitting,  I  want  you  to  try  to  grab  the  most  obvious  sensation
that’s there. If you can’t yet feel the supporting – there is kind of a flow that’s
coming up from underneath supporting the body, it’s natural floating there
\DUrole{pdfpage}{90}  all the time – if you pull your mind inside, you’ll be able to experience this
supporting  sensation. There’s  also  vibrations. There’s  also  the  temperature
aspect. There’s also just the aspect of the sitting posture.

\sphinxAtStartPar
I would like you all to do the three or four stage sitting meditation. It
will be busy – because it is busy – but it will be very beneficial. With this
extra  work,  the  mind  has  no  time  to  wander,  go  thinking  or  planing.  We
sharpen our aim with directing consciousness – we are using awareness and
wisdom to direct our consciousness – to the place where the physical sensation is most clear. And we are following it. From beginning to end. And
then we’re broadening our awareness capturing the whole body. Then we’re
coming  back  in  again.  Keep  circulating,  keep  looping  around  and  around.
Don’t worry if it is not clear for a first few minutes, just keep doing the work
and then the mind will follow and start to know what to do. It understands
during the breath it needs to be here. At the end of the breath it becomes wide
and takes in the whole body. You train the mind with various objects. Like
going to the gym.

\sphinxAtStartPar
Make sure, there are no gaps during the noting process! From beginning to end, from beginning to end. If you become aware – and you don’t
have to search for these sensations, but if you become aware of some vibration, if that’s the predominant sensation, make a note ‘vibrating, vibrating’.
Try to make that an objective object. So we can step back from it and stop
identifying with it. If there is some pressure, or if you feel some pushing,
make a note of that. ‘Pressure, pressure’. ‘Pushing, pushing’. If you feel it’s
kind of twisting. Make a note of that. If we can already follow the breath as
it comes out, in, it starts to penetrate, it starts to go in into each other. The
mind and the object become very close to each other. They come very close
right up in your face, in your awareness, in your consciousness.

\sphinxAtStartPar
Whatever is the predominant sensation, must be observed and noticed.
We  should  spend  80  or  90\%  of  our  time  doing  this  rising,  falling,  sitting,
touching, rising, falling, sitting, touching. If you want to do the fourth object:
rising, falling, sitting, hearing, rising, falling, sitting, hearing, rising, falling,
sitting,  touching,  rising,  falling,  sitting,  touching. You  keep  going  around
and around like that. Only noting the other external objects when they come
to interfere with our cycling. Only when they interfere. We want to stay as
\DUrole{pdfpage}{91}  much as we can with the main object, as much as we can in the present, as
much as we can internalizing.

\sphinxAtStartPar
In the first two days, when you’re a little bit tired, you will find this
kind of hard work. For the first few days you’re doing it and no results are
occurring. But what we’re doing is putting the conditions in place for insight
to unfold.

\sphinxAtStartPar
This  is  how  we  practice  with  the  body  door. This  afternoon  we  will
go into examining feeling, the first object of the mental door. But first we’ll
stick with the body.

\sphinxAtStartPar
Also  whilst  we’re  practicing  in  this  way,  during  the  walking  meditation, you can become aware of the arising of various sense consciousnesses.
Sense consciousness arises at the eye door, the ear door, the nose door, the
tongue door and the body door. It’s very impermanent. It arises for a fraction
of a second and then it passes away.

\sphinxAtStartPar
As our Vipassana practice develops, we are going to go through four
foundations of mindfulness. We are going to look at the physical sensations,
which we have been talking about for the last few days. We’re going to look
at  feeling  which  is  either  pleasant  or  unpleasant.  We  are  going  to  look  at
various mind states. The state of mind with greed or hatred or delusion. And
we are also going to be looking at various reactions and thought processes
that we may have. So there are four foundations of mindfulness to establish
our  awareness  in  the  present  moment,  to  block  craving  from  being  from
entering. And we’ll be going through these as the week goes on.


\section{True Vipassana technique}
\label{\detokenize{2-a:true-vipassana-technique}}
\sphinxAtStartPar
When we are paying attention to the rising, the falling and the sitting,
sense  consciousnesses  will  be  activated.  Sometimes  we  will  hear  things.
We’ll start seeing things or smelling things. These sense consciousnesses are
arising and passing away. You can make a note of them. In fact, true Vipassana practice includes all six doors. We note and know them as they arise and
pass away at the six doors. As my old teacher used to say, note it, know it and
let it go. Whatever is arising. The instructions this morning have been for the
body door. That’s just one of the doors. We are just paying attention to the
body, its posture and its sensations. But the other doors are also opening and
\DUrole{pdfpage}{92}  closing continuously. So we’ll need to start to become aware of them as well.
In the beginning, we start with a selected small group of things to train the
mind on. The rising, the falling, the sitting, the touching, the hearing. This
is a select group so that we can start to train the mind to be present {[}
\sphinxstyleemphasis{directed}
\sphinxstyleemphasis{awareness}
{]}. As our practice develops, we start to let go of this selected group
of meditation objects and our awareness opens up and becomes choiceless
{[}
\sphinxstyleemphasis{choiceless awareness}
{]}. We start to watch whatever is arising, whenever it
is arising at the eye door, the ear, nose, tongue and body doors and the mind
door as well. Once we’ve established a certain level of stability in the mind,
once  our  awareness  and  wisdom  have  become  concentrated  or  stabilized,
then you will be able to bounce between the various doors.

\sphinxAtStartPar
For the beginning stage, we try to stay at the main object as much as
possible, as I said. If any of these other doors are activated, you can quickly
note them and come back. When our practice is fully developed, we won’t
need to stick with the rising, falling, sitting. We will be able to be aware in
the present moment to whatever is going on. Arising and passing away.

\sphinxAtStartPar
So  we  try  to  maintain  our  awareness  on  the  main  object,  but  for  the
sake  of  completeness,  the  body  and  mind  receives  all  its  information  into
through these six doors moment after moment. One door opens and closes,
another  door  opens  and  closes.  Consciousness  can  only  arise  in  one  location. And then it passes away. It does so very rapidly, so that we build up a
perception of the world. We hear the dog barking and see the dog, maybe
through two doors (hearing, seeing) or three doors (with smelling), then our
mind goes through our recognition system, our perception, we recognize it
as a dog. We have seen it and we can smell it and we can hear it. So we paint
a picture of what it actually is. A conceptual view of what is actually there.
When we start to do this, that means our mind is going out into the story of
that barking. It’s going out through the eye door.

\sphinxAtStartPar
What we want to be able to do is to maintain our awareness internally
and allow consciousness to arise and pass away, arise and pass away, arise
and  pass  away  with  us  just  observing,  just  witnessing  what  is  going  on.
We’re not going into the story. We are stepping back from the story. When
there is a sound outside and you have an ear, ear consciousness arises. When
there is a visible form and you have some eyes, eye consciousness arises.
\DUrole{pdfpage}{93}  When you have a nose and there is a scent, nose consciousness arises. We
normally call this hearing, seeing and smelling. I want you to make a note
when you notice that the body has been activated in this way. If you notice
that the eye door has been activated, make a note, seeing. Don’t get stuck
in  what  you  are  seeing.  That  is  called  looking,  watching.  You  have  gone
through the door and you’ve gone to the object and you’re starting to play
with it. You have become engaged with it. So our meditation practice is to
come inside, to notice that one of the sense doors has been triggered and just
make a note. Seeing is occurring. This unit is sitting here, its eyes are open,
seeing is occurring. Hearing is occurring. Smelling is occurring. There is a
physical sensation of touching, touching is occurring. Sometimes it’s going
into  the  past,  remembering  is  occurring.  Sometimes  it  goes  to  the  future.
Planning is occurring. This is what we want to get down and note. We are
not interested in the content of what we are hearing and seeing. We don’t care
if it’s a yellow dog or white dog. Or a brown dog, or a tree, or a rock. We’re
not interested in what we are seeing. We are interested in the fact that seeing
is occurring. We’re not interested in that someone’s T\sphinxhyphen{}shirt is pink or green.
If we are noticing details like that, that means we’re going out through the
doors. We don’t make a note green or pink or whatever, T\sphinxhyphen{}shirt or person.
We’re not making a note like that. We’re coming back before that starts to be
identified with. We’re coming back if we note, «oh, the process of seeing is
occurring and finishes». «Oh, hearing is occurring.» Finished. Smelling. It’s
finished. Rising, falling is occurring. Hearing again. Planning. Future.

\sphinxAtStartPar
The  six  doors  are  opening  and  closing  continuously  throughout  the
day. If we are unmindful of them, each of those events will be subjectified.
Each of those events will be appropriated and identified with either as me,
mine, I or some variation of that ego experience. But if our awareness and
wisdom is able to note and know, if we’re able to keep up with the doors
as they arise and pass away, we won’t go into the stories of the content. We
won’t start to identify with the story and pictures and ideas, that we create
when we see and hear and smell and taste things. We will be able to be very
present and aware and able to see exactly what is going on. This body does
see things, does hear things. There are physical sensations going on. These
are our objects. Don’t think of physical pain as stopping you from meditation. \DUrole{pdfpage}{94}  That pain is the object of the present moment. That wandering mind is
the object of the present moment. So it doesn’t matter whether you can keep
your mind continuously with the breath – of course, it’s preferred. Of course,
that’s what we’re trying to do, what we’re aiming for – but if your mind is
wandering for the first few days of practice, or the physical sensations, the
numbness, the sore back, or the tiredness, whatever your experience is in the
present moment, that can also be an object of our meditation. What we’re
most  importantly  doing,  is  turning  our  experience,  our  life,  into  a  field  of
objects. And  we’re  stepping  back  from  those  fields  of  objects  so  that  we
can see them just arising and passing away without becoming involved with
them, without identifying with them. When we see in this way, it’s called
clear seeing – Vipassana.

\sphinxAtStartPar
We have to see the physical objects, we have to see the mental objects,
we have to see causes and effects and linkages between these two, and if we
can practice in real time with this real data – things that are actually occurring  in  the  present  moment  –  then  we  will  break  through  and  stop  seeing
the individual characteristics of each event and we’ll start to see the general
characteristics  inherent  in  all  conditioned  phenomena. All  the  phenomena
arising and passing away at the six sense doors are impermanent. All phenomena are suffering, when it’s identified with. And all phenomena is nonself,  it’s  out  of  control.  It’s  not  happening  to  anyone.  It’s  just  happening.
There’s no see\sphinxhyphen{}er, but there is seeing occurring. There’s no hear\sphinxhyphen{}er, but there
is hearing occurring. Physical sensations are occurring in the body, but they
are not yours. They don’t belong to you. If you identify with them – suffering
will be your friend.

\sphinxAtStartPar
You have a physical pain in the body and you identify with it, it gives
you suffering. If you don’t identify with it and see it clearly for what it actually is, it doesn’t give you any suffering! It is just a physical sensation, it’s
just a mental state. If we become sad for example, if we identify with the
sadness and start to think, «oh, I am sad, I am unhappy, I am depressed, I am
worried», we will give ourselves suffering. If we can step back from it, note
it, know it and let it go, then we’ve stopped identifying with it. That sadness
may still be there, but it doesn’t cause any dukkha. It doesn’t give us any suffering because it doesn’t belong to us. We haven’t accepted it. – So we step
\DUrole{pdfpage}{95}  away from that stuff.

\sphinxAtStartPar
For the moment we’re doing it with the physical sensations, but we’ll
start to move to the other doors as well. We observe the knowing that’s arising in the present moment. As Ajahn Sumedo says: «The practice is, now is
the knowing.» This is the state of the mind when it’s just knowing what is
arising and passing away in the present moment. Have a think about that!
Now is the knowing. We are just knowing whatever is there without getting
involved in it. When we don’t get involved, it passes away and then a new
object arises. We don’t get involved, it passes away and then a new object
arises. And a new object arises. If we can do this effectively and efficiently,
this  process  will  become  extremely  rapid. We  will  start  to  see  things  in  a
very radically different way. It kind of looks like it’s all so solid and happening. That’s because we’re identifying with the objects of our consciousness. For long periods of time. For days and years, we have been identifying
with our body, with our emotional states. We have been identifying with our
reactions and judgments, with our views and opinions. Until they stay for a
while. We don’t really get to see them passing so rapidly. But when we disconnect it and step back, when we put the car in neutral, and, yes, we are still
rolling down the hill. There is still the rising and passing away but we are
detached. Dispassionate towards what is going on. We’re starting to let go of
things. Meditation is the practice of letting go. So we increase our focus on
letting go to sense doors.

\sphinxAtStartPar
Don’t get trapped in sense doors! Don’t get trapped in sight and sounds.
Don’t start delighting! We are not doing the walking meditation to look to
the jungle or look at other people’s bodies. Don’t allow the mind to go out
during the food time. When you sit there just be objective and aware that
you’re putting food into your head that’s being chewed. Watch the lower jaw
as it goes up and down. Then there is some swallowing. There is some tasting occurring. There may be some pleasantness, some liking. These things
are all going on. But don’t get absorbed into them, especially into your own
thoughts, into your own emotional states. Stop identifying with them.

\sphinxAtStartPar
If you are feeling tired and lazy, recognize tiredness is there. Laziness
is there. Try to disconnect from it. Don’t identify with it. When you can do
that, the tiredness will be there, but it won’t be unpleasant and you won’t be
\DUrole{pdfpage}{96}  affected by it. It’ll just be there as an object. As soon as it is seen clearly, then
it can no longer form a base for the sense of self. The object has been seen
clearly. You will not identify with that. It cannot be a base of me, mine and I.
It cannot be a base of self. So there is cessation in that moment.

\sphinxAtStartPar
You only need to do that continuously. Object after object. Of course,
we will miss a few and we note a few. We will get stuck in some old thought
patterns. Then they start to come up. We will get stuck in them. Try to pull
yourself out as quickly as possible.

\sphinxAtStartPar
If you are starting having conversations with yourself or with others,
imaginary conversations, stop that as quickly as you can. If you notice you’re
having a conversation with yourself, ‘talking, talking’. ‘Madness, madness’.
Just be aware that there is some delight in that conversation. Even though
you want to be meditating, «I just think to the end of this one. Just allow
me to think a little bit more and I will just get to the end of this one.» Stop
it quickly! Train your mind to stop that from going into the content. Come
back and observe the structure. The content will take us away. The content
has led us on a wild goose chase. Our whole lives! The content we’re following around, we’re chasing after craving’s desires, wishes and wants, we are
following that. So now is the time to just drop that. «Well, it started thinking.»  ‘Thinking’.  «Conversing.»  ‘Conversing’.  «Thinking  again.»  ‘Thinking’. «Again.» ‘Thinking. Thinking’. And then come back to your rising and
falling.  ‘Rising,  falling.  Rising,  falling.  Thinking.  Rising,  falling.  Sitting.
Remembering’.

\sphinxAtStartPar
When you’re thinking this over and over again, this is not meditation.
You’re  wasting  your  time. You  are  allowing  your  mind  to  get  absorbed  in
the object. We don’t want to get stuck in the object. There is your thoughts,
your past, your future, the sound of coughing, the sound of dogs, the sound
of insects. Whether it’s the sights of the jungle or other people. Don’t allow
yourself to go into your judging mind or your comparing mind or your liking
mind or your disliking mind. Don’t go into the story! Vipassana practice is
all about stepping out of the story. It’s about detaching. We are detaching,
turning, transforming our fast flowing experience into a field of objects that
can be noted, known and let go. Noting, knowing, letting go. Noting, knowing, letting go. When you do this continuously, object after object, your mind
\DUrole{pdfpage}{97}  will  fall  into  a  nice  state  of  freedom,  of  detachment,  a  state  of  dispassion
where you won’t find any of the content of the dependently arisen, conditioned phenomena of the mind and body. They no longer hold any interest
for you. You will have seen through them, through the magic show that the
mind presents to us. You will be able to let it go. You will be able to live
freely. You won’t be a slave of your emotional states for your own thought
reaction patterns. You can break out of your reaction patterns. You can break
out of emotional states that you have been experiencing for a long time. It’s
quite  possible.  In  fact,  people  have  been  doing  it  for  26  centuries. This  is
how we get a happy and peaceful life. When we step back from the noise and
the nonsense of the mind, step back and have a look at that stuff, the mind
won’t become interested in it at all. It sees it only as objects for the arising
of self. It’s just a continuous show of me, mine, I, this is for me, that’s not for
me, that’s mine, identifying with stuff, identifying with the emotional states,
identifying with everything. Creating a dichotomy in the world, a duality, a
separation from everything else. When our Vipassana practice is successful,
when  we’ve  managed  to  remove  this  sense  of  I  from  our  experience,  we
go back to nature, we go back to what the Buddha called dhamma. We’ve
experienced dhamma for ourselves. We’ve experienced the truth. We don’t
need anybody to tell us what the truth is. We don’t have to believe anybody,
because we have experienced it for ourselves. There is a state of calmness,
coolness  and  peacefulness  watching  this  never  ending  flow  of  mental  and
physical phenomena. It’s all conditioned and dependently arisen. It doesn’t
belong  to  you.  It’s  old  karma  manifesting  from  the  past. When  the  conditions are right, that old karma can manifest. We are then reacting to it. We’re
engaging in it, we’re involving ourselves with it creating fresh karma, which
there goes and gets recycled again. Our experience is the manifestation of
our old intentions and our reactions to that, our creation of new karma, which
is going on all the time if we are unaware. Of course, with mindfulness and
wisdom in the present moment, we can stop this process from happening. We
kind of short\sphinxhyphen{}circuit it. And we can free ourselves from the mind.

\sphinxAtStartPar
Just a little exercise, when you go to the viewpoint. Be still and aware
and watch the six sense doors opening and closing. Don’t get stuck in the
content. We don’t want to know what we are smelling but only that smelling
\DUrole{pdfpage}{98}  is occurring. Smell consciousness is arising in the present moment. That’s
our experience right now. That’s our experience of eating right now. Tasting
is  occurring.  Hearing  is  occurring. And  then  it  passes  away.  It  arises  and
passes away. So go to the viewpoint and just rapidly note ‘hearing, seeing,
touching, hearing, seeing, touching, smelling,…’ try to bounce between the
doors without getting into the content. See how far you can go with it. See
how fast you can note it.

\sphinxAtStartPar
The faster you note it, the bigger the gap becomes, the greater the space
created around the object. If your noting is kind of slow and sporadically,
sometimes you note and sometimes you don’t, then you will be opening a
little gap and then closing again. Opening a little gap and closing again. But
if you can note consistently and continuously, you’ll open the gap until eventually it stays open for quite some time and you will start to experience the
arising and passing away of mental and physical phenomena independent of
desire. You’ll see it in the present moment. And that is accompanied with a
great deal of bliss, a great deal of happiness.

\sphinxAtStartPar
Now  we’ll  practice  some  walking  mediation.  As  far  as  the  walking
meditation is concerned, for those of you who would like to increase their
walking  meditation  you  can  as  well. You  can  also  increase  to  four  stages.
‘Lifting, rising, moving, placing’. Four steps. ‘Lifting, rising, moving, placing’. Four stages just like in our sitting meditation! ‘Rising, falling, sitting,
touching’.  Do  you  see  the  similarity? We’re  using  a  structure  to  train  our
awareness in the present moment.

\sphinxAtStartPar
I  also  suggested  that  you  stand  still  in  one  spot  and  just  move  your
foot back and forth at least five or 10 times each foot, so that we get used
to the sensation on the sole of the foot. It’s the sensation on the sole of the
foot that we’re following here. You should not have an image of your foot in
your mind, you should not look at your foot. These are all conceptual things.
\sphinxstyleemphasis{We want our meditation to break through to ultimate reality. So we are only}
\sphinxstyleemphasis{paying  attention  to  the  physical  sensations  in  these  for  stages.}
There  are
different  sensations  in  each  stage. When  you  lift  the  foot  up,  what  is  that
sensation  lifting  up?  Is  it  heavy  or  does  it  feel  light?  When  we  place  the
foot down, what does that feel like? Is it heavy or does it feel light? You can
investigate that for yourself.

\sphinxstepscope


\chapter{Day 2, afternoon}
\label{\detokenize{2-b:day-2-afternoon}}\label{\detokenize{2-b::doc}}
\LOCALaudiolink{https://www.mixcloud.com/anthonymarkwell/day-2-afternoon-talk/}

\sphinxAtStartPar
We  are  going  to  have  a  look  at  the  mental  aspect  of  our  experience.
We are going to look at what is known as
\sphinxstyleemphasis{vedana}
or feeling. It plays a very
important role in the Buddha’s teaching. Vedana or feeling is one of the four
satipatthanas. It’s the second satipatthana, the second foundation of mindfulness.

\sphinxAtStartPar
We use the analogy of the Shwedagon pagoda in Yangon, Myanmar.
There is 100 meter tall pagoda on top of a huge platform. In order to get to
that platform, you need to walk up one of four gigantic staircases. You can
use any of the stairways to walk up to the platform, to walk up and see the
pagoda.

\sphinxAtStartPar
Here we are using the pagoda as an analogy for our present moment
experience  and  we’re  using  the  four  staircases  for  the  four  foundations  of
mindfulness. We can establish awareness in the present moment using one of
the four foundations. You can use the body door, that we have been looking
at already – the physical sensations and postures of the body. We can also
use feeling – pleasantness and unpleasantness. We can use our mind state in
that  present  moment  experience  as  a  foundation  for  mindfulness. And  we
can also use thoughts or reactions, that come in the present moment experience as a foundation for mindfulness. Depending on your skill, you can use
anyone  of  the  four  foundations  of  mindfulness  to  establish  awareness  and
\DUrole{pdfpage}{100}  wisdom in the present moment. You can choose. All four are available. Just
like  the  stairways  that  lead  to  the  pagoda  platform,  they  are  all  available
for us. We can use the body, feelings, emotions or thoughts to establish our
awareness in the present moment.


\section{Feeling}
\label{\detokenize{2-b:feeling}}
\sphinxAtStartPar
So feeling is one of the four foundations of mindfulness. Feeling has
been  isolated  because  of  importance  of  it’s  role  in  the  Buddha’s  teaching.
Feeling is also one of the five aggregates. Again it’s isolated from the other
mental aggregates of perception and sankharas, conditional formations. It’s
part of nama, the mind, but it’s separated. Feeling also plays an important
role in the
\sphinxstyleemphasis{paticca samuppada}, dependent origination, as feeling is the cause
and condition for craving. Craving is the cause and condition for dukkha. So
feeling plays a very important role in the teachings of the Buddha. So we
need to understand precisely what feeling is.

\sphinxAtStartPar
Feeling or
\sphinxstyleemphasis{vedana}
is the cause and condition for delight and lust. It’s
based on feeling, particularly pleasant feeling, that we start liking things, we
start delighting in things. Lust arises in the mind. Lust for being, craving for
being arises.

\sphinxAtStartPar
Feeling  is  simply  the  affective  tone,  or  affective  component  of  our
experience.  It’s  the  flavor  of  our  experience.  Feeling  is  either  pleasant,
unpleasant  or  neutral.  There  are  three  types  of  feeling:  pleasant,  unpleasant,  neutral.  Structurally  one  type  of  feeling  will  be  arising  in  the  present
moment,  continuously,  with  the  other  of  the  five  aggregates,  and  passing
away. Arises and passes away. It is just the flavor. It’s not the physical sensations in the body – they are the four elements, they are physical – it is not
the emotional state, that we find ourselves in sometimes, that is a mind state.
Feeling is something different. Feeling is just the unpleasantness of an experience or pleasantness of an experience, just that. Yet it plays an enormous
role,  because  feeling  is  known  as  the
\sphinxstyleemphasis{citta  sankhara}.  Or  the  mind  conditioner. Our minds have been conditioned by feeling. We are going to need to
understand feeling, not only on a general level, but on an intuitional level,
on a meditational level. We are going to have to see for ourselves how pleasant feeling conditions liking and craving that leads to dukkha, and unpleasant \DUrole{pdfpage}{101}  feeling conditions disliking and aversion which also leads to dukkha, if
feeling is not noted and known. Then it will be appropriated and identified
with, it will be subjectified, it will be for me. As it’s arising in nature, it’s just
pleasantness or unpleasantness arising and passing away. It doesn’t belong to
anybody. It’s just a feature of the mind and body process. It is just one part,
the tone or the flavor of the experience whether it’s pleasant or unpleasant.

\sphinxAtStartPar
We need to really note it quickly, when it starts to arise, before it starts
to condition the mind and before we loose our awareness and wisdom in the
present moment, before we get taken away by a pleasant story or an unpleasant experience. We need to be able to note this in the present moment clearly.
Feelings are impermanent, they arise and pass away, extremely rapidly,
they  are  conditioned,  they  are  dependently  arisen. They  are  the  subject  to
fading away and ceasing.

\sphinxAtStartPar
There are six types of feeling: feeling born of eye contact, of ear contact, of nose contact, of tongue contact, of body contact, and feeling born of
mind contact. Contact is that type of state where our internal base and the
external objects, that relate to that base, contact each other and consciousness  arises.  Consciousness  arises  joined  with  feeling. They  arise  and  pass
away.
\sphinxstyleemphasis{Mind and matter and consciousness arise and pass away together at}
\sphinxstyleemphasis{the six sense doors.}
So there is feeling which can be pleasant, unpleasant or
neutral at the eye door, that can be pleasant, unpleasant or neutral at the ear
door, mind door and so on.

\sphinxAtStartPar
If you feel a pleasant feeling, you won’t feel an unpleasant or neutral
feeling at that moment. Feeling is mutually exclusive to the moment. It is
either pleasant, unpleasant or neutral. And then it passes away. Of course,
you  can  have  all  kinds  of  thoughts  or  emotional  states  towards  a  feeling.
Feeling will arise and pass away, rapidly replacing one another.

\sphinxAtStartPar
The Buddha was asked by the wanderer Pataliputta:
\sphinxstyleemphasis{«Why does feeling arise in the present moment? Why in every single moment is it pleasant,
unpleasant, neutral? Why is that part of the structure of our experience, the}
\sphinxstyleemphasis{structure, the nature that we’re subjected to?»}
–
\sphinxstyleemphasis{«Having done an intentional
action through body, speech and mind whose resultant is pleasant, one feels
pleasure. Having done an intentional action through body, speech and mind
– that’s karma, having created some karma, some intentional action, some
\DUrole{pdfpage}{102}  cetana; cetana, intention, is karma – whose resultant is unpleasant, one feels}
\sphinxstyleemphasis{unpleasantness.»}
So the feelings that we experience in the present moment
are the resultants of old intentional structures. Old karma, that we have created. That remains latent.
\sphinxstyleemphasis{Kama sati}, karmic energy, as a potentiality, only
has an opportunity to arise when its conditions are in place. If there is some
condition in place for it to hatch and manifest, to bloom, then it will! And
it will give its fruit. Either pleasant or unpleasant or neutral. If the karma
that was created was wholesome or unwholesome than the resultant will be
experienced as pleasant or unpleasant feeling, respectively.

\sphinxAtStartPar
When they bubble up, they don’t belong to anybody. It’s not you, you
didn’t create an intentional structure.
\sphinxstyleemphasis{There is nobody here that is creating}
\sphinxstyleemphasis{karma. Karma is creating itself.}
Reaction. The five aggregates are creating
themselves  every  moment. And  when  that  resultant  bubbles  up,  if  we  are
unmindful, if we fail to activate our awareness and wisdom in the present
moment and note and know it, see it clearly and let go of it, so that it is not
a base for the arising of self, then that feeling will simply be appropriated
and identified with. The feeling will happen to me, it will be mine, the feeling will be subjectified. Then we just react to it with liking if it’s pleasant or
with disliking if it’s unpleasant. Of course that reaction is new karma. That’s
a new intentional structure through body, speech and mind which will give
a resultant to be experienced as pleasant or unpleasant feeling. So it keeps
looping. Around and around like this. This is what ignorance is. That is the
nature of ignorance. Unawareness in the present moment. If we are unable to
see this structure happening in real\sphinxhyphen{}time in our mediation practice, we won’t
fully understand what it is to be stuck in samsara. This is a very tight and
concrete structure. It fits together very well. It has been operating successfully and managing itself for a very long time. This is its latest manifestation
sitting here on the meditation mat.

\sphinxAtStartPar
Receiving  resultants,  reacting  towards  them,  creating  new  karma,
receiving  resultants,  reacting  towards  them,  creating  new  karma,  receiving resultants, reacting, creating new karma – moment after moment. Each
of  these  moments  giving  us  a  little  taste  of  me  or  mine. When  these  little
moments are added up together they produce a sense of I, a self, a me, an
identity. They give this mind and body process, which is flowing on in nature,
\DUrole{pdfpage}{103}  to believe it is somebody. It’s alive. It’s calling itself something. In fact, our
parents called it something. We started to maintain that name as well.

\sphinxAtStartPar
We need to be very careful about pleasantness or unpleasantness. Pleasantness and unpleasantness is the trigger. Whenever you find this triggering
you, whenever you find yourself reacting, if you feel you’re about to explode
in anger, or you can feel an addiction urge arising, a liking and a disliking, a
pushing and pulling away, have a look at the feeling. Start to note the pleasantness or unpleasantness of the situation and you will be able to free your
mind from those states. The most important thing is not to appropriate feeling or identify with it. The Buddha tells us, that the way leading to the origination of identity view, to the arising of identity view, we regard the internal
base, the external base, the consciousness that arises when these two meet
together,  and  the  feeling  born  from  those  bases  touching  together.  When
we take these things as mine, this I am, this is myself, that’s when identity
view arises. Identity view ceases when we stop appropriating and identifying the internal base, the external object, the consciousness that arises in that
moment (the knowing), the contact of that base and the feeling of that base.
These five things. If we can stop appropriating and identifying with them at
the six sense bases, then we are free. We are free of identity view. Identity
view is the large leash that keeps us bound to samsara. It’s the view that the
mind and body process comes to believe that it is someone. The Buddha’s
teaching is to remove that sense of I from the picture. Remove the sense of
me from our experience so that we can go back to nature. The Buddha says,
that all the conditioned phenomena, all the mental and physical phenomena,
that we are identifying with are dukkha. The whole lot. All the physical sensations, all the mental sensations, the thoughts, the emotions, the pleasant,
the unpleasant, all of them. It’s all conditioned nonsense arising and passing
away. It doesn’t belong to anybody. We need to be able to see this if we want
to breakthrough. If you want to evolve as a spiritual being and remove the
sense of self, remove the ego, remove the duality, that we have put into the
moment, where there is a me and a world. That’s how we are operating. With
separation. When we remove the sense of self, there is no separation at all.
There is a knowing. And it’s blissful.

\sphinxAtStartPar
We need to understand and see these feelings as impermanent, as suffering \DUrole{pdfpage}{104}  and as non\sphinxhyphen{}self. Of course pleasantness is going to arise. The Buddha
still calls it dukkha. It’s important to understand here, what the Buddha is
asking  us  to  let  go  of,  is  not  the  pleasant  feeling  but  the  attachment  and
identification that we developed towards pleasant feeling. Pleasant feelings
will arise, it is their nature to arise, old karma is unfolding. You will experience some pleasantness and some unpleasantness in your life. And there
is nothing wrong with that. That is nature. What the Buddha’s teaching is
all about, is stopping the identification with the feelings. That gives us the
problem.  When  we  start  identifying  with  something  unpleasant,  than  it  is
dukkha right away. When we start identifying with something pleasant, then
it is dukkha when that object passes away, when we loose it. Dukkha arises
through attachment. Freedom arises through non\sphinxhyphen{}attachment.

\sphinxAtStartPar
But don’t think that you have to give up all those special activities that
you like to do. You don’t have to give up your life to practice the Buddha’s
teaching.  Pleasantness,  pleasant  activities,  friends  who  are  nice,  these  are
all ok things. You can still eat, go on holidays, listen to music but don’t get
attached to the pleasantness. Stop identifying with it and free yourself from
it. Then you will be able to enjoy the world without an ego getting in the
way. It’s a much higher, more fulfilling sense of enjoyment. There is a bliss
that arises when you can let go of the pleasure that’s created through sensuality. At the moment we don’t see anything higher than the pleasures that arise
through the five physical sense doors. They give us so much pleasure! Or we
pleasure ourselves through our own thinking, our own rumination.

\sphinxAtStartPar
In  the  dependent  origination,  between  the  six  sense  bases  and  feeling is what is known as contact or
\sphinxstyleemphasis{phassa}.
\sphinxstyleemphasis{Phassa is the point in the sense
perception  process,  where  we  activate  our  awareness  and  note,  know  and}
\sphinxstyleemphasis{let  go  right  at  the  point  where  consciousness  is  arising.}
When  there  is  an
internal base and an object, consciousness arises. Ear consciousness arises
when there is an ear and a sound. Body consciousness arises when there is
a body and something touching the body. Sight consciousness arises when
you have an eye ball and there is sufficient light and visible forms available.
When these two touch together, it is called contact. They cause the arising
of consciousness in the present moment. We need to be able to note right at
that point of contact. Right at the six sense bases where they’re arising and
\DUrole{pdfpage}{105}  passing away. It’s right at that point of contact, if it’s unnoted and unknown,
that craving enters the sense perception process. Craving for being gets in
there. This very strong desire to be something, to become somebody. Unfortunately it’s rife in our society. Our whole education system is pushing us to
become something. The subjects we have to choose. The society, our family,
friends  all  are  pushing  us  to  become  something. This  craving  for  being  is
defilement. It is the cause of dukkha. The Buddha told us to note it, know
it and let it go. Free our mind from it. Craving for being has an opportunity
to enter the sense perception process right at the point of contact. It’s conditioned by the feeling that’s arising there.

\sphinxAtStartPar
In the
\sphinxstyleemphasis{Madupindika sutta}, the Buddha gives a very succinct description of the sense perception process.
\sphinxstyleemphasis{«When there is eye and physical forms,
eye consciousness arises, the meeting of the three is contact, with contact}
\sphinxstyleemphasis{as condition feeling.}
(The grammar then changes in the Pali text.)
\sphinxstyleemphasis{What one
feels,  that  one  perceives.  What  one  perceives  that  one  thinks  about.  What}
\sphinxstyleemphasis{one thinks about that one mentally proliferates.»}
Mental proliferation means
appropriation and identification. It means taking things as mine, this I am,
this  is  myself.  Up  to  the  point  of  feeling,  the  sense  perception  process  is
objective. There is no subjectivity in there. Eye, plus form, eye consciousness  arises,  the  meeting  of  the  three  is  contact,  with  contact  as  condition,
feeling. This is an impersonal process that is happening to nobody. If there is
no awareness and wisdom in that moment, when this contact is taking place,
then craving sneaks in. What one feels, that one perceives. What one perceives, that one thinks about. It is already starting happening to somebody.
The feelings and the perceptions and the thoughts are for somebody, happening to somebody. They are mine. So subjectification has taken place. And
then mental proliferation besets a person with respect to past, present and
future visible forms cognized through the eye. It happens extremely rapidly.
You will need to be continuous in your meditation practice to be able
to catch it. You won’t catch it quickly or easily. But if you are diligent, if
you  are  practicing  correctly,  moment  after  moment,  without  any  desire  to
try and get any results – we need to drop that desire, we need to check our
attitude – if you are just practicing continuously, «bum», and you will drop
into the present moment, and you will see things as they are. The mental and
\DUrole{pdfpage}{106}  physical phenomena will become quite clear to you. You will be able to see
them arising and passing in the moment, you will be able to see mental and
physical phenomena, that have been subjectified and are under the control
of craving, and you will be able to see mental and physical phenomena, that
have escaped the subjectification process, that are free data, that pass away,
that are no longer useful or able to continue the illusion that there is somebody here. It is only because we are not looking that we think we are someone. As soon as you start looking, you won’t be able to find anyone there. It
is impersonal. It’s just a natural process unfolding, like a tree growing in the
forest! It starts small, then grows up, then gives fruit, and then it gets old and
falls over. Nothing much different than this body here. It gets born, it grows
old, it gets sick and dies. All the time we’re holding on to it preciously thinking it’s me. Actually it’s a huge burden.

\sphinxAtStartPar
It is a real hassle to have a body. You have to take it everywhere, you
have to feed it, you have to wash it, clean it and cloth it, make it look fancy
sometimes. It is like a 30 kg backpack. You have to take it all over the place
with  you.  It’s  got  six  radars  out  there  collecting  information  through  the
sense doors. We spend most of our lives taking care of this body. We think
it is really important. It is just earth, water, fire and air. It’s just the four elements arising and passing away. Because it has got a nice conceptual structure around it, we have come to think it is me. We are willing to work for 50
years, to borrow large amounts of money from the bank to buy a concrete
room to keep it at night. Why are we doing that? We need a fancy room to
keep it, this rotten log of festeringness. If we don’t see it clearly we’ll take it
very seriously, we take our life very seriously. We’ll take ourselves very seriously. We get upset when people challenge it. When we start to see it more
clearly though, it starts to loose its importance to us.

\sphinxAtStartPar
So the six sense bases of contact are impermanent and are subject to
change. A pleasant feeling or an unpleasant feeling that arises from impermanent bases, do you think that such a feeling can possibly be permanent?
No! It is also impermanent. It also passes away very, very rapidly.

\sphinxAtStartPar
Notice  when  the  dog  barks.  It  takes  half  a  second.  It’s  just  a  quick
amount of noise and then it’s gone. Finished, it’s passed away. But our mind
takes this fraction of a second of noise and starts to spin. «What’s that dog
\DUrole{pdfpage}{107}  doing here? Why is there a dog here? Which dog is it? I don’t like dogs. My
dog at home…» – and then you get into a fantasy about dog walking. You’re
gone  for  five  minutes…bring  yourself  back  to  the  present  and  start  again.
The early days on the meditation retreat are very much like that. We need to
activate our awareness continuously. Keep activating, keep pulling yourself
back inside. Use the ear door to pull you back in. Sound is always there. We
are not listening to sounds. That means we have gone out if we are listening. We are holding our awareness inside the body, just noting that hearing
is occurring. If you can do that, you’ll find that you can come back to the
present very rapidly. Hold it there for a bit. Bring your mind into the body.
Start to pay attention to the rise and fall. Get a few in, and then, wooop, it’s
gone again. Activate your awareness, come back to the present, keep doing
it. We are just training. We are training the mind as it runs out. We bring it
back. Like when we’re fishing. Bring it back. Than it runs out. Bring it back.
Finally, it is so tired, that we can just bring it in. It won’t find any interest
in going out anymore and we can hold it inside. That’s when our meditation
really starts to stabilize, when we can develop awareness of the body. Full
awareness of the body is what we want to develop.

\sphinxAtStartPar
The importance of feeling and its place in the Buddha’s teaching is also
shown to us by the
\sphinxstyleemphasis{Bramachalla sutta}
(first discourse in the large collection
Diga Nikaya).
\sphinxstyleemphasis{Brama}
means the highest,
\sphinxstyleemphasis{challa}
is a net. The Buddha throws
a net across 62 views. In the Buddha’s time, he identified 62 different views.
There  were  various  philosophers,  recluses  and  meditators  and  priests  and
brahmins. All of them had their own ideas how the world was functioning,
they all had their philosophical standpoint. They all had their different views
on  how  the  sense  of  self  arises  and  what  it  is. All  the  monks  and  nuns  in
Burma have to study it for their exams. It is an important sutta.

\sphinxAtStartPar
It goes something like this:
\sphinxstyleemphasis{«When those ascetics and brahmins, who
are  speculators  about  the  self  in  the  past,  who  are  speculators  about  self
in  the  future,  who  have  fixed  views  about  the  self,  who  put  forward  views
about the self in these 62 ways, it is merely the feeling of those who do not
know and do not see, the delight, worry and vacillation of those immersed}
\sphinxstyleemphasis{in craving.»}
These 62 views are all conditioned by contact. Present moment,
contact,  contact  arising  at  the  physical  doors.
\sphinxstyleemphasis{«That  all  these  speculators
\DUrole{pdfpage}{108}  should  experience  feeling  without  contact  that  is  impossible.  They  experienced  the  feelings,  pleasant,  unpleasant  and  neutral  by  repeated  contact}
\sphinxstyleemphasis{through the six sense basis. With feeling as condition craving arises.»}
Luckily  the  Buddha  gives  us  some  instruction.
\sphinxstyleemphasis{«When  a  monk  understands  as
they really are, the arising and passing away of these six sense bases, when
he understands the satisfaction of the six sense bases, when he understands
the danger of the six sense bases, and the escape from them, then he knows
that which goes beyond all these views of self.»}

\sphinxAtStartPar
So when we understand the point of contact, where feeling is arising,
we understand the arising and passing away of feeling, when we know it to
be an impermanent, dependently arisen, conditioned phenomena. We see it
arising, we see it passing. We see the satisfaction that we get from feeling,
the pleasantness, the joy, the delight and lust we get from feeling, and also
see the danger that feeling presents to us. In fact, that feeling is the cause
and condition to delight and lust and craving. If we see the danger of feeling,
than we are able to let it go. We move beyond feeling. We see the escape
from feeling. So these five things are to be seen and understood. We know
the  arising  of  feeling,  we  make  a  note  when  it’s  arising,  when  it’s  there.
We make a note when it’s passed away. Pleasantness there. Now it’s gone.
We know the arising and passing. We know the satisfaction. We know the
danger. And then we see the escape from it. When we have completely let go
of feeling, in that moment, we’re free. No self arises in that moment. That’s
a moment of freedom.

\sphinxAtStartPar
Our job as meditators is to just extend that moment out indefinitely as
much as we can. In the beginning stages we only get a fraction of a second
as a look but then the curtain opens and stays open for a little bit longer, for
seconds or minutes, maybe for hours where we’re experiencing the mind and
body process without a sense of self. And it’s gone when we are right there
in the present moment. We are free from dukkha.

\sphinxAtStartPar
\sphinxstyleemphasis{«That one shall here and now make an end of suffering, without abandoning the underlying tendency to lust or pleasant feeling, without abolishing the underlying tendency to aversion for painful feeling, without removing
the underlying tendency to ignorance with regard to neutral feeling, without
abandoning ignorance and arousing true knowledge, this is impossible.}

\sphinxAtStartPar
\sphinxstyleemphasis{\DUrole{pdfpage}{109}  When one is touched by a pleasant feeling, if one delights in it, welcomes  it,  remains  holding  to  it,  then  the  underlying  tendency  to  lust  lies
within one.}

\sphinxAtStartPar
\sphinxstyleemphasis{When one is touched by a painful feeling, if one sorrows and grieves
and laments, weeps, beating ones breast and becoming distraught, then the
underlying tendency to aversion still lies within one.}

\sphinxAtStartPar
\sphinxstyleemphasis{When  one  is  touched  by  a  neutral  feeling,  if  one  doesn’t  understand
as  it  actually  is,  the  arising,  passing,  the  satisfaction,  the  danger  and  the
escape in regard to that feeling, then the underlying tendency to ignorance
lies within one.»}

\sphinxAtStartPar
So  you  see  that  coming  to  understand  feeling,  noting  the  pleasantness or unpleasantness of a particular situation is very important if we’re to
remove ourselves from these defilements. If we are to let go of lust, aversion
and ignorance. Greed, hatred and delusion. When we can remove these from
our  present  moment  experience,  then  we  see  that  which  is  beyond,  when
our Vipassana  insight  becomes  strong.
\sphinxstyleemphasis{Phassa}
or  contact  does  not  yield  a
self. This is the point where the knowledge of arising and passing away is
observed. The Buddha called it the deathless state. He called it the
\sphinxstyleemphasis{amata dathu}.
\sphinxstyleemphasis{Amata}, there is no death occurring in that present moment experience
because there is nobody there who dies. It’s the deathless because we moved
beyond the realm of birth and death. Birth and death are concepts that relate
to a person, that relate to an individual. There needs to be somebody before
there  can  be  birth  and  death.  If  the  mind  and  body  process  are  not  thinking of themselves as somebody, if they have escaped that delusion, if they
have uprooted that ignorance in that present moment, if they have developed
awareness and wisdom in that present moment, then they can escape feeling
and they escape dukkha. And they realize the deathless, the arising and passing away of phenomena.

\sphinxAtStartPar
When we are considering feeling, it is important that we don’t get confused between feelings and sensations in the body or mind states in the mind.
To give you a little example, the four foundations of mindfulness, the four
walkways to the pagoda platform, are always available for us to use. We can
walk to the present moment using anyone of the staircases. For example a
mosquito comes and bites you on the arm:
\begin{itemize}
\item {} 
\sphinxAtStartPar
\DUrole{pdfpage}{110} In that present moment experience there can be some tingling, or itching, maybe a little bit of heat occurring. That is a physical sensation.
That’s the first foundation, mindfulness of the body or
\sphinxstyleemphasis{kaya\sphinxhyphen{}nupassana}. We can direct our awareness and wisdom in the present moment on
that physical sensation. We have established mindfulness on the body,
on a physical sensation.

\item {} 
\sphinxAtStartPar
If we choose to, we can establish our awareness on the unpleasantness
of that situation. There is also an experience of unpleasantness. We can
establish  our  awareness  just  on  the  unpleasantness.  It’s  called  mindfulness on feeling,
\sphinxstyleemphasis{vedana\sphinxhyphen{}nupassana}. Just on the unpleasantness, not
the physical sensation on what our mind is complaining and whinging
about, just the unpleasantness. You can establish awareness on feeling.

\item {} 
\sphinxAtStartPar
If we want to have another look at this situation, we can look at our
mind,  the  third  foundation  of  mindfulness.  The
\sphinxstyleemphasis{citta\sphinxhyphen{}nupassana}.  We
can establish our awareness on the mind state, which is most probably
aversion. Aversion is that type of dissatisfied mind. We are bitten by
a  mosquito,  the  mind  is  full  of  aversion. That’s  the  third  foundation.
We can use that to establish our awareness on, to note and know that
present moment experience. To blow it apart with our awareness and
wisdom so that it is not used as the base for the arising of me, mine or
I. We don’t go into it and start to think, «Oh, my itchy arm. Why does
it happen to me all the time?»

\item {} 
\sphinxAtStartPar
We can establish our awareness on the fourth foundation of mindfulness,
\sphinxstyleemphasis{dhamma\sphinxhyphen{}nupassana},  which  is  the  reaction  process,  the  thought
process,  the  karmically  active  process.  That’s  the  actively  pushing
away, that kind of energy of disliking, pushing. It’s also available to us
as a foundation for mindfulness.

\end{itemize}

\sphinxAtStartPar
The physical sensation – the itchiness, or feeling – the pleasantness,
or the mind state – the aversion in the mind, or mind objects – our reaction
of disliking towards it. So in four ways we can become mindful of a present
moment experience. This is a very simple example. It happens moment after
moment, after moment. That is what our experience is. The five aggregates
arising  together  and  ceasing  together,  in  every  moment  of  our  conscious
experience,  there  will  be  a  physical  sensation,  there  will  be  some  feeling,
\DUrole{pdfpage}{111}  there will be some mind state and there will be some reaction going on.

\sphinxAtStartPar
We are establishing our awareness stepping back from what is going
on. We are establishing awareness in the present moment so that we can see
things as they really are. When we can see things as they really are, we can
begin to let go of them. We do let go of them. If we are unable to do this,
then we’re just subjected to sensuality. If we’re unable to note and know in
the present moment, if we can’t activate our awareness and be aware of our
own mind states, then we have nothing else to do but simply follow along the
flow of craving wherever it takes us. We become the slave of craving. The
slave of our own mind. And the only way we know to escape from dukkha,
is through sensuality looking for something pleasant, something enjoyable,
forms cognizable by the eye, or sounds cognizable by the ear, sent cognized
through the nose, tastes cognized through the tongue, sensations cognized
through the body. That will be our only escape to our dukkha when it’s arising. Seeking out one of these things. The pleasure and joy that arises, dependent on the cords of sensual pleasure is simply called sensual pleasure. The
Buddha regarded it as a filthy pleasure, a false pleasure, an ignoble pleasure.
\sphinxstyleemphasis{«I say this kind of pleasure that it should not be pursued, that it should not}
\sphinxstyleemphasis{be developed, that it should not be cultivated, that it should be feared.»}
He
knew the danger this sensual pleasure leads to. It certainly doesn’t lead to
awakening. It leads to burying deep into samsara. We are delighting in the
cause of dukkha.

\sphinxAtStartPar
So  controlling  our  reactions  to  these  three  feelings,  the  pleasant,
unpleasant  and  the  neutral.  Neutral  feeling  probably  takes  up  98\%  of  our
time.  There  is  maybe  1\%  pleasantness  and  1\%  unpleasantness.  But  those
little 1\% are what we live for. Desiring something delightful and pleasant.
Trying to escape from the unpleasantness. If it’s slightly unpleasant we try to
change it, try to get rid of it, searching for something pleasant.

\sphinxAtStartPar
When  we  are  in  stress  or  a  difficult  situation  such  as  an  emotional
break\sphinxhyphen{}down, we go for a snack, chocolate, ice cream, or some other kind of
snack, just to try to distract ourselves from the dukkha that we got ourselves
in.  But  the  correct  strategy  is  to  use  awareness  and  wisdom,  to  note  and
know and let go. Don’t go looking for outside options to cure yourself. The
best cure we have, is right here, the mind and body process itself, awareness
\DUrole{pdfpage}{112}  and wisdom.

\sphinxAtStartPar
\sphinxstyleemphasis{So controlling our reactions to these three feelings through attention
and  observation  at  the  very  point  where  defilement  arises,  is  the  heart  of}
\sphinxstyleemphasis{mind training.}
That’s what we are doing here. We are training the mind in
non\sphinxhyphen{}reaction. This  does  not  mean  that  we  become  boring  or  a  rock  in  the
forest, sitting there completely unmoved and unchallenged by anything. Of
course, you can still have some fun in your life. The Buddha is not advocating the death of all fun, just don’t get attached to it. Or as Ajahn Chah says,
«you can enjoy as much salt as you like as long as you don’t find it salty».

\sphinxAtStartPar
So  we  restrain  our  senses. We  keep  an  observation  on  our  six  sense
doors. We are watching, we are noting, knowing and letting go each moment
activating our awareness and seeing what’s there. When we have activated
our awareness, there is always going to be an object. Either a physical object
or  a  mental  object  –  we  can  choose. All  we  have  to  do  is  to  maintain  our
awareness and wisdom. Awareness will take us to the object and penetrate it
clearly. Wisdom allows us to disengage from it and step out of the identification process. In that case it is let go of. Dukkha ceases. Dukkha ceases when
wisdom  removes  ignorance. When  wisdom  is  unable  to  uproot  ignorance,
then  the  flow  of  paticca  samuppada,  dependent  orientation,  must  result  in
dukkha. Awareness and wisdom do have the power over ignorance. If they
didn’t  have  the  power,  no\sphinxhyphen{}one  would  be  able  to  get  enlightened.  Because
they do have the power, it is possible that we can break free of this mind and
body process. We can allow it to do just what it does and we can enjoy the
space and freedom of consciousness, which is not addicted to the mind and
body process.

\sphinxstepscope


\chapter{Day 3, morning}
\label{\detokenize{3-a:day-3-morning}}\label{\detokenize{3-a::doc}}
\LOCALaudiolink{https://www.mixcloud.com/anthonymarkwell/day-3-morning-talk-five-hindrances/}

\sphinxAtStartPar
Day  three  is  sometimes  characterized  as  being  a  difficult  day.  But  I
often feel that the first and second days are more difficult myself. You will be
feeling very tired. You will be feeling quite exhausted. Don’t start following
your thoughts or getting trapped in your thinking, in your own whispering
mind. Of course it’s challenging. Everyone is facing their own challenges. It
won’t be long before you can pull yourself out of these mind state. It’s what
we are here for. We are training to pull ourself out of difficult mind states
such as stress, worry, anguish. We are learning how to come out of it. If we
never  learn  how  to  come  out  of  that  through  meditation,  then  you  always
come out of your mind states using something else. Wether it be some kind
of sensuality or some kind of distraction, some kind of technology or whatever it may be. So why not use our own mind, our awareness and wisdom to
pull us out of these mind states?

\sphinxAtStartPar
This  morning  we  are  going  to  talk  about  the  five  hindrances  or  the
\sphinxstyleemphasis{panca  nivarana}.  These  are  mind  states  which  are  contained  in  the  fourth
section  of  the  satipatthana  text,  in  the  section  called
\sphinxstyleemphasis{dhamma\sphinxhyphen{}nupassana}
or  recollection  of  mind  objects. These  are  thought  processes  that  are  arising  and  passing  away.  The  Buddha  asked  us  to  observe  them,  to  become
aware of them, to fully understand them. To say these five hindrances are an
impediment to our meditation would be an understatement. They do disturb
\DUrole{pdfpage}{114}  our practice. They make it difficult to do the mediation practice, however,
these mental states are also an important vehicle for insight. You can learn a
lot about the nature of the mind through observing your reactions and your
thought patterns to things.

\sphinxAtStartPar
We can temporarily overcome these hindrances in our meditation practice, but they are only fully removed at the noble eightfold path. When we
reach  the  level  of  stream  entry,  some  of  the  hindrances  will  be  cut  off,  in
particular doubt will be cut off.

\sphinxAtStartPar
Turning our attention to these hindrances at the time of a meditation
retreat can sometimes be a little bit late. These hindrances are always arising in our mind. They’re always there. In fact, they are much of our mental
life which is spent floating around in those hindrances. It’s because of these
hindrances that our mind does not become concentrated at deeper levels of
samadhi.  Because  these  hindrances  are  always  getting  in  our  way,  they’re
always blocking us, always disturbing our meditation development.

\sphinxAtStartPar
So what we will be doing here is identifying these mental states. You
will all be experiencing one or two of the hindrances or unfortunately maybe
five of the hindrances. Listen to them and see if you can identify these mental
states going on and see if they’re causing you any problem.
\sphinxstyleemphasis{We just need to
note them, know and let them go! The same technique is prescribed to the}
\sphinxstyleemphasis{mental objects and to mental states.}
They are constantly with us, these five
hindrances. The five hindrances are constantly coming to disturb us.

\sphinxAtStartPar
The five hindrances are legitimate objects of our satipatthana practice.
They are presently arising, they are dependently arisen, conditioned and they
are impermanent and they are unsatisfactory and they are impersonal. They
don’t belong to anyone. So see if you can notice these defilements. These
five hindrances are types of defilements, types of imperfections of the mind
that  come  to  cloud  the  mind  in  their  own  way  so  that  the  mind  can’t  see
things clearly. They kind of get in the way by the five hindrances attacking
the mind.

\sphinxAtStartPar
So in our meditation practice we need to keep alert, be aware. If you
find  that  you  are  being  attacked  by  one  of  these  five  hindrances,  you  will
need to address the problem. You will need to have a close look at it, make
a note of which hindrance is occurring and try to step away from it. Try to
\DUrole{pdfpage}{115}  disconnect or disengage from the mental state. It will still be there, it may
have some unpleasantness or it may have some pleasantness associated with
it. Try to step away, stop engaging with that particular state. Watch it pass
away. Watch it dissolve under the light of awareness and wisdom.

\sphinxAtStartPar
There’s a wonderful quote the Buddha gives us.
\sphinxstyleemphasis{«Whatever a bikkhu
frequently thinks and ponders upon, that will become the inclination of his}
\sphinxstyleemphasis{mind.»}
Whatever we think about, in the way we are thinking whilst we are
thinking, we are actually conditioning our own mind until we become that
person! If we are constantly getting frustrated and agitated, constantly developing little states of aversion all the time, then chances are we are developing that and that we turn into that person. We are going to turn into an angry
person! We’re  going  to  turn  into  a  person  with  a  lot  of  issues  about  other
people or other things. So be aware of this tendency, be aware of this conditioning process. Whatever you frequently think and ponder upon, that will
become the inclination of your mind. So be very careful!

\sphinxAtStartPar
If you constantly worry about things, you are conditioning your mind
to be in a state of fear and then you become that person. You start to live that
reality because you have been conditioning your mind. Frequently thinking
and pondering upon. Every moment you’re conditioning. So be very careful
with what kind of thoughts you start to condition the mind with, because that
is what you end up as. That is the reality you start to live.

\sphinxAtStartPar
If you noticed that you have come under the sway of a hindrance, don’t
feed  it  with  any  more  thoughts  about  it.  We  are  not  here  to  continuously
think  about  things.  Thinking  is  not  meditation.  Thinking  feeds  the  mind
states. Just like the dog that comes to visit your house. If you feed it, it is
likely to come back the next day. If you feed it again, it will come back and
end up sleeping on your couch. That’s what happens when we start to feed
things. They come back and they become part of the furniture. They become
part of our experience of life. So be careful with these five hindrances.

\sphinxAtStartPar
\sphinxstyleemphasis{These are the five things you need to let go in order to enter jhana, to}
\sphinxstyleemphasis{enter the states of samadhi, necessary for unfolding insight.}
You will need
to  remove  these.  We  don’t  try  and  push  them  away.  This  is  important  to
remember. We are not trying to push away anything or get rid of anything.
We are not trying to get anything, or to become anything. We are checking
\DUrole{pdfpage}{116}  our attitude in how we are meditating and we are just observing whatever is
there. We are just accepting that it is there. We may not like it, but we accept
it as this present moment experience contained with this. That’s what’s there.
You accept it for what it is. You note it’s there, you disengage from it, step
back from it and just let it go. It drops by itself as soon as you’ve disengaged
from it.

\sphinxAtStartPar
These five hindrances are very habitual states of mind. They are constantly  coming  back  and  we  are  constantly  reacting  in  the  same  way  to
things. It becomes a habit. We start to condition ourselves with things. If you
don’t like tomato sauce, you start to condition yourself and will never like
it. As far as our mind is concerned we are just reacting. We’ve developed a
pattern of disliking. Whenever we see a bottle of tomato sauce, we avoid it.
If someone puts tomato sauce on our food, we become upset, because we
don’t like it. We haven’t been eating tomato sauce for 10 years and still react
in the same way. There is some disliking there because we have conditioned
the mind. We have done this in many, many ways throughout our lives. We
have developed little habits, tendencies. So we need to have a look at these.
We need to see the suffering that these tendencies have caused us and then
we will be able to let them go, to be able to break the cycle of reaction. We
don’t have to be reacting machines. We can be alert and aware in the present
moment, allowing whatever it is to arise noting it, knowing it and letting it
go.


\section{First hindrance kāmacchanda}
\label{\detokenize{3-a:first-hindrance-kamacchanda}}
\sphinxAtStartPar
So the first hindrance is known as
\sphinxstyleemphasis{kāmacchanda}
or sensual desire. It
means desiring things of the sense doors, wanting sensual pleasures, wanting pleasurable objects that arise through the eye, ear, nose, tongue and body
doors. It is a state of mind of wanting, seeking, wanting to get something,
wanting to change something. When you’re wanting things. When the mind
goes off desiring, yearning, a strong desire to get a higher education degree
or a strong desire to try and buy an expensive house or get a fancy car or
whatever it may be. Have a look if your mind is inclining towards this. «Oh I
like that yoga mat, oh that is a nice top I want to get», whatever it is that you
start looking at and start liking and wanting. It is amazing how the mind will
\DUrole{pdfpage}{117}  find something even in the barren environment to start liking, to start wanting. We are looking at the tendency. Not so much the objects. We want to
know if the mind energy of sensual desire is there. We want to see if there is
any inclination of the mind wanting to go out and enjoy sensuality. Passion.
Lusting. Maybe you have spotted someone on the retreat and had a look at
their body having lustful thoughts. Or you’re thinking about when lunch is
going to be. There can be many other different things that we crave for, that
we wish for. So have a look at this kind of coveting mind.

\sphinxAtStartPar
Especially  for  beautiful  things.  We  hope  and  desire  and  want  things
which are beautiful. We think they will make us happy in some way. Desirous, we work hard to get the things that we think are beautiful.

\sphinxAtStartPar
Sometimes  we  are  just  after  refinements  in  sensuality.  You  have  a
cotton  shirt  and  want  to  get  a  silk  shirt.  You  want  other  flip\sphinxhyphen{}flops.  Other
sunglasses, thinking of upgrading, to get things better. Maybe you are thinking  of  upgrading  your  boyfriend  or  maybe  of  upgrading  to  a  new  kind  of
relationship. You’re trying to get things better. This kind of energy that flows
through the mind of wanting. It’s reaching out to things and desiring them.
This is a hindrance that comes to play in our meditation practice. When we
sit  and  meditate  and  it  starts  to  want  something.  Maybe  it’s  something  at
home, maybe some kind of recognition, maybe it wants some power, or it
wants to be somebody famous. It wants to be something.

\sphinxAtStartPar
Looking  for  nice  food  when  getting  out  of  here,  wanting  to  go  to  a
party, dancing, fun activities. That’s fine but just have a look at that craving,
wanting energy. We normally go for these kind of objects when we are experiencing dukkha, when there is some unpleasantness in our life. When we are
maybe not too comfortable in a dorm room, when we have been sitting on
the floor for many hours, when we are extremely tired or frustrated. Perhaps
the sound of my voice starts to annoy you thinking this is unbearable for you.
Whatever it is that you’re kind of finding that you want to change. You’re
wanting something to be different. Look at that! There is nothing wrong here
but there may be some resistance in your mind.

\sphinxAtStartPar
You  may  be  missing  the  normal  objects  that  you  delight  yourself.
Maybe  you’re  used  to  having  an  evening  beer  and  a  joint.  Maybe  you’re
missing  your  partner,  your  lover.  Maybe  you  want  to  have  certain  food,  a
\DUrole{pdfpage}{118}  certain  environment.  Maybe  you  are  missing  a  comfortable  bed.  «I  want
air\sphinxhyphen{}con.» These are simple things that we start to think about, but they come
to disturb our meditation. Particularly if we start looping and thinking about
them again and again. Be careful because you start to condition the mind.

\sphinxAtStartPar
The  Buddha  gives  us  a  simile  of  dyed  water.  The  water  is  normally
crystal\sphinxhyphen{}clear but the water has become all colored and we cannot see through
the water too clearly. This is what the mind with sensual desire is like. The
normal state of the mind without the five hindrances is bright and clear. It
knows things. It sees things quite clearly. Because of these hindrances it has
become clouded with color so that we can’t see into it. We can’t see things as
they really are. We are seeing our projections and our conceptualizations that
we paint on top of everything.

\sphinxAtStartPar
So  we’re  bringing  ourselves  into  the  present  moment,  activate  our
awareness, become aware of this sensual desire in our mind. Try not to get
caught up in the story of the sensual desire. That means don’t think about the
objects of your sensual desire – pizza, girls, going on a diving trip, whatever it is that you’re wanting. You can be sure that everyone will have some
sensual desire, something we can enjoy through the eye, ear, nose, tongue
or  the  body  doors. This  is  very  much  how  we  spend  our  time  here  in  the
human realm. Most of our time we spend chasing something delightful that
we  can  enjoy. We  want  things.  Have  a  look  at  that. When  you  notice  that
there is any desiring of something, make a note ‚desiring, desiring, wanting,
wanting, lusting, lusting,…’ nothing is too small to note. No thought that is
arising is too small, too insignificant to note. We need to note continuously
because  what  is  happening  is  the  mind  is  going  into  the  objects,  and  then
we note, and we’re coming out of it. We go in and then we come out. We go
in and then we come out. This is because our awareness and wisdom is not
continuous enough yet. So we can note something, then we wander a little
bit, note something and wander again a little bit. We want to get to the stage
where we can hold back the whole flood of objects coming in through the
six sense doors with our awareness and our wisdom so that we don’t go out
of our meditation practice. So that we can create a void area, a demilitarized
zone, if you like around all these mind objects. So we can stay back and not
let them touch us or disturb us in any way.


\section{Second hindrance vyāpāda}
\label{\detokenize{3-a:second-hindrance-vyapada}}
\sphinxAtStartPar
\DUrole{pdfpage}{119} The second hindrance is known as
\sphinxstyleemphasis{vyāpāda}. It means ill will or aversion. This is the mind state which does not like stuff. The opposite of the first
one. It is always trying to get rid of something. This is the type of mental
reaction if we don’t like something. We develop some kind of aversion or
anger it can escalate to, states of irritation, where the mind becomes annoyed
with something, maybe it’s some kind of frustration state. Once it arises, if
we  don’t  note  it  and  know  it,  it’s  going  to  start  looping,  it’s  going  to  start
turning around again and again, irritating us, getting us more upset, getting
us more unmindful and then going into it again, thinking about it again, not
wanting this, anything but this. To get out of here. These kind of mind states
start to arise.

\sphinxAtStartPar
We start judging, comparing, complaining, resisting things. We want to
push them away. We don’t like it. Look at this type of energy, this frustration
or  annoyance  or  aversion  or  judgment.  Maybe  you’re  judging  others.  Out
there on the walking path, you’re looking at other people, judging: «Oh, they
are not walking very fast. Walking too fast. They’re being slack. They’re not
sitting. Oh, they’re leaning against a wall. Oh, she’s sitting very straight.»
Maybe  you’re  judging  people  like  that. You  start  comparing  yourself  with
others.

\sphinxAtStartPar
Maybe  you  have  some  unfulfilled  expectations.  Maybe  you  thought
you’re  going  to  come  here,  sitting  nice  and  peaceful,  with  candles  and
incense, with Spa music in the background. No, your expectations have not
been fulfilled!

\sphinxAtStartPar
Maybe you started blaming others: «I can’t meditate because that girl
is always coughing. I can’t do the walking meditation, because that guy is
moving his arms around all time. I can’t get any sleep because lizards are in
my room.» Maybe you start to blame the whole world for all your troubles
here.
\sphinxstyleemphasis{There’s  nothing  wrong  with  the  world!  There  may  be  some  issues  in}
\sphinxstyleemphasis{your mind}. So have a look at this stuff as it starts to bubble up.

\sphinxAtStartPar
Sometimes it can be funny to have a look at the mind. Again and again
to see how ridiculous our reactions are. We find it amusing if we can pull
ourselves out it. If we can’t pull ourselves out of it, we start looping, we start
to cause ourselves a lot of dukkha.

\sphinxAtStartPar
\DUrole{pdfpage}{120}  Have a look at that pushing away energy. Have a look at the mind, the
image of your reaction. That’s all it’s doing. It’s reacting to things. It’s like a
two year old. It’s not getting what it wants. «I want it to be otherwise. I don’t
wanted to be like this. I want to be like that.» See how you’re reacting. If
you are reacting like that and identifying with those mind states, is it causing happiness or suffering? Can you feel the suffering flowing through your
veins of the body? Can you feel being upset with it? If you don’t deal with it
here, where are you going to deal with it? It’s just going to keep coming up.
You’re going to keep reacting in the same way until you set yourself up to
be a constant reactor. Maybe a nuclear reactor if you get upset about people.
There are some explosions awaiting to happen.

\sphinxAtStartPar
Maybe you are comparing. Some people have been to different meditation retreats, done a few different techniques and they’ve started to compare.
«On  this  retreat  we  were  allowed  to  do  that.  Here  we’re  not  allowed  to.»
You start to get aversion toward something. Comparing, judging. «This one
is better for me. The food is really good here but the silence is really good
there.»

\sphinxAtStartPar
So have a look at the complaining or criticizing mind. Your reflections
or your criticism of things that are going on, are just a reflection of your own
mind. It’s just displaying how you are reacting in the world. When we are
here in silence, it’s a very good chance to see what we are. It’s a very good
chance to see how our mind reacts and to see how we can handle how our
mind reacts to things.
\sphinxstyleemphasis{You’re going to have to come to terms with your mental
states at some time. You have to watch them.}

\sphinxAtStartPar
\sphinxstyleemphasis{If you are not aware of them or don’t know how to deal with them, you}
\sphinxstyleemphasis{will be constantly in trouble.}
Constantly causing yourself stress. You are a
‘the grass is always greener over there’\sphinxhyphen{}person. Constantly shuffling. «It’s
not good enough here, I have to go over there.»

\sphinxAtStartPar
Sometimes people run around looking for enlightenment. They think
happiness is going to be outside somewhere. Maybe in a particular ashram in
the Himalaya, maybe on a beach in South America. But that is not how it is.
These things are happening in the mind.
\sphinxstyleemphasis{Our problems are in the mind.}
Happiness is right here and dukkha is right here as well – right here in this very
body!
\sphinxstyleemphasis{«In this fathom length body is dukkha, the arising of dukkha, the ces\sphinxhyphen{}}
\sphinxstyleemphasis{sation of dukkha and the path leading to the cessation do dukkha.»}
\DUrole{pdfpage}{121}   We don’t
need to look outside anywhere for our supreme happiness. It is right here!

\sphinxAtStartPar
External  stuff  is  just  a  distraction  for  us.  It’s  just  sensuality.  Just  to
kind of keep us going so the whole world doesn’t implode in some enormous
dukkha state. If we didn’t have any sensuality, imagine what an existence we
would be living in. If we didn’t have awareness and wisdom and couldn’t
engage in sensuality, then our life would become a never\sphinxhyphen{}ending nightmare.
– So start to have a look at this. Start to have a look at your mind state.

\sphinxAtStartPar
See  if  you  are  pushing  something  away,  if  you’re  not  satisfied  with
something. Have a look at the dissatisfied mind. Sometimes it can lead to
hatred, sometimes it can lead to anger, sometimes it can even lead to thoughts
of violence. «If that guy bangs that dorm room door one more time….». You
start to have thoughts like that.

\sphinxAtStartPar
So make a note that your mind starts judging, starts comparing, starts
complaining, having negative experiences towards things. Take note of that.
Step out of that looping mind state. Free yourself from it. Just make a little
bit of effort, ‘comparing, judging, judging, dissatisfied, dissatisfied’ and step
back from that.

\sphinxAtStartPar
Who is it who is dissatisfied? Who is complaining here? What’s going
on? There is a mental state that is bouncing around inside your head. And
you are identifying with it. Holding it, claiming it to be you. That of course
causes you dukkha. If you can stop identifying with it, then it stops bouncing
around. It disappears.

\sphinxAtStartPar
It only stays, if we’re identifying with it. Objects remain in the present, if we identify with them. When we don’t identify, they pass! They pass
very quickly and they’re gone. And they come back again, and they are gone.
Come back again, gone again. This is the power of awareness and wisdom.
So try not to get stuck or caught in the story of your own aversions, of your
own dissatisfactions, of your judgments.

\sphinxAtStartPar
The Buddha said, that this type of mind state, is like boiling water. You
can’t see into the water. It’s boiling, it’s bubbling. So we need to turn off the
gas. Allow the water to calm down and then we will be able to see into this
particular mind state.


\section{Third hindrance thīnamiddha}
\label{\detokenize{3-a:third-hindrance-thinamiddha}}
\sphinxAtStartPar
\DUrole{pdfpage}{122}  The  third  hindrance,  that  we  all  have  been  experiencing,  is  called
\sphinxstyleemphasis{thīnamiddha}.  It  is  normally  translated  as  sloth  and  torpor,  but  because
nobody knows what that means, we translate it as laziness and boredom.

\sphinxAtStartPar
This is the kind of mind state which comes to infect us, when we start
to become «uuhf», lazy. We start to become bored of things. We can’t raise
enough energy even to lift our head up. Our body is slump.

\sphinxAtStartPar
In sitting meditation, you think, «I’ll just sit here and pretend», waiting
for the bell, waiting for lunchtime. Maybe the head’s on the wall. You started
nodding away. Sleeping. Have a look if you have been infected by this kind
of laziness energy.

\sphinxAtStartPar
It can be quite troubling. It starts as tiredness and then you start thinking about being tired. You start to feed it. You need to note it clearly, «oh,
there is some tiredness occurring». Don’t continue thinking «I’m so tired, I
want to sleep», «my meditation will go much better if I had a sleep now», «I
will have a good sitting after lunch». Don’t start listening to this nonsense!
There is some tiredness occurring. «I am so tired, I really need to sleep, my
meditation will go much better if I sleep. I need 9 hours a day. I can’t function.» Have a look if your mind starts going into that.

\sphinxAtStartPar
Have a look if your mind starts having states of boredom. If you are
not noting and knowing, of course, you will be experiencing some boredom.
Who wants to sit in a hall for nine hours a day and not doing anything?
\sphinxstyleemphasis{Of}
\sphinxstyleemphasis{course it’s boring, so you will need to start meditating!}
If you start to attach
with states of boredom, what does it mean? «I am boring.» So you become a
boring person if you start to identify with your mind states. There’s nothing
boring here, but there maybe some boredom occurring in your mind. Please
be aware of that. See it for what it is. See dullness, see the lethargy, see the
flatness and the apathy.

\sphinxAtStartPar
If you started thinking in that way, if you started looping in that way,
then you’re going to be causing yourself a lot of distress. You’re going to be
getting yourself upset. You’re going to start freaking out. So stop it! Just stop
it and have a look at the mind. Stop it from being slack and inactive. Start
noting.  Starting  being  aware  of  whatever  physical  posture  you  are  in.  Be
aware of the rising and falling. Be aware of sounds and smells and sights. Be
\DUrole{pdfpage}{123}  aware of all the stuff that is going on.

\sphinxAtStartPar
In the beginning stages it might seem a bit boring to watch the breath,
but once we start to connect with it, wow, you will see that it is anything
but boring. In fact, it will become the most fascinating thing you have ever
experienced once you connect with the breath, once your mind goes into the
breath and becomes one with the breath truly able to follow it in the present
moment. You will find that it is an extremely fascinating thing.

\sphinxAtStartPar
It’s just a matter of checking your attitude and then redirecting yourself  to  the  practice.  Directing  yourself  away  from  your  looping,  habitual
thoughts of negativity. Just step out of that pattern that’s going on. Step out
of it and start doing the practice! In that way you will be able to overcome
that state of sleepiness, of sluggishness or boredom.

\sphinxAtStartPar
Practical  things  we  can  do  to  overcome  sleepiness  and  tiredness
include: take a shower, splash water on yourself, look up into the bright sky,
fill your mind with light, pull your ears. In your walking meditation you can
walk a bit faster for the first five minutes. Build up some momentum, build
some energy up. If you’re creeping around super, super slow for 40 minutes,
you may not be becoming invigorating enough, you may not have enough
energy to keep noting.

\sphinxAtStartPar
If you are feeling tired in the sitting meditation, open your eyes. Or if
you are feeling so tired that you’ve started nodding, just stand up and meditate and then sit down again. We are just transitioning between postures –
meditation takes place in the mind. It is mind training. It doesn’t matter what
posture your body is in. Traditionally we sit in the cross legged posture but
it’s quite possible to meditate in chairs, to meditate while standing or walking. It’s even possible to meditate while we are lying down. So use all these
postures and develop continuous awareness. This will help you to overcome
the third hindrance.

\sphinxAtStartPar
When you’re becoming weary or slack, notice when the mind is tired.
Vigorously  note  it  when  the  mind  is  feeling  tired  until  the  point  where  it
steps back and the tiredness is over there and your awareness and wisdom
are over here. Two different things. Two completely separate things and you
can choose whether you want to go into the tiredness or whether you want
to stay in alertness. Tiredness just becomes another object. When we are not
\DUrole{pdfpage}{124}  in it, when we are not identifying with it, it’s not you that’s tired, there is
tiredness! Start to view these hindrances through the prism of the four noble
truths.

\sphinxAtStartPar
The  four  noble  truths  always  start  with
\sphinxstyleemphasis{«there  is..  »}.  That’s  the  way
the  Buddha  instructed  us  to  see  the  four  noble  truths.  There  is  suffering.
There is tiredness. There is wanting. There is aversion. There is disliking.
\sphinxstyleemphasis{It is not you!}
But that is what is arising in the present moment right now.
You can step away from that. Stop conditioning your mind with those states.
Normally, we go into them and start conditioning the mind and then we’ll
become  those  states. A  craving  person,  an  angry  person  or  a  lazy  person.
We keep fueling ourselves with this. We keep programming ourselves with
these hindrances and then that’s what we become. So be very aware and very
careful not to start indulging in these mental states when they are coming up
and disturb our meditation.

\sphinxAtStartPar
The  Buddha  gives  a  simile  of  a  pond,  that’s  covered  by  some  green
vegetation. Pond scum. When that stuff is growing on the water, you can’t
see into the water. You can’t see the clear water of the pond. It’s temporarily
covered. So just push away that pond scum.

\sphinxAtStartPar
Make a note: ‘tiredness, laziness, boredom’ is there. Make a note and
let it go. It may take five or six notings, it may take 10 times to note for a
particular mental state. In the initial stage we do it once or twice and it might
disappear a little bit then it comes back. We note it again. And we know it
again. We keep noting it. We keep hammering it away until it gets the idea:
this is not mine, this I am not, this is not myself. We start to separate from
the mental state.

\sphinxAtStartPar
If we stop identifying with these mental states, they loose all power.
They loose their conditioning power. They loose their ability to act as a base
or a foundation for the sense of self. Me, mine and I simply can’t arise when
we are noting and knowing persistently in the present moment.


\section{Fourth hindrance uddhachakukkucca}
\label{\detokenize{3-a:fourth-hindrance-uddhachakukkucca}}
\sphinxAtStartPar
The  fourth  hindrance  is  known  as
\sphinxstyleemphasis{uddhachakukkucca},  which  means
restlessness and worry. This is the type of mind state that comes upon us,
when we start to worry about things. We become restless about things. You
\DUrole{pdfpage}{125}  have all experienced this wandering mind. It’s wandering here, it’s wandering there. It’s unrest. It’s wanting something else. It’s desiring to anything. It
wants to read something. It’s getting anxious about something. Maybe you’re
starting to having withdrawals from information. We are all used to being
bombarded with so much information. We haven’t got our devices or books
with us. So the mind starts getting a little bit restless. It hasn’t got its normal
toys to play with. So it’s becoming more and more restless.

\sphinxAtStartPar
Planning  the  future  –  it’s  wandering.  Thinking  about  what  it’s  going
to be doing next week. Planning to make your life a success in someway.
Maybe you are thinking about the past, long forgotten memories are coming
up. You’re starting to play to go through old conversations. Recognize when
the mind is wandering. Make a note ‘wandering, wandering, remembering,
remembering, planning, planning’. Quite often we spend most of our time
just wandering in thoughts in the future, planning things that will never come
true. Just nonsense! Of course, you all need to make decisions at some point
in your life and we can do that with wise attention. We can decide how we
make our decisions wisely depending on what the causes or purposes of our
motivation is. We can see which is the best decision to be made in that way.
Become  aware  if  you  are  having  concern  for  outside  things.  If  your
mind is wandering to the outside. Maybe you’re becoming restless. Maybe
you’re  becoming  upset.  Maybe  you  are  feeling  guilty  about  something.
Maybe you are experiencing some fear or maybe you are scared of something.

\sphinxAtStartPar
Be sure not to condition yourself with these kind of thoughts, sometimes irrational thoughts. We’ll get in all kind of phobias, all kinds of belief
systems when we start to believe our own thoughts. We start to listen to these
very impermanent, dependently arisen, conditioned thoughts that come up
and pass away. They arise and pass away. Arise and pass away. We take them
very seriously. We start to identify with them. Of course, they cause us much
trouble.

\sphinxAtStartPar
Maybe you are worrying about the retreat. «I am trying hard, but it’s
not  working.  I’m  not  going  to  get  any  results.  I  am  hopeless.  It  is  day  3
already  and  I  am  still  not  enlightened.  What’s  wrong?»  You  are  used  to
achieving, setting your goals but it’s not working. Your expectations are not
\DUrole{pdfpage}{126}  fulfilled.  You  are  becoming  restless  and  upset  or  you  are  worrying  about
your meditation results. Maybe you’re worrying about what other people are
going to think about you, when you tell them about your meditation results.
All kinds of things that people worry about and get stressed about. No\sphinxhyphen{}one
is interested in what your meditation is doing, except me. I’m of course very
interested and hope that you’re along the path. But don’t think that you have
to try and achieve something or get something by you or for you.

\sphinxAtStartPar
This  can  be  big  source  of  problems  for  people.  They  want  so  very
much to get it. And then of course after two days of tiredness, it’s not working, you become upset. «It’s not working. This system doesn’t work. I have
tried now.»

\sphinxAtStartPar
No! It is just like going to the supermarket and buying the ingredients
and then say the cake does not work. But you have to make the cake first,
you have to do the cooking. Buying the ingredients is not enough. You actually  have  to  do  some  work. Vipassana  is  exactly  the  same.  Just  sitting  on
the mat with your eyes closed and waiting for some insight to fall from the
heaven – it just does not happen like that! No! Insights are conditioned, it’s
conditioned phenomena. You all need to put the conditions in place and we
are all working towards that. The more efficiently and effectively you can
line up all the conditions, that is when the magic starts to happen. If you are
neglecting  some  particular  condition,  then  start  to  work  on  that  one.  The
things I have been saying, all need to be put into the actions. They are all
conditions for the arising of insights.

\sphinxAtStartPar
The wandering mind can be very useful. It teaches us that things are out
of control. It teaches us that thoughts just keep coming, they just keep arising. Meditation practice is not to stop the thinking. We are trying to observe
it, to watch it and detach from it. You’re not interested in perpetuating the
thinking. Don’t fuel it by thinking about it again and again. What happens
when you do that? We all know! You become upset. When there is an object
of resentment and then you keep thinking about it again and again, then you
get yourself upset. This is how we become upset, through constantly thinking about something that we think is negative. If we get in a serious habit
of that, then we get depression. Or, we end up with anxiety. We end up with
some  serious  mental  disorder,  because  of  the  thinking.  It’s  just  thoughts!
\DUrole{pdfpage}{127}  There is nothing wrong with you, but no\sphinxhyphen{}one taught us to have a look at these
thoughts before. Step back from them, they are not you. They don’t belong
to you, nothing to do with you! It’s conditioned phenomena. If you identify
with them, dukkha! If you don’t identify, freedom! So start to step back from
your  own  mind,  from  the  thoughts,  the  restlessness  and  worrying.  Don’t
follow the story. That is what this restlessness is all about. We haven’t been
fast enough in noting the thoughts. A thought has come up and then it started
blossoming into a story. And we started following along with that story. So
try to note the thoughts before they start to turn into stories. Before they take
you away for 10 minutes to another country, to another conversation with
another person, and then you wake up when you hear the bell.

\sphinxAtStartPar
Knowing that we are thinking, that those five hindrances are within us
is already a good start. We need to be able to recognize them, see when we
are being attacked by them, and then step away from them. It is not good
enough to recognize that they are there and then continue to allow them to
keep going. We want to cut that behavior. It is not easy! Because our reaction
patterns have become habitual, we have probably reacted in the same way
for 20, 30 years. It has become a habit. So we are addicted to our reaction
patterns. We think that is who we are. «I don’t like this, this, this and this.»
It’s  already  been  decided  that  these  things  are  bad  and  need  to  be  pushed
away. So every time, when we encounter them, reaction. Dukkha. Reaction.
Dukkha. Every time we come in contact with those things, dukkha. Try to
step  away  from  that  position,  even  if  you  have  been  holding  that  position
for a long time. Try to step away from your views, opinions. These views
and opinions are just rooted in wrong view. They don’t belong to anybody.
Don’t become a person that has become fixed in their wrong view. The older
people get, the more and more they are fixed in the ways of doing things.
«This  is  the  exact  way  you  need  to  make  a  cup  of  tea.  First  tea  bag,  then
water, then milk.» If you put first the milk in, then they start reacting in this
way. It’s just a simple habit that we got used to. And then we think the tea
tastes somehow differently. They can get quite upset, and even not drink the
cup of tea.

\sphinxAtStartPar
So have a look at those kinds of little patterns that we get ourselves
into the trouble that these patterns make. Everything is up for auction in the
\DUrole{pdfpage}{128}  mind. All the mind states can be let go of! None of them belong to you. Get
rid of it. Let it go. Let go of the thoughts and patterns. Free the mind from its
tendency to be in a certain way.

\sphinxAtStartPar
The Buddha gives us a simile of wind whipped water. It’s like water out
there on the ocean on a windy day. Lots of little white caps. You can’t see
down to the sand. The surface of the water is uncalm. It’s disturbed.


\section{Fifth hindrance vicikicchā}
\label{\detokenize{3-a:fifth-hindrance-vicikiccha}}
\sphinxAtStartPar
The fifth hindrance is known as skeptical doubt or
\sphinxstyleemphasis{vicikicchā}
or skepticism. It’s uncertainty, doubt in the mind. Sometimes we start experiencing
doubt. We are unsure about the practice. We are hesitating. We are uncertain
about  what  we  are  doing.  Perhaps  you  are  uncertain  about  the  Buddha’s
teaching.  Maybe  you’re  not  quite  sure  if  the  Buddha  was  correct  or  not.
Maybe you are doubting about the meditation technique that we are teaching here. Maybe you’re having doubt about me as being a teacher. Maybe
you’re having doubts about your own ability to do the practice. Maybe you’re
having thoughts of low self\sphinxhyphen{}confidence. You’re doubting the practice and you
are  doubting  yourself.  You  are  switching  between  meditation  techniques,
sometimes at the nose, sometimes at the abdomen. «I do it this way then I
do it that way.» The mind is not confident enough to fulfill the training. The
faculty of confidence or faith, the faculty of
\sphinxstyleemphasis{saddha}
is weak.

\sphinxAtStartPar
You  need  to  boost  up  that  a  little  bit. You  need  to  become  confident
with what we’re doing here. People have been doing this for a long time and
experiencing  the  results.  Your  ability  to  put  the  instructions  into  practice
will  determine  the  results  of  your  practice. You  shouldn’t  have  any  doubt
about that. This is a conditioned, impersonal process. It has got nothing to
do with you! We are putting in place various conditions and then it happens
by itself. It’s beyond our control. So have no doubt about that! Just put the
instructions in place!

\sphinxAtStartPar
Have a look at this mind state of doubt. It can really destroy your practice. If you start to think «this technique is hopeless», or «I can not do it, this
is not what I want», you’ve tipped yourself out. You’ve fallen over. See if
you are entertaining thoughts of doubt.

\sphinxAtStartPar
\DUrole{pdfpage}{129}  So these are the five hindrances. Sensual desire and ill will, laziness
and  boredom,  restlessness  and  worry  and  skeptical  doubt. These  five  hindrances will come and attack us in our meditation practice. Many things we
do on this retreat are specifically designed to overcome these five hindrances.
We are working on restraining our sense doors. We are restraining our
food intake. We are keeping the eight precepts. If we have a look at the eight
precepts, we will see that actually the sixth, seventh and eighth precepts are
all about restraining the sense doors. We are restraining by eating food only
in the morning. We are restraining the delight that can arise at the tongue
door. We are not dancing, singing, listening to music. We are restraining the
ear door and the body door in those activities. We are not watching shows,
DVDs  or  films.  So  we  are  restraining  the  eye  door. We  are  not  using  perfumes and flowers so that we can restrain the nose door. We are not sleeping
on high and luxurious beds. So we’re restraining the body door.

\sphinxAtStartPar
This  is  a  restraint  mechanism  so  that  we  can  practice.  We  need  to
restrain these five hindrances from arising and we need to restrain the sense
doors so sensuality or sensual desire doesn’t start to arise. This is all to overcome the first hindrance of sensual desire.

\sphinxAtStartPar
We  are  practicing  the  loving  kindness  meditation  as  an  antidote  to
anger or frustration or irritations so that we don’t have ill will and aversion
in our mind.

\sphinxAtStartPar
We are being moderate in our eating taking just two meals a day. We
are devoted to wakefulness, getting up very early and going to bed quite late.
We’re devoted to being awake and alert while on the retreat here. This is so
that we can overcome the third hindrance of laziness and boredom.

\sphinxAtStartPar
All the activities on the retreat here are to overcome something. They
are an aid to us. So try to join in the restraints for your benefit!

\sphinxAtStartPar
We have a fixed schedule so we don’t need to worry about what to do.
We are removing those kind of distractions of choosing where we can go or
what we can do. There is no choice. We are just following along here. We are
following the precepts that we don’t have to worry that our sila or morality
or virtue isn’t at the right level for our meditation practice.

\sphinxAtStartPar
And of course, we are doing the chanting and bowing to develop faith
and confidence in the Buddha’s teachings and the sangha. And we developed
\DUrole{pdfpage}{130}  a measure of confidence within ourselves as well. We gain a reflective acceptance of the teaching. We are listening to the talks, understanding, putting
the practice into action. We are developing a measure of confidence and faith
in the Buddha’s teaching.

\sphinxAtStartPar
Have a look at the translation on your chanting sheets. There are nine
qualities of the Buddha, six qualities of the dhamma and nine qualities of the
sangha that we can recollect.
\sphinxstyleemphasis{Buddhanusati, dhammanusati, sanghanusati}.
That word sati at the end means mindfulness. Mindfulness of the qualities of
the Buddha, of the dhamma and the sangha. Start to appreciate the gloriousness of the Buddha’s teaching! Start to appreciate how deep and profound
the  dhamma  is.  And  appreciate  those  people  who  have  come  across  the
teaching before us and have done the practice and have realized the results
for themselves. They are the ones who have passed the knowledge down to
us! So we can also come in touch with that teaching. Faith and confidence is
as such developed to remove doubt.

\sphinxAtStartPar
It is important to know that these mind states are objects of our mediation. They are not just disturbances. Don’t think you can’t meditate because
of the hindrances. They are the objects, the things we need to be watching.
They are the things which cause troubles. That’s where the dukkha is coming
from. These five hindrances. Sensual desire – not getting it, unhappy. Anger
– getting it, not happy. When we are lazy and bored – not happy. When we
are restless and agitated – we’re not happy. When we are doubting and lacking  confidence  –  not  happy.  Dukkha  is  arising  when  we  are  entertaining
these five hindrances. So you will need to note and know them and let them
go. None of them are you! They don’t belong to you. If you identify with
them,  dukkha!  If  you  can  note,  know  and  let  them  go,  than  you  are  free!
You have freed the mind from them. And do it again! Train the mind to be
free – that is all what our mediation practice is about. We are training the
mind to be free in any situation! Whatever is arising in the present moment
can be noted, known and let go of! All conditioned objects have the same
structure. They all arise, get clung to, stop getting clung to, and pass away.
Now, how long they stay, depends on how long you identify with that state.
If you’re identifying with it strongly, then you will have all day this state of
agitation. You pick it up in the morning and it will be disturbing you all the
\DUrole{pdfpage}{131}  way through to lunchtime until you find something else to be distracted by.
But  if  you  can  note  it  quickly  –  dissolved! Very  rapidly! And  then  a  new
object – and dissolved. Dissolved. Dissolved. If we can get our mind into a
stable mode of perception, with awareness and wisdom continuously arising, noting, knowing every object – then we get ourselves into a state which
is called
\sphinxstyleemphasis{udayabbaya ñāṇa}, the knowledge of arising and passing away. We
start  to  see  all  mental  and  physical  phenomena  arising  and  passing  away
very, very rapidly because nothing is being identified with. It is all being let
go of! Very, very quickly. Until the mind drops. Until the objects stop passing away and the mind goes into cessation. It stops identifying with all the
rapidly arising and passing away phenomena, that we have been able to note
continuously – nothing getting identified with that state of mind, there is no
sense of self arising, just conditioned phenomena until – whoop, this thing
goes out, «wfuuh», like a candle that you put out with the wind of your hand,
«wfuuh». It’s gone. The mind and body process ceases. It turns into cessation
mode and things temporarily cease. Dukkha ceases. This is a little taste of
the deathless. Not a full taste of nirvana but it is something quite close. And
that is all within reach of our meditation retreat this week.

\sphinxAtStartPar
Don’t  beat  yourself  up,  if  you  find  yourself  under  attack  by  the  five
hindrances. We are all. We all have to work through them. We just need to
deal with them patiently, have tolerance. Don’t become tricked by the mind
into becoming upset by your mind state. Have a look around you. It’s a so
delightful environment here and the food is very good. All you need to do
is to have a look at your mind and see what is really going on there. Don’t
attach to these hindrances. Don’t pick them up when you notice that they are
floating around. Our job is to note them, to know them and to let them go.

\sphinxstepscope


\chapter{Day 3, afternoon}
\label{\detokenize{3-b:day-3-afternoon}}\label{\detokenize{3-b::doc}}
\LOCALaudiolink{https://www.mixcloud.com/anthonymarkwell/day-3-afternoon-talk-enlightenment-factors-mindfulness-energy/}

\sphinxAtStartPar
We are going to have a look at a few of the enlightenment factors, the
\sphinxstyleemphasis{bojjhangas}.  These  enlightenment  factors  are  given  by  the  Buddha  in  the
fourth section in the satipatthana text, the section under dhamma\sphinxhyphen{}nupassana
or contemplation of mind objects. The five hindrances are also in the same
section. They are the unwholesome mind states that come to surround consciousness and cause us so many difficulties. As we develop those four foundations of mindfulness in our meditation practice, these seven enlightenment
factors start to be developed as well. These seven enlightenment factors are
mental states of mind that we develop. They are wholesome states.

\sphinxAtStartPar
The first one is mindfulness,
\sphinxstyleemphasis{sati sambojjhanga}.
\sphinxstyleemphasis{Bo}
means ‚to know’,
Bodhi  or  Buddha.  To  know,  to  become  enlightened.
\sphinxstyleemphasis{Anga}
is  a  causative
factor. So bojjhanga is a causative factor of enlightenment, of knowing.

\sphinxAtStartPar
The seven enlightenment factors will need to be developed during our
meditation practice. Consciousness, the knowing gets surrounded by these
seven enlightenment factors. They are the ones that are doing our meditation
work and they will eventually elevate consciousness out of samsara, out of
dukkha and into cessation.


\section{First enlightenment factor sati}
\label{\detokenize{3-b:first-enlightenment-factor-sati}}
\sphinxAtStartPar
\DUrole{pdfpage}{133}   Mindfulness is the generally accepted translation of sati. But we can
also  use  awareness.  One  of  my  meditation  masters  used  to  use  the  words
\sphinxstyleemphasis{observing power}. He thought that was much better to describe the faculty of
mindfulness.

\sphinxAtStartPar
Mindfulness  is  dynamic  and  confronting.  Mindfulness  leaps  onto  its
objects,  covering  them  completely,  penetrating  them  and  not  missing  any
part of the objects.

\sphinxAtStartPar
The  characteristic  of  mindfulness,  this  state  we  are  activating  in  the
present  moment,  is  non\sphinxhyphen{}superficiality.  Mindfulness  sinks  into  its  objects.
When we direct our awareness to an object, the task of mindfulness is to drag
consciousness to the place of knowing. If we’re looking at the rise and fall
of the abdomen, awareness will drag our consciousness, our knowing, to the
place and sink into the movement of the abdomen. It sinks in, it doesn’t float
on the surface. It’s non\sphinxhyphen{}superficial. It’s penetrative. Awareness takes us into
the object so that we penetrate it. Some may have started to experience it
with the breath, coming closer, closer, closer, until it comes up right in front
of it. You feel like you’re inside the breath. It’s confronting. Up in your face
like that. The object and the mind start to join together in our experience.

\sphinxAtStartPar
Mindfulness is non\sphinxhyphen{}superficial. It’s deep and profound. It doesn’t float
on the surface. A cork thrown into a river just floats on the surface, a stone
thrown sinks into it, into the object. This is what mindfulness does. It sinks
into it so that other mental factors, such as wisdom can do their job, specifically wisdom does seeing things clearly and letting them go. Mindfulness
ensures that the mind will sink deeply into the object.

\sphinxAtStartPar
The  function  of  mindfulness  is  non\sphinxhyphen{}disappearance.  That  means  we
don’t forget what we are doing. It doesn’t disappear from the object. We’re
trying  to  be  firm. We’re  not  forgetting  or  allowing  our  object  to  disappear
from our frames of reference. The object is noticed, and noticed, and noted
and noted and not forgotten about. We don’t forget in the middle of a breath
and start wandering off. Mindfulness is not yet established if that is happening. We sink into the object, we stay there and the object doesn’t disappear
from our view. Mindfulness drags our consciousness into the present so that
it  knows  just  that  object  that  mindfulness  has  selected  and  gone  into.  We
\DUrole{pdfpage}{134}  need to be firm to put our mind in the abdomen area so it doesn’t slip off.
We may have to repeatedly put down our mind into the abdomen area. We
have to reestablish it. We have to constantly reactivate our awareness at the
abdomen. We  are  training  our  mind  so  it  knows  what  it’s  doing. We  keep
sending the mind there, keep sending it there, keep sending it there. Eventually, it knows exactly what it’s doing and it will start to go there by itself.
Like taking a dog to the park, the dog starts to know the way, it doesn’t need
a leash nor to be told. Our mindfulness is the same. We train the mind. Our
mind starts to go there, to the abdomen automatically. Even when we do our
daily activities, you may find your awareness coming spontaneously to the
rise and fall of the abdomen.

\sphinxAtStartPar
The  manifestation  of  mindfulness  is  confrontation. We  will  know  it,
because the object is right up there, right close to us. We’ve come right in
clear to it. You will feel that you are deep down inside the body surrounded
by your abdomen. You are in there! You are confronting it. In the beginning
stages we start to get kind of a feeling, well now it’s extended, now it’s contracted, it’s expanding and contracting and then we start to see a few more
subtle  things. And  then  finally  we  start  to  see  it  very  clearly  indeed. This
is what is happening whilst we’re directing our awareness to the abdomen.

\sphinxAtStartPar
We are constantly sending it there, we are focusing, we are being firm,
and  we  are  trying  not  to  slip  off  it.  When  the  mind  penetrates  these  processes, when it can stay with the movement, then it starts to really understand the nature of the four elements. It sees the hardness, it sees the pressure, it sees the tension and the vibration. All these elements are manifesting
quite clearly at this stage to the mind. You will come to the understanding
that  you  are  definitely  not  this  body. This  body  is  something  but  it  is  not
you. You will come to an understanding that it’s not mine, this I am not, this
is  not  myself. You  will  come  to  an  understanding  this  is  actually  quite  an
unsatisfactory predicament we have got ourselves in. You will see the nature
of what it really means to be infested with a body. We had it foisted upon us.
We have to have this thing! So we start to see more clearly. We start to see
its impermanence, we see the unsatisfactory nature of it, and we actually see
that it is not us anyway. It is this thing we have to carry around with us and
care for. Clean, feed, sleep. We start to see it as it really is.

\sphinxAtStartPar
\DUrole{pdfpage}{135}  When  mindfulness  confronts  an  object,  it’s  like  it  doesn’t  see  it  so
clearly in the beginning stages. It’s like there’s somebody at the end of the
road. You can see that there’s someone there but you can’t make out if it’s
a man or woman. You can’t see their face at all. But as you get closer and
closer, then the details start to become clear. You can start to see its face. You
can start to see if it’s someone you know. As you get closer and closer you
get to see more details about that person.

\sphinxAtStartPar
This is what mindfulness is like. It’s not superficial. It’s not vague or
blurry. It’s quite clear and quite distinct. Mindfulness sinks in its object. This
is the function of sati sambojjhanga. To bring us into the present moment
and sink into the object. It’s that mental factor which pulls consciousness, the
knowing, into the present moment and sinks into an object so it can realize
the nature of that object, note it, know it and let it go.

\sphinxAtStartPar
So we have to apply a sharp degree of care. We have to look meticulously at our object of observation to understand its true nature. When we
can bring ourselves face\sphinxhyphen{}to\sphinxhyphen{}face to the rising and falling of the abdomen, the
details will start to appear by themselves and the practice will start to bear
some fruit.

\sphinxAtStartPar
Once we repeatedly face the object, and when there is no misses, the
object won’t be easily forgotten, you’ll understand its true nature. And you
will be able to, not only develop mindfulness, but hopefully be able to stabilize that mindfulness. We’ll make that mindfulness so continuous, that it
becomes a state of samadhi. Continuous awareness, stability, or a concentrated state of mind.

\sphinxAtStartPar
When  we’re  developing  this  awareness,  this  mindfulness,  it  builds  a
very  strong  barrier  to  the  arising  of  other  defilements.  Greed,  hatred  and
delusion and the five hindrances seem to get swept away, once mindfulness
starts to show up on the scene. It has a lot of power in the mental scene and
those hindrances, that arise continuously, start to be suppressed for a little
bit. We start to move them out of the way. When we move them out of the
way, we can start to see what is actually going on with the mind. It’s because
of these five hindrances, that the mind is covered by all kinds of distractions,
all kinds of foliage that we can’t see it clearly.

\sphinxAtStartPar
You will start to feel a state of ease and comfort once mindfulness is
\DUrole{pdfpage}{136}  being established because the defilements won’t have a chance to enter. They
are temporarily suppressed. You will find that just that by itself is a pleasant
experience. Just being free form the five hindrances is a wonderful experience. Just that. And we go even further. Once we start to develop, once the
hindrances are gone, then you will start to see the true nature of the mind.

\sphinxAtStartPar
When  mindfulness  is  persistently  and  repeatedly  activated,  then
wisdom can do its job as well. We start to have insight into the true nature
of  the  mind  and  body. We  see  the  individual  phenomena  and  we  also  see
the general characteristics of all phenomena. Impermanence starts to show
itself. Dukkha is revealed. Non\sphinxhyphen{}self, a frightening experience, reveals. When
you realize, that actually this body is not me, it’s quite a shock. I shouldn’t
say it’s frightening, but it can be quite a shock. It’s a deep insight with profound implications. You start to realize that this thing is not you. You stop
identifying with it. Even if it’s just for a moment, you start to see things quite
differently.

\sphinxAtStartPar
So the cause of mindfulness is nothing more than mindfulness itself.
The more we practice it, the more we can develop it, the more it gets developed by itself until it becomes continuous, or what the Burmese would call
you reach a stage of momentum, where mindfulness, wisdom and awareness
are almost automatic. We almost don’t have to activate them anymore. They
are just there, staying there. The Thais call it
\sphinxstyleemphasis{maha sati}. The Burmese call it
\sphinxstyleemphasis{momentum}. We can call it
\sphinxstyleemphasis{continuity of awareness}.

\sphinxAtStartPar
So  the  development  of  mindfulness  is  simply  a  continuum.  One
moment after the next moment, moment after moment. We are inclining our
mind  towards  activating  this  state. At  this  stage  of  the  meditation  retreat,
this has been pretty much what we have been doing for the last few days.
We have been activating our awareness to the present moment, bringing it
in our body. We are trying to establish our awareness on the body. We are
trying to establish our awareness in the body so that we can see it for what it
really is. How we can do that is by accessing these sensations that are arising throughout the body. When we start examining these sensations, then we
come to a state of concentration if we can practice continuously. This state of
observation, this state of mindfulness starts to become concentrated. Things
start to become clear.

\sphinxAtStartPar
\sphinxstyleemphasis{\DUrole{pdfpage}{137}  As a meditator the only task is to be aware of whatever is arising in}
\sphinxstyleemphasis{the present moment.}
This is our main thing, this is what we are doing here.
Everything else is very much a secondary activity. Activating and maintaining our awareness in the present moment is our priority until this enlightenment factor of mindfulness can be fully developed. So I encourage you to
put aside any other activities.


\section{Third enlightenment factor viriya}
\label{\detokenize{3-b:third-enlightenment-factor-viriya}}
\sphinxAtStartPar
The third enlightenment factor is known as
\sphinxstyleemphasis{viriya sambojjhanga}. Viriya
means  energy.  The  first  three  enlightenment  factors,  mindfulness,  energy
and  investigation  are  the  mental  states  that  are  doing  the  work  of  meditation. They are doing the job of activating awareness, sinking into the object
and then knowing it by letting it go, letting it go after it’s been known. But
the third enlightenment factor energy is the energy expended, that we use in
activating our awareness into the present. Mindfulness pulls us to the present
but we need to activate it. And this effort that we make of keeping coming
back to the present moment is what we mean by energy here. It is a mental
effort. It is the effort to be constantly here and now. The effort to be meditating. The effort to be aware. It is the effort that we make when we notice
that the mind has wandered, started thinking about stuff, we bring it back
to the present. This is strong effort when we are constantly doing that, then
we are going to develop this enlightenment factor that will take us all the
way. If we are still not doing that, if we’re still allowing the mind to wander
here and there some times, if we are still not being a strict parent parenting
our mind correctly, then we are going to end up with a mind that wanders
and goes here and there. We have to take care of the mind. We have to train
the mind. This is what energy is about, directing our mind consistently and
persistently and continuously towards the present moment and the object of
our observation.

\sphinxAtStartPar
In the old Pali books, viriya is described as the state of the heroic ones.
People who are hard working, who have the capacity of doing stuff, they can
be heroic in whatever they do, wherever. We need to have a lot of effort in
meditation. It’s not a matter of just coming into the hall, sitting down, closing our eyes, and «ok, come on insight please arrive and land on me.» That’s
\DUrole{pdfpage}{138}  not going to happen. We need to put the right conditions in place. And those
right conditions contain effort. So it’s effort itself that gives people a heroic
capacity. It’s just their effort that makes them successful.

\sphinxAtStartPar
The characteristic of effort is an enduring patience in the face of difficulty. When we experience difficulties or troubles, it is effort that stands
up to the trouble. We put forth effort, we try to train the mind, we try to be
patient. We try to tolerate. It’s the ability to see to the end whatever one sets
one’s heart on. We decide we are going to do something and we are going to
work and we are going to do it. This kind of effort. Even when it is difficult.
Even when we are tired, bored and lonely. It’s the effort that we put forth to
achieve our goal.

\sphinxAtStartPar
So  patience  and  acceptance  from  the  very  beginning  of  the  practice
is very important. We need to accept that it is not going to be an easy ride.
Some of us had easy lives. Things come to us easily. We are intelligent, we
cruised through school. Someone gave us a good job. We collected a bit of
money. Things have been relatively easy for us. The only way to find out
how tough we are is when we try to stand\sphinxhyphen{}up to the difficulties of life. When
we  start  to  have  some  stress. When  we  start  to  have  some  worry  or  some
concern or our mind starts freaking out a little bit. That’s when we need the
effort and the energy to be tolerant to endure what is going on in the mind.
It’s the ability to grip one’s teeth right through to the end what one starts.

\sphinxAtStartPar
On a retreat like this we sleep much less, sitting motionless on the floor
in the heat cross\sphinxhyphen{}legged. Your mind is dissatisfied thinking about all those
sensual pleasures that you’re missing out. Lots of things going on here in the
mind whilst we are on retreat. You need courageous effort to forebear these
difficulties. If you raise your energy level than your mind gains strength to
bear with it. Of course, you can swing around your leg when you have pain,
but try to keep it a little bit longer. Train your mind in patience and forbearance.

\sphinxAtStartPar
Effort has the power to refresh the mind and keep it powered. There
is a lot of energy that we use in our meditation practice but once it starts,
mindfulness starts, you may not be able to sleep before midnight. You are
lying there wondering what is going on. So tired all the time and now you
can’t sleep. This is just mindfulness starting to come into practice. It’s starting \DUrole{pdfpage}{139}  to be continuous. Once mindfulness is turned on, it’s difficult to turn it
off. Thankfully! This is particularly drawn out in a longer retreat. You will
see how energized the mind can become. Sleep, whilst still necessary is very
much a secondary interest of the mind of a long\sphinxhyphen{}term meditator. The mind
becomes sharp, alert and clear. In fact it gets a lot of rest from just being in
samadhi. It replenishes itself. We don’t need so much sleep.

\sphinxAtStartPar
The enlightenment factor of effort has the function of supporting the
practice. It supports the mind when it is under attack. It is very important to
have some encouragement and some inspiration. Not only from yourself but
from the group here. You will find that meditating in a group can be much
more powerful.

\sphinxAtStartPar
Our meditation practice does take a great deal of energy. We really do
have to work to establish continuity of our awareness. We do need to keep
activating our awareness. Keep pulling ourselves back inside the body. Keep
pulling ourselves back into the present moment. As soon as we’ve noticed
we have gone out, bring it back! Don’t hesitate. Chop off the conversation
that  you’re  having  with  yourself  or  your  friend.  Stop  thinking  about  those
plans that you have been making. Tell yourself that you will be able to think
about this stuff later. Now is not the time! Now you have deadlines to meet!
You have work to do. It’s not the time for idle fantasies or imaginative journeys. Thinking and pondering about stuff is not going to take you anywhere.
That’s not what we’re here to do.

\sphinxAtStartPar
When the Buddha spoke of energy as being a kind of heat, when the
mind is filled with energy, it becomes hot to dry out those hindrances. When
we are noting and knowing and letting go, noting, knowing and letting go,
the mind becomes very hot. Any of those defilements or any of those hindrances in the mind are just blown away. It dries out like a stick of wood, as
a simile in the text goes. Our defilements simply can’t compete with an energized  mind.  When  we’re  activating  our  awareness  in  the  present  moment
continuously,  these  five  hindrances  are  blown  away.  We  have  the  tool  to
remove them. And the tool is in our own mind.

\sphinxAtStartPar
If our effort is strong and the mind can vaporize defilements, the mind
will  vaporize  defilements  before  they  can  really  touch  us.  In  putting  forth
\sphinxstyleemphasis{right effort}, the sixth factor of the noble eightfold path, we are doing four
\DUrole{pdfpage}{140}  things: two on the negative or unwholesome spectrum, two on the positive
or  wholesome  spectrum.  Our  energy  or  right  effort  is  to  ensure  that  any
unwholesome  states,  any  of  those  hindrance  if  you  like,  that  have  arisen
already, we get rid of them. Note them, know them and let them go. If they
haven’t arisen yet, we block them from arising by maintaining our awareness in the present. And on the wholesome side, if these enlightenment factors haven’t yet arisen, we put forth effort and put the conditions in place for
them to arise. And if they have already arisen, we put forth effort to develop
them further and strengthen them and grow them until they come to complete fulfillment.

\sphinxAtStartPar
So  we  are  overcoming  and  avoiding  the  unwholesome  and  we  are
arousing  and  developing  the  wholesome  states.  And  these  enlightenment
factors are on the side of the wholesome. An energized mind can jump from
one object to the other with ease and quickness. Once we develop the right
level  of  energy,  we  will  be  able  to  easily  do  the  practice  of  ‚rising,  falling, sitting, touching, rising, falling, sitting, touching’. The mind will know
where  it  is  going,  it  will  be  internalized,  it  will  be  present,  and  it  will  be
able to jump from object to object. In fact following it around quite closely.
So our awareness and these objects of the body come quite close. We don’t
loose track of them. The mind is very interested in what it’s doing. There is
a great deal of enthusiasm and interest, when our energy is fully activated,
activating our awareness into the present moment. This is the type of energy
we are talking about here.

\sphinxAtStartPar
When we can do this, defilements are blown away. The mind becomes
clear.  Clarity  increases.  There  is  a  brightness  in  the  mind.  It  starts  to  see
things  as  they  really  are.  Energetic  mindfulness  allows  us  to  deeply  penetrate into the object of observation. It doesn’t allow the mind to scatter and
wander way.

\sphinxAtStartPar
With energy we are activating our awareness, noting what’s there. The
second enlightenment factor, investigation can be activated quite clearly. We
are activating our energy, we are increasing our mindfulness and so it penetrates, and then wisdom does the job of letting go. We are noting, knowing
and letting go. We are activating our awareness, noting what is there, allowing wisdom to wiggle ourselves free from our attachment to that object, and
\DUrole{pdfpage}{141}  it’s passing away. Then we’re repeating the process over and over again until
the mind stabilizes and gets still in this way of watching the mind and body.
It’s not a reactionary process. We’re not reacting with liking or disliking. We
are just observing and allowing the mind to disengage from that stuff, from
the  mental  and  physical  phenomena  that  keep  arising  and  passing  away.
That’s stuff that doesn’t belong to anyone and yet, we are very, very busy of
trying to hold on to it and try to create an identity. We are trying to create a
life out of it. Craving for being is trying to establish itself. That’s all craving
wants to do. Craving wants to be. It very much wants to be something. You
may have been able to see that in your life already. In various stages of our
life we really want to be something.

\sphinxAtStartPar
So energy is at times essential. If a meditator can’t muster the effort to
confront his own defilements, or confront a difficult mind state, if you can’t
really sit there and look what is going on in your mind, you’re going to get
stuck! You’re  going  to  start  cringing  and  cowering. You  can’t  sit  and  look
at the nature of your own mind. You need to be able to do this. You need
to be able to observe it objectively without getting caught up in it. Without
being freaked out by the nature of your own thoughts. Without following the
thoughts. Very often we get stuck in this story of our thoughts. We totally
believe  our  own  stories.  In  fact,  we  make  up  stories  for  us  to  believe  and
delude ourselves. We create nonsense and then try to believe it. We try to
think that there is some happiness in that. Of course, it very rarely is. If it
is, it’s only very momentary before it passes away and we are on the search
again for something else to give us some happiness, to give us some meaning, to give us some being.

\sphinxAtStartPar
Craving for being is the cause for suffering, and yet, all we do our lives
is entertain this notion of craving to be, wanting to be, becoming. On this we
have focussed our life so far. Constantly going after new things to become.
«I want to become a sailor. I want to become this, I want to be this, I want
to  be  that…»  because  that’s  the  only  way  we  know  how  to  get  pleasure.
Our only way – up to now! Real pleasure comes from a concentrated mind,
a mind that is still and bright, knowing. There is great deal of pleasure into
that. The pleasure of a sensual world fades in comparison. It’s like comparing a bowl of dog food on the floor that’s been there for a few days with one
\DUrole{pdfpage}{142}  of our nice meals. There is really no choice, is there?

\sphinxAtStartPar
So for effort to be developed to the point of being a factor of enlightenment, it must have the quality of persistence. We need to be persistent in our
effort, we need to keep going, keep being persistent even when troubles or
blockages come, even when we have a little freak out, when an emotional
state really throws us. Be persistent! Keep noting and knowing and you will
break  through  it.  Sometimes  it  takes  six  or  ten  times  to  note  a  particular
emotional state to note it and truly let it go. Sometimes it takes longer. You
have to keep working with some states because they have become habitual.
Our long ingrained habits that we have been reacting in the same way to the
same object for 20 years now. So we have a reaction cycle and process in
place. It takes some effort to break this, but it’s evidently breakable. We can
move beyond the trap that we have set up for ourselves. So with persistent
effort the mind can be protected from its wrong thoughts.

\sphinxAtStartPar
This enlightenment factor blows the third hindrance, laziness and boredom, out of the water. Once energy is established, we can really start functioning quite clearly in our practice.

\sphinxAtStartPar
The Buddha was quite brief in describing the cause how energy arises.
He says it is from wise attention. Paying attention to the cause of energy creates energy. When we realize the cause of what we’re doing, when we see
our motivation for what we are doing – that is desire to escape from samsara
– when you see the impermanence and the unsatisfactoriness of this mind
and body process, it does arise in the mind a state known as
\sphinxstyleemphasis{samvega}. It is
the desire to escape from the conditioned realm, the desire to move beyond
this  very  impermanent,  dependently  arisen,  conditioned,  impersonal  mind
and body process. It’s the desire to escape from it. To move out of it. This
is called evolution. This is the evolution of our species. And this is what the
Buddha was hailed for in his time. Someone who broke through, broke out
of the mind and body process, broke out of dukkha, broke out of the identification with the mind and body process.

\sphinxAtStartPar
So this is the type of effort or energy that we need if we’re going to go
to practice wholeheartedly. I wholly recommend, that you put forth as much
effort and energy as you can to just activating your awareness in the present moment. Just keep coming back. It doesn’t matter if you’re getting any
\DUrole{pdfpage}{143}  results. It doesn’t matter if anything is happening. Don’t be discouraged. I
assure you the conditions that we have been putting in place for the last few
days, will start to show their results in a very short period of time – if you
have been doing, if you have been consistent in the practice.

\sphinxAtStartPar
The three enlightenment factors energy, mindfulness and investigation
start to spin on each other. Noting, knowing and letting go. These start to do
the work and then the other enlightenment factors will start to be developed.


\section{Painful sensations}
\label{\detokenize{3-b:painful-sensations}}
\sphinxAtStartPar
Some of you will be experiencing some aches and pains in your body
and  you  may  be  wondering  how  you  can  deal  with  them  in  this  meditation practice. You may be thinking, «I can’t mediate because of the pain. I
really can’t concentrate my mind.» Those physical sensations are part of the
practice as well. This is a good way to develop the enlightenment factor of
energy. The Buddha said, that unwholesome intention leads to resultants that
are experienced as painful feeling. So if you have been experiencing a few
painful feelings or a few painful sensations, I think it is save to say that there
has been some unwholesomeness going on. And this is karma.

\sphinxAtStartPar
Our experience of physical sensations is not so bad when compared to
a dog for example. Imagine having a body like that. It is covered in fur and
has got creatures living in it. When we get a single ant walking on our arm,
we start screaming. Imagine having hundreds of flees living on your body
consistently. So our physical sensations are not so bad. Others have it much
worse than us. But still physical sensations will have to be dealt with, if we
are to be successful in our meditation practice.

\sphinxAtStartPar
By now you should have all had a few physical sensation that are painful. If we keep our posture changing all the time, then we will keep disrupting the flow of our practice. So we have to learn to sit still. We are going to
have to learn to sit for 45 minutes or longer. On the first couple of days, you
may feel it will be impossible. But towards the end it will be better.

\sphinxAtStartPar
Try to manage your pain, that we don’t constantly react to it, that we
don’t constantly shuffle around. Stay with a physical sensation that is painful
for a while and just see if you can note it. See if you can become aware of it.
What  is  it  that  makes  these  sensations  unpleasant?  It  is  just  the  four
\DUrole{pdfpage}{144}  elements  balancing  themselves.  Just  the  ratio  is  changing.  Various  things
are  going  on  in  the  body  all  the  time,  changing  here  and  there.  The  four
elements are just manifesting as physical sensations in the body. Investigating a very strong or powerful sensation can be a very wonderful meditation
object, can be a very unique experience in the life of a meditator, when we
start to practice with physical sensations. Because the five aggregates arise
together and pass together, awareness can take different perspectives of the
same present moment experience. We can look at the sensations that are created or are occurring in the body. We can have a look at the unpleasantness
which is a mental state. This is also occurring at the same time. These are
two different things. There is a physical sensation of aching, or throbbing
or hardness – that is one thing. And there is an unpleasantness that is in the
mind.
\sphinxstyleemphasis{The sensation is in the body, the pain is in the mind.}
These two things
arise together, exist together and pass away together. They are two separate
things. They are mind and matter. Consciousness is that which knows what
is going on. These three are arising and passing away together. We can activate  our  awareness  and  wisdom  in  that  present  moment  and  see  how  this
physical sensation and how this painful feeling do arise together. And in fact
we can separate them and see them clearly. Indeed we will need to do this in
our Vipassana practice to progress.

\sphinxAtStartPar
So if we have been slow in noting a physical sensation, it very quickly
turns into something painful. It becomes accompanied by a mental feeling.
We appropriate it and call it mine. «My painful knee!» Three things going
on there. Knee is a concept. Pain is unpleasant, we start to call it mine. We
start to hold on to something that is unpleasant. We started to identify with
something that is unpleasant. So we shouldn’t wonder why we are experiencing suffering if we are holding on to something that is associated with
unpleasantness. All we need to do is drop that. Drop the holding, we won’t
experience the feeling as unpleasant anymore.

\sphinxAtStartPar
If someone puts a ball of hot metal in your hand, we shouldn’t keep
holding it and complain, «oh it is hurting». If we hold on to it: «Oh it is mine,
it is my emotional state, and I will keep on holding to it as long as I can and
give myself dukkha as long a I can.» That is quite often what we do with
emotional states. They arise and we hold on to them, burning ourselves. We
\DUrole{pdfpage}{145}  need to learn the practice of letting them go. And this practice with physical
sensations can help us to do that. There are two different strategies to deal
with painful sensations.

\sphinxAtStartPar
The first one is as follows. You are noting, knowing letting go, ’rising,
falling, sitting, touching, hearing’. We do this cycle 90\% of our time. But
when something distracting comes, we need to note that object. We put our
mind  into  the  knee  and  go  to  the  most  concentrated  point  of  that  sensations and burry the mind into the knee. And we make a note of whatever the
sensation is, ’throbbing, bouncing, hot’. Just acknowledge that this is your
present moment experience. Don’t go into the knee and identify with it. «Oh
it’s  my  knee.» And  then  just  come  back  ’rising,  falling,  sitting,  touching,
rising, falling sitting, touching,…’ and then go back again to the knee, where
the sensation is strongest. ’Aching, aching, aching’. And then stay there for
a  moment  and  see  what  happens  after  you  made  the  note.  There  are  four
possibilities that can happen: a) the sensation will increase, b) it decreases,
c) it completely disappears, d) it changes places (e.g. it goes from the knee
to your back. I don’t know why it does this, how it does this; it is scared to
the noting mind perhaps). Then continue ’rising, falling, sitting, touching,
rising, falling sitting, touching’, then come back. Until the mind let’s it go.
Do  this  four,  six,  maybe  ten  times.  Note,  know,  observe  the  change!  We
incorporate the physical sensation into our practice.

\sphinxAtStartPar
The second requires more effort and concentration and is for dharma
warriors. It should only be done on painful physical sensations of our practice not on old injuries! Meditation pain you recognize that it goes away after
standing up. In case of old injuries don’t do this kind of practice. ’Rising,
falling,  sitting,  touching,  rising,  falling  sitting,  touching,…’  then  we  send
our  mind  to  the  place  where  the  sensation  is  the  strongest.  We  stay  there
and drill in, like an electric drill into the physical sensation, bbrrr. We are
not identifying nor getting angry. We stay with the physical sensation. We
are using it as an object for awareness and wisdom. Hold the mind there, it
can  be  a  strong  sensation.  Hold  your  mind  there,  ’aching,  aching,  aching,
aching’. Just watch what happens. Then go in there again in the most concentrated part of the physical sensation. We are keeping going into it, holding it there, holding it there, you may start to tremble, to shake or to sweat
\DUrole{pdfpage}{146}  a little bit. Don’t worry about that. Stay there, stay there. If it’s too much
come to the ’rising, falling, sitting, touching, rising, falling sitting, touching’. Then go in again, drill into it, like an electric drill, bbrrrrr. Stay there,
keep gently noting. Don’t identify my pain – this is very important. Noting
it, noting it, noting it – until boom! (slap) it disappears. It snaps. Our experience shatters. The physical sensation is still in the body but our experience
of the sensation comes out of the body as an ephemeral experience. There is
a separation of the mind and the body. You will see that the sensation and the
feeling are two separate things. The feeling sensation changes into a neutral
one. You are not attached to it and let it go in the present moment. You will
be left with three things: the physical sensation doing its thing, the mental
feeling doing its thing and the knowing, the consciousness, observing what
is going on. This is a Vipassana insight that we can break through to. Do it
only  on  meditation  pains!  If  this  happens,  you  should  make  a  note  in  the
present moment:
\sphinxstyleemphasis{Knowing, knowing, knowing}.
\sphinxstyleemphasis{A state of knowing is occur\sphinxhyphen{}}
\sphinxstyleemphasis{ring in the present and it is quite unique, a Vipassana insight knowing.}
Once
you have passed this stage, you may find that sitting will become easier and
easier for you. It only takes one or two breakthroughs like this to really break
through the false view that we have to be subjected to these sensations all
the time. We actually become to free the mind from physical sensations that
arise in the body. We become quite quick in it as well. So that we can get
to the point a physical painful sensation arises, and we quickly turn off the
mind so that we don’t have to react.

\sphinxstepscope


\chapter{Day 4, morning}
\label{\detokenize{4-a:day-4-morning}}\label{\detokenize{4-a::doc}}
\LOCALaudiolink{https://www.mixcloud.com/anthonymarkwell/day-4-morning-talk-enlightenment-factors-investigation/}


\section{Second enlightenment factor dhamma vicaya}
\label{\detokenize{4-a:second-enlightenment-factor-dhamma-vicaya}}
\sphinxAtStartPar
The second enlightenment factor is called
\sphinxstyleemphasis{dhamma vicaya}
or investigation of state. This is a part of wisdom. There is a simile: like being in a dark
room and you get a flashlight. Investigation is the wisdom factor that joins
with  the  consciousness,  with  the  knowing.  With  the  flashlight  we  start  to
see things in the room. Dhamma vicaya is the same. We activate our awareness in the present moment, allow mindfulness to penetrate the object, then
wisdom comes to play and knows what it is. This clear seeing enables the
letting go mechanism.

\sphinxAtStartPar
Mindfulness,  energy  and  investigation  spin  on  each  other.  Investigation takes up the role of wisdom.

\sphinxAtStartPar
A further four enlightenment factors start to arise when the conditions
are in place, that is the first three enlightenment factors are properly working
as a team, activating, noting, knowing, letting go, then we start to experience
other enlightenment factors. The next four rapture, tranquility, concentration
and equanimity start to arise in our practice.

\sphinxAtStartPar
Investigation  of  states  is  not  carried  out  by  the  means  of  the  thinking  process,  you  can’t  think  yourself  to  wisdom. Thinking  is  an  object  of
wisdom, an object of knowing. You can note it, know it and let it go. It is part
\DUrole{pdfpage}{148}  of the field of conditioned phenomena that we are observing, noting, knowing and letting go. Thinking itself doesn’t lead intuitively to insights. It plays
a role in bringing us to some understanding, to the practice if you like. But
the investigation we are talking about is an intuitive discerning insight that
distinguishes the characteristic of the phenomena. And what are those phenomena? There is mind and body processes that we have been talking about
this week. When we see the dhamma, we see rupa, matter – earth, water, fire
and air and when we see nama, the mind, we see feelings, perceptions, intentions, attention and contact – vedanā, saññā, cetanā, manasikāra and phassa
is referred to as nama. So we are noting these physical and mental phenomena, we are seeing the individual characteristics of these things, the content
of these things, but we are not going into the content. We are much more
interested in noting the structure in which the content is operating. When we
do that we start to understand our insight. We start to understand the conditioning process that is happening in the mind and body unit. Sometimes
the body is the cause and the mind is the effect, sometimes the body is the
cause and the body is the effect, sometimes mind is the cause and mind is the
effect, or mind is the cause and body is the effect. Four ways. Actually, there
are many other ways the mind and the body are conditioning each other.

\sphinxAtStartPar
Mind,  matter,  cause  and  effect:  these  are  the  raw  data.  When  our
awareness and wisdom is able to note and know this continuously, we will
come to an even further understanding in our wisdom. We will move beyond
the content of our experience, the conceptualization of our experience, we
will move beyond the individual characteristics of each of these mental and
physical  phenomena.  We  will  go  into  another  mode  of  perception,  a  very
particular mode of perception. It is a Vipassana perception which allows us
to  see  all  mental  and  physical  phenomena  having  the  same  characteristic
of  impermanence,  unsatisfactoriness  –  dukkha  –  and  non\sphinxhyphen{}self.  This  is  the
insight,  that  dhamma  vicaya  presents  to  us  when  we  are  developing  this
enlightenment factor. It allows us to see not only the field of our experience
but we go beyond that. We see the characteristics of all phenomena, of all
conditioned phenomena. They are all impermanent, dukkha and non\sphinxhyphen{}self.

\sphinxAtStartPar
The  characteristic  of  investigation  is  just  to  know  and  see  things
clearly, not through intellectual investigation but through an intuitive mode
\DUrole{pdfpage}{149}  of experience. You can’t make insights arise, you can’t force them to arise.
They only arise spontaneously when the conditions for them to arise are perfectly arranged: keeping the mind in the present moment, keeping the mind
stabilized, watching the objects as they arise and pass away, and then ‘bum’,
the  curtain  opens  and  insight  spontaneously  appears,  an  understanding,  a
knowledge comes into your mind and you know things. You see things very
differently. You come to an experience of the dhamma by yourself and for
yourself. You won’t need me to explain it to you, you will be independent
of other people’s explanations of what just happened. You will have seen for
yourself very clearly the nature of the body and the nature of the mind. You
will have known that these two things are two very separate and different
things. And  you  will  know  that  because  of  ignorance  and  unawareness  of
the present we don’t see things clearly. Because we don’t see things clearly,
we are caught in a habitual reaction mechanism of creating karma and then
being  exposed  to  the  resultants  which  just  keeps  looping  around. We  will
come to an understanding that this has been going on for a very long time.
We  will  understand  that  all  this  mental  and  physical  phenomena  doesn’t
belong  to  anybody  and  that  it  is  just  arising  and  passing  away.  We  come
about the insight of
\sphinxstyleemphasis{udayabbaya ñāṇa}
– the knowledge of arising and passing away. We see all phenomena arising and passing away, quite rapidly and
in a detached manner. We are not interfering with anything. Our awareness
and wisdom have been becoming quite sharp, quite consistent, quite continuous. We really even needn’t activate our awareness at this stage. We don’t
need to direct the mind anywhere. We don’t need to evaluate anything. The
mind stays still and holds itself in this mode of knowing, patiently observing and watching, witnessing the flow of physical and mental phenomena as
they bubble up and disappear. We come to an understanding of this Vipassana insight.

\sphinxAtStartPar
Now the function of investigation is to see things clearly. We come to
know this. The function is to totally remove this darkness from the mind so
that it remains illuminated and aware, so that it remains present. An enlightening  experience  is  a  continuous  arising  of  this  presence  in  the  moment
without  being  interested  to,  without  being  attracted  to  or  repelled  by  the
mental  and  physical  phenomena.  The  mind  becomes  completely  equanimous. \DUrole{pdfpage}{150}  Detached, non\sphinxhyphen{}identifying with things that we used to take quite seriously, thoughts and emotional states that we would identify with and hold
to. You see that you can’t create an identity of that stuff you see. You know
it  is  not  you. This  is  the  function  of  insight,  that  we  let  things  go.  Greed,
hatred,  delusion,  the  hindrances.  We  can  free  our  consciousness  from  the
subjectivation  process  and  can  live  in  a  state  of  peace  and  freedom.  Free
from attachment and clinging.

\sphinxAtStartPar
The manifestation of investigation is the dissipation of confusion. There
is no longer doubt about what the nature of the body and mind is. We don’t
rely on other people telling us what the body and mind is. We have seen it for
ourselves. The dhamma, six qualities, discernible by the wise, to be seen for
oneself, dhamma leads one on. When we have seen it, all becomes dustness,
this is how it is rolling through, this is how the karma is rolling out. And
we become content with that. This is what is happening today. Don’t worry
about  yesterday,  don’t  think  about  tomorrow.  This  is  our  present  moment
experience. Try to make yourself alert and awake, each morning you wake
up, observe. This is a new day. Let’s see what the karma will bring. Is there
going to be pleasantness or is there going to be unpleasantness? Either way
we note, know and let go of it. We don’t get caught up. We enjoy it for what
it is without attachment, and then it’s gone. When it’s gone we are not upset,
because  we  understand  that  that  is  its  nature.  Its  nature  is  to  pass  away.
When we don’t embrace the characteristics of conditioned phenomena, we
can save ourselves of a lot of dukkha. Of course it’s gone! That’s its nature.
Don’t  hold  on  so  tightly  that  when  it’s  gone  you  experience  dukkha.  It  is
only the holding that gives us dukkha. It’s just mental and physical phenomena, just doing its thing. Consciousness is knowing it, that is all.

\sphinxAtStartPar
Matter is made up of the four elements. We know that they have their
individual characteristics. We experience these characteristics and can sink
into them by noting and knowing continuously and let go of them. Then we
might break through and have a vision of the ultimate reality of the four elements, a vision of what this physical body is. It’s not what you think it is, but
that’s up to you to examine for yourself.

\sphinxAtStartPar
When we bring our attention to the rise and fall of the abdomen, it is
composed of sensations. Try to let go of the conceptual image of the body.
\DUrole{pdfpage}{151}  We have had an image of our body in our minds all our lives. We have a look
at it. Don’t do your meditation on the abdomen seeing moving a body part.
It is a field of sensations. That is what it is! We need to move beyond our
concepts of stomach, rising, abdomen, falling. We need to go into the mode
of perceptions where we are just perceiving physical sensations. Tightness,
the pushing, the hardness, the softness, the warmth, the flow, the vibration,
tingling, all these physical sensations that are arising there. That is what we
want to slide into, if you like.

\sphinxAtStartPar
When  we  see  this  in  the  rising  and  falling  process,  two  distinct  processes are going on: it is the physical phenomena of the tension, of the pushing, of the pulling, movement of the abdomen and there is also consciousness, the knowing is there as well. These are two different processes. Both
of them are arising and passing away.
\sphinxstyleemphasis{Consciousness is not a stable thing}
\sphinxstyleemphasis{either!}
It  arises  when  its  conditions  are  in  place  –  internal  base,  external
base, meeting together, consciousness arises – and then it cognizes the experience and passes away as well. The whole thing has no solidity in it at all.
All is completely unstable, all is completely out of control. It’s impersonal.
Consciousness arising, mind and matter arising, they have a symbiotic relationship,  they  cause  each  other,  they  lean  upon  each  other.  Without  consciousness  nama\sphinxhyphen{}rupa  cannot  be  known.  It  doesn’t  have  a  space  to  arise.
Without nama\sphinxhyphen{}rupa consciousness doesn’t have a condition upon which to
arise. So the two are mutually conditioning each other. The whole thing is
arising and passing away the whole time.

\sphinxAtStartPar
So if you realize this for you, your mind is balanced and you experience equanimity. Things are arising and passing away. The first insight can
be illuminating but also shocking, you may feel foolished, tricked about the
thing sitting on the mat. What is so appealing of the body, head hair, body
hair, nails, teeth, skin, eyes? What is so attractive? Maybe the nails, but only
when they are painted not when they are in the soup. The hair on the floor is
no more me, mine, I. Have a look at the nature of the body! What is it really
that  we  are  attaching  to? These  things  are  completely  out  of  control. You
can’t say I wish my hair has a different color. The nails are just growing, we
can’t stop them, out of control! The heart pumps blood through our system.
The body takes care of itself, air is coming in and out. Sounds are operating,
\DUrole{pdfpage}{152}  sights are going on. Your teeth are decaying. It is all happening by itself. It’s
not happening to anyone and it’s not happening to you. No one is controlling
it, and it is not under the control of an external power. There is nobody sitting on the clouds, looking down and controlling things. It’s out of control.
It is going on according to its own cause and effect mechanism.

\sphinxAtStartPar
And this is the mechanism that the Buddha discovered. He didn’t invent
the teaching, he uncovered the truth. He uncovered how things are actually
operating and then disclosed that to the world. The world has been listening
to it, practicing it and realizing it for 2500 years. That’s what the Buddha
is, the one who knows. Somebody who broke through and had the ability to
transmit the knowledge what the body\sphinxhyphen{}mind process is. Seeing it so clearly
giving us a formula of the dependent origination so that we also can come
to the understanding of the nature of the reality and remove this craving for
being.
\sphinxstyleemphasis{Baba niroda nirvana}
– the cessation of being is nirvana. When that
being, that wish to be someone, that desire, that craving to become, to get, to
achieve, when that is completely let go of, that is what the Buddha is teaching is nirvana. It is the removal of the sense of self from your experience,
from the experience of the mind and body process. It goes back completely
to full nature, it goes back to dhamma. The consciousness has become freed
no  longer  finding  a  foundation  on  which  it  can  establish  a  sense  of  self.
All foundations have been covered by mindfulness. The four foundations of
mindfulness have been used correctly to establish awareness and wisdom in
the present moment. All holes have been blocked, we are holding back the
tide of the sense of self. We freed the mind, we become very light, no stress,
no worrying, thinking about future plans because there is no\sphinxhyphen{}one it’s going
to happen to. Future is just a thought arising in the present. We don’t have
to worry about it. Just wake up every morning and be glorious. We have no
other work to do than to activate our awareness and wisdom in the present
moment. Nothing else really makes sense once you have seen the nature of
the mind and the body. We can go on and perform various tasks and do different things, eventually we just have to go back and do our spiritual work,
because the rest of it, what for? Unless you have a philosophy like a good
friend of mine. The meaning of life according to him is, that the one with the
most toys wins.

\sphinxAtStartPar
\DUrole{pdfpage}{153}  Spontaneous insight causes insight to happen. Insight begets insight.
Once  we  start  to  see  things,  we  start  to  practice  more,  we  start  to  put  the
conditions  in  place  even  more  effectively.  Further  insights  start  to  unfold.
We activate our awareness repeatedly, not judging, not getting upset by what
we find by activating that awareness, whatever mind state is lurking in the
dark,  wherever  our  mind  has  wandered  to,  whatever  emotional  cloud  has
descended upon us, we don’t judge it, we don’t get upset, we don’t push it
away. Wisdom is that which allows it just to be there. We accept whatever
is there. When we accept it, then we let it go. We let go of things through
accepting them. We don’t let go of things through pushing them away. When
we accept how things are, then we become content toward that experience. It
is what it is. We see it clearly. If there is something that has become unpleasant, we don’t get upset. We activate our awareness, switch on our wisdom,
see  that  object  clearly,  the  mind  disengages  and  steps  back  from  it,  stops
identifying with it, knows it as a dependently arisen, conditioned phenomena
subject to the three characteristics of impermanence, dukkha and non\sphinxhyphen{}self,
and that object passes away. It’s seen. It can no longer be a base for me and
mine. It’s purpose has been foiled. It doesn’t have an effect on our experience.  It’s  just  stuff  lofting  by,  just  like  clouds  floating  through  the  sky.  It
just passes through. Mental and physical phenomena start to arise and pass
away. Arise and pass away. Arise and pass away. Nothing is sticky. We don’t
get caught in the story of any of the mind states or any of the thoughts. We
become like teflon. We don’t attach to anything. And the dukkha just evaporates. It just disappears. We start to see things as they really are.

\sphinxAtStartPar
This second enlightenment factor is very important. We have been calling it wisdom. The first enlightenment factor we have been calling awareness.
\sphinxstyleemphasis{Awareness and wisdom and energy.}
These three are the workers activating  noting,  knowing,  letting  go,  activating  noting,  knowing,  letting  go
–  pulling  us  into  the  present  moment,  penetrating  the  object,  disengaging
from it and watching it pass away. We can do this over and over. If we can
stabilize this activity, if we can still the mind and still keep doing this meditation practice, then our samadhi starts to increase. Samadhi with awareness
and wisdom. Not the kind of samadhi with concentration that falls down a
well on the ground, kind of pleasant but dull not really knowing anything.
\DUrole{pdfpage}{154}  Awareness and wisdom with concentration, this is how our insights start to
arise. This is how we start to let go of the defilements, of the cravings, of the
aversions, of the frustrations and annoyances – all those types of mind states.


\section{Intention}
\label{\detokenize{4-a:intention}}
\sphinxAtStartPar
I also want to talk about intention, particularly in our walking meditation practice we are going to introduce that aspect of intention. Intention to
act or to do something implies a forethought. We think before we act. First
there is a mental action before the physical or before we speak. Structurally
there must always be an intention before a physical or verbal action takes
place. Before we do anything, we must instruct the body to move, to twist,
to open, to close, to look. Before any of these physical actions, there must
be a thought or order and that thought we call
\sphinxstyleemphasis{cetana}. The Buddha said,
\sphinxstyleemphasis{«I}
\sphinxstyleemphasis{declare to you that cetana is karma.»}

\sphinxAtStartPar
The  mind  really  does  cause  matter.  If  you  think  you  want  to  raise
your hand, you think, «raise your hand». It is through thinking that it happens. The mind animates this meat\sphinxhyphen{}puppet. The nature of the body is kind
of dumb. It is just the elements, the same elements as outside, the external
elements  are  the  same  as  the  internal  four  elements.  But  when  consciousness has left the body, it just lies on the floor like a rotten log. We call that
dead. Even when people are unconscious they are like dead. Consciousness
animates the body.
\sphinxstyleemphasis{It opens its mannerisms}, all manifestations of the mind
states. If we become upset, the eyes start crying, we start to shake, our heart
starts to pump a little stronger, breath becomes shorter. The mind affects how
the body is operating.

\sphinxAtStartPar
In our walking meditation, we are going to expand our awareness to
include  awareness  of  intention  of  the  four  stages  in  the  walking.  So  you
should pay attention to the moment just before you lift the foot, before you
raise the foot, before you move the foot and before you place the foot. This
intention will be difficult for you to see in the beginning stages. So we make
a little thought, it is a little contrived but effective. ‘Intending to lift – lifting,
intending to raise – raising, intending to move – moving, intending to place –
placing’. Before each of these stages you should become aware of the desire
to move your foot. Become aware of your mental thought to move your foot.
\DUrole{pdfpage}{155}  We are paying attention to the thought before we lift and then the lifting. The
intention to lift the foot is nama, mind. Moving the foot is rupa, matter. So
the intention and the movement are mind and matter, arising simultaneously.
They arise together and cease together. If you can drop your mind into the
present moment by continuously noting this action, independent of desire,
independent of any wanting to see anything or for any kind of experience, if
you can drop into it, you can have an experience of mind and matter in the
present moment. And you will be able to see things as they really are. And
you  will  also  be  able  to  see  the  conditioning  process  that  occurs  between
mind and matter. You will see that the mind is actually conditioning the body
in this circumstance. Later on we will see all kinds of conditioning patterns.
Intention  is  difficult  to  note  at  first  if  the  mind  is  not  continuously
aware of the present moment. It takes effort to maintain our awareness continuously.

\sphinxAtStartPar
Keep your mind in the moving foot. Lower your head, keep your eyes
three quarters closed. Don’t become distracted by the jungle or other people’s body. Begin your walking meditation practice at a little bit faster pace,
so that you can build up some rhythm. Then after 5 or 10 minutes slow down
to the super\sphinxhyphen{}slow walking if you wish to. You should build up momentum
and get used to your walking path. Center yourself in the practice that you’re
doing and then start to slow down. If you’re having difficulty with the walking meditation, try walking on a solid floor. A stone floor or tiles will make
it easier. The best way is to walk closely to your sitting mat so that you have
no  breaks  in  between  the  walking  and  sitting.  So  you  stand\sphinxhyphen{}up,  you  walk
and  then  you  sit  down  again  immediately.  No  gaps,  no  breaks.  Really  try
to tighten up those gaps or breaks that you may have given yourself during
the walking and sitting meditation. Try to take that hour and a half period
as one continuous time zone of awareness in the present moment regardless
of whether you’re walking or sitting or transitioning between the two. Try to
keep it together.

\sphinxAtStartPar
The systematic practice of this four stage walking meditation will lead
us to seeing the nature of
\sphinxstyleemphasis{sabhava}
or
\sphinxstyleemphasis{ultimate reality, intrinsic reality of mind}
\sphinxstyleemphasis{and matter}. We will be able to see that intention actually moves the body.
You will see this for yourself. So note the consciousness consisting of the
\DUrole{pdfpage}{156}  intentions. You can do this in other practices as well not only the walking but
in the sitting or do it in the dining hall when you are intending to reach for
the cup of coffee or something. ‘Intending to reach, intending’, then reach
out, when you touch the cup, ‘hardness, hardness’, when intending to bend
your  arm,  ‘intending  to  bend,  intending  to  bend,  bending,  bending’,  then
you’re bending it up, ’intending to flip the head back’, we flip it back and
pour it in, pour the coffee in our face. We are tasting. Tasting is occurring.

\sphinxAtStartPar
Be aware of the intentions before you start to make your movements.
Opening the door to the dorm room, go through the whole sequence. Going
to the bathroom. Brushing your teeth – be mindful when moving the brush
up, ‘up’ or down, ‘down’ instead of moving it in your face and thinking of
all other things.

\sphinxAtStartPar
So the inclusion of intention can slow things down but be very fascinating. You may start to feel walking like a robot. You may feel all kinds of
different energies flowing through the body. The four elements will start to
get clearer as well.

\sphinxAtStartPar
But most of all you start to see the cause and effect of things and that is
the most important. We want to see ourselves that actually these mind states
are causing the body to move. These states not only cause the body to move
but can cause all other kinds of effects on the body as well. Some of you
might be familiar with the term psycho\sphinxhyphen{}somatic. Diseases that are caused in
the body by the mind. Through repeatedly thinking in one way, we can actually make the body sick. The mind has a very powerful effect over the body.
Start to see this connection.

\sphinxAtStartPar
We  can  witness  the  power  that  the  mind  has  in  the  walking  process.
If  we  just  observe  the  simple  things  in  our  live,  such  as  walking,  we  can
come to a profound understanding that intention has power over the body.
Intention makes the body walk around, it can make it sit down, it can make
it stand up. Intention makes it do and say all the things it does. When we
start to realize the power intention has over the body, you start to think that
you better take care over these mind states that you are generating in your
mind.  Mind  states  do  actually  have  power.  There  is  a  karmic  energy  produced as well through intention. Not only does intention affect the body in
the present moment but it creates karmic energy that remains latent waiting
\DUrole{pdfpage}{157}  for  an  opportunity  to  manifest. Waiting  for  the  conditions  to  be  right  that
that karma can give its fruit. We have to be careful of what kind of intentions
we are making because it is generating both wholesome and unwholesome
karma. We have to be aware that we are actually conditioning our existence
through our intentional structure. We are creating our world, we create the
world that we live in through our thoughts. We are all responsible for our
own thoughts and the world we live in. Our experience on a day to day basis
is  the  result  of  our  karma. We  are  born  of  our  karma,  heirs  of  our  karma,
related to each other through our karma. Whatever we do for good or bad
that we will forebear. Everything that is going on in the mind and body process is karma related. We are either creating karma or we are observing the
results of old karma. This is a natural process, it doesn’t require anyone to
do anything, either internal or external agents. It is like gravity. It doesn’t
need to be anyone there to turn on the gravity switch. It is just like that. It is
a natural law. It is a physical law. So karma is a mental law. It is just like that.
That is what the Buddha discovered. That is how he uncovered the nature of
the mind and body process.

\sphinxAtStartPar
So  in  our  walking  mediation  we  sharpen  up  our  processes.  We  consciously  generate  the  thought  lifting  before  lifting. Think,  «I  am  going  to
lift the foot now», and then lift it. «I am going to move it», and then move
it. «I am going to place it», then place it. We can say these words to ourselves, it it’s helpful, we don’t need to. ‘Going to lift, lifting, going to move,
moving,  going  to  place,  placing’. Then  move  your  awareness  to  the  other
foot. Always keep your awareness in the moving foot. It is really important
that it is on the sole of the foot, following the physical sensation. Try to bring
this intention and the movement closer and closer together. See how they are
influencing each other. Actually they are happening at the same time in real
time. In the beginning of our practice, when we are generating a thought or
intention to lift or move or place, then there will be some succession. First
we think, then we move. As you get used to doing it – keep going around
and around and around – you start to find that the intention and the movement start to correlate with each other. And you start to think that it moves,
it moves automatically. You start to see that the physical body is animated, it
starts to move because of intention. We can witness this for ourselves in the
\DUrole{pdfpage}{158}  walking meditation practice.

\sphinxAtStartPar
So try to see very clearly the different stages in the walking meditation.
Don’t mix them up. They are four individual discreet events. Lifting, raising, moving, placing. When we keep a gap for each stage, we see that each
stage arises and passes away. And when that has happened, it is finished, we
can move on to the next event. The next stage, it’s finished. The next stage,
it’s  finished.  Finished,  finished,  finished.  We  start  to  see  that  these  stages
in the walking meditation are very impermanent. They arise and pass away
quite rapidly even when we are walking slowly. They arise and pass, arise
and pass. You should take a great deal of care in observing, be meticulous.
Don’t think that the walking meditation is secondary to the sitting meditation.  Bring  the  energy,  the  mindfulness  and  the  wisdom  developed  in  that
walking meditation into the meditation hall. Use it in your sitting meditation.
Don’t waste it by wandering here and there after the walking meditation.

\sphinxAtStartPar
The realization of this cause and effect mechanism can be witnessed
for ourselves in the present moment.

\sphinxAtStartPar
Don’t give in in any thoughts or emotions during the walking meditation.  Cut  it,  bring  it  back  into  the  present.  But  don’t  walk  wanting  results
to occur. That will cause a lot of frustration. It ends always in tears. In the
back of our mind there is some craving, we want to become a mediator, getting results. We must check our attitude.
\sphinxstyleemphasis{We are not walking to get results,}
\sphinxstyleemphasis{but  to  put  the  conditions  in  place.}
We  follow  the  instructions,  don’t  leave
anything away. Follow them as closely as you can. If you notice wishing,
imagine results, the mind expanding to the universe and generating all kinds
of knowledge, then we only generated ideas. Let go of these! Put aside all
theories and just do the practice in the present moment. That is the only way
how you will truly understand the nature of the mind and body process.

\sphinxAtStartPar
A deep insight is an intuitive expression of the mind revealing the reality  of  nature  in  the  present  moment  to  us.  When  you  least  expect  it,  the
dhamma presents itself to us. The only thing you need to do is to get out of
the way – literally, you. The only reason why Vipassana is not arising is that
we may have expectations or a desire for something to arise. Drop all your
expectations  and  wishes  for  success  and  achievements.  Meditation  is  the
complete opposite of achievement, it is letting go of everything that we have
\DUrole{pdfpage}{159}  ever learned in school. We are not here to achieve anything, we are here to
relinquish and let go of everything.

\sphinxstepscope


\chapter{Day 4, afternoon}
\label{\detokenize{4-b:day-4-afternoon}}\label{\detokenize{4-b::doc}}
\LOCALaudiolink{https://www.mixcloud.com/anthonymarkwell/day-4-afternoon-talk-contemplation-of-mind-states/}


\section{Third foundation of mindfulness citta nupassana}
\label{\detokenize{4-b:third-foundation-of-mindfulness-citta-nupassana}}
\sphinxAtStartPar
The third foundation of mindfulness is contemplation of the mind or
mind states. Meditation is mind development. As our practice develops we
give  more  and  more  emphasis  to  the  mind,  to  watching  the  mind.  In  particular, in watching defiled mind states. We are training the mind. Training
to do what? To observe, weaken and destroy these defilements which enter
our  sense  experience  and  cause  dukkha.  The  Buddha’s  teaching  is  full  of
groupings  of  these  defilements  or  imperfections.  In  particular  the  satipatthana tells us to look at greed, hatred and delusion.
\sphinxstyleemphasis{Lobha},
\sphinxstyleemphasis{dosa}
and
\sphinxstyleemphasis{moha}.
The
\sphinxstyleemphasis{ten fetters}
is another group. Or the seven latent tendencies,
\sphinxstyleemphasis{anusaia}. The
three taints, the
\sphinxstyleemphasis{asavas}. The five hindrances,
\sphinxstyleemphasis{nivanas}. All of these different
groupings  of  defiled  mind  states,  imperfections  or  corruptions  come  from
three roots. Lobha, dosa and moha; greed, hatred and delusion. These three
roots have their base in ignorance, in unknowing, in unawareness,
\sphinxstyleemphasis{avija}. If
we are unaware of the present moment experience, than that experience is an
experience with ignorance. Ignorance arises and starts to condition the mind
and body process in that moment. In fact, in that moment the whole experience stands on ignorance. That is an unaware moment. It is a moment which
is going to be dukkha. When there is ignorance and craving, there is dukkha.
When ignorance and craving ceases, dukkha ceases.
\sphinxstyleemphasis{\DUrole{pdfpage}{161}  And this doesn’t happen}
\sphinxstyleemphasis{anywhere or for any time except for right here and right now.}
These things
are occurring right now, this is a present moment experience. It is either one
with awareness, knowledge, or with unawareness, ignorance.

\sphinxAtStartPar
The aim this afternoon is not to give you a catalogue of the impurities
of  the  mind. We  are  just  going  to  have  a  look  at  a  few  ones,  in  particular
those  which  pull  us  out  into  wanting  states  and  those  modes  that  push  us
away: attracting and aversion. Our mindfulness practice endeavors to experience  them,  to  get  to  know  them,  so  that  they  can  be  let  go  of.  Without
knowing and seeing them, it is not possible to let them go. In the first of the
four noble truths ‘there is dukkha’, the Buddha said,
\sphinxstyleemphasis{«dukkha must be under\sphinxhyphen{}}
\sphinxstyleemphasis{stood»}. It must be understood. We must realize it through our own wisdom.
For us to do this we must have right attitude. We need to make sure that we
are meditating with the right intentions. Right attitude allows us to observe
and accept whatever is happening, whether it is pleasant or unpleasant, in a
relaxed and alert manner, not over focussing – we are not causing stress or
tension in the mind – we are open and comfortable, we are spacious in the
mind,  we  are  just  spaciously  observing,  we  are  just  watching. We  are  not
interfering with things. Most of all we are accepting, we have to accept that
good  and  bad  experiences  occur.  Every  experience  whether  good  or  bad,
gives us an experience to learn something about the nature of that particular
conditioned phenomena. If it is a good experience, we probably attach to it
and want to keep it. If it is a bad experience we probably do not want it and
try to push it away. In both cases we are attaching, we are identifying with
that particular phenomena taking it as mine, happening to me or taking that
experience  happening  for  me.  When  we  do  this  with  all  our  experiences
continuously again and again, subjectifying, taking them as mine or me than
we build a sense of I. All these little moments of me and mine crystallize
and become more concrete in a stronger sense of subjectivity, a more intense
form of the self, an I. An I becomes to be born. If all these things are happening to me and all this stuff is mine, than it is very natural for the idea of
«I am». It occurs. If this is all me and mine, well then «I am», I have a self.
And so the mind and body process start to believe it is someone. It has come
under the full sway of ignorance.


\section{Craving}
\label{\detokenize{4-b:craving}}
\sphinxAtStartPar
\DUrole{pdfpage}{162}  Craving  is  manifesting  in  every  moment,  craving  for  being,
\sphinxstyleemphasis{bhava tanha},  entering  into  the  sense  experience,  flooding  our  experience  from
moment  after  moment  with  infection  of  the  self,  infection  of  the  ego. We
need  to  notice  whether  our  mind  accepts  things  how  they  are,  or  whether
there is liking or disliking, whether there is some judging or reactions. We
need  to  watch  ourselves  with  our  motivations  for  our  meditation  practice.
We need to keep checking to see with what kind of mind state we are practicing. If we are practicing with wishing, wanting – «I have to get in this meditation this far, I have got to get…» – then you are blocking your meditation
from developing. That wanting is a little bit of desire and is stopping us from
entering into the present moment.

\sphinxAtStartPar
Whatever  happens  in  the  body  is  dhamma,  whatever  happens  in  the
mind is dhamma. It is all dhamma\sphinxhyphen{}nature. Nothing is happening to you or
for you. We are appropriating it. It is taking ownership itself. That is only
because it does not know and not see. It just decides to take ownership of
things. It seeks an identity and finds one in our daily lives in our daily sense
experiences. Feeling hot is just feeling hot, it is just dhamma\sphinxhyphen{}nature. We will
feel much hotter if we start to identify with that heat. In fact we can give ourselves dukkha. When the body starts to feel hot, like it is today, we can make
a note ‘warmth, heat, sweating’, whatever it is, whatever the physical sensation is at this moment, make a note of it, know it, stop identifying with it, so
that the heat is let go – it is your attachment we are talking of here, we are
not turning the weather off – the body still remains hot but we have stopped
attaching to it so you don’t find it that unpleasant. It is still hotness, there is
still some heat but you have turned off your attachment so that we turned
off the dukkha. It is only if we hold things that dukkha comes to destroy us.
\sphinxstyleemphasis{Allow it to be as it is! Accept things as they are! Let them roll\sphinxhyphen{}out.}
They are
impermanent and will pass away very rapidly anyhow. Why ruin your morning, why ruining your afternoon by a little bit of attachment to something,
cause some dukkha for yourself. Don’t write any dukkha in your schedule.
We  don’t  need  it.  –  Everything  that  is  happening  is  happening  because  of
cause and effect. Our work is to have the right attitude to maintain awareness, to use our intelligence, to have interest in the practice.


\section{Ignorance}
\label{\detokenize{4-b:ignorance}}
\sphinxAtStartPar
\DUrole{pdfpage}{163} Ignorance of these defiled states of mind affects our attitude. When we
don’t understand the nature of these corruptions in the mind, when we have
wrong attitude, we can’t help being affected by them. They do affect our life.
It  is  important  that  we  recognize  and  investigate  our  wrong  attitudes  and
straighten them up. At this time of the retreat, if you are trying to meditate
to get some results, please stop that. Drop any expectational desire for any
results. Just do your work of putting the conditions in place. The results will
take care of themselves. Try to understand how these wrong attitudes really
affect our practice when we have some strong expectations. When we have
expectations  and  these  expectations  are  not  fulfilled  –  what  is  that  state?
Dukkha!  So  drop  your  expectations  and  guess  what  disappears  –  dukkha
disappears.  So  don’t  try  to  create  or  expect  anything.  Trying  to  create  or
expect anything to happen is greed, lobha. Don’t reject what is happening.
Rejecting what is happening is aversion, dosa. Not activating our awareness
in the present is not\sphinxhyphen{}knowing, it is ignorance, unawareness, moha, delusion.
So we don’t try to make things the way that we want them to be. We don’t
try  to  make  things  happen,  the  way  we  want  them  to  happen. We  are  just
trying to know what is happening as it is happening. Wanting this to happen
or not to happen is expectations. Expectation leads to anxiety and that can
lead to aversion.

\sphinxAtStartPar
It is important that we are aware of our expectations, our wanting to
become a meditator, get meditation to be put in our CV, another thing we
can do, just another activity that we can use to create an identity, for others
to look at. We are not in the identity or personality creating game here. That
is not what the retreat or meditation is about. It is about freeing ourselves
exactly from that thing. Getting all excited about becoming a meditator, a
retreatant. Have a look at that if you start going into that mind\sphinxhyphen{}set. Don’t use
your practice as another foundation for the creating of a self. The practice
is meant for removing the sense of self. It is not for building up a spiritual
credential. We are learning to let go of that mental attitude of trying to create
and generate a personality.

\sphinxAtStartPar
Our societies generally promote developing a very strong sense of self.
We come from competitive societies. We have created social and economic
\DUrole{pdfpage}{164}  systems based on competition. We are competitive. Instead of cooperating
with each other we are competing with each other, fighting and scrambling
who can be the strongest, the fastest, the richest, the funniest, the smartest.
We are competing. Our societies drive us to create an identity. The creator on
our website, the owner on our business cards, to be able to tell people who
we are or what we are. We don’t want to be nothing, do we? That is one of
the nastiest things that anybody can say to us. «You are just nothing!» – If
someone says that to me, «oh, thank you very much!» The practice is working!


\section{Aversion}
\label{\detokenize{4-b:aversion}}
\sphinxAtStartPar
Aversion  either  arises  from  the  things  that  are  not  the  way  we  think
they should be, or, from a desire that they should be different. Or it arises
simply from unawareness of what right practice is, not quite sure of what we
meant to be doing in the Buddha’s teaching. We received a few instructions
from here and there, but it is not quite clear yet exactly what we should be
doing except for sitting on the floor and breathing. What else is it all about?
We just get little bits and pieces here and there. We need to understand what
the practice is all about. It is about letting go of our attachments! Our attachments that generate a sense of self, a me, mine and I, ego, egoic development, if you like.

\sphinxAtStartPar
Try to recognize dissatisfaction when it arises in the mind. Try to fully
accept it, watch it, alertly, when the mind becomes dissatisfied or frustrated
or irritated with something. Stop and observe, just feel and sense that experience. Don’t go into it. Don’t try to push it away. Just accept it while it is
there. The dissatisfied mind is a wonderful learning tool, the mind of twisting and turning, not happy, complaining, wanting to escape, get out of here.
Have a look at that mind state. It is wonderful to come in contact with it.
Become to really see it what it is. When we see it clearly, dissatisfaction dissolves! Completely disappears. And we are left with comfort, with patience
and with peace.

\sphinxAtStartPar
During the process of observation and exploration of dissatisfaction,
its causes become clear. Our understanding dissolved the dissatisfaction. It
may  return!  But  we  can  note  it,  know  it  and  let  it  go  again!  Dissatisfaction \DUrole{pdfpage}{165}   comes  from  unfulfilled  expectations,  unfulfilled  desires,  craving.  We
see more and more clearly the harm that dissatisfaction causes the mind and
body, we can watch it, we can see the trouble that it makes. But when we
step back of it, we can see how easy it is to step out of this dissatisfied mind
and stop identifying with it. Stop getting caught up in the story that is being
generated. When we get caught up to it and attach to our own thinking, our
own  whinging  and  complaining,  when  we  use  whinging  and  complaining
for a base of self, as a foundation for the arising of self, then it shouldn’t be
a surprise to you that you aren’t so happy. We just generated a self\sphinxhyphen{}moment
based on whinging and complaining. Of course it is going to be a dissatisfying moment. It is dukkha.


\section{Wrong attitude}
\label{\detokenize{4-b:wrong-attitude}}
\sphinxAtStartPar
Wrong attitudes are caused by delusion. Not truly knowing what the
practice is and how to practice. We all have these delusions and wrong attitudes.  Defilements  of  craving  and  aversion  and  all  the  relatives  they  may
have. Not accepting defilements only strengthens them. We have to accept
that they exist and arise. Old karma is manifesting, cause and effect set in
operation. These  defilements  do  come  and  hinder  our  meditation  practice,
they prevent us form living fully in the world. They shut us down. They turn
our world into a duality.

\sphinxAtStartPar
So  we  need  to  understand  those  defiled  states  of  mind.  We  need  to
watch  out  for  them  when  they  arise,  accept  them  that  they  are  there,  note
them, know them and let them go. Don’t attach to them, don’t reject them,
don’t  ignore  them.  Most  importantly,  don’t  identify  with  them. Try  to  see
them  for  what  they  are:  impermanent,  dukkha  and  non\sphinxhyphen{}self. When  we  see
our emotional states or defiled states of mind arising, we look at them this
way so that we can let them go completely.

\sphinxAtStartPar
As  we  stop  attaching,  we’ll  stop  identifying  with  these  defiled  mind
states,  their  strength  slowly  diminishes.  Things  that  used  to  stir  us  up,  to
frustrate and irritate us, start to loose their power. We understand the irritated
mind, we understand that it is dukkha. We know why it is happening so we
can adjust our behavior accordingly. We can adjust our experience, we just
need to activate our awareness more continuously in the present. Whenever
\DUrole{pdfpage}{166}  we  find  ourselves  attacked  by  an  emotional  state,  that  is  the  time  to  start
noting,  that’s  the  time  to  start  stepping  back  and  stop  identifying  with  it
before it becomes a habit, before we start developing a habit of identifying
with  particular  mind  states. They  keep  coming  back,  again  and  again,  we
keep looping on them and reacting on them. Craving for being is finding a
lovely launching pad for it to gain a sense of self. Craving for being doesn’t
mind if it is a pleasant or unpleasant experience. It just wants to be. Its only
goal in the present moment experience is to infiltrate that experience, to get
in there to infect the present with a sense of self. It’s a virus! The sense of
self is a virus that comes to attack us. We have six vulnerable places: eye,
ear, nose, tongue, body and mind door. We have to protect our doors from
moment after moment to stop this virus entering. In the beginning stages of
the practice it is difficult, there will be lots of gaps that it will keep leaking
through. But if we note very rapidly, if we increase the speed at which we are
noting the moment, then we can start to build up a barrier – like somebody to
hold back the sea. You have to be very efficient to hold back the waves from
coming  up  the  beach  because  those  waves  just  keep  coming.  Block  them
with mindfulness and wisdom. Note every moment. Then those waves won’t
be waves of dukkha but waves of equanimity, waves of bliss.

\sphinxAtStartPar
So Vipassana meditation is much more than simply observing things
with a receptive mind. We also have to use wisdom, investigation of dhamma.
We need to create space for wisdom to operate. Having mindfulness, bringing ourselves into the present and noting what’s there is not enough! If you
are just going into the moment seeing what is there and noting it, noting it,
noting it, if you are not using wisdom to detach and step back from it, then
you are probably noting going into the story. You are noting and the story
is  becoming  stronger.  In  our  satipatthana  text  it  says  very  clearly,  that  we
have  to  practice  with
\sphinxstyleemphasis{satisampajana},  with  mindfulness  and  clear  comprehension,  or  as  we  have  been  calling  it  this  week,  awareness  and  wisdom.
Two different faculties. If we put energy in the mix, these three do the work
of  noting,  knowing  and  letting  go  repeatedly  and  continuously.  We  can’t
practice blindly or mechanically without thinking. Our meditation needs to
be simple, not complicated. We have to come to understand the structure of
our experience so that the letting go process can take place repeatedly.

\sphinxAtStartPar
\DUrole{pdfpage}{167}  When we are watching the mind, when we are turning our attention to
the mind, it’s really talking about emotional states. If you become aware that
an emotional state is there, just make a note such as ‘frustrated, frustrated’
and then come back to ‚rising, falling, sitting, touching, rising, falling, sitting, touching, rising, falling, sitting, touching, frustrated, frustrated’. Just
take a note that it is there. Just observe, don’t interfere with it, don’t push it
away. Don’t become upset that frustration is occurring. Don’t get annoyed
with  the  frustration.  Don’t  find  it  unpleasant,  just  know  that  frustration  is
occurring. This is not me, this is not mine, this is not my self. We have to
understand that it is just a mental state that has arisen and that it will pass
away very rapidly if it is not identified with. If it is identified with, then you
have  to  take  it  on  board,  carry  it  with  your  bag  with  you.  It  is  a  piece  of
dukkha. And you keep it until whatever causes the frustration passes away
and then your frustration passes away and with it your dukkha. You don’t
have to pick up every piece of dukkha that comes along your way. You do
have a choice in the present moment, either awareness or dukkha! As simple
as  that.  Which  do  you  want?  Up  until  now,  you  have  been  experiencing
dukkha all your life. Now we have the opportunity to apply a slightly different strategy in our experience of the world. Instead of going into the story
from our mind, we are stepping back from it. Letting go of it. We are creating
space around the stories of our lives so that it passes away. Every moment is
a new and fresh moment. Every moment is a new moment. Whatever dukkha
has arisen before this particular moment, it has now passed, it’s history, it’s
gone.  Don’t  bring  it  up  as  a  memory  again  and  start  dukkhering  yourself.
It is gone! Why bring it into the present? Why bring your old stories in the
form  of  memories  into  the  present  moment  and  attaching  and  identifying
with them? It is because we do it, because these old stories give us a sense of
self. «The poor me», «this is what happened to me». Think about it, we are
generating thoughts about me! «How unfortunate I am» or «how sad it was
that it happened to me, for me». Now these memories are just trigger points,
new foundations, or should we say old foundations of self that have come
back  to  revisit  again.  – The  future  is  the  same. The  planning  is  the  same.
Why do we like to plan? Why do we like to prepare? Thinking about it over
and over again. It gives us a sense of self. The planning for somebody. It is
\DUrole{pdfpage}{168}  my planning, it’s planning that I am going to do. These events, these states
of mind are the foundations for self. Self is very tricky! Craving has been
on the top of the tree for a very long time. It’s very tricky and very stable.
To break it, to crack it, you need a jackhammer of awareness and wisdom.
That is the only thing that can break it. Nothing else has been successful in
breaking through ignorance.

\sphinxAtStartPar
Have  a  look  in  your  mind,  whether  you  are  having  some  blemishes,
some emotional states, that you are attaching too. See whether you have any
stains in your mind. If you do, like most of us, you got some work to do. Some
spiritual work – that is what our life is about. The evolution of our mind, the
development of our mind. If we see any faults, we need to be alerted to them,
watch them and observe these faults. If you see a stain in your clothes, you
wash it out. Luckily for us all the stains are the same. They just need a bit of
soap powder to wash it out. Some are a little bit tougher than others, that is
true. But the technique is the same: washing. It may take two or three times
to wash the stains out. Some stains are stronger than others. Have a look at
your clothes. See whether you have any of these mind states, that come to
visit us, hang around, cause some trouble and leave again. Like a bunch of
drunken  guys  coming  into  a  car  park  late  night  after  the  pub. They  cause
some trouble and leave. If we get upset about that, it’s only because we are
attaching. My place, my pleasantness, my sleeping time. We have attached
to something and so somebody’s activity is annoying us. My town – what are
these guys doing here? If we don’t mind what is going on, then it’s just what
it is. When we mature in our view, hands off, ok, let it go. Those guys will
go before morning and you have saved yourself a night of stress, worry and
discomfort because you have been able to let go. If you haven’t let go then
you had an awful night listening all night to these guys.

\sphinxAtStartPar
Have a look if there is any anger or resentment. These cause a lot of
trouble. Maybe there is some ill will, hatred towards somebody. Maybe you
look down on others just judging – maybe that’s your stain. Maybe there is
some envy or avarice. We are jealous of what some people have got, what
they have become or how their life has turned out. Maybe we really, really,
really  want  something  –  avarice.  A  very  strong  form  of  wanting.  Maybe
you are fraudulent and deceitful. You don’t always present the truth of who
\DUrole{pdfpage}{169}  you are and what you are. Maybe you praise yourself and disparage others.
Judging others? Maybe some comparing? Maybe there is some complaining
going  on?  Complaining  and  whinging.  Have  a  look  to  see  if  these  mental
states are arising, note them, know them and let them go. If you don’t find
your  stain,  it  is  not  only  you  that  has  to  live  with  these  states,  but  your
friends,  your  family,  your  lover,  everybody  has  to  live  with  it.  You  don’t
want  to  be  the  last  person  in  the  room  to  find  out  that  you  are  angry,  do
you? When we’re unaware and we become triggered, than we become angry.
Everyone else can see it, why can’t we see it? Why don’t we stop it? Why
do we allow it to escalate and become a big problem? Why do we allow a
tiny irritation to turn into a frustration, get annoyed by that frustration and
become  angry?  We  may  say  something  unpleasant  to  somebody  else.  We
may  do  something  unpleasant  to  somebody  else.  Just  because  we  haven’t
looked at our own mind. Just because we have reacted to an infection or a
virus, which got into our head. And then what do we do? We blame others.
«I only got angry because of you!» We blame others, when we don’t see our
own defilements, our own imperfections in the mind. Have a look and see
if you have a temper or whether you get angry. When we experience anger
or resentment, it means that we have a mental resistance towards the object.
We are trying to push it away. If we are angry with someone, if we look at
that person again and again, we might even become angrier because we have
resistance towards that person already. It doesn’t matter what they say or do,
we are going to find a fault in their speech or their actions.

\sphinxAtStartPar
If we see any triggering, it is the perfect chance to work on ourselves.
That person who is annoying us has just become our new teacher. We should
be grateful to them, even though it is difficult. People who give us annoyance  can  really  lead  us  to  understanding  something  important.  We  create
our own dukkha through our attachments and reactions. Sure, there may be
someone who says something or does something unpleasant or foolish, but
we alone have reacted to it!
\sphinxstyleemphasis{We alone are responsible for our dukkha!}
It is
possible to live without dukkha if we are very mindful and aware. We are not
observing anger to lessen it or to go away. We are observing to understand
the connection between our mind state and our reactions to the mind state. It
is our reactions which is our main point of interest. We want to know when
\DUrole{pdfpage}{170}  our mind has become dyed with a stain of emotion.

\sphinxAtStartPar
Mind  states  are  satipatthana\sphinxhyphen{}objects.  We  can  use  them.  Craving  is
using them to develop a sense of self. Awareness and wisdom can use them
to develop the mind, to free the mind. All objects are possible. When craving
can use it for a sense of self, then we can use it for a moment of knowing. We
can use it to bring us even to cessation. These are legitimate objects of meditation practice. We need to know when our mind has been infected with lust,
by hatred or delusion. We need to understand these mind states very clearly.
More importantly, we need to understand our reactions to these mind states.
We need to check our attitude. Wishing anger to disappear, to decrease or go
away is wrong attitude. If we find ourselves in an angry state, we don’t want
it wishing to go away, we don’t want to find ourselves in pushing it away.
That is just further reaction! We’re trying to use aversion to get rid of anger.
And that doesn’t work. So how do we deal with things if we find ourselves
in an angry state for example? We just have to observe the moment. We just
have to be with it. It doesn’t matter if the anger goes away or not. Anger is
not the problem. This is conditioned phenomena. Anger is not the problem.
Our  negative  mental  reactions  when  we  experience  anger  is  the  problem.
How we react when we find ourselves in this state is what we need to look
at here. We can react with attachment and identifying going into the story
further  and  becoming  angrier,  completely  loosing  control,  or  we  can  step
back  and  develop  awareness  and  wisdom  in  the  moment  having  a  look  at
that anger, short circuiting the reaction mechanism that happens. Break it,
shut it off! When we stop reacting to it, that anger stops being dukkha! It’s
just anger, it’s not me or mine, it doesn’t belong to me, it’s impermanent, it
arises and passes away. As soon as you hold it or identify with it – dukkha!
If you can note and know it, disengage and separate from it, using wisdom,
then dukkha ceases in that moment and there arises a moment of wisdom,
a moment of clear seeing. Vipassana practice can pull us out of reactionary
emotional states and pull us out of habitual looping thoughts and tendencies.
It has the power to do that. We just need to practice it. Putting it in practice,
we  have  come  here  to  this  retreat  to  learn  how  to  do  this  in  an  intensive
environment. We try to do this as much as we can in seven silent days. Try
to get to the point as much, as much as you can. Then when you go down the
\DUrole{pdfpage}{171}  hill to the 7Eleven\sphinxhyphen{}village you can wander around the market and enjoy the
delights there without getting yourself into too much trouble. Of course, you
still react, but we have now a technique to step out of our reaction mechanism. We can cool ourselves down, we can calm ourselves rapidly.

\sphinxAtStartPar
So this is important: anger does arise and is part of our conditioning.
Mind states come up. Emotional states do arise. It’s karma, playing its role in
our life. No\sphinxhyphen{}one can stop karma from giving its resultant. Things are going to
manifest. It’s not all going to be rainbows and dolphins. Sometimes there are
troubles that arise. But we can be wise with our reactions toward anger. And
this is our practice in the present moment. It’s a gradual training and a gradual practice. We are learning how to do it in this retreat, but the real practice
begins for the rest of your life. See if you can do it, if you find yourself in
a state of unrest, when you’ve started freaking out about something. That’s
the time to employ the technique. If we do it in times where the mind is not
freaking out, then the mind will become very used to it. We’ll have the skill
to turn and observe something objectively, step back from it and allow the
let go\sphinxhyphen{}mechanism to do its job. It’s all built in, this is how we are designed.
This is how the mind and body process is set up. It’s very efficient noting
and  knowing  and  letting  go.  You  can  create  quite  a  wide  space  between
the knowing and all the mental and physical phenomena that arise and pass
away not getting sucked into any of them. We can have periods of meditation where nothing comes to stick, it’s all just doing its thing. As we start
to see the benefits of the practice, there is a great deal of joy in practicing
in this way. When there is anger and resentment, you’ll find that the feeling
associated with that anger is very strong – unpleasant feeling. There will be
some unpleasantness. That’s the aroma, the flavor of unpleasantness. There
will also be physical sensations occurring in the body when we get angry.
Because there is strong feeling, they are relatively easy to observe; subtle
states of craving and addiction are more difficult. When it is something that
is very unpleasant, of course it’s easier to let go of it. When it’s something
that is very, very pleasant – more difficult!

\sphinxAtStartPar
Letting go happens in stages: we let go of the past and the future, we
let go of the external world, we move beyond sensuality, we move beyond
the five hindrances, we let go of the painful physical sensations, we let go of
\DUrole{pdfpage}{172}  the painful mental emotions. Then we have to let go of the pleasant physical
sensations, and yes, we have to let go of the pleasant mental states that arise.
Everything needs to be let go. The whole lot needs to be chopped up, the
whole lot needs to be let go. We can’t allow any attachment to stand. If we
do, it will be a base for dukkha to arise.

\sphinxAtStartPar
There is a direct link between our mind state, mental feeling and the
physical  sensations  that  arise.  They  are  linked  together,  our  reactions  are
also linked to them. We described it earlier as the four big stairways leading
to the pagoda. You can decide which stairway you are going to use if going
to the pagoda platform.

\sphinxAtStartPar
In our meditation practice we can also choose, if we want to look at
our emotional state of mind. When it is really strong and we are really struggling with it, we can turn our attention and look at the feelings. The unpleasantness of the emotional state – that’s another stairway. If one stairway is
blocked, when we are having trouble with it, go around the corner, try to go
up another stairway. Try to reach the present moment, awareness of the present moment by going through the feeling foundation. If it is too unpleasant,
have a look at the physical sensations. When we become angry, the physical
sensations become quite obvious. Our face turns read, we feel warm inside,
our heart starts beating faster, our breathing quickens and becomes shorter,
we may start frothing at the mouth. We can be aware of these physical sensations  as  well.  We  have  to  choose  whatever  for  us  is  appropriate  in  the
moment. Sometimes, if it is a big event or a big experience, we’ll need to
note all the aspects of it, we need to walk all the stairways for us to let go
of the experience completely. We’ll have to let go of the physical sensations
that arise, and the feeling, we’ll have to let go of the emotional state itself,
and we’ll have to let go of our attachments and reactions of liking and disliking. The fact that it is there. All these things will have to be let go of. From
the moment you become aware of anger, no matter how weak or strong it is,
check your mind and body for tension. See if you feel any tensing going on.
Feel it in the body, it will start to become tense. The muscles start to contract
a little bit. So hardness starts to occur in the body. The earth element starts to
manifest in its strong aspect. Hardness, the body hardens.

\sphinxAtStartPar
When we are able to be with anger, that doesn’t mean that we are equanimous. \DUrole{pdfpage}{173}  Just because we can sit and tolerate it, «hm, it’s going on», that is
not equanimity. Sitting there and forcing ourselves to sit through an angry
mind state, forcing ourselves to sit through an aching knee even, that’s not
equanimity when we force ourselves. Equanimity has to do with reaction to
things. True equanimity arises when we are not pushing or pulling, we are
not attracted nor repelled by the experience. We are completely cool to it.
It doesn’t matter to us if it arises or passes away. If it’s there and stays for a
while or if it stays for a week. We have no interest in trying to get rid of it.
Our only interest is to know it, to see it clearly. We are not trying to push it
away in any way. We are not just blindly sitting with it. Our practice is going
beyond that.

\sphinxAtStartPar
So in this process, if we neglect to watch the mind, if we are not aware
of our mental reaction to an emotional state that is arising, if we fail to realize  that  sitting  with  anger  is  not  the  same  as  developing  equanimity,  then
we are just going to be dukkhering ourselves, holding ourselves in dukkha,
maintaining the base that craving has created. We are holding it there.

\sphinxAtStartPar
So instead start to watch the mental reactions to any emotional state
that start to arise. Try to make some space between the anger and you. It’s
not yours. Try to use the prism of the four noble truths. The Buddha says,
\sphinxstyleemphasis{«there is dukkha»}. This is his first noble truth. We can use these words «there
is». We can put them in front of any emotional state. «There is anger», «there
is envy», «there is jealousy». Whatever it may be, we can dis\sphinxhyphen{}identify with
it. We can stop identifying with it. We can objectify «there it is», there is an
angry mind state that has just arisen. It is very different from «I am angry».
«There is anger». The same experience, one with mindfulness and wisdom,
the other just blindly reacting and appropriating and identifying. When these
mind states come up, they are not you. They don’t belong to you. You don’t
need to take them so seriously. People take their mind states very seriously!
They  take  themselves  very  seriously!  And  of  course,  they  cause  a  lot  of
problems those people. Problems to themselves and problems to the people
around  them.  Making  space  around  our  mental  states,  around  our  mind
objects is the heart of our meditation practice. We are using wisdom to step
back. When  we  see  it  as  it  is,  wisdom  allows  consciousness  to  step  back.
Consciousness is not in there addicted, subjectified, it’s out of that. It’s in the
\DUrole{pdfpage}{174}  space around it. It’s in the freedom that surrounds the mind state. If we’re in
the mind state, we’re in the dukkha. If we’re noting it and letting it go, than
we’re in the space of freedom. Freedom from our own mental afflictions.

\sphinxAtStartPar
So we start to watch our mental reactions to resistance and the anger
gradually decrease and fade away. When we are not giving it any attention, it
goes away like a little kid, when we are not entertaining it, not giving it any
attention, it will find something else to do.

\sphinxAtStartPar
Equanimity  is  the  result  of  truly  understanding  the  nature  of  liking
and disliking. And truly understanding liking and disliking comes from the
observation of it, of our evidence, from our own experience, not from someone else telling that you have to stop liking and disliking. That’s not going to
be powerful enough to do any transformation. That’s a little bit of information you got, a little information contained in knowledge, but it is not strong
enough for a transformation to occur.

\sphinxAtStartPar
So understanding the difference between equanimity and to bear with
a mind that is seizing in anger, is very important. When we look at anger or
resentment directly, and there is no resistance to it, then there is no problem
in the mind. It’s just a mind state that is there. We are neither upset by it, nor
disturbed by it. And because it’s not getting any fuel, because it’s not fed by
craving, it can’t be used as a base for the arising of self, it just passes away.
It has become redundant. That mental experience is useless to the mind and
body process. It can’t generate a sense of being and that is all that craving
wants. So we put a wedge in between our experience and craving. We create
some space. Then we can truly see the stain in the cloth and wash it out. The
mind becomes equanimous and sensitive.

\sphinxAtStartPar
When  we  truly  have  equanimity  towards  a  mental  state  like  anger,
resentment or jealousy – the mind state may still remain there – but there
won’t  be  any  dukkha  related  to  it. We  won’t  find  it  uncomfortable. Anger
will be purely an object. It will have been objectified, put out there, it’s not
me, it’s not mine. It looses its importance and its function and it fades away
while we look on with equanimity.

\sphinxAtStartPar
So remember that you are not looking at the reactions of the mind to
make them go away. We are always investigating the nature of our reactions
so that we can understand them. When we understand them, when we create
\DUrole{pdfpage}{175}  space around them, then they pass away by themselves. We don’t need to
push them away.

\sphinxAtStartPar
So try to apply these points, try to observe any emotional states that
you are going through. Spend the next few days here learning how to look
at them objectively. When you get back to the busy world, it is much more
difficult to look at them than in silence. See if you can notice any stains in
your clothes that you have to do any work with. When we notice what needs
to be done, this is the beginning of a spiritual life, this is the beginning of
becoming a spiritual person. We’re not talking about a religious one, we are
talking about somebody who is doing some work on themselves, who makes
themselves and the world a more pleasant place to be.

\sphinxAtStartPar
Whatever we learn from dealing with a single mind state like anger or
resentment can be applied to other mind states as well. Emotions or looping thought patterns, all their relatives and different emotional states can be
dealt with in the same way, because all have the same structure. Structurally
they’re the same, their content is different. When we dealt with a certain set
of circumstances, we have seen that content, the way of letting it go is by
seeing its structure that holds that content in place. Once we start to observe
the structure in our present moment experience, it doesn’t matter what the
picture on the screen is. It doesn’t matter what the emotional state is. It can
be one of dozens. That’s kind of irrelevant to us, the content is not important.
Everything needs to be let go of. Of course we can start with the really damaging ones, and then we can start to work on the more subtle ones to refine
our  experience  of  life. Work  on  the  big  stories  first  that  cause  us  anxiety,
stress or depression. Fear or worry or concern. Then work on the more subtle
ones that give us states of unpleasantness.

\sphinxAtStartPar
\sphinxstyleemphasis{It’s important to remind ourselves that emotions are natural phenom\sphinxhyphen{}}
\sphinxstyleemphasis{ena  but  they  are  not  your  emotions.}
They  are  happening  in  this  mind  and
body process. If you identify with them, then they are yours and you have
to experience them! You dukkher yourself. But if you have awareness and
the wisdom in the moment they arise, then you don’t have to identify with
them. You objectified it, they’re removed through the power of awareness
and wisdom.

\sphinxAtStartPar
When you feel yourself under attack by a strong emotional state, sometimes \DUrole{pdfpage}{176}  it’s best not to even look at that emotional state. This may cause even
more trouble, if our awareness and wisdom are not up to the task. In going
closer  and  closer  we  may  become  completely  engulfed. That’s  the  time  –
when a very strong emotion, like when a family member dies, you have very
strong grief, or you break up with a girlfriend or boyfriend, a strong emotional state – to use another stairway. Have a look at the body. Have a look
at feeling. Have a look at the unpleasantness or pleasantness of the situation.
When it all becomes completely too much to look at, when all of the
four foundations seem too much to observe, in that case you can turn your
mind to a neutral object. Just try to distract yourself from that state. Start
to watch the breath or start to do some walking meditation. Do a good hour
of walking meditation, maybe fast walking to clear it out. Pay attention to
something  else  until  the  emotion  calms  down  a  little  bit.  Not  completely
disappears, but when it calms down, then you may be able to have a look
at it properly. When you have weakened it a little bit. When you distracted
yourself from being in the story so much, then you can have a look. Go and
jump on your bike and go for a ride, or go for a swim. Calm down and then
have a look at it. Then you can learn something, it’s all about learning. It’s
all about knowledge and understanding. That’s how letting go takes place.
Through wisdom.

\sphinxAtStartPar
The more skillfully you can do this, the more skillfully you will understand that physical and mental phenomena are related with each other, the
more skillfully you will be able to deal with your experience of life and the
more satisfying that experience will be.

\sphinxstepscope


\chapter{Day 5, morning}
\label{\detokenize{5-a:day-5-morning}}\label{\detokenize{5-a::doc}}
\LOCALaudiolink{https://www.mixcloud.com/anthonymarkwell/day-5-morning-talk-desires-intentions-and-the-five-faculties/}

\sphinxAtStartPar
This morning we are going to look more at intention and desire. And
we are going to have a look at who is doing the meditating. If there is no
me,  mine,  no  I,  if  there  is  nobody  here,  then  who  is  doing  the  meditation
practice?

\sphinxAtStartPar
Our  meditation  experience  will  be  different  for  everybody,  so  don’t
compare to anybody or with your own expectations. Just settle back in the
moment.  Be  here,  noting,  knowing,  trying  not  to  get  caught  in  your  own
stories that take us away.


\section{Intention and desire}
\label{\detokenize{5-a:intention-and-desire}}
\sphinxAtStartPar
I  want  you  to  start  to  have  a  look  at  your  little  wants  or  your  little
desires. Sometimes we have the mind with just wanting something a little,
little things we want to do, «would it be ok if I just did this». No! Look at
these  little  wants  and  little  needs  we  have  been  catering  for  all  our  lives.
We become the slaves of little wanting. It becomes habitual in our life. Our
mind wants things to be just a little different than how they are. We start to
tweak things, to interfere with what is functioning quite nicely but we are
not satisfied with. We need to change it a little bit to satisfy our craving, to
satisfy our wanting.

\sphinxAtStartPar
So all our lives we have been catering to these little perceived desires,
\DUrole{pdfpage}{178}  all these things that we think make our life better. A little better for me. Actually these wishes and wants, these desires and cravings are the cause of suffering. It’s those little wishes that give us so much stress and worry. Because
when they are not fulfilled, they give us a lot of dukkha. In fact it’s in the
definition  of  dukkha,  «not  getting  what  one  wants». Any  discontent  with
the way things are. If we are constantly reflecting upon it, we can become
a miserable person, complaining about things, criticizing things. Things are
not quite the way we want them to be, we want them to be a little bit different
because that would be better for me. It’s always a question of what’s better
for me. We can make ourselves more comfortable even when we are doing
things for others.

\sphinxAtStartPar
So this content needs to be observed, needs to be noted and known and
let go of. Try to become aware of any states of discontent in the mind. Have a
look at it closely and the trouble that is going on. If we can understand this, if
we can change this, we can change our life quite dramatically. If we come to
understand the difference of what we want and what we need – necessity, we
do need certain stuff, a place to sleep, food every now and then, some clothing  –  these  are  needs.  Everything  else  are  wants.  If  you  don’t  understand
your mind wanting stuff, if you can’t see this, than you are going to have to
run around after it. You are going to have to fulfill all its wants. Or worse,
you are going to have to run around to fulfill other people’s wants! We do a
lot of that, don’t we? We run around to fulfill what other people want, what
other people expect. It can be tricky. The different groups in our lives want
different things. Our parents, our boss, our girlfriend, our lover – each wants
something.  They  are  all  wanting  different  things.  It  causes  a  lot  of  stress
if  we  don’t  see  this  wanting  mind,  this  mind  which  is  not  quite  satisfied.
The tweaking mind. It wants to tweak itself. Making itself busy. Business is
somehow relegated to a big thing in our societies. It’s kind of good. We want
to be busy! Busy doing what? – Creating a sense of self!

\sphinxAtStartPar
Just this little discontent. «Kow Tahm is here to change you. You are
not here to change Kow Tahm.» This was a sign of Steve and Rosemary here.
People want to change things! Why? – To satisfy themselves. To feed their
self, to make sure their self is getting a little pat, «it’s all going to be all right,
it’s all going to be good», just trying to feel a little more comfortable. All
\DUrole{pdfpage}{179}  we are really doing, is running around trying to satisfy this state of me. It’s
defilement. It’s corruption of our present moment experience. An imperfection of the mind.

\sphinxAtStartPar
If we constantly spend our lives running around catering to our wishes
and wants, being their slaves, being their soldiers, then we run around doing
craving’s work. It only leads to dukkha. So why not chill out and relax. Yes,
relax a little bit – but no relaxing here, we are on a retreat – but when you
leave you can relax a little bit.

\sphinxAtStartPar
Try to let go of some of your wishes and wants. Try to identify them in
your mind when they’re swirling. I am sure there’s a few things coming up.
Little things and wants. What is that you have been planning in this retreat
here? Have a look at those little things. We are constantly catering to them.
We are constantly trying to become content but just don’t know the way. We
think that if we just get that done then this will be it. «That’s the last room
I’m going to extend the house with. Five extensions is enough, there will be
no more.» Maybe a garage. We have to stop adding on that stuff. It’s only if
we stop getting into craving, we just say no, we can’t fulfill your wishes at
the moment. And then we may fall into some contentment.


\section{Contentment}
\label{\detokenize{5-a:contentment}}
\sphinxAtStartPar
Talking about contentment here, being content is right here in the present moment. Contentment is really important in our meditation practice. And
it’s really important for our daily live if we want to be sane. We want to be
content. We need to understand and be cool with how things are. We develop
a state of mind, this is how it is now, this is what is occurring right now. It’s
just like this.
\sphinxstyleemphasis{This is dustness or suchness},
\sphinxstyleemphasis{tatata}
in the Pali language. We
are just allowing things to be such. We’re just allowing things to be dust. We
are letting go of cravings little wants and wishes. We are not going to feed it
anymore. We think by feeding it constantly, finally we are going to become
content. No! It’s never satisfied. That desire is unsatiable. Contentment only
happens when we give up. Trying to change everything when we settle down
into the moment, say well, «this is how it is, hurray! It is just like this at the
moment». Today is now. Today it’s like this. Then we’re just ourselves and
start accepting things. We acknowledge, we observe how it is. We wake up
\DUrole{pdfpage}{180}  every morning and say, well, this is how it is today. And we have a good day.
All we have to do is to extend our moment of contentment out a few
moments. Keep extending it into the future. If you’re feeling unhappy, just
make yourself content with how things are and than, it makes moment content,  makes  moment  content,  makes  moment  content,  whoever  you  meet,
wherever  you  go.  It’s  all  cool.  It’s  all  good.  There’s  nothing  wrong  with
people  of  the  world.  But  sometimes  our  minds  can  create  problems,  our
mind just sees things as good or as bad. Things are not inherently good or
inherently bad. We make ourselves quite unhappy by not recognizing contentment is right here in now. Contentment is accepting the way things are.
Try to notice these little intentions because they take us out of contentment.
Contentment  is  the  proximate  cause  of  happiness.  Happiness  is  the
proximate cause for concentration. And concentration is the proximate cause
for insight, for Vipassana. So we need to be content if we want to get Vipassana.  It’s  as  simple  as  that.  It’s  a  conditioned  process,  contentment  makes
us  happy,  happiness  makes  us  concentrated  in  the  moment,  when  we  are
concentrated and happy, we see things as they really are. If we are discontent
and unhappy and moving and switching and dissatisfied you’re never going
to concentrate the mind. You’re going to be restless, it’s going to be uncomfortable. So we need to become comfortable every moment of the day that
we are meditating. We need to develop contentment with the way things are
for  our  meditation  practice  to  come  to  fulfillment. We  have  to  understand
our inability to stand up to defilements, the craving sense who attacks us. It’s
really not that dissimilar to little children wanting more of this, more of that.
More, more, we start to learn this from a young age. Observe the children
and watch the craving grow in them. It’s quite bizarre if you start to observe
that. Watch the little ones to get their sense of ego if they want to get stuff
and to become.

\sphinxAtStartPar
If we don’t make an effort to note these little desires and wants, and to
let them go, then we are going to remain spiritually immature. If you want
to evolve as a spiritual being you will have to become content with the way
things are. This is how your karmically conditioned world is operating. You
are the heir, you are the owner of your own karma. It’s unfolding, don’t be
upset about it. We only think our karma is really bad when we attach to it. If
\DUrole{pdfpage}{181}  we don’t attach to all that is unfolding, it’s all just fine. Nothing wrong with
it at all. Just conditioned processes – mind and matter arising and passing
away. It doesn’t belong to anyone. Just going its way. We do make enormous
problems  for  ourselves,  if  we  do  become  dissatisfied  or  when  we  want  to
change  things,  constantly  changing  and  tweaking,  making  ourselves  busy,
busy with the thought of me, making the world a better place for me to live
in. And it doesn’t stop. When you are doing it in your 20’s, in your 30’s, you
are doing it in your 60’s. Constantly tweaking, constantly not satisfied, looking for something else to do.

\sphinxAtStartPar
So there is much dukkha in the mind that is discontent, which is unsatisfied. To be aware of this mind state, when it’s coming up – if you are not
happy  with  something,  if  you  are  resisting  something,  if  you  are  pushing
against – have a look at that mind state. That is where your work has to be
done. That is what you need to realize to become content with. Right there!
There is no\sphinxhyphen{}one else disturbing your mediation except for craving. Craving
is the one that gets in the way.

\sphinxAtStartPar
Let’s  have  a  look  at  this  cause  of  suffering.  If  you  are  experiencing
some dukkha, the cause of it is craving. You are not content with the way
things are right now. We want things to be different and so there is dukkha.
When we can open our heart and open our minds into the present and accept
whatever is there, not identifying with it, and allow it to be and let it pass
away, than there is no dukkha in that moment. It is just a ceaseless flow of
mental and physical phenomena.


\section{The five faculties}
\label{\detokenize{5-a:the-five-faculties}}
\sphinxAtStartPar
On  the  first  evening  of  our  mediation  retreat,  we  asked  the  question
what is meditation. We said meditation is the development of five spiritual
faculties. They are known as the
\sphinxstyleemphasis{indrya}. Our Vipassana practice can be seen
as a process developing certain positive mental factors, certain positive faculties until they are powerful enough to dominate the mind quite continuously. Until they are powerful enough to take control of the present moment
away from craving and its mission to be. Then these faculties hand things
over to awareness and wisdom so that they can do their job. On an intensive
retreat like this, our practice will start to develop these controlling faculties.
\begin{itemize}
\item {} 
\sphinxAtStartPar
\DUrole{pdfpage}{182} They  are  strong  endurable  faith  and  confidence,  it’s  faith  and  confidence in yourself and the teaching and the practice.

\item {} 
\sphinxAtStartPar
We  develop  a  continuous  energy,  activating  mindfulness,  activating    awareness in the present.

\item {} 
\sphinxAtStartPar
We  develop  a  penetrative  mindfulness  or  awareness  where  the  mind really sinks into the object, it doesn’t slip off. Mindfulness keeps us in the present being here.

\item {} 
\sphinxAtStartPar
We also develop different stages of concentration. The mind becomes
stable. We develop a type of concentration which allows for the fifth
faculty to unfold – wisdom. We develop a profound insight or wisdom
into the nature of things.

\end{itemize}

\sphinxAtStartPar
These five faculties are what we are developing on the retreat. In fact,
these five faculties are doing the meditation practice. There is no\sphinxhyphen{}one doing
it, they are meditating themselves. We activate them to get them going, so
the mind starts to do its training. As we do the practice, these five faculties
become  stronger  and  more  powerful.  On  a  longer  retreat,  one  month,  two
months  or  three  months,  these  faculties  become  quite  developed  and  they
take over. They take over control of the mind and body process. Our mindfulness becomes continuous with sustained momentum. We don’t slip off of
the object, we don’t have to activate our awareness so much, we are right
there  all  the  time. We  are  in  the  moment,  noting,  knowing  and  letting  go.
That is when the faculties are fully developed, they turn into powers. These
powers have the ability to cut through defilement. Not just suppress them,
cut them off! Cut off the nasties from arising again.

\sphinxAtStartPar
So these mental states do the meditating. They combine to watch the
mind and body. They are the mind and body, but they are watching themselves. The thing is doing it itself. Please don’t inform craving what the faculties are doing. It will get upset. And please don’t tell the faculties that they
are meditating themselves out of existence. That’s what it’s doing. The thing
will meditate, this mind and body process, and the faculties are developed, it
will mediate and meditate and meditate, until it sees clearly how it is coming
into  existence  and  then  it  will  remove  its  cause  for  coming  into  existence
and it will come to cessation. There is nobody here. It is an impermanent,
impersonal process that we have been trapped in, that we believe is our life.
\DUrole{pdfpage}{183}  Very much like the matrix. That’s what we are living in.

\sphinxAtStartPar
So these five faculties have the power to shut down the mind and body
process, to bring it to cessation, to free consciousness from the bondage of
mind and matter. And the bondage to the rounds of being and existence, to
the bondage of birth, old age, sickness and death. We can free the mind, free
consciousness  from  being  addicted  to  these  narrow,  small,  four  elements.
And the mind, it goes with it. At the moment it is so strongly identifying with
the body, it can hardly do anything else, constantly busy catering to these
body’s wishes and wants trying to make it happy, trying to make it secure,
trying  to  give  it  love  and  attention,  looking  at  it  in  the  mirror,  smiling  at
itself, liking itself. It thinks about it constantly. It’s what it’s doing. So these
five faculties have the ability to turn off the system so it comes to cessation
and then the consciousness is free, unlimited. The knowing is not bound by
the  confines  of  its  former  addiction.  It  stepped  out  of  its  addiction  to  this
mind and body process. It stepped out of its addiction to being, its wanting to
be something, its craving for existence, for recognition, for power, for fame,
for identity, for personality, for grooviness. It does all kinds of things, quite
dependent upon what other people think about it. What other people think
about  the  mind  and  body  process  can  make  us  quite  upset.  Some  people
don’t like us or scold us for some reason. We get upset with them, upset of
this mind and body process. We identify with other people’s craving, other
people’s defilements. We let that upset us. So other people are in the mess
and we get pulled in the mess as well. We have identified with their mess.

\sphinxAtStartPar
So we develop those five faculties. That is what our mediation practice
is all about.
\sphinxstyleemphasis{There is no\sphinxhyphen{}one that is meditating, there is no meditator. Medita\sphinxhyphen{}}
\sphinxstyleemphasis{tion is occurring}. There is no meditator. Seeing is occurring but there is no
seer. Hearing is occurring but there is no hearer. These things are going on.
If we don’t train the mind, it will just be passive and receptive to whatever
is there, and then reactive, passively reacting – if there can be such a thing –
passive reaction. Sitting there and let the thing constantly just react: liking,
disliking, wanting, not wanting. And we just go through our life like that.
You  think  that  is  life!  We  think  we  have  to  conform  to  craving’s  desires.
Trying to conform how our societies’ craving is. As Krishnamurti says, «it is
by no means a sign of good health, to be well adjusted to a profoundly sick
\DUrole{pdfpage}{184}  society». We don’t have to try and conform to craving’s desires and wishes.
That’s  not  our  purpose  to  become  one  of  the  herd,  to  become  one  of  the
sheep. Our purpose is evolution, evolve as spiritual beings. That is the quest
of  life.  That’s  our  great  fortunate  circumstance,  that  we  have  been  given
the very fortunate opportunity of rebirth into a human body. A sense\sphinxhyphen{}sphere
being that experiences dukkha and sukkha, experiences suffering and happiness, is fit and healthy and has come upon the Buddha’s teaching, is one
of the lucky ones on the planet. Who has come upon this teaching? There
is  seven  billion  people  on  the  planet.  How  many  of  them  get  to  actually
practice  the  art  of  evolution?  Consider  yourselves  as  extremely  fortunate.
To be fit and healthy and to have come upon the Buddha’s teaching. Allow
the meditation process to evolve. And so that you can bring yourself out of
dukkha.

\sphinxAtStartPar
As  we  are  doing  our  practice,  we  are  noting  our  ‚rising,  falling,  sitting, touching, hearing, smelling, tasting, remembering, planning, thinking
angry, dissatisfied, frustrated, lusting’ – all the things that are going on in
the mind. As we start to note and know, and let go – as we all have put the
conditions  in  place  during  the  last  couple  of  days  –  you  start  to  see  some
results of your efforts. Maybe minor, a few thrills and spills here and there,
maybe some understanding is coming to you, the nature of what life is all
about. Of course, your insight gets deeper and deeper the more concentrated
you become.

\sphinxAtStartPar
We are doing our meditation and have our first good sitting, or something good happens, we become kind of satisfied. «Oh, what he is talking
about  happened.»  «Oh,  good»,  and  so  we  develop  the  first  faculty,
\sphinxstyleemphasis{sadha}
or  faith.  «This  teaching  works,  maybe  the  Buddha  was  right,  maybe  if  I
follow  this  technique  it  will  take  me  somewhere.»  So  we  develop  faith
and  confidence  in  the  method  and  in  ourselves.  «I  can  do  it!»  What  does
that  do?  It  makes  us  want  to  work  more,  get  the  whole,  sit  again. We  put
forth  more  energy.  When  we  put  forth  more  energy  the  facility  of  energy
starts  to  develop. We  are  actively  generating  our  awareness  in  the  present
moment. We are bringing our mind back into the present moment after it has
been wandering somewhere. It’s the effort and the energy to be present, to
bring ourselves back, that we are talking about here. When we have a little
\DUrole{pdfpage}{185}  bit  of  success  in  our  practice,  it  builds  our  confidence,  we  put  forth  more
effort, when we are putting forth more effort, then our mindfulness becomes
more continuous. Our presence becomes more continuous. We sink into the
objects  that  are  being  noted.  Our  mindfulness  becomes  more  confronted,
more penetrative. It stays longer. It is able to discern the nature of reality
of whatever it is that we are noting. And so the faculty of mindfulness gets
developed. It’s the third faculty.

\sphinxAtStartPar
As we do this, our mindfulness is continuous and penetrative, then the
fourth faculty comes into being. The mind stabilizes, it will become quiet. It
becomes still. – When you are in the back of a car wanting to take a photo, it
becomes still fuzzy, not clear. As our concentration develops things become
stable. We can take a photo and will see that that is crisp and clear. We are
seeing things more clearly as our mind is more stable, becoming more concentrated. So the fourth faculty becomes developed. So when these first four
faculties – faith, energy, mindfulness and concentration – when their work
has been done, when they’ve been developed, then insight flows. We’ll have
a taste of some dhamma. The curtain will open, knowledge will be revealed
intuitively, you won’t come and need to ask me what happened, you’ll know
it for yourself. You’ll see things as they are, you have a first taste of Vipassana knowledge.

\sphinxAtStartPar
Of course, when an insight arises there is a great deal of excitement
and  determination  for  the  practitioner.  We  become  enthusiastic. And  then
our faith faculty deepens again. We get more faith and confidence, that leads
to more energy, which leads to a more penetrative mindfulness, a more continuous  awareness,  which  leads  to  greater  stability  and  concentration,  and
another insight unfolds. And then we get more faith and confidence…so it
starts  again.  These  faculties  start  to  grow.  The  Buddha  says,  we  need  to
develop these faculties. We build them until we can see things arising and
passing away, until the knowledge of udayabbaya ñāṇa, the knowledge of
arising and passing away. When we reach this, the mind will feel very fulfilled,  very  content  with  its  progress,  in  its  evolution.  It  sees  all  things  as
conditioned  and  impermanent,  unwilling  to  appropriate  and  identify  with
any  of  the  stuff.  It’s  no  longer  sticky. Awareness  doesn’t  get  stuck  in  the
objects. They’ve all become coated in teflon. They arise and pass away, they
\DUrole{pdfpage}{186}  arise and pass away. None of the stories stick. None of them really have any
interest. We can’t even be bothered thinking about them. We don’t want to
go  into  any  stories  of  analyzing  or  identifying,  things  are  just  arising  and
passing away and we are very content in that moment. We start to see things
as they really are. Vipassana insight arises.

\sphinxAtStartPar
This  picture  of  Vipassana  is  given  by  the  Buddha  in  the  old  texts:
Vipassana can be compared to sharpening a knife. There are a lot of different conditions that we need to put in place to sharpening a knife. First of all
the sharpening stone needs to be properly oiled, we need to hold the knife
just at the right angle, we have to press with just the right amount of pressure, we have to go in a single direction, and we need to do it continuously,
make sure we don’t blur any of the edges. Then we have to flip it over and do
the other side of the blade. When we do this, when we develop our faculties
in this way, the knife becomes very sharp. It cuts through defilement. It cuts
through to wisdom.

\sphinxAtStartPar
We  can  look  at  the  development  of  our  practice  just  like  sharpening
a  knife.  To  be  observing  all  the  different  parts  of  the  sharpening  process.
The  right  oil,  angle,  pressure,  the  right  direction,  all  these  things  need  to
be monitored while we are watching, while we are noting and knowing in
the  present  moment.  We  need  to  observe  these  faculties  as  well  to  make
sure they are not getting unbalanced in some way. Learning to balance these
faculties is an important part in our development. If we are lucky that we
can practice with a meditation master who can do that, that’s good! I was
fortunate enough to have a meditation teacher who paid a lot of attention to
the balancing of the faculties making sure that the faculties are not becoming
unbalanced but a condition for the arising of wisdom – wise attention. The
voice of another. When we attend wisely to things, when we see the reasons
and causes for things, we can see the reasons and causes for our meditation
practice  becoming  successful,  we  can  make  appropriate  decisions  on  how
we do our practice.


\section{Balancing the five faculties}
\label{\detokenize{5-a:balancing-the-five-faculties}}
\sphinxAtStartPar
\DUrole{pdfpage}{187} So the five faculties need to be balanced. Mindfulness is in the middle.
Awareness  is  there,  watching  the  show,  observing,  taking  note  of  things.
Energy  and  concentration  need  to  be  balanced  with  each  other.  Faith  and
wisdom need to be balanced with each other. And then all five need to be
balanced together. In our practice we need to monitor how our mind is doing.
If the faculty of energy becomes too strong, stronger than concentration,  then  the  mind  becomes  restless. You  can’t  sit  still,  not  concentrated.
In this case we need to sit more than walking. Our walking meditation and
sitting meditation can help us balance the faculties.

\sphinxAtStartPar
If we found ourselves not agitated but in a state of dullness, when our
meditation  is  leading  to  a  quiet  state  of  dullness,  when  nothing  is  really
happening, no further development, but is kind of nice and pleasant, when
we’ve slipped into a dullness, then our faculty of concentration is stronger
than our faculty of energy. We are not noting enough. The mind is spinning
quite well and then it’s going into a kind of calmness. It just becomes dull
and  calm.  So  we’ll  need  to  balance  that.  We’ll  need  to  do  more  energyfaculty work, more walking meditation, more noting. When we note more
continuously, this builds the energy faculty.

\sphinxAtStartPar
So we are trying to balance them. Too much slackness, do some more
walking! Too much concentration, when it’s becoming dull, more walking!
If we become agitated in our mind, we need to do more sitting. On a short
retreat like this, we do just 45 minutes of each. On a longer retreat, we may
monitor  that.  If  we  find  ourself  becoming  restless,  we  just  walk  for  half
an  hour  and  sit  for  an  hour  to  develop  the  faculty  of  concentration.  If  we
become dull and sleepy, we do an hour of walking meditation and half an
hour of sitting trying to make more energy in the mind, more noting. You
will notice when you try to note four objects, ’rising, falling, sitting, touching’, the mind is quite busy. Just noting like this for five minutes over and
over again, takes a lot of effort, a lot of energy. When we do that our mind
becomes very energized and then we can balance it with concentration.

\sphinxAtStartPar
On the other side, we have to balance faith with wisdom. If our faculty
of faith and confidence becomes too strong then we may become gullible, we
start believing any kind of story that people tell us. Or we may become overdevotional. \DUrole{pdfpage}{188}  Or we may become obsessed collecting flowers for the Buddha,
spending six hours of the day doing that. Well, devotional activities, nothing wrong with them, I like them very much. We do a little bit, that’s nice
and  inspiring. We  don’t  go  overboard. We  don’t  allow  our  faculty  of  faith
to become so strong that wisdom and reason disappear. We need to balance
our intelligence, our reason, our wisdom with faith. If we don’t have enough
faith, the mind thinks it knows everything. If it doesn’t have faith, it relies
purely on science and on reason, only these things are true and everything
else is nonsense. It starts getting into that mode of operation. And this can
be a problem as well; if we start to think we know everything already, no
further work to be done. We maybe start to admonish others, we start to look
down on other people. So we need to develop the faculty of wisdom to be
balanced with the faculty of faith. We don’t want too much faith, then we
become gullible, and we don’t want too much wisdom or intelligence so that
we start to think we are the smartest person in the class.

\sphinxAtStartPar
So these five faculties are to be balanced. Balancing the faculties is an
aspect of meditation that needs to be understood clearly. You won’t always
be on a retreat\sphinxhyphen{}setting like this. You won’t always be able to ask me, what
should I do with my sitting, with my walking. You’ll have to take care of
yourself. We need to be able to adjust and know. You’ll need to be able to
read your own mind. «Oh, now the mind is feeling slack. It’s the time to do
a long walking session and then I’ll do a short sit.» Or if we are feeling restless, worried and stressed, then it’s the time to do more sitting; just have a
quick walking session before you have a long sitting to calm things down.
You need to adjust the faculties yourself with your own practice. This is how
we develop the faculties. These are the meditators. So we need to observe
and watch that.


\section{Developing the five faculties}
\label{\detokenize{5-a:developing-the-five-faculties}}
\sphinxAtStartPar
A few tips for developing the five faculties: The first of them is, pay
attention  to  impermanence.  Really  have  a  good  look  at  things  arising  and
passing  away.  See  it  in  the  breath,  it  arises  and  passes  away.  Each  breath
is  just  a  single  momentary  unit,  coming  into  existence  and  passing  away,
never returning again. That breath is gone – finished! It’s arisen and passed
\DUrole{pdfpage}{189}  away.  Our  stages  in  the  walking,  they  arise  and  finish,  arise  and  finish,
finish, finish. Sounds are finishing, our thoughts and ideas are finishing. Our
memory will come and go. A pain in the body will come and go. Sensations,
feelings, emotions, thoughts – they are all arising and passing away. Have
a  look  at  this,  pay  attention  to  this  aspect  of  your  experience.  Try  to  see
things as being impermanent. Nothing lasts, nothing lasts forever. That’s an
understatement! Nothing lasts for a second! It’s all gone, if you are noting
and knowing and letting go. If you are not noting and knowing, then you are
holding, you are attaching and clinging. Things stay around a little bit longer
because they are being used by craving to develop a sense of self. Craving
found a nice little bed for itself, a nice little sofa to sit on and starting its job
of  causing  problems. Also  observe  mind  and  matter,  mental  and  physical
phenomena arising and passing away at the six sense doors. This is where
impermanence  is  seen.  We  see  impermanence  in  our  own  mind  and  body
processes. Impermanence externally is also interesting to pay attention to.
The leaves falling, the light fading, the sun moving. Those things are impermanent but we are much more interested in looking at our own internal features. This is where dukkha arises, and this is where dukkha ceases. And this
idea of impermanence can only be confirmed through our own observations.
So your own watching, your own noting and knowing when you see it for
yourself. Reading about it, doesn’t really do anything. You can have a nice
read about impermanence, but that’s nice, close the book and go to sleep. It
doesn’t really change you. When you see it in your own mind, happening in
your own body, then you get a good taste of what the dhamma is all about.
It’s visible in the here and now.

\sphinxAtStartPar
Secondly,  we  should  have  care  and  respect. This  is  a  good  basis  for
developing the faculties. Care and respect for the practice, for the training.
An  attitude  of  great  care,  we  are  trying  to  be  meticulous  in  our  practice.
Have a look at the qualities of the sangha – those enlightened disciples of the
Buddha. The Buddha has practiced straight forwardly, methodically, persistently. This is the qualities of the practice of the enlightened ones. We’ll need
to develop this as well. We’ll need to have respect, have reverence for the
training. It’s a noble training. It’s a training that has been passed down for
26 centuries. It’s very effective. Such a teaching has had a profound effect
\DUrole{pdfpage}{190}  on human civilization in the last 2000 years. Now it’s our chance to witness
this. You can also reflect on the benefits you are likely to enjoy from doing
this  practice.  Having  care  means  slowing  down.  Slow  down  your  movements. Your  transitioning  from  the  sitting  hall,  to  the  walking  area,  to  the
dinning hall, to your dorm. When you go to the toilet, slow down, be mindful and keep your awareness inside the body as much as you can. This is a
second factor we can use to strengthen our faculties.

\sphinxAtStartPar
A  third  factor  is  unbroken  continuity.  Be  continuous  in  your  practice. This develops the faculties very well. Preserving the continuity of our
activated awareness. Be in the moment as much as possible. Moment after
moment.  Don’t  allow  any  breaks  in  your  practice. This  really  strengthens
the practice.

\sphinxAtStartPar
These spiritual faculties produce spiritual warriors, people who are on
the  path  to  enlightenment,  who  have  let  go  the  worldly  associations.  Not
giving  up  all  fun  in  life!  But  letting  go  of  the  nonsense  that  is  not  useful
for our evolution. We start to recognize what is useful and what is unuseful
in our life, what is beneficial and what is unbeneficial, what is suitable and
what is unsuitable for walking the path. When we recognize what is suitable
and  what  is  unsuitable  we  can  make  wise  decisions  and  our  practice  will
develop from there.

\sphinxAtStartPar
So  having  unbroken  continuity  throughout  the  day,  apart  from  the
hours  we  are  sleeping,  we  are  paying  attention  from  the  moment  we  hear
that  bell  in  the  morning,  becoming  aware  of  our  head  on  the  pillow,  our
‘very strong desire’ to leap up and get to the meditation hall as quickly as
possible – some of you may have been thinking about sleeping in the meditation hall, maybe – so our continuity should be as strong as possible.
\sphinxstyleemphasis{That is}
\sphinxstyleemphasis{actually the secret of developing these five faculties.}
Continuity, the fire stick
rubbing! You have to keep it going so the friction remains warm. When the
friction remains warm, we never know when the flames and fire are going
to jump out. We never know when that spontaneous flame is going to flare
up. We never know when an insight is going to arise. We know, that it arises
when the mind is continuously aware of the present moment but we don’t
know when. So we just have to keep putting it into the moment and trust that
all conditions are going to be in place and so then it pops up by itself. An
\DUrole{pdfpage}{191}  insight, an understanding about the nature of the mind and the body process,
that has a transformational effect that allows us to let go of that particular
object. We no longer identify with it as me or mine. It’s seen clearly and it’s
moved on from. We shouldn’t spend any time reflecting on how to practice
when we leave this place. If you find yourself thinking about next week, or
next  month,  make  a  note  of  that  planning.  Bring  you  back  to  the  present.
Be strict with yourself. We have to be quite firm, kind of like when you are
training an animal. You need to be kind, but set the rules, set the discipline
and then you find that the dog becomes a very well trained friend. Just like
the mind will become a very well trained friend as well after we have done
our mindfulness training.

\sphinxAtStartPar
And fourthly, the development of patience, perseverance and commitment to our practice strengthens the five faculties, strengthens these meditators.  Staying  committed  to  what  we  are  doing  and  being  patient.  In  fact
the Buddha says, patience is the highest training. Practice patience – it’s a
practice! It’s not that some people are patient and others are impatient. No,
it  is  something  that  we  practice  with.  When  we  find  ourselves  becoming
impatient, find ourselves getting frustrated because we are waiting for something or someone, that’s the time to be patient. When we are being attacked
by a mental state or an emotional state, that’s the time to be patient. To ride
it through, ride through the storm. The storm is coming, you can’t go outside to escape it, you can only go inside, go inside and become patient and
watch the storm blow over without reacting to it. And it’s gone. This is very
useful for the development of the faculties. We practice with effort, we don’t
entertain any attachments. We don’t take any prisons. Everything is let go
of, everything is to be noted and known and let go of. If we find ourselves
getting a little close to something, then we got some work to do there. If we
started attaching to it, there is work to be done. Things need to be let go of.
If we are not prepared to confront our defilements, if we are not prepared to
go in the battle, with our fears, with our anxieties, then we have just to keep
attaching to them and dukkhering ourselves. But if we can just stand\sphinxhyphen{}up once
with mindfulness and awareness and look at this emotional state or this fear
or this concern, we look it right into the face, watch our problem and note
it, know it. We stand up to it, we will see it fade away. In front of wisdom.
\DUrole{pdfpage}{192}  Wisdom  is  much  tougher  than  the  defilements.  The  defilements  are  more
tricky. They have been ruling the show for a long time. They’ve had control
for many, many years. So they are well exercised and well trained. Wisdom
is a little bit weak, still in school, still learning the rules. Sometimes it wins,
sometimes it looses. As long as we are developing, that’s fine. As long as
we put forth effort, as long as we know what we’re doing, then we’re doing
the training, the practice. The Buddha says, it’s a gradual training, a gradual
practice and a gradual process. This is not the kind of practice that we do for
a week and then put it on the shelf and say it was good. Finished. – Meditation  is  something  that  we  continue  in  our  lives.  It  goes  much  further  than
sitting on the mat or walking like a zombie in the forest. It goes much further
than that, our meditation practice.

\sphinxAtStartPar
So these faculties, when we develop them, when we balance them, get
us to deep understanding of nature and life. They are the meditators. So take
care of them. They are the ones doing the work for you.

\sphinxstepscope


\chapter{Day 5, afternoon}
\label{\detokenize{5-b:day-5-afternoon}}\label{\detokenize{5-b::doc}}
\LOCALaudiolink{https://www.mixcloud.com/anthonymarkwell/day-5-morning-talk-first-noble-truth/}

\sphinxAtStartPar
This afternoon we are going to have a look at the first noble truth, have
a look at
\sphinxstyleemphasis{dukkha}. There is dukkha.


\section{First noble truth}
\label{\detokenize{5-b:first-noble-truth}}
\sphinxAtStartPar
We’ve  been  using  this  word  dukkha  this  week.  It’s  a  word  that  the
Buddha appropriated from his local language at that time. The word dukkha
means pain. There is pain. He was using it in a different way.
\sphinxstyleemphasis{Dukkha sacha},
or the truth of dukkha. The noble truth of dukkha or the noble truth of suffering as it is normally translated. Suffering is kind of an unfortunate translation because it really gives a pessimistic view of the Buddha’s teaching.
People sometimes think, life is suffering, the world is suffering, I am suffering, the Buddha is suffering – the Buddha is not suffering! The Buddha is
free from suffering.

\sphinxAtStartPar
The first noble truth, like all four truths are given to us, is in the fourth
section of the satipatthana text, the contemplation on dhammas. And we get
a definition there of dukkha in the text. The Buddha asks what is dukkha.
\sphinxstyleemphasis{«Birth,  old  age,  sickness,  death,  sorrow,  lamentation,  pain,  grief,  and
despair – these are all dukkha. Association with disliked things, separation}
\sphinxstyleemphasis{from liked things, not getting what one wants – these are all dukkha.»}

\sphinxAtStartPar
The  Buddha  then  gives  a  very  interesting  ending  to  his  definition  of
dukkha. In summary, «
\sphinxstyleemphasis{the five aggregates affected by clinging are dukkha}
\DUrole{pdfpage}{194}  ».
The  five  aggregates  is  the  Buddha’s  most  common  way  of  analyzing  the
human condition, the mind and body process. He divides us into five groups,
that are joined together, that arise together and cease together. They are very
briefly: (1) physical form, (2) feeling, (3) perception, (4) conditional formations and (5) consciousness.

\sphinxAtStartPar
A few of the translations like suffering, unsatisfactoriness, stress – they
all  bring  with  them  their  problems.  In  countries  like  Thailand  or  Burma,
this  teaching  isn’t  translated. They  just  take  the  word  into  their  language.
Dukkha  is  Thai.  In  the  Myanmar  it’s  the  same.  Very  difficult  to  translate
a concept like dukkha, the noble truth of dukkha and bring it into another
language. This  is  because  other  languages  don’t  contain  this  concept. The
teaching in the Pali text is quite a unique one. Other people didn’t come up
with it. Other people didn’t uncover it. So they don’t have the concepts and
the words to be able to express it. We have to turn to the Buddha’s own word
for him to express the nature of his teaching. Dukkha in our modern world
includes all kinds of depressions, anxieties, stresses and worries, all kinds of
disorders and illnesses.

\sphinxAtStartPar
Essentially, what the Buddha is saying, is, that dukkha is when these
five  aggregates  have  been  affected  by  clinging.  Clinging  is  an  intensified
form of craving; craving brings us to me and mine, clinging brings us to I
– an intensified form with the five aggregates having been not only infected
with craving, but they started to calcify, they started to solidify. More concrete,  these  independent  momentary  appropriations  of  things  as  being  me
and mine come to be joined together, they come to be so continuous that an
image of an I springs up. Just like in a film when it is running through the
projector. Individual frames of me and mine, me and mine, me and mine, me
and mine, me and mine, taking things as happening to me or for me, being
me and mine. Individual events that add up when we run the film through
the projector, we get a movie. And that is what we get, we get a movie of
our  life.  It’s  flowing  through  the  projector.  It’s  hard  to  see  the  individual
moments, where we’re appropriating and identifying, infecting the mind and
body process with this idea of a self. It’s not an easy thing to see, to witness.
That’s what we are here to do. This is what Vipassana is all about, to see this
\DUrole{pdfpage}{195}  happening in real time with real data. The real data being your own mind
and body process.

\sphinxAtStartPar
We find in the Pali texts certain usages of the word dukkha. When the
Buddha used the word
\sphinxstyleemphasis{dukkha vedana}, he is talking about painful feeling, as
opposed to
\sphinxstyleemphasis{sukha vedana}, pleasing feeling. These two often go together. A
third one
\sphinxstyleemphasis{adukkhama\sphinxhyphen{}asukhama}, a\sphinxhyphen{}dukkha, a\sphinxhyphen{}sukha, not painful, not pleasant feeling. We call that neutral, adukkhama\sphinxhyphen{}asukhama vedana. Three types
of vedana.

\sphinxAtStartPar
Sometimes  the  Buddha  uses  the  word
\sphinxstyleemphasis{sankhara  dukkha}.  This  is  the
dukkha  that  comes  from  formations.  This  is  the  dukkha  that  we  see  by
having a body. Just the fact that we have this conditioned body arising in the
moment, it’s the dukkha that we understand from the conditioning process
from the mind and body process it is. Just the fact that we have this body is
dukkha, inherent dukkha. It’s the dukkha we get to see from changing posture. It’s the dukkha that forces us to change, forces us to move, forces us
to cure. That’s the type of dukkha that we are curing all the time. Sankhara
dukkha.

\sphinxAtStartPar
Then  there  is
\sphinxstyleemphasis{dukkha  lakana}.  Dukkha  lakana  is  the  characteristic  of
dukkha such as when we’re looking at impermanence\sphinxhyphen{}dukkha and non\sphinxhyphen{}self.
One of the three characteristics. It’s a conditioned, dependently arisen phenomena. This is the dukkha we realize in our Vipassana practice when we are
noting and knowing the individual characteristics of the mental and physical
phenomena.  When  we’re  occupied  doing  this  continuously,  moment  after
moment, fully absorbed into doing this activity, noting and knowing, developing equanimity towards the objects without a desire to change anything or
own anything. When that desire drops away, we see dukkha. We see it, that’s
the one we see in our Vipassana.

\sphinxAtStartPar
Finally, there is
\sphinxstyleemphasis{dukkha saccha}. Dukkha saccha is the truth of suffering. This is the noble truth, inherent suffering. We can’t change the situation.
When things are appropriated and identified with, that’s dukkha saccha, the
truth of suffering. The Buddha asked us to observe it, to know it, fully. He
asks us to know it fully.

\sphinxAtStartPar
In fact that this nama\sphinxhyphen{}rupa, the mind and body process is impermanent
and vulnerable to pain, and completely is unable to provide us any security,
\DUrole{pdfpage}{196}  or  any  lasting  satisfaction.  It  is  by  its  nature  unsecure  and  impermanent.
It’s arising and passing away. How can we find any security, any long lasting safety in something which is so impermanent, something so changeable,
something so conditioned and dependently arisen, something so impersonal
that  it  is  out  of  control?  It  is  like  trying  to  get  safety  in  an  out  of  control
car shuttling down the highway. Find some safety and security in that! Of
course, there isn’t! And if we do try to hold on to it, dukkha! Dukkha is there
to greet us when we start to own the body.

\sphinxAtStartPar
The Buddha likes to explain and analyze the mind and body process.
And  he  does  so  using  two  major  categories. There  are  other  categories  as
well  but  the  two  major  ones:  the  six  sense  bases  and  the  five  aggregates.
The five aggregates we just briefly touched on but the six sense bases we’ve
been talking about this week, the bases of the eye, ear, nose, tongue, body
and  mind.
\sphinxstyleemphasis{Suffering  arises  through  these  six  bases.}
This  is  where  dukkha
arises in the present moment and this is where dukkha ceases in the present  moment.  If  objects  have  been  appropriated  at  the  six  sense  doors  and
infected with the virus of me and mine, then dukkha is arising at those six
sense  doors.  The  Buddha  said,  the  eye  and  visible  forms  are  dukkha,  the
ear and sounds are dukkha, the nose and smells are dukkha, the tongue and
tastes are dukkha, if they are appropriated and identified with. As they are
themselves they are fine, they are neutral. That’s rupa, neither good nor bad,
it’s just a body, it’s just a manifestation of old karmic intentions arising in
the present moment. It’s physicality. It’s only dukkha if we own it. There is
my physicality. As soon as we take ownership over it, we have dukkhered
it. We are living in dukkha when we are owning the body and mind process.
The Buddha’s first noble truth points this out to us. Have a look, here,
in this body. There is dukkha, appropriated mental and physical phenomena.
He often teaches uses the phrases
\sphinxstyleemphasis{«the six sense bases are imperma\sphinxhyphen{}}
\sphinxstyleemphasis{nent».}
What is impermanent that is dukkha. What is dukkha that is non\sphinxhyphen{}self.
How can you claim ownership, how can you claim to be something which
is impermanent and dependently arisen. When we see it with our meditation
wisdom, we see that the body and the mind is extremely impermanent. It is
laboring under a conditionality process. It is fixed in its dependency. How
can we find any happiness in this thing? And the Buddha is not saying that
\DUrole{pdfpage}{197}  life is all terrible. He is saying, when you hold on to it, when you attach and
cling to the mind and body process, then this thing called dukkha arises.

\sphinxAtStartPar
Luckily for us, he gave us the second noble truth as well. He told us,
that dukkha is arising from craving.

\sphinxAtStartPar
And then the third noble truth: this dukkha ceases, this dukkha ends.
It’s possible to put out the flame of dukkha. We can move beyond this first
noble truth. We can realize the third noble truth. There is an end from dukkha,
there is freedom from dukkha. The Buddha called that nirvana.

\sphinxAtStartPar
And the fourth noble truth, that’s the causes and conditions for the arising of cessation, for the arising of the end of dukkha.

\sphinxAtStartPar
The  Buddha  wasn’t  particularly  concerned  with  full  philosophical
completeness.  He  was  much  more  interested  in  a  practical  and  pragmatic
way of practicing. His teaching constantly directs us to the truth. Rarely is it,
that he comes out with a statement exclaiming the truth. He much prefers to
just point out the path that leads to the truth. He wants us to walk that path.
He wants us to understand for ourselves, having realized the truth for ourselves. He was a very skillful teacher in the way he delivered his teachings
for 45 years. On the age of 35, after his enlightenment, until the age of 80,
he wandered the northern area of India teaching. He says,
\sphinxstyleemphasis{«now as before, I}
\sphinxstyleemphasis{teach dukkha and the end of dukkha»}. That’s it. 45 years wandering around
teaching about dukkha. Beings understanding those teachings and becoming
enlightened themselves. If there is something that we truly need to understand in this life, it is dukkha! He tells us, that this dukkha needs to be fully
understood in all ways. He tells us that consciousness is arising at these six
sense bases.
\sphinxstyleemphasis{He calls these six sense bases the ALL. This is all there is.}
There
is nothing outside of this. This six sense basis presents the world to us. We
call it «ALL».
\sphinxstyleemphasis{«This is the ALL, monks.»}
And he says, the ALL is subject to
birth, sickness and death. When the six sense bases are appropriated – there
is  nothing  wrong  with  the  six  sense  bases  itself,  they  are  actually  fine,  if
we leave them alone – dukkha is produced when we appropriate, when we
identify, when we take ownership of, when we start to build our identity or
our personality based upon this basis, based upon these objects. The mind
has been doing this for a very long time, efficiently and effectively. So completely that there are rarely any gaps. Very rarely does anyone ever have a
\DUrole{pdfpage}{198}  gap. Very rarely does anyone get to experience the world without a self being
in  the  way.  Some  people  have  spontaneous  realizations.  Sometimes  great
traumas can trigger this in people. In states of depressions, or great physical
traumas where people get injured they can break through. Sometimes people
have  a  spontaneous  experience  when  they  are  in  nature,  when  they  are  in
the jungle or in the dessert or when they are alone. An experience without a
self in it. When the world is not divided, when it hasn’t slipped into duality,
before it goes to duality, there is a oneness.
\sphinxstyleemphasis{Before the sense of self comes}
\sphinxstyleemphasis{into to divide «now» into «my» and «that» there is a singleness that’s there.}
This is the cessation the Buddha is talking about. Some people have random
and spontaneous experiences of it. Not the full enlightenment, not the full
pulling out of the defilements but just a quick glimpse. Some psychedelic
substances give us a weird taste of this experience. Unsustaining though, it
slips away. The Buddha’s teaching is a systematic method and technique that
brings us into the present moment without a sense of self. We can see things
as they really are, freeing our consciousness from this infection.


\section{Craving, conceit and views}
\label{\detokenize{5-b:craving-conceit-and-views}}
\sphinxAtStartPar
The Buddha calls the six sense bases the «world». So this is the world,
the whole world. The whole world is empty of self and that which belongs
to self. He calls the six internal bases old karma. The manifestation of these
six internal bases is the manifestation of old karma. He also says that these
are the six bases of conceivings. Once we are in, once a sense door has been
activated, if consciousness has arisen at that door, feeling has arisen with it,
pleasant or unpleasant – internal object, external object, contacting so that
consciousness arises in the present moment and knows that experience in the
present moment – if this has gone unnoted, if awareness hasn’t been sharp
and  particular  at  this  exact  point  of  contact,  than  that  sense  experience  is
infected by me or mine. It’s infected by craving, conceits and views.

\sphinxAtStartPar
Craving is what takes this experience as being mine. Conceit takes the
experience as «this I am». And wrong views take the experiences as «this
is my self». There are three different types of conceiving, all conceiving in
a  subjectified  way. All  of  different  intensities.  Craving,  conceit  and  views
arise at the six sense doors.

\sphinxAtStartPar
\DUrole{pdfpage}{199}  Our Vipassana meditation practice is the systematic observation of the
six sense doors. Noting, knowing and letting go what’s going on. We begin
our practice at the body door, watching the rise and fall of the abdomen or
the breath sensation as it passes in and out through the nostril, paying attention to our posture as well. It’s a difficult door. Then we bring the other doors
in, the seeing, the hearing, the smelling and the tasting. We are aware when
these doors are activated. We’re noting every moment we can, to stop the
infection. And then the mind door! In its wanderings, in its versions of the
truth, in its memory, grand plans and dreams. These are also being infected.
These are also being used as base for the arising of self.

\sphinxAtStartPar
So, our practice leads to an abandoning of craving, conceit and views,
leads us to an abandoning of conceiving of what the Buddha called
\sphinxstyleemphasis{papanca},
conceptualization.


\section{The six sense bases}
\label{\detokenize{5-b:the-six-sense-bases}}
\sphinxAtStartPar
The six sense bases are our access to the world. This is how the body
interacts. We are calling it
\sphinxstyleemphasis{the six doors}
sometimes,
\sphinxstyleemphasis{the six sense bases}
sometimes. The real definition is the
\sphinxstyleemphasis{chah ayatana}, the
\sphinxstyleemphasis{six sense spheres}. These
doorways are actually spheres of experience. The ear door for example: the
sphere of the ear is only sound. The ear cannot see, it cannot smell, it only
cognizes sound. The eye door only cognizes visible forms. It doesn’t taste
or smell either. The nose door has its sphere of experience. The tongue has
its sphere of experience. The body door its own. The mind door experiences
all  five  physical  doors  and  its  own  door,  its  own  sphere.  For  example  we
can listen to some music while we are sitting here in the hall. It’s the song
in our head. That’s the mind door open – but we are listening probably to
some  awful  sound  that  you  can’t  get  rid  of,  repeating  and  looping  around
and around.

\sphinxAtStartPar
So  these  spheres,  the  six  sense  bases  are  like  little  radars,  whenever
they are activated. And these six sense spheres, their information is collected
and compiled together. The feeling from all of them is assessed and analyzed
very rapidly by the mind. We very rapidly build a picture or conceptualization of what’s going on, our perception or our recognition. We draw on our
resources  of  previous  recognitions  and  perceptions  to  paint  a  picture.  We
\DUrole{pdfpage}{200}  very  rapidly  paint  a  conceptualization  of  what  the  experience  is,  what  the
event is. Of course, built into that experience is the fact that the experience
is happening to me. That’s also another layer of paint that we put on. The
subjective layer of paint. This is happening to me. This is something for me,
this is mine. And it’s through doing this continuously over and over and over
again, that a picture is painted. A very strong identity view arises. It doesn’t
want to be told that it’s nothing. If I tried to tell you, that you are nothing, it’s
not happening to you – «of course it’s happening to me», people say. They
are very unwilling to let go of their sense of self. We are often quite scared!
If there is no me, what will I do? Well, you are free, there is nothing to do
for you. You are lying in the hammock. You are done. Done is what had to
be done. There is no more coming to any state of being. That’s what our goal
as humans is. We meditate ourselves out of existence so that the mind finds
no longer a base upon which it can rest and call «me» and «mine» a home.
Consciousness  unfixed,  consciousness  unsettled,  consciousness  released.
It’s the mind unentangled with the world. The world of mental and physical
phenomena. When consciousness separates, steps away from that, then it is
released. It’s free.

\sphinxAtStartPar
In the Samyutta Nikaya we have a wonderful sutta, called the Malunkyaputta  sutta.  Malunkya  was  a  guy  that  came  to  visit  the  Buddha  several
times. His conversations have been recorded with the Buddha. He came on
many occasions. We can trace his development actually. His first few questions he was quite unsure of what he was talking about. The Buddha set him
straight, gave him a few instructions. He came back, asking deeper and more
profound questions. Finally he came to the Buddha and said, look, just give
me the short version. I don’t have much time. I’m an old man. I just want
to know how to do it really quickly. I just want enlightenment, kind of now,
don’t we all? And so the Buddha gave him this teaching:
\sphinxstyleemphasis{«Well, Malunkyaputta, regarding things seen, regarding things heard and sensed and things
cognized by you, in the seen there should be just the seen, in the heard there
should be just the heard, in the sensed there should be just the sensed and in
the cognized there should be just the cognized. When, Malunkyaputta, in the
seen there is just the seen, in the heard there is just the heard, in the sensed
there is just the sensed and in the cognized there is just the cognized – nothing \DUrole{pdfpage}{201}  more added – then you will not be ‚by that‘. You will not be ‚therein‘ and
you will not be ‚anywhere between the two‘».}

\sphinxAtStartPar
When you are not ‘therein’ and not ‘by that’, then you will be neither
here nor there nor beyond the two. Just this is the end of suffering. So when
we  are  no  longer  taking  things  –  things  seen,  things  heard,  things  sensed
(sensed is the nose door, tongue door and body door), things cognized is the
mind door, when you are no longer building your sense of self so that you no
longer be by that or be therein, then we will no longer be anywhere. Being
comes to an end, being ceases. We will no longer be between the two. This
is the end of suffering. Malunkyaputta listened to that carefully and became
fully enlightened. Maybe it will happen to you, I hope so (audience laughing).


\section{The five aggregates}
\label{\detokenize{5-b:the-five-aggregates}}
\sphinxAtStartPar
The  Buddha’s  main  way  of  analyzing  the  mind  and  body  process  is
the  five  aggregates.  It’s  given  to  us  in  the  definition  of  dukkha,  the  first
noble  truth.  In  short,  the  five  aggregates  affected  by  clinging  are  dukkha.
Aggregate means something joined together, groups compounded together,
things  held  together.  This  looks  at  one  thing  but  it’s  actually  five  things.
Actually five individual things joined together, arising together and ceasing
together.  The  Buddha’s  primary  scheme  for  analyzing  sentient  existence.
It’s the ultimate referent to the first noble truth. The purpose of analyzing
the body in this way is to let go of the mind and body process. The Buddha
wasn’t analyzing the body and mind process into these five groups, so that
we can come upon some scientific discoveries. He is not trying to analyze
to the amino\sphinxhyphen{}acids and protons and neutrons and electrons. He is not trying
to analyze every single type of nerve or brain pulsation that we may have.
He is giving us a scheme with which we can practice in the present moment.
He says, the five aggregates need to be fully inspected and fully understood.
We need to know the five aggregates in the past, the present and the future.
We need to know them internally and externally. The gross manifestations
and the subtle manifestations. The inferior and superior manifestations. The
far and near. The purpose of this investigation is liberation, not kind of analytical scientific study where we pigeonhole everything, put the information
\DUrole{pdfpage}{202}  over there. That’s the scientific method. Take something, cut it apart, analyze
it, split it up, analyze it and try to understand it in that way. The Buddha is
doing that on the internal bases with the mind, not with the eyes. You don’t
divide the body and mind up with your eyes. How can you divide your mind
with  the  eyes? You  can’t  even  see  the  mind  with  your  eyes.  The  mind  is
immaterial. It is not part of the body. It’s an immaterial experience. We need
the mind to be able to understand the mind, to understand feeling, perception, formations and consciousness. In fact, we need awareness and wisdom
for us to fully understand it.

\sphinxAtStartPar
The  Buddha  tells  us  that  these  five  aggregates  tend  to  affliction  and
cannot be made to conform to our desires. They are out of control, they are
impersonal! He tells us that attachment to them leads to stress, sorrow and
depression  and  that  their  change  leads  to  fear,  anxiety  and  distress. When
we are attaching to it we get depressed, and when we are worried about it,
we develop anxiety. Normally we are attached and concerned about the past.
When we are obsessed with the past, it leads us to a sadness and to a depression. When  we  are  obsessed  with  the  future,  it  leads  us  to  fear  and  worry
and  concern  and  anxiety.  So  we  know,  where  depression  and  anxiety  are
coming from – spending too much time in the past or too much time in the
future. Of course, the past and future are just thoughts arising in the present
moment. Just a thought pattern that we started looping, started to persistently
think about.
\sphinxstyleemphasis{«Whatever a bikkhu frequently thinks and ponders upon, that}
\sphinxstyleemphasis{will become the inclination of his mind.»}
If we are frequently thinking and
pondering upon a sad event, then guess what happens, we become sad! If we
are frequently thinking and pondering and worrying upon the future, guess
what happens, we become anxious. We condition ourselves into that state by
constantly allowing the mind to loop in a particular thought pattern. We need
to step out of that looping system otherwise we cause ourselves an enormous
amount of problems. Our Vipassana technique is to dissolve this looping in
the  present  moment. That  we  can  move  beyond  habitual  thought  patterns.
We’ll move beyond habitual emotional states.

\sphinxAtStartPar
Sometimes  we  have  an  emotional  state  that  has  been  arising  for  so
long, we don’t know any escape from it. It’s been going on for so long we
don’t  know  any  difference.  People  get  addicted  to  things  for  20,  30,  40,
\DUrole{pdfpage}{203}  50 years. Their experience is, that when that object comes into their mind,
wanting, desiring and they go into it. There is a strong connection there. As
soon as the object becomes an object of consciousness, they are straightforward identifying and wanting it. That is just a pattern they picked up. It’s
an addiction. It’s a pattern way of thinking. Our awareness meditation can
break through this. It won’t happen immediately but if you continue noting,
knowing and letting go every time that these impulses arise, you start to see
them as they are. You witness them and you won’t be fooled by them. You
will have strength over them so that we can let them go and move beyond
our attachment and identification with things.

\sphinxAtStartPar
Because  the  five  aggregates  are  impermanent,  they  never  fulfill  our
hopes of permanence and security, while they do give us some pleasure –
that’s for sure! The Buddha says, there is happiness! There is joy going on, it
does arise. He doesn’t say that the world is a barren place of evil and nothing
is happy. That’s not what he is saying. He’s just saying there is dukkha here.
Look out for it. He says whilst the five aggregates do give us pleasure and
joy, they must change, they must alter, they must pass away. And if you are
attached and holding to something that’s passed away – dukkha! If you are
attached to your wallet and loose it – dukkha! Attachment gives the problem.
It’s because of attachment that dukkha arises.

\sphinxAtStartPar
The  story  of  a  guy  who  wakes  up  in  the  morning  after  a  big  storm,
he looks out to the window, to the left, sees a big tree that has fallen on his
neighbor’s car, smashed it completely. «Oh, that’s unfortunate. Dave’s care
has  been  smashed.  Oh  well,  I  hope  he’s  got  insurance.» Then  he  looks  to
the  right  and  sees  another  tree  has  smashed  his  car.  He  completely  freaks
out. «Oh, my car!» Why? The events are exactly the same but the level of
attachment is different. He’s attached to his own car. He gets a lot of suffering from that. «Just lost 20 grant.» Insurance had expired, oops. We can see
it’s attachment that causes us problems. When it’s not my car, «oh, well, I
hope it’s all right». When it’s my car it’s an enormous amount of dukkha.
Problems  and  issues  come  up.  Problems  and  issues  arise  by  taking  things
seriously, by taking them as «me» and «mine». When we do this, our only
companion is dukkha. When you start to release this identification process,
the world opens up, it becomes spacious. How wonderful it is to be homeless. \DUrole{pdfpage}{204}  Not have to worry about paying a rent or a mortgage, having to do the
garden or clean the house. All those things, all those problems just disappear
when we are not attached to our dwelling, when we can let them go. When
we let go, we know how it is, when we have a big clean up at home. You go
through and throw away 25 black bags of garbage that you collected. How
freeing that can be! Finally, all those things that we have been holding on to
can be let go of. It’s a clutter. To clear out the clutter of our lives.

\sphinxAtStartPar
It’s  attachment  that  causes  the  problem.  It’s  not  the  object.  Pleasure
does arise in this world. The Buddha says, that we need to understand the
satisfaction, and the danger, and the escape of these five aggregates. We have
to understand, yes they do bring satisfaction. Of course, they are bringing
satisfaction. We wouldn’t identify with them if it wasn’t satisfying. If it was
all total misery, then we wouldn’t be interested in it at all. But because, there
is some pleasantness that’s there, we get sucked into it.

\sphinxAtStartPar
Habitually  we  assume  control  over  the  five  aggregates. We  think  we
are in control but in reality we are devoured by them moment after moment.
These are the domain of defilements. This is where the nasty mental states
arise. Appropriating and identifying.


\section{Identity view}
\label{\detokenize{5-b:identity-view}}
\sphinxAtStartPar
When we do this, when we take the five aggregates to be a self, there
arises what is known as
\sphinxstyleemphasis{sakaya ditti}.
\sphinxstyleemphasis{Sakaya}
means identity or personality.
\sphinxstyleemphasis{Ditti}
means a view. Our five aggregates become infected with sakaya ditti, or
identity view, or personality view, distorted view of personality. We say that
the five aggregates have come to view. They have attained to view. The five
aggregates have now become an identity instead of being an impersonal flow
of mental and physical phenomena, just cruising along according to karma,
according  to  cause  and  effect,  arising  and  passing.  When  we  don’t  allow
awareness  to  note  and  know  the  present,  when  we  are  out  of  the  present,
when we are unaware, when ignorance is arising, then the five aggregates
are identified with. They come under identity view. And this identity view
happens in four ways over the five aggregates. So there is 20 different types
of identity view.
\begin{itemize}
\item {} 
\sphinxAtStartPar
When we regard the body as a self, when we identify a particular aggregate,
\DUrole{pdfpage}{205} one of the aggregates as a self. If we regard the body as a self,
or a feeling as self, or a thought or consciousness as self, this is one
way of appropriation. Taking things as a self.

\item {} 
\sphinxAtStartPar
The second way is taking the experience as happening to me. It’s possessing the experience. We possess the body, the feeling, the perception, the formation or the consciousness. We become the owner. So we
start to identify in that way. Self possesses one of the aggregates.

\item {} 
\sphinxAtStartPar
The third way in which identity view arises is when we start to think
we contain the aggregate. That the aggregate, the body, the feeling or a
thought is somehow in self. It’s in the self. It’s part of the self.

\item {} 
\sphinxAtStartPar
And the fourth way is when we take the self as being in the aggregate.
We say the self is in the body, or in the feeling or in the consciousness.

\end{itemize}

\sphinxAtStartPar
So in these four ways we may regard material form as self, or material
form  as  possessed  of  self,  or  material  form  as  in  self,  or  self  as  in  material  form. And  so  on  with  the  other  aggregates.  We  regard  consciousness
as self, or we regard consciousness as possessed of self, or we regard self
as in consciousness, or we regard consciousness in self. So in these ways,
the  mind  and  body  process  come  to  be.  «Being»  is  established,  from  lots
of «me»\sphinxhyphen{} and «mine»\sphinxhyphen{}experiences, «I» is established, with the sense if «I»,
being has come to be. The mind and body process have become infected by
being. Becoming. There is an identity formed in there. They believe they are
someone. Strongly believe. And when we do this, we start to attend to our
experiences unwisely. We don’t see them as they really are. We start to have
questions when the sense of I arises. Maybe we get a little freaked out, we
start to ask these questions: «Was I in the past? Was I not in the past? How
was I in the past? Where was I in the past? What was I in the past? – I am
now». So it starts to reflect what I was before. It starts to reflect unwisely. Or
it starts to reflect into the future. Identity view starts to be projected forward.
«Shall I be in the future? What shall I be in the future? Shall I not be in the
future? How shall I be in the future? Where will I be in the future?» So once
the mind and body starts to believe it is someone, it starts to ask these kinds
of questions. And this gives just more fuel to the sense of I! It believes it has
some kind of past and it has some kind of future. And so it starts to map out
its live. It starts to dream up its personality, it starts to become somebody.
\DUrole{pdfpage}{206}  That has happened to all of us! It’s been happening for a long time.

\sphinxAtStartPar
Fortunately the Buddha taught us the way that we can see this process
in action and that we can let it go, remove ourselves from it.

\sphinxAtStartPar
If it’s not in the past and not in the future, it can just be inwardly confused in the present moment. When it doesn’t see things as they are, it starts
to think: «Am I? What am I? Who am I? Where have I come from? Where
am I going? Am I a being?» It starts to think in all these different ways. It
starts to attend unwisely to the present moment in this way. It’s subjectified
to the present moment. A duality has been created. There is a dichotomy in
the world, separation into two halves: the world of me and out there.

\sphinxAtStartPar
When  this  happens  all  kind  of  views  start  to  arise.  Stronger  views.
When  he  attends  unwisely,  these  views  arise  in  him.  «Oh,  it’s  this  self  of
mine that speaks and feels. This self of mine experiences here and now the
results of good and bad actions. This self of mine is really permanent, really
everlasting and not subject to change and it will endure as long as eternity.»
The identity view has become so solid, that the being starts to think of itself
in solid terms. «I mean, I am permanent. All this mind and matter, that stuff
is passing away but the experiencer of it that’s the real self.» That’s the self
with  the  big  «S». That’s  my  true  self,  that’s  what’s  beyond  or  underneath
all the swirling flow of mental and physical phenomena. The view of a big
self arises. A really concrete and solid form. A full blown delusion. In that it
starts to believe that it’s attached and identified with the five aggregates but
then it starts to believe that it is something even beyond the five aggregates.
It has really been infected by a sense of self.

\sphinxAtStartPar
The Buddha says when this happens, when he attends unwisely, these
views arise. He says these views are a speculative view. It’s called the thicket
of views, a wilderness of views, a contortion of views, a facilitation of views,
a  fetter  of  views.
\sphinxstyleemphasis{«Fettered  by  this  fetter  of  views,  the  untaught  ordinary
person is not freed from birth, aging and death, sorrow, lamentation, pain,}
\sphinxstyleemphasis{grief and despair. They are not freed from dukkha»}, I say.

\sphinxAtStartPar
So identity view is the leash that keeps us bound to existence. It keeps
us bound to this circle of being, of craving to be, wanting to be – of suffering! It’s this identity view, this constant need for recognition, this constant
need that it has for being «me». We spend so much time doing this in our
\DUrole{pdfpage}{207}  lives. We create an identity trying to develop an ego. Partly the result of our
family, partly the result of our community and our society structure. Everyone  is  craving  to  be  something.  We  are  taught  it  from  a  very  young  age.
«What are you going to be when you are going to grow up?» In our education system they want us to become something. It’s geared to get trained a
particular thing for a particular task, with a label, with a hat on, with a name
tag. «I’m a doctor, I’m a lawyer, I’m a plumber, carpenter, nurse, teacher,
truck driver, psychiatrist.» We want to be something. When we don’t want,
when we don’t get what we want to be – dukkha! Even worse, if we don’t
know what we want to be, that’s even more confusing. Everyone’s telling us
that we have to be something, that we have to get something, that we have
to become something.
\sphinxstyleemphasis{The nature of reality is that we are nothing, and yet}
\sphinxstyleemphasis{everyone is saying that we have to be something.}
There’s a huge gap between
those two things. There’s a cognitive dissonance. Our expectations and our
reality  are  two  different  things.  It’s  no  wonder  that  we  go  crazy  trying  to
think what we have to become. It’s no wonder that we have spent time in
institutions of learning to try to become something. We are not really interested in counting other people’s money, but I want to be an accountant. It’s
the being, we strive and try to become. This the cause of dukkha.

\sphinxAtStartPar
So  identity  view  is  removed  when  we  reach  the  first  path. When  we
arrive  at  the  path  of  the  stream\sphinxhyphen{}enterer  or  a
\sphinxstyleemphasis{sottapanna}. The  first  level  of
enlightenment,  the  first  level  of  awakening.  Where  this  gross  self,  this
identify view is pulled out by the roots and discarded. Wisdom completely
removes it. The sense of self will no longer arise. It is impossible for it to
arise again. Doubt also disappears or that the believe in rights and rituals can
lead to awakening. This wrong view is also removed. Any doubt about what
is the correct way to practice, the noble eightfold path, that’s also done away
with. So identity view, believe in rights and rituals being efficacious leading
to the desired result is thrown away, and also doubt is let go of. These three
fetters are removed at the first path of awakening.

\sphinxAtStartPar
The path of stream entry is reachable in this very life. This is our goal
as  humans  to  enter  the  stream  that  flows  to  nirvana.  If  we  can  go  further
than  that,  even  better.  The  heart  of  our  Vipassana  practice  is  to  note  our
five  aggregates,  to  see  them  clearly  as  impermanent,  dukkha,  dependently
\DUrole{pdfpage}{208}  arisen, conditioned, out of control, impersonal processes. Dukkha is finally
removed by contemplating the rise and the fall of the aggregates.

\sphinxAtStartPar
\sphinxstyleemphasis{«Venerable sir, how does one know and how does one see, so that in
regard to this body with its consciousness and all the external conceptions,
there  is  no  I\sphinxhyphen{}making  and  no  mine\sphinxhyphen{}making  and  no  underlying  tendency  to
conceit?» – «Well, bikkhu, any kind of material form, any kind of feeling, any
kind of perception, any kind of mental formation, any kind of consciousness
that arises, whether it be past, present or future, whether it be internal or
external, whether it’s gross or subtle, superior or inferior, near or far, one
sees all these five aggregates as they actually are with proper wisdom thus,
this is not mine, this is not I, this is not myself. When he knows and sees thus,
he becomes disenchanted with the five aggregates. Being disenchanted, he
becomes dispassionate. And through dispassion the mind is liberated. When
it  is  liberated,  there  comes  the  knowledge  it  is  liberated.  He  understands
birth is destroyed, the holy life has been lived, what had to be done, has been}
\sphinxstyleemphasis{done. There is no more coming to any state of being.»}

\sphinxAtStartPar
And this is how we remove dukkha, this is how we contemplate to pull
ourselves out of this conditioned existence and transition into the unconditioned.

\sphinxstepscope


\chapter{Day 6, morning}
\label{\detokenize{6-a:day-6-morning}}\label{\detokenize{6-a::doc}}
\LOCALaudiolink{https://www.mixcloud.com/anthonymarkwell/day-5-morning-talk-enlightenment-factors-rapture-tranquility-concentration/}

\sphinxAtStartPar
This morning we are going to continue to look at the bojjhangas, the
enlightenment factors.


\section{Enlightenment factors, bojjhangas}
\label{\detokenize{6-a:enlightenment-factors-bojjhangas}}
\sphinxAtStartPar
We’ve had a look at three of the enlightenment factors already. Energy,
mindfulness  and  investigation.  Those  three  states  spinning  on  each  other,
doing the work. Activating awareness in the present, seeing things clearly,
letting them go – this is the work of the first three enlightenment factors. As
we develop these enlightenment factors, the other four come to be fulfilled
or get developed. We don’t become enlightened by reading or studying the
scriptures or by thinking or by wishing that enlightenment just burst one’s
mind. There need to be certain conditions in place for this to happen. And
those conditions are the seven enlightenment factors.

\sphinxAtStartPar
These seven bojjhangas can be seen in all stages of our Vipassana practice. They don’t fully develop and become quite clear to the mind until the
stage of the knowledge of the arising of passing away, until we can tune in
our mind into that frequency of arising and passing, seeing things just arising and passing away, the mind not clinging or holding to anything. At the
point of contact where consciousness is arising at the six bases, we are able
to activate awareness so it joins with consciousness. It’s able to know that
\DUrole{pdfpage}{210}  present moment experience.

\sphinxAtStartPar
The  development  of  the  seven  enlightenment  factors  can  be  looked
upon as a president with the ministers, like a king or queen surrounded by
the seven ministers. All those ministers having their functions and their role
to play within the cabinet, within the structure of the organization.

\sphinxAtStartPar
Consciousness is just the knowing. That’s all it does. It knows stuff. If
the mental factors that are surrounding consciousness are the hindrances, for
example, that’s what we know. That’s what our experience is. If the seven
enlightenment  factors  surround  the  consciousness  than  our  experience  is
something  quite  different.  So  here  we  are  developing  these  seven  enlightenment factors. We are practicing, activating our awareness, noting what’s
there,  stepping  back  from  the  phenomena  –  mental  or  physical,  pleasant
or  unpleasant  –  and  allowing  it  to  pass  away.  Of  course,  it’s  very  quickly
replaced  by  a  new  object.  Something  new  comes  to  the  attention  of  consciousness. We are not going into that story. We are not sinking into it, we
are not appropriating it, we are not holding it. We are letting it go. We are
knowing them, knowing them clearly.

\sphinxAtStartPar
These  enlightenment  factors  allow  consciousness  to  know  in  a  very
particular way. In fact, they surround consciousness to elevate it onto a very
particular type of knowing, a path knowing, knowing in the manner of the
noble eightfold path.

\sphinxAtStartPar
We are practicing the four foundations of mindfulness. It doesn’t mean
studying or thinking or listening to talks about the practice. It doesn’t mean
discussing them or arguing about the final points. It means sitting down and
doing the practice!

\sphinxAtStartPar
We have to directly and experientially become aware of the four bases
on which mindfulness is established and we should practice with this awareness not intermittedly, not sporadically, not with gaps but continuously, persistently and unremittingly. And this is exactly what we do in our Vipassana
meditation.  As  we  are  noting  and  knowing  and  letting  go,  the  first  three
enlightenment factors are doing their job.


\section{Fourth enlightenment factor piti}
\label{\detokenize{6-a:fourth-enlightenment-factor-piti}}
\sphinxAtStartPar
\DUrole{pdfpage}{211} The fourth enlightenment factor starts to come into our experience. It
is known as piti. It means rapture. It’s a mental state. It’s a state of mind that
arises in the meditator when the first three enlightenment factors are up and
running, when they are working, when we’ve started to note and know and
let go. This fourth enlightenment factor comes on the scene.

\sphinxAtStartPar
It has the characteristics of zest, or enthusiasm or intense interest. There
is also some delight. Some satisfaction and some happiness that comes to the
practice.  Piti’s  role  in  the  practice  is  to  make  the  practice  interesting  and
exciting and it certainly does that. Some of you will have started to experience the side effects of piti. When piti is in the mind it also causes the body
to react in certain ways. Physical sensations arise in the body because of the
mental state of piti. It’s cause and effect there.

\sphinxAtStartPar
The  function  of  this  rapture  is  to  fill  the  mind  with  lightness,  to  fill
it  with  energy  and  agility,  to  make  it  sharp  and  functioning,  to  make  the
meditation fun and exciting. It’s bringing a deep sense of satisfaction into
the practice as well. When this rapture occurs, the cause and uncomfortable
sensations, that were arising in the body – the painful knee, the soar hips, the
stabbing in the back – those sensations disappear and they are replaced by
something that is much softer, much gentler, smooth and light.

\sphinxAtStartPar
Piti manifests in many different ways. For some people it starts to manifest as a lightness throughout the body. You will feel very light, very buoyant in the sitting meditation and also in the walking meditation it happens.
The body will start to feel like it’s very light, to lift and elevate. Sometimes
you  might  think  you  are  floating  in  the  air.  It  might  feel  so  light,  that  the
whole  body  disappears. You  can’t  really  feel  anything.  It  becomes  supercomfortable.  Those  uncomfortable  states  of  just  a  few  days  ago,  kind  of
disappear when piti is in the mind and is having its effect on the body. Sometimes we feel, when piti is arising, that the body is pushed and pulled. It’s
being swayed. It starts to rock from front to back, or side to side, sometimes
it starts to get quite rapid. Sometimes it whirls, it starts to turn, the body is
turning. The mind is meditating but the body is starting to move around.

\sphinxAtStartPar
If this starts to happen, make a note of that. If you feel some lightness
in the body, make a note lightness. If you feel it’s being swayed or rocked,
\DUrole{pdfpage}{212}  make a note of those sensations of rocking. Stay with it, don’t get absorbed
into the story of piti. Piti can be very enticing. It can be very exciting. When
it starts to happen we kind of like it. We want it to happen again! Just note
it when it’s arising. It’s not something that we want to get attached to. It is
something that we want to develop but we don’t want to be attached. Sometimes it will make us feel as if we are off balance. You are going to tip backwards, you’re going to feel like you’re going to about fall over. You probably
won’t! Don’t worry. All kind of weird sensation that arise in the body. Sometimes there’s thrills and chills. An excitement. Sometimes the body has little
twitches, little shakes or you have a little vibration session. It starts tingling
and shaking and twisting. Sometimes it gets quite violent. It feels like some
lightening balls are coming through the body. Other times it just feels like
the body’s just plugged in, a kind of buzz going on. We can think it’s quite
nice when it starts to happen. There’s flashes like lightening or sparks like
electricity.  Sometimes  you  feel  so  light,  that  you  feel  like  you’re  floating
away. Am I still on the ground? It might feel like you are flying. But you do
feel fantastically comfortable and there is no real wish to change position or
move. When I ring the bell, «I’m not moving from this (audience laughing).
This is nice!» «But that nice curry is there for lunch.» – «I don’t care. I’m
going to meditate through this.» Very comfortable this state of piti.

\sphinxAtStartPar
The important thing about piti, it gives us great interest in the practice.
We start to see some results about our hard work. When it starts to happen,
we feel validated. «Ok, it starts to happen.» Confidence arises, we put in an
extra five or ten percent effort. And we draw down a lot of extra on that little
bit of effort.

\sphinxAtStartPar
The Buddha told us, there is only one cause for rapture: wise attention.
The arising of rapture is due to bringing in wholesome, rapturous feelings.
When we are developing rapture, we can do so in a number of ways. Rapture arises when we do our Vipassana practice. Rapture arises when we’re
contemplating the Buddha, the dhamma and sangha. When we start to think
about  the  glory  of  the  Buddha,  how  deep  and  wonderful  his  teaching  is,
when we start to think about all those monks and nuns of hundreds generations that did the practice and passed the teaching down to our current generation, we start to have great respect for what was going on. The teaching
\DUrole{pdfpage}{213}  that we have been exposed to. This can also bring feelings of rapture. I have
experienced that myself, when going to see an old teacher. Sometimes the
rapture can be so strong that we start crying. Some of you may have felt it
in  the  loving  kindness  meditation. The  body  starts  to  feel  so  comfortable,
it’s just tears of joy, or tears of relief, or tears of release. Sometimes piti can
become so strong that the whole body starts to lean forward, it starts to go
down like that, you can see people’s body rocking, backwards and forwards.
They are not asleep! This is just the effect of piti on the body.

\sphinxAtStartPar
There  are  many  examples  of  people  who  have  gained  enlightenment
in the old texts after contemplating the Buddha, dhamma and sangha. This
is a wonderful way of activating piti in our lives! When we think and reflect
upon,  when  we  consider  about  the  Buddha,  dhamma  and  sangha  –  that’s
what we’ve been doing in our evening chanting.

\sphinxAtStartPar
So this piti, when it starts to arise, make sure you note it. Make sure
you  note  the  manifestations  of  it,  the  physical  sensations.  Make  sure  you
note the pleasantness of the experience. Make sure you note the actual mind
state  itself. Try  to  feel  the  enthusiasm  and  the  buzz,  the  electricity  of  piti
that’s going through the mind. It will be very clear, the actual sensations and
a lot of joy happening. Be careful not to attach to this state. Be careful not to
get addicted to this mind state. Sometimes, it can happen on the first couple
of days of a retreat, people get very excited and then they spend the next five
days trying to get it back. Trying hard! Meditating with craving! Meditating
with wanting! So at this stage of the retreat you should all be letting go your
resistances, and you should be letting go of the past and the future coming
into  the  present. When  these  types  of  mind  state  arise,  just  note  them.  Be
aware that they are there. We are not clinging or holding to them. We don’t
have to get super excited. Although you can be pleased, knowing that your
practice is developing along the path, that has been very well mapped out.
Piti is the fourth enlightenment factor that starts to arise.


\section{Fifth enlightenment factor passaddhi}
\label{\detokenize{6-a:fifth-enlightenment-factor-passaddhi}}
\sphinxAtStartPar
Leading  on  from  this  stage  of  piti,  is  the  fifth  enlightenment  factor,
that is known as
\sphinxstyleemphasis{passaddhi}
which means tranquility. It means coolness of
the mind, or calmness or stillness. When things become very tranquil, they
\DUrole{pdfpage}{214}  become  extremely  quiet,  extremely  still  like  the  water  on  the  surface  of  a
lake with no wind thereout. Completely still and silent. The buzz, the electricity of piti starts to give way to this calmness and coolness in the mind. We
become very, very still. Passaddhi arises when mental agitation or restlessness and worry have been put aside, that fourth hindrance. When we stop
worrying about things, when we’ve stopped wanting and desiring something
to happen, then tranquility starts to be activated. The body must be silent and
still and the mind becomes silent and still as well.

\sphinxAtStartPar
The  characteristic  of  tranquility  is  to  calm  the  body  and  to  silence
and tranquilize agitation. As our meditation develops, energy activates our
awareness,  turns  it  on,  mindfulness  pays  attention  to  the  present  moment,
penetrating  that  object  that  is  there,  investigation  thoroughly  understands
what this object is, that causes the letting go to take place, there is a great
deal of excitement and buzz when this starts to occur, when we start to free
the mind from sensuality and start to free the mind from the five hindrances,
freeing the mind from past and future, so we are internalized with the mind
and  body  states  and  we’re  not  fussed  about  them,  we’re  not  disturbed  by
them,  we  are  quite  intent  on  just  watching,  the  body  feels  quite  comfortable – then as this starts to happen, the electric buzz of piti starts to decline
a little bit and tranquility takes its place. They are both still there! They are
both still functioning but tranquility becomes stronger and so things start to
become very quiet, very still. Your body won’t need to move at all. You’ll be
completely silent and completely comfortable. That four elements, that are
normally changing in their ratios, normally giving some physical sensations
in  the  body,  will  just  be  completely  equalized.  There  won’t  be  any  tough
spots in the body, nothing uncomfortable. It will be perfectly calm. And a
wonderful state of mind, which will allow us to sit for long periods of time.
Two hours, three hours, comfortably, silent and still, observing, watching,
noting, knowing and letting go.

\sphinxAtStartPar
The function of tranquility is to take the heat out of the mind, to suppress  the  heat  of  the  defilements  and  the  heat  of  sensual  objects  that  normally attract us, so the mind can be released from remorse and worry and
restlessness. When the mind is assaulted by these kinds of harmful states, it
becomes hot and we become restless. When we become worried – «only two
\DUrole{pdfpage}{215}  days left of the retreat and still I’m not enlightened» – you start to worry,
to think about that. When we give that up, when we resign ourselves to the
fact that we are here and we are doing the practice, then tranquility can start
and  come  to  play.  Tranquility  of  mind  extinguishes  the  heat  and  replaces
it with a calm coolness. Tranquility is probably the state you recognize as
being meditation. When you think of meditation, sitting on top of a hill on
a tropical island, it’s going to be still and quiet and beautiful. It’s probably
what you are thinking about this state of tranquility. Do a little bit more work
and it’s yours to experience. That is what meditation is like. It’s a little bit
more effort and these states start to expose themselves. Often it’s not just the
effort, but it’s just time – time for aligning all the conditions. We need put
all the conditions in place and then these mental states start to arise. Without
them being in place, it will never arise. We’ll just have thinking or exploring
mentally ourselves, what we imagine those states of mind to be.

\sphinxAtStartPar
The manifestation of passaddhi is non\sphinxhyphen{}agitation of the mind and body.
Complete stillness, great calm, great clarity, great tranquility. Our mind is
normally  in  a  state  of  agitation,  jumping  between  the  past  and  the  future,
going here and there. There isn’t any time for calmness or coolness. There’s
no silence or peace. There are always thoughts coming into the mind, we’re
always distracted by external objects, it gets very busy in the head! We normally  don’t  notice  it,  how  busy  things  are  until  we  come  to  a  place  like
this  and  try  to  watch  it.  Often  it  can  be  quite  horrifying  to  see  the  nature
of our own mind, to see how it continuously thinks about itself. To watch
that.  – When  our  mind  becomes  scattered,  it  starts  going  into  the  objects,
into  the  mental  and  physical  phenomena.  When  it  starts  to  identify  with
them in some way or enjoy them or delighting them, gain a sense of being
from them, then we can perform unskillful actions or say unskillful things.
Speech and actions can lead to remorseful states of mind. We feel, «I probably shouldn’t have said that», we feel we should apologize. We should, if
we’ve been rude! When the mind gets assaulted by remorse and regret, there
is no happiness. Tranquility has the ability of removing this remorse and this
regret, this worry. When we still and calm things down, these states of mind
evaporate. And then all the ripples in the mind cease. The eye pad becomes
completely still. You can see very clearly the details of a good photograph.

\sphinxAtStartPar
\DUrole{pdfpage}{216}  So this factor of enlightenment follows on rapture, from piti. It arises
when the previous one is already in place. Thus we need to get excited and
enthusiastic  about  the  practice.  That  will  lead  us  to  keep  coming  back  to
the mat. It leads us to keep coming back to the meditation hall or even keep
coming back to the retreat, month after month. When these different kinds of
pleasantness start to arise, it can be quite addictive. We want to experience
that more often. The strongest rapture, the most pervasive rapture is associated with strong tranquility later on. As our rapture becomes stronger, our
tranquility becomes calmer and stiller. In fact, tranquility starts to overcome
the rapture. The coolness and calmness of tranquility kind of suppresses the
electricity  buzz  of  piti.  Both  of  them  are  working  together. They  are  both
doing their jobs. One is making the mind enthusiastic, one is calming things
down. They both have a function to play. They both have a role to play in
balancing and managing the work of consciousness, the work of knowing.

\sphinxAtStartPar
All  these  mental  factors  have  their  job  to  do. We  have  to  make  sure
they’re fully developed so they can perform their jobs correctly and properly
when we need them to. So that they can assist consciousness from letting go
of the mental and physical phenomena. These are the factors of enlightenment, the bojjhanga, these causative factors that lead to enlightenment.

\sphinxAtStartPar
Arousing tranquility, we need to reflect on the causes of tranquility. We
need to have wise attention. We need to take note after a good sitting. After
a good sitting, we should explore what just happened. What were the conditions that you just put in place? What was the time of the day? Had you eaten
nothing? Had you eaten little? What posture were you sitting in? How did
you begin the practice? Did you do some walking meditation or yoga before
you sat down? Had you been listening to a talk before going into meditation? Explore those events before the state of tranquility arises. And try to
replicate them, try to copy them so you get the same results. We are putting
the conditions in place for things to unfold. So there’s a few tips for you, for
arousing tranquility.

\sphinxAtStartPar
The first one is, take nutritious and sensible food. Make sure it’s the
type of food not only satisfying the necessity of eating but it’s suitable for
you. Make sure it’s actual food that’s been grown and produced and cooked
properly,  not  only  processed  food  or  food  from  factories.  Try  to  get  food
\DUrole{pdfpage}{217}  from farms and gardens. Here on Koh Phangan, we have a great supply of
good food. It should be the kind of food, we feel comfortable with. If you
like to eat garlic and onion and chilli, if that has been your normal eating
pattern for the last five or ten years, then you feel comfortable with that. It
won’t be annoying or agitating. If you are used to eating fruit, if you are used
to eating bland food or flavorless food, you have to organize that when you
are doing your meditation practice. Make sure the body is comfortable with
what kind of food it’s receiving. This will make the body calm and comfortable.

\sphinxAtStartPar
Secondly,  we  should  meditate  in  an  environment  where  the  weather
is good. We don’t want it too hot, we don’t want it too cold. We don’t want
to have to wear ski\sphinxhyphen{}jackets meditating and big boots when we are walking
around, but we also don’t want to be sweating continuously. So here on the
island we get this good weather. It should be comfortable and convenient for
you to meditate.

\sphinxAtStartPar
Thirdly, adopt a comfortable posture. To get the mind tranquil, you are
going to have to make the body tranquil. And this can take time. Don’t pressure yourself to start siting in the full lotus, thinking, I will never meditate
until I can do this posture. You’re maybe twisting yourself for quite some
time  before  you  start  meditating.  Make  sure  you  find  yourself  a  comfortable posture. We are used to sitting on a chair. That’s fine. I do recommend
stretching your hips, your legs so that you can establish a comfortable meditation posture. Get it worked out in your first year of your meditation life so
that you can become comfortable whenever you want to do a sitting.

\sphinxAtStartPar
You  need  to  maintain  a  balanced  effort.  We  shouldn’t  become  over
enthusiastic. Work super\sphinxhyphen{}hard, continuously and then stop and just become
bored and fed up with it and start again with super\sphinxhyphen{}enthusiasm. That’s stopping and starting. This is not great. We want to be continuous in our efforts.
Have a balanced effort.

\sphinxAtStartPar
We should also avoid bad tempered or rough people if we are tying to
develop tranquility. You know somebody who’s often excited or animated
or often gets angry or voices their opinion strongly, people start to mouth
at  you,  or  talk  rapidly  or  loudly  at  you.  These  are  not  the  kind  of  people
we  want  to  associate  to  when  we  want  to  develop  tranquility. We  want to
\DUrole{pdfpage}{218}  be around tranquil and calm people. This will be very beneficial for you to
develop your practice. Avoiding associating with people that are not calm,
are not balanced, are not stable.

\sphinxAtStartPar
And lastly, we should incline our mind toward tranquility. We should
be trying to incline our mind to quietness. While we are in the meditation
center,  calm  down.  Lower  your  eyes,  lower  your  head.  Come  into  a  quiet
zone by yourself. You will find that this is very beneficial for your sitting
practice.  In  particular  we  want  to  be  in  the  present  moment,  tranquil  and
calm. Holding the mind still and calm.

\sphinxAtStartPar
So this is the fifth enlightenment factor, passaddhi. Once piti and passaddhi have started to enter in your meditation scene, then the sixth enlightenment factor starts to be developed.


\section{Sixth enlightenment factor samadhi}
\label{\detokenize{6-a:sixth-enlightenment-factor-samadhi}}
\sphinxAtStartPar
The  sixth  enlightenment  factor  is  known  as
\sphinxstyleemphasis{samadhi}.  It’s  normally
translated as concentration but I prefer the translation of
\sphinxstyleemphasis{stability}. Stillness.
Samadhi is a factor of mind which lands on the object of observation
and boroughs into it. Mindfulness penetrates the object, samadhi holds the
mind there, keeps it still, stops the mind from leaving the scene. Mindfulness
will bring us to the present, hold us in the present, penetrate the object so that
we are there. When we do this continuously the mind stabilizes in this mode
of operation. We call that samadhi or stillness or calmness. It’s a quietness
of mind. It means that the mind sticks on the present object of observation. It
sinks into it and it remains there with it so that it sees and knows things very
clearly. Just like with the camera in the back of the truck, when it’s driving
along and bouncing around, it’s difficult to get a clear image. So too, when
the mind is still agitated and wandering, unstable with different thoughts and
memories coming to create ripples on the surface of the water. Samadhi is
that state of mind which will calm down the different mental factors. It also
brings them together. It has this quality of pulling things together.

\sphinxAtStartPar
It has a characteristic of non\sphinxhyphen{}dispersal. Things don’t disperse or spreadout.  Things  are  coming  together,  concentrating.  But  this  word  of  concentration  is  often  misunderstood  as  a  state  of  intense  directiveness,  intense
state of super\sphinxhyphen{}focusedness. But samadhi is a bit more than that. Our awareness \DUrole{pdfpage}{219}  can still be broad but our attention can be very still at the same time.
Samadhi has this characteristic of non\sphinxhyphen{}scattering. It’s the mental factor that
pulls the other mental factors in together. It brings the energy and mindfulness and investigation, the rapture and the tranquility, it pulls these mental
states together, closer to consciousness. So it gets a more unified experience.
That’s when insight starts to really arise. That’s when we start to see things
as they really are – Vipassana\sphinxhyphen{}knowledge.

\sphinxAtStartPar
Its  function  is  to  bring  things  together,  collecting  the  mind  together.
And that’s a translation that is sometimes used for samadhi – collectedness.
Things are becoming collected.

\sphinxAtStartPar
There are many different types of samadhi. When we are talking about
samadhi  today,  we  are  talking  about  the  enlightenment  factor  of  samadhi,
\sphinxstyleemphasis{samadhi bojjhanga}. This leads to
\sphinxstyleemphasis{samma\sphinxhyphen{}samadhi}, or noble right concentration. We  should  not  confuse  samadhi  with  other  types  of  samadhis. There
are  many  different  samadhis  from  many  different  traditions.  Certainly  the
hindu traditions have many different types of samadhi. The Buddha’s tradition does as well. In fact we’ll find that at the heart of most spiritual and religious traditions, there is some kind of samadhi involved. Some kind of collectedness, some kind of one\sphinxhyphen{}pointedness, some kind of stillness that leads
to an experience of something. Normally, a unification of mind, normally, a
oneness are some descriptions. We should be careful that we don’t take the
definitions of other traditions and try to apply those definitions to the Buddha’s teaching. The definition of samadhi in the Buddha’s teaching is quite
clear and it doesn’t need to be added to by other religious traditions, whether
it be the desert fathers from the Christian tradition or by Meister Eckhart,
or from the Islamic tradition, the Kabhala or from the Jewish tradition, the
Hindu tradition or any of the indigenous cultures that have their own types
of samadhi. Putting the mind in a state of internal one\sphinxhyphen{}pointedness is not a
new thing for humans. We’ve been doing it for a very long time. Granted
in our western culture we don’t pay any attention to it at all. It hasn’t been
something we’ve been interested in. But that’s changing.

\sphinxAtStartPar
So be aware when we are talking about samadhi in this context, we are
talking  about  a  very  specific  definition.  Not  the  kind  of  samadhis  that  are
available in the various tantric traditions or the other meditative traditions.
\DUrole{pdfpage}{220}  This is a kind of samadhi we find explained in the Pali texts and practiced
throughout the Buddha’s world.

\sphinxAtStartPar
The particular type of samadhi that we are after, is one that leads to
insight. It’s the samadhi which co\sphinxhyphen{}joins with our awareness so that we penetrate and  see  the  three characteristics, the three general characteristics of
all  physical  and  mental  phenomena.  This  is  where  our  samadhi  leads  to.
This kind of samadhi has its bases in the four foundations of mindfulness.
If you are developing your concentration based on the four foundations of
mindfulness using real objects that arise – not conceptual objects, using real
objects of mental and physical phenomena that are presently arising – then
we start to develop the samadhi the Buddha is talking about. The samadhi
that leads us to seeing things clearly, that allows wisdom to arise, wisdom
into  the  three  characteristics  of  impermanence,  dukkha  and  non\sphinxhyphen{}self.  For
sure it’s wisdom that sees these characteristics, but the proximate cause of
wisdom is samadhi. And this is the kind of samadhi that we are developing.
The one that happens quite calmly and automatically whilst we are developing the four foundations of mindfulness. We should take great care in ensuring that our mind is always open and feels spacious. We shouldn’t feel tight.
It shouldn’t feel condensed, it shouldn’t feel tense, it shouldn’t be dark. It
should be light, airy, open and spacious the feeling of being concentrated in
the present moment. This type of concentration leads us to seeing things as
they really are.

\sphinxAtStartPar
Of course, there is other types of samadhi practiced in the Buddha’s
teaching  as  well.  There  are  various  states  of  liberation  and  various  states
of  concentrations  using  both  ultimate  objects  and  conceptual  objects.  But
the  particular  type  of  concentration  we  are  talking  about,  as  I  said,  is  this
enlightenment factor of samadhi.

\sphinxAtStartPar
Vipassana practice is aimed at developing wisdom and the completion
of  various  insights,  finally  arriving  at  equanimity,  developing  the  mind  to
this seventh enlightenment factor. When we’re watching the rise and the fall
of the abdomen, when we’re being mindful to the process of the movement,
the  physical  sensations  that  are  arising,  with  each  moment  of  energy  and
effort, you expand in cultivating the present moment awareness, there is a
corresponding mental activity of penetrating the object. So we’re activating
\DUrole{pdfpage}{221}  and penetrating the present moment doing this over and over again until the
mind stabilizes in this mode of watching. This is the type of samadhi that
we’re after. The type of samadhi that stabilizes and is able to see the rise and
fall of natural mental and physical phenomena. When it sees in this way, it
will start to let them go. Wisdom will be activated and the letting go process
begins. We start to purify our mind and body process. We start to purify the
moment. We can have a state of purification from moment to moment. Our
job  is  to  extend  that  out  as  much  as  we  can.  The  mind  becomes  purified
because the hindrances can’t enter. Those defiled states are having troubles
entering  our  experience  because  the  enlightenment  factors  have  taken  the
place of the five hindrances and we’re on the road to development. Samadhi
has the power to gather together all the other mental factors into a unified
force  so  that  all  the  factors  can  do  their  job. They  are  all  arising  together
and they are all ceasing together but they have their own jobs and their own
functions. And we need to understand what these jobs and functions are and
to make sure that we are developing them. Quite often people forget a few
different factors and so we leave a few out. Our enlightenment factors are
not fully developed and the faculties are not balanced. Wisdom is neglected,
or extra energy is neglected, awareness is not penetrative, but concentration
can  be  strong,  the  mind  just  becomes  still,  not  really  knowing  anything.
Awareness and wisdom is unable to keep up with the state of concentration.
So the mind just goes…it falls down a well. It stays there in stillness and
happiness  and  a  little  bit  of  brightness.  It  stays  still  and  calm.  People  can
become quite attached to this state of mind. In fact we can develop it that
it becomes a very firm base for the arising of self. «I am a person that gets
concentrated and here it is.» We can start to use that base of samadhi not for
awareness and wisdom but as a foundation for the sense of self. We start to
appropriate and identify with that state of samadhi. We believe we’ve got it!
We are the owner of samadhi or we can do it or this is my samadhi! We start
to identify with this state of stillness and calmness.

\sphinxAtStartPar
When we practice correctly with all the enlightenment factors developing, and when we are looking out for balancing the faculties, our awareness
and wisdom will be able to keep up with each different stage of samadhi. As
our samadhi deepens, our awareness and wisdom need to be sharp enough to
\DUrole{pdfpage}{222}  catch up. They need to be able to note and know every stage that the mind
gets itself into. We need to note and know from the wandering stages, when
we are in the past and future and wandering and thinking, we need to understand when the body is becoming uncomfortable through those stages, we
need to understand when piti and tranquility are starting to arise. We need
to  note  and  know  and  let  go  of  them  even  though  they  are  very  pleasant.
We  need  to  stay  objective  towards  everything.  If  we  start  to  identify  with
anything the object gets sucked in and becomes a story of self. So we need
to  be  highly  alert,  highly  aware.  Be  on  the  lookout  for  any  stages  or  any
platforms on which the sense of self can establish itself. It’s very tricky this
craving! It’s been infiltrating and overcoming mental and physical states for
a long time. So our job is to watch out where it may land even in our states
of samadhi, even when the mind is becoming still. We still have to have a
mind  that  is  non\sphinxhyphen{}attached.  In  fact,  as  we  develop  and  let  go,  develop  and
let go, we realize that this is the only way for further development. To let
go of what has already been achieved. To relinquish and let go and release
ourselves even from deep states of concentration. This too has to be let go.
The wisdom tells us. It’s nice, you can stay here for a while, but even this
needs to be let go of.

\sphinxAtStartPar
Concentration is the proximate cause for the unfolding of wisdom. And
once the mind is quiet and still, there is space for wisdom to arise. The gap,
the space between the knowing and the physical and mental phenomena gets
wider and wider to the point where we are not attaching to the mental and
physical states as they arise and pass away.

\sphinxAtStartPar
The  Buddha  tells  us,  it’s  continuous  wise  attention  that  leads  to  this
state of samadhi aimed at developing concentration. The aim of concentration is the cause of developing concentration. It works on itself. One stage of
concentration leads to the next stage. So we have to incline our mind consistently to become stable – not only aware and present, but making our mind
still and calm as well. And this happens over time. As we start to practice, we
become more and more skillful at entering the state of calmness, maintaining
our awareness there, and then, releasing the mind from the calmness, coming
out. This is developed over time.


\section{First jhana}
\label{\detokenize{6-a:first-jhana}}
\sphinxAtStartPar
\DUrole{pdfpage}{223} When these enlightenment factors are starting to be developed in this
way, the stage of concentration starts to be developed. The eight factor of
the  noble  eightfold  path,  the  last  factor  is  samma\sphinxhyphen{}samadhi.  This
\sphinxstyleemphasis{samma\sphinxhyphen{}}
\sphinxstyleemphasis{samadhi}
has  four  parts  to  it.  The  first  jhana,  the  second  jhana,  the  third
jhana  and  the  fourth  jhana.  These  are  stages  of  concentration  or  shall  we
say, they are stages of letting go. This is how far and how much you have
been able to let go. That determines those states of mind your samadhi is in.
We said this week that our meditation practice is just the practice of letting
go. When we’ve let go of thoughts of sensuality, when we’ve let go of the
hindrances, when we’ve let go of the painful physical sensations that arise
in the body, then we start to enter into the first jhana. And the Buddha gives
us some directions on how to do this and what we should do when this state
arises. There is a case where,
\sphinxstyleemphasis{«a monk quite secluded from sensual pleasure,
secluded from unwholesome states enters into and abides in the first jhana
which has rapture and pleasure born of seclusion accompanied by applied}
\sphinxstyleemphasis{and sustained thought.»}
There are a few things in there:
\begin{itemize}
\item {} 
\sphinxAtStartPar
Secluded from sensual pleasures, secluded from unwholesome states –
they have been let go of.

\item {} 
\sphinxAtStartPar
We develop initial and sustained application – this is direct thought and
evaluative  thought. We  develop  these  in  our  meditation  practice. We
are aiming and sustaining our attention on various objects. The «initial thought» to send the mind to the rise and the fall. The «sustained
thought» to keep our mind right there. Initial and sustained application.

\item {} 
\sphinxAtStartPar
Piti starts to arise and happiness starts to arise.

\end{itemize}

\sphinxAtStartPar
These mind states are found in the first jhana. When this starts to happen,
we become secluded, secluded from unwholesome states. The Buddha gives
us instructions on what to do with this:
\sphinxstyleemphasis{«He permeates, pervades, suffuses
and  fills  this  very  body  with  the  rapture  and  pleasure  born  of  seclusion.
There is nothing of the entire body unpervaded by the rapture and pleasure}
\sphinxstyleemphasis{born of seclusion.»}
So when we notice that rapture or piti and happiness is
starting  to  arise  in  our  meditation  practice,  the  Buddha’s  directions  are  to
fill  the  body  with  this  rapture,  fill  the  mind  with  this  rapture.  Completely.
Permeate, pervade, suffuse and fill – which is not fear these mental states!
\DUrole{pdfpage}{224}  The Buddha is telling us to directly use them, to contemplate it. He gives
us a simile:
\sphinxstyleemphasis{«Just as if a skilled bathman, or a bathman’s apprentice would
pour bath powder into a brass basin with a little bit of water, kneading it
together, sprinkling it with some water and so that the ball of bath powder
would become wet, it would become saturated, moisture\sphinxhyphen{}ladden, permeated}
\sphinxstyleemphasis{within and permeated without, but would nevertheless not drip.»}
We got a
ball of clothe\sphinxhyphen{}washing powder, if you like, made into a ball with a little bit
of water. The ball is wet and moist, but it is not so wet and so moist that it
falls apart. And that’s what we do with our body at the states of rapture and
pleasantness when they start to arise. We don’t get all excited about it and
start identifying with it. We don’t get excited and say, «oh I can do it». We
don’t  start  to  subjectify  the  experience. And  we  also  don’t  become  scared
of the experience. When the mind starts to become very, very still and the
breath gets very, very short, becomes very quick, don’t get any fear. Don’t
buy into the fear. If fear starts to arise in the mind, make a note of it: ‘worrying, worrying, scared, fear’, whatever word you like to use. Step away from
that fear and allow the qualities of the samadhi experience to fill the mind
and body process completely.

\sphinxAtStartPar
When we do this, we enter into the first jhana, the first jhana experience.  For  us  to  progress  any  further  from  this,  we’re  going  to  have  to  do
some more letting go. When we let go, when we establish our awareness and
we get to a stage of momentum where we don’t have to activate our awareness in the present and send it to the abdomen and sustain it at the abdomen
– we’ve done that practice so often that the mind knows how to do it itself. It
does it automatically. It can do it by itself to the stage of momentum, where
we are still and present continuously in the present moment. When we get
to this stage of observation, it becomes a little bit effortless. We don’t have
to try so much. This particularly happens on longer retreats. Something after
a month or two months, when we have been constantly activating and the
mind knows what it’s doing it becomes automatic.


\section{Second jhana}
\label{\detokenize{6-a:second-jhana}}
\sphinxAtStartPar
\DUrole{pdfpage}{225} We start to enter into a deeper state of samadhi, deeper concentration,
when we let go of the directed and sustained thinking that’s needed to keep
us in the present moment with the first jhana. When we let them go, then we
go into a more stable concentration. We enter into the second jhana.
\sphinxstyleemphasis{«And
furthermore, with the stilling of directed thought and evaluation, he enters
into and abides in the second jhana which has rapture and pleasure born of}
\sphinxstyleemphasis{concentration.»}
A unification of awareness which is free from directed and
evaluating thought. There is internal assurance or confidence.
\sphinxstyleemphasis{«He permeates, pervades, suffuses and fills this very body with the rapture and pleasure
born of concentration. There is nothing of the entire body unpervaded by the}
\sphinxstyleemphasis{rapture and pleasure born of concentration.»}

\sphinxAtStartPar
So in the second jhana we really start to experience samadhi. The rapture  and  pleasure  are  born  of  concentration.  In  the  first  jhana  the  rapture
and  pleasure  are  born  of  seclusion.  There  is  a  certain  level  of  happiness
from  being  secluded  from  the  hindrances.  As  soon  as  we  suppress  those
hindrances our mind completely changes. The hindrances keep us busy with
non\sphinxhyphen{}sense. Sensual desire, aversion, laziness, boredom, restlessness, worry
and doubt. These five hindrances really block us. The first jhana is an experience of being secluded from them. The second jhana is an experience of
being  secluded  from  initial  and  sustained  thinking.  We  don’t  need  to  use
thinking in our meditation anymore. We become still. The mind has become
concentrated and we start to see things as they really are. At this stage in our
practice,  the  knowledge  of  arising  and  passing  away  becomes  quite  clear.
We  see  mental  and  physical  phenomena  arising  and  passing  and  we  don’t
hold on to them, we don’t identify with them. It’s just stuff that is arising and
passing – very, very rapidly. And the mind settles in to a very nice state of
observation. There is an internal assurance about that. We are quite confident
in what we are doing. Be careful in this stage of the concentration practice,
that you don’t become addicted to it. The piti can become very strong in this
stage.  For  many  people  there  is  a  lot  of  light.  Light  is  shining  throughout
the body coming out of the mind. Things are very calm and very peaceful.
We feel an enormous amount of faith and confidence in the Buddha and his
teaching. We think it’s miraculous and wonderful. There is a lot of strength
\DUrole{pdfpage}{226}  in the practice. At this stage we start to think, «well, it’s time to shave the
head and become a nun or a monk». We start to really enjoy the practice and
we start to see things as they really are.

\sphinxAtStartPar
The Buddha gives us a simile. He says,
\sphinxstyleemphasis{«just like a lake with spring
water, welling up from within, having no inflow from the east, west, north
or south – the sky is periodically supplying some showers – so that a cool}
\sphinxstyleemphasis{font  of  water  would  be  welling  up  from  within.»}
It’s  coming  from  inside.
There is a cool water coming to fill the lake, so that the cool water comes to
completely dominate the lake.
\sphinxstyleemphasis{«There being no part of the lake unpervaded
by the cool water. So too, a monk pervades, suffuses, fills and permeates this}
\sphinxstyleemphasis{very  body  with  the  rapture  and  pleasure  born  of  concentration.»}
There  is
nothing of the entire body unpervaded by this concentration.

\sphinxAtStartPar
So  this  is  the  second  jhana.  This  is  as  far  as  we  go  today  in  examining  this  enlightenment  factor  of  concentration.  Tomorrow  we’ll  start  to
have  a  look  at  the  seventh  enlightenment  factor,  the  factor  of  equanimity
or evenness. This factor of equanimity plays an important role in the third
jhana and the fourth jhana. We’ll start to see how the development of these
enlightenment  factors  and  their  completion  and  fulfillment,  fulfill  the  last
factor of the noble eightfold path. The four jhanas come to completion when
these  enlightenment  factors  have  been  successfully  developed.  When  the
enlightenment factors are full and complete and stabilized, the fourth noble
truth  arises.  It’s  the  path!  The  noble  eightfold  path  arises  in  the  mind  of
the meditator. Right view is established, right attention is there, previously
established right speech, right action and right livelihood, we’ve been putting forth effort and mindfulness continuously in the present so that the four
jhanas start to arise. And this is how the eightfold path develops. We’ll talk
more about that tomorrow.

\sphinxAtStartPar
For this morning I think that should be enough for us. We are going to
continue with some walking mediation.

\sphinxstepscope


\chapter{Day 6, afternoon}
\label{\detokenize{6-b:day-6-afternoon}}\label{\detokenize{6-b::doc}}
\LOCALaudiolink{https://www.mixcloud.com/anthonymarkwell/day-6-afternoon-talk-second-and-third-noble-truths/}

\sphinxAtStartPar
This afternoon we’re going to have a look a the second and the third
noble truths, the noble truth of the cause of suffering, and the noble truth of
the cessation of suffering.

\sphinxAtStartPar
Yesterday we had a look at suffering. There is dukkha. We had a look at
how the Buddha analyzes our mental and physical phenomena. He analyzed
it into the six sense bases and the five aggregates.

\sphinxAtStartPar
This afternoon, we’re going to have a look how the second and third
noble truths are closely related to each other. In fact, one of them, the cause
of dukkha, and the other, the cessation of suffering, are really two sides of
one coin. When we look at these noble truths together, we are looking at the
dependent origination, or paticca samuppada.


\section{Second noble truth}
\label{\detokenize{6-b:second-noble-truth}}
\sphinxAtStartPar
The Buddha tells us, that the cause of dukkha needs to be seen in order
to be eradicated. What is the second noble truth of the arising of suffering?
– It’s just this craving that occurs again and again bound up with delight and
lust. Seeking delight here and there. Sensual desire, craving for being and
craving for non\sphinxhyphen{}being.

\sphinxAtStartPar
The cause of suffering is craving for being. It’s just this craving that
occurs again and again that is bound up with delight and lust. There is some
\DUrole{pdfpage}{228}  delighting. That’s why craving can hit its mark so often. We like it. We enjoy
being.  It  enjoys  being.  It  loves  to  be  somebody. That’s  why  craving’s  job
is so easy. It’s delighting in itself. And it’s seeking delight here and there.
\sphinxstyleemphasis{«Now here, now there, seeking fresh delight»}
at the six sense doors. Craving
is seeking delight. It’s seeking being at the six sense doors.

\sphinxAtStartPar
Where does this craving arise? Where does it get established? –
\sphinxstyleemphasis{«Wherever  in  the  world  there  is  something  enticing  and  pleasurable,  there  this}
\sphinxstyleemphasis{craving arises and is established.»}
Wherever there is an object in which it
can seek delight, there craving arises. It’s the six sense based nama\sphinxhyphen{}rupa.

\sphinxAtStartPar
\sphinxstyleemphasis{The ultimate cause of dukkha is ignorance.}
Ignorance of the four noble
truths. Unawareness of the four noble truths. Not knowing, not understanding  dukkha  and  its  cause,  the  cessation  and  the  path.  Through  not  understanding these four noble truths, beings are bound by the leash of craving to
samsara. Bound to dukkha.

\sphinxAtStartPar
If we’re to have the ultimate breakthrough, we’ll need to understand,
not only dukkha but the cause of it, its fuel, this craving that arises at the
six sense bases attaching and delighting in the objects. Delighting in them
means finding some pleasure in the sense of self. It delights in it.

\sphinxAtStartPar
The practical real cause of dukkha for us, is the sense of self. Whenever there’s a self, there’s dukkha. We have a lot of self, there’s a lot of selfishness in us, then we’re going to have a lot of dukkha. When our sense of
self reduces, then our dukkha reduces. Reduce it a little bit, dukkha reduces a
little bit. Reduce it lot, dukkha reduces a lot. Completely reduced, complete
reduction of dukkha. It depends on how much self we have, how selfish we
are, how big our ego is. This dependence determines how much dukkha we
experience. We want to get around with a huge ego and huge sense of self,
appropriating and identifying everything we come across, dividing the world
into things that are mine and for me and those that are not, then we experience dukkha in every moment. If we want to have a good life, then we need
to be aware of this. We need to notice selfishness. It leads to dukkha. Reducing our sense of self leads to a reduction in dukkha. It’s only because there is
a «me», there is suffering. When there is no «me», then there is no dukkha.
There is no dukkha that applies to anybody. How can it apply to someone
that doesn’t exist? It’s only applying when there is a sense of «me». When
\DUrole{pdfpage}{229}  the «me» is gone, dukkha is also gone. It vanishes. It’s a subjective state.

\sphinxAtStartPar
According  to  our  satipatthana  text,  craving  arises  in  sixty  ways.  It
arises at the internal base, where it arises in ten different ways. According
to the six different doors. So in 60 ways craving manages to penetrate the
mind and body process. A little parasite has 60 ways it can get in in a sense
experience moment. 60 ways! Our mindfulness needs to be sharp, to be activated to stop this! It’s a very strong and rigid system that’s been set up. It’s
difficult to turn it off. Craving always hits its mark unless we have supreme
mindfulness and awareness or we can nail the present moment clearly in our
mind, see things as they really are and not be trapped or being caught by the
infection when it comes to us. We can note it, know it and let it go.
\begin{itemize}
\item {} 
\sphinxAtStartPar
Craving  arises  at  the  internal  bases,  the  eye,  ear,  nose,  tongue,  body
and mind. We take them as being mine or as being me.

\item {} 
\sphinxAtStartPar
Or the external objects that correspond to those doors, the forms, the
sounds, the smells, the tastes, the touch and the various thoughts and
ideas that come up. These are the objects of the internal bases. These
are taken as being for me, or happening to me or being mine.

\item {} 
\sphinxAtStartPar
There is also the consciousness. Six different types of consciousness
that arise. The eye consciousness, the ear consciousness, nose, tongue,
body consciousness, the mind consciousness. These also are taken as
happening to me, or that is me or they are mine in some way.

\item {} 
\sphinxAtStartPar
The contact that happens between these three that is sometimes appropriated and identified with.

\item {} 
\sphinxAtStartPar
Sometimes the feeling that is borne from this, structurally, in the moment, when there is a contact between an external object and an internal base, consciousness arises in this moment of contact, structurally
there is also feeling arising in that moment, either pleasant or unpleasant. We are taking that as happening to me. This is my feeling. We are
all over it.

\item {} 
\sphinxAtStartPar
Craving enters the perception process as well, the recognition. We pull
up old data from the past to try and recognize what this experience of
the moment is. We can identify it, recognize it, perceive it. We build a
little picture of what’s going on and we start to think about it.

\item {} 
\sphinxAtStartPar
\DUrole{pdfpage}{230} Through thinking at the six doors.

\item {} 
\sphinxAtStartPar
Six different types of volitional intention. Volitional intention is a karma process.

\item {} 
\sphinxAtStartPar
Six types of craving, six types of initial thought.

\item {} 
\sphinxAtStartPar
And six types of sustained, evaluative thoughts.

\end{itemize}

\sphinxAtStartPar
We  are  trying  to  subectify  anything.  Craving  tries  to  get  whatever  it
can.  Wherever  there’s  a  gap,  wherever,  there’s  an  unmindful  spot,  that’s
where craving can enter. It’s like a rock climber, trying to seek any kind of
refuge it can. Wherever there’s a place, it will grab and get hold of trying to
keep its existence going. It doesn’t want to let go of the rock face. It wants to
be. Craving wants to be something. This is its primordial energy. It’s a subjective state of being. It’s trying to hit a mark. That energy is flowing trying
to manifest as someone, as something. It wants to be something. This time
around, it’s manifesting as a human! As a being in the sense sphere world. A
being that manifests with six sense spheres. And it has been finding its mark
pretty well. For our entire lives, craving has been quite successful in infiltrating the moment and building up an idea. It’s become a solid and enduring idea. Lasting for decades. We’re constantly backing it up, supporting it,
performing different activities and collecting various things so that we can
add to or enhance the sense of self, the sense of me, our subjective state, our
being. We’re adding to it trying to reinforce it! If it gets weakened anyway,
we feel mortally wounded. Nobody wants to have a weakened sense of self.
We are striving to make it stronger. The biggest, the strongest, the fastest, the
richest, the funniest – all those states. We’ve been doing that our whole lives.
So  in  60  ways,  10  by  six  doors,  craving  has  an  opportunity  to  arise.
These  are  the  objects  that  we  need  to  be  noting  and  knowing,  we  need  to
be guarding them from the infiltration of craving to be. We need to get our
awareness and wisdom operating and spinning so quickly, that wisdom takes
control  of  the  moments,  the  point  of  contact,  after  point  of  contact,  after
point of contact. Awareness and wisdom is there, observing, noting, knowing. When it knows, craving can’t land on to that object. Craving can’t do
with it as it likes. That object has now been clearly seen by Vipassana. It’s no
longer an object which can be subjectified. It can’t be taken as this is mine,
this I am, this is my self. It’s out of the game if you like. It’s been neutralized
\DUrole{pdfpage}{231}  by awareness and wisdom in the present moment. There is a little gap, a little
glitch in the matrix. Our job is to make these often and regular. We need to
make these gaps more continuous and more sustained. We need to start to
break down this very well constructed mechanism turning the neutral environment into one that has been subjectified – dukkhering it. Craving dukkhas
the mind and matter process.

\sphinxAtStartPar
A fully enlightened being doesn’t have any dukkha. They’ve removed
dukkha, and yet, they still exist. They still manifest. The mind and body process still continues even though craving has been abandoned, uprooted. So
it’s possible to still have a mind and body process, still functioning, all the
enlightened ones, and the Buddha after his enlightenment, existed dukkhafree. And yet he was still able to perform functions, eating, sleeping, talking,
walking, a lot of teaching, all those things.

\sphinxAtStartPar
Now the cessation of craving doesn’t mean that we disappear. It means
that our mind and body process is no longer infected by delusion, by ignorance. Greed, hatred and delusion don’t have much chance of arising because
there is no self that generates that. It takes a self to be greedy. «I’m greedy, I
want.» It’s a subjectified state. «I’m not happy with. I’m angry. I don’t know.
I’m deluded.»

\sphinxAtStartPar
In 60 ways appropriating and identifying, craving mistakenly subjectifies  the  moment  and  after  this  is  done  for  many  years  or  decades,  there
comes to be a strong of sense of «I am». A strong sense of being manifests
within the creature. In the mind and body process it comes to a very strong
understanding that «I am». A self is established. It seems concrete, it seems
continuous. We never question it. If someone asks us, if we really are someone, it seems obvious to us. «Of course, I’m someone, what are you talking
about?» If somebody says, you’re no\sphinxhyphen{}one. «No, I am someone, I am». Are
you? Where are you? I want to see that self. Where is it arising? – It’s just an
idea! It’s a dream that infects our reality. It turns our experience of objective
nature, of dhamma, into a subjectified experience. An experience of dukkha.
When the Buddha is pointing this out, he declares,
\sphinxstyleemphasis{«there is dukkha»}. This is
how it’s happening. Luckily for us, he also says,
\sphinxstyleemphasis{«there is an end of dukkha»}.
There is the experience that is the mind and body process which is not undergoing this process of subjectification. The cessation of dukkha is when craving \DUrole{pdfpage}{232}  doesn’t enter this stream of experience. That’s the third noble truth. We’ll
talk about it very shortly.

\sphinxAtStartPar
So this thing believes it is someone but it’s just a natural impersonal
flow of mental and physical phenomena dependently arisen and conditioned.
Traveling along on its own journey, naturally unfolding. It’s only because we
don’t see it as it really is, that we come to believe this false story that we’ve
been told for years and years. Our parents didn’t know and understand, they
played along with the game. They may have even enhanced craving’s ability
to get into the stream of our flow.

\sphinxAtStartPar
We are just quietly appropriating each sense experience moment building  and  constructing  an  ever  more  present  sense  of  I.  We’re  developing
being.  If  we  are  not  noting  and  knowing,  we’re  developing  being.  Being
is the cause of dukkha. Once there’s being, there’s birth. When there’s no
being, there’s no birth. If there’s no personality being constructed than the
word  «birth»  doesn’t  apply.  Birth  is  a  concept  that  relates  to  a  person.  If
there’s no person manifest in that moment, then that’s the end of birth. The
cessation  of  birth.  Birth  doesn’t  take  place.  Here  we’re  talking  about  the
birth of the self, the birth of the ego. It’s the idea that I am someone. This is
birth. «I am, I’m here».

\sphinxAtStartPar
So the heart of the teaching is to destroy this idea of me, and mine and
I, to destroy this idea of subjectivity and to enter the void of
\sphinxstyleemphasis{suññata}, where
the mind and matter process is not infected by craving to be, the mind and
matter process is empty of self and things pertaining to self. This is an experience of voidness. That world in which a sense of self doesn’t make itself
apparent. Empty.

\sphinxAtStartPar
So that’s the second noble truth, the cause of dukkha. Dukkha is produced  by  this  energy,  this  craving,  this  wanting,  this  strong  desire  to  be
something. See if you can notice and recognize this in your own mind. See
if you notice when it wants to be something. Check your activities and the
motivations for your activities. Why are you performing and doing things? Is
it to generate a stronger sense of self? Are you deliberately trying to develop
a personality? If so, have a look what happens when you do that. Understand
that the development of your personality is the development of your dukkha.
You’re  actually  helping  craving. You’re  opening  the  door  and  preparing  a
\DUrole{pdfpage}{233}  seat.  «Please  come  in,  sir.  Subjectify  this  experience.  Come  in  and  cause
some dukkha here. We are all happy here. Come and disturb us!» You don’t
want to invite craving into your life! When you notice it, make a note ‘craving, wanting, desiring, wanting to be’.

\sphinxAtStartPar
It seems just like a simple thing. It seems obvious but all of us have
experienced that craving to be when we have to make that decision what we
are going to be in our life. What are you going to be when you grow up? We
think we’ve worked it out in our 20s when we get some job, secure training
or get a degree of some sort. «Oh, that’s what I am.» But it doesn’t really
stop. Well into your 40s and 50s, we’re still asking the question, what am
I going to be when I grow up? – Yes, it doesn’t stop. What am I going to
be? And then we become someone and you are not really satisfied with that
identity. You want a new one. Something a little bit more funky. Something
a little bit cooler. We want to have our own unique niche in the world. Our
own unique business card. A CV that is wild and interesting. We can show
others. See, how interesting I am! – It’s all to create a sense of self. We think
we need to have a strong sense of self to be happy. It’s exactly the opposite.
The more we think and spin about ourselves, the more dukkha we get.

\sphinxAtStartPar
We can even cause mental illnesses through obsessive thinking about
ourselves. Obsessive thinking about the past and ourself – depression states.
Obsessive thinking and planning and worrying about the future and ourself
– anxiety states! These states of mind occur because of craving! Because we
don’t see the trouble that it’s making. Our society and community is telling
us, you have to become something. You have to be someone. And when we
don’t know what we want to be, it’s awful. We need to work it out. What am
I going to be? Everyone is asking you. What are you going to be? You feel
obliged. What is it that we want to be, that we are searching for? Try to see
that this energy is craving, is wanting. It really is the cause of dukkha in your
life. This is the one, that’s causing the problems. When we can drop it and
let it go, when we don’t have to fulfill its desires, running around trying to
fulfill craving’s appointments, craving’s wishes – when we’re not doing that,
then we’re free. We are free in that moment. We are not a slave of craving.
We’re giving up the job of building our personality and our identity. We’re
throwing that out of the window.

\sphinxAtStartPar
\DUrole{pdfpage}{234}  When  we  note  that  we  start  doing  that,  or  when  we  note,  what  our
motivation for doing a particular action or saying a particular thing is, when
we notice that this activity is actually just a self\sphinxhyphen{}producing one, then we give
it up. We let it go. We change our view, or we change our ability to perform
that function. We change our attitude to performing that function. – We can
still be helpful, we can still do different things, but we just don’t do it with
such a sense of self that it causes identity view to arise.


\section{Third noble truth}
\label{\detokenize{6-b:third-noble-truth}}
\sphinxAtStartPar
The third noble truth, the cessation of dukkha or
\sphinxstyleemphasis{dukkha nirodha}, is the
other side of the coin. It’s the end of suffering. That’s what cessation means.
The finish of it, the end of suffering. Where suffering ceases to exist, where
it no longer appears.

\sphinxAtStartPar
The  Buddha  said  repeatedly:
\sphinxstyleemphasis{«Now  as  before  monks,  I  teach  suffer\sphinxhyphen{}}
\sphinxstyleemphasis{ing and the cessation of suffering»}. That’s what the Buddha’s project is. He
talks a lot about dukkha and the cessation of dukkha. And this is important
to keep in mind, when we investigate the dhamma, when we investigate the
Buddha’s teaching for ourselves. When we’re looking to see what the actual
teaching  is.  He  was  just  focussed  on  these  things,  dukkha  and  the  end  of
dukkha. That’s all.

\sphinxAtStartPar
In fact, there was a time, when he was sitting in the forest with a large
group of monks and he picked up a hand full of leaves and said,
\sphinxstyleemphasis{«monks,
what is more, the leaves in my hand or the leaves in this great forest?» «Oh,
venerable, sir, the leaves in this great forest are many. The leaves in your
hand are so few.» «So it is, monks, so it is. The knowledge I have gained is
vast, like the leaves in this forest, but the knowledge I’m giving you is just}
\sphinxstyleemphasis{this.»}
Dukkha  and  the  end  of  dukkha.  Of  all  the  things  the  Buddha  came
to  understand  and  came  to  know,  he  taught  us  this:  dukkha,  the  cause  of
dukkha, the end of dukkha and the path leading to the cessation of dukkha.
Only this was worthy in his talks.

\sphinxAtStartPar
These  days  people  want  to  know  what  it  is  that  Buddhism  teaches.
What  it  teaches  in  relation  to  a  whole  lot  of  modern  day  dilemmas. What
does  Buddhism  say  about  environmental  warming,  climate  change,  vegetarianism, poverty, global corporate profit taking, same sex marriage, about
\DUrole{pdfpage}{235}  a  whole  range  of  different  things.  The  truth  is,  the  Buddha  didn’t  really
address any of these subjects. It wasn’t really his thing. We can take some
ideas  or  interference  from  his  teaching  and  apply  it  to  the  modern  problems but his project was singular. He was only interested in this problem of
dukkha and he was very keen for us to see it. He had the unique ability to
teach this dukkha in many different ways to many different people. He had
the  ability  to  look  at  somebody  and  know  how  their  craving  was  arising,
where  they  were  stuck.  He  pointed  out  to  them  directly.  There  are  many
instances of beings becoming enlightened, quite rapidly, after an intervention  from  the  Buddha.  He  was  pointing  out  where  they  had  gone  wrong,
what they’re not seeing, how they were clouded by the matrix. And so while
he  doesn’t  deviate  from  his  central  concern,  he  did  approach  the  subject
from many different angles according to who he was talking about. All these
angles have been recorded in the old Pali texts. We can investigate them for
ourselves and see how the Buddha approaches this problem of dukkha and
the cessation of dukkha. The myriads of ways, the dozens of ways that he
teaches and tells us to look. If we understand what he is talking about and try
to follow what he teaches, we will experience for ourselves the nature of the
dhamma. People have been doing so, for a thousand years. We will also have
that opportunity. The Buddha’s teaching has one taste. He says,
\sphinxstyleemphasis{«just as the
ocean has one taste, the taste of salt, so too this dhamma has one taste, the}
\sphinxstyleemphasis{taste of release.»}
He wants us to understand dhamma, because he wants us to
be free from the dukkherizing process. He wants us to be free from dukkha.
So  what  is  this  third  noble  truth?
\sphinxstyleemphasis{«It  is  just  the  reminderless  fading
away and ceasing, the giving up, the relinquishing, the letting go and reject\sphinxhyphen{}}
\sphinxstyleemphasis{ing of that same craving.»}
It’s the giving up of that craving. Whenever you
notice it arising, make a note and give it up. Start to do the work of cessation. Whenever you notice it, relinquish it. Let it go. Start to walk the path.
Whenever you note it, there is craving for being starting to play its games
and becomes strong in your life, you can quickly remove the suffering that
it causes by just noting and knowing it, stepping back. Don’t be fooled by
craving any longer. Fading away, ceasing, giving up, relinquishing, letting
go and rejecting of that craving to be.


\section{Nirvana}
\label{\detokenize{6-b:nirvana}}
\sphinxAtStartPar
\DUrole{pdfpage}{236} The cessation of dukkha is something that the Buddha called
\sphinxstyleemphasis{nirvana}.
He had a very special name to use for this experience where the mind and
body  process  are  freed  from  the  craving  to  be.  When  we  free  it  through
awareness  and  wisdom,  than  that  experience  is  called  nirvana.  Nirvana  is
not a wonderful palace up in the clouds somewhere.
\sphinxstyleemphasis{Nirvana is the mind and}
\sphinxstyleemphasis{body process in the present moment unaffected by craving to be.}
This is how
cessation takes place. That’s an experience of nirvana in the here and now.
This  nirvana  the  Buddha  talks  about  in  the  texts,  he  called  it  the  unborn.
It’s the base, the unborn, the unbecome, the unmade, the unconditioned, the
transcendent, supra\sphinxhyphen{}mundane, ageless, deathless, sorrowless, supreme security from bondage. ‘Bondage’ is the state of being. ‘Supreme security from’
is letting go that craving. We’ll become secure. We reach the base, nirvana,
the  unconditioned  element.  It’s  not  impermanent,  it’s  permanent.  It’s  not
dukkha, it’s the end of dukkha. It’s still impersonal, it’s not a self. It’s still a
state of impersonality, it’s still a state of non\sphinxhyphen{}self. You don’t become all of a
sudden nirvana. That’s not your true self. It just is what it is. The cessation
of dukkha.

\sphinxAtStartPar
Nirvana is also described as the stilling of all formations. The quieting
of conditioned phenomena. The relinquishing of all acquisitions. All those
things we come to acquire and get – not only the physical things but also
our own attributes that we believe we are – it’s the relinquishing of those
acquisitions. All  the  time  and  all  the  effort  we’ve  gone  into  creating  that
special identity that we have today, it’s letting all that go and not caring what
other people think about us letting it go. We’re evolving in a spiritual way.
We are moving beyond our normal mundane lives of eating, sleeping, seeking sensual pleasures, from one sensuality to another, searching for stuff to
identify with, to have fun to. As the Buddha says,
\sphinxstyleemphasis{«delighting»}. We’re seeking delight. Craving is seeking delight. That’s its speciality. As soon as there
is something pleasurable, craving rubs its hand and glee, «uhmm, good, I’ll
get easily in here». There is some delighting, creating a sense of being, right
here, right now. It loves to be. It loves to develop a self. It likes to work hard
on it. It loves dukkha. It’s the cause of dukkha.

\sphinxAtStartPar
Nirvana is described as the destruction of craving and dispassion. So
\DUrole{pdfpage}{237}  there  are  many  passages  throughout  the  Pali  texts  describing  this  state  of
the bliss of the release when the mind is released from craving. Some of the
most beautiful are in the old poetry books. The texts of the terigata and teragata verses, of the elder monks and nuns, where they exclaim in beautiful
poetry the bliss of release. Hundreds of monks and nuns explaining, or exalting their experience of nirvana. All of them using slightly different words to
explain this experience they uncovered as member of the sangha, the noble
sangha, the aria\sphinxhyphen{}sangha. Beings who have come across the Buddha’s teaching, had enough intelligence to recognize its importance, put aside a period
of their live to intensify their practice and break through, break through to
the further shore. And then they tell others about it. Beautiful. Very beautiful. I recommend it if you have a chance to read those texts and translations.
Nirvana  is  sometimes  referred  to  in  the  texts  as  the  unfashioned.  It
hasn’t  undergone  fashioning  yet.  Sculpting.  Creating.  It’s  called  the  end.
The effluentness. The truth, the beyond, the subtle. The very hard to see. The
permanent.  The  undecaying.  The  surfacelessness.  The  non\sphinxhyphen{}objectification.
Peace. The exquisite. The solace. The exhaustion of craving. The wonderful. The marvelous. The secure. The unafflicted. The passionlessness. The
pure. The release. Non\sphinxhyphen{}attachment. All these terms signify this state which is
beyond craving, where craving can’t enter. The end of craving. The shelter.
The harbor. The refuge.

\sphinxAtStartPar
The  term  «nirvana»  was  actually  a  word  that  the  Buddha  took  from
the local language. It didn’t have the meaning of the complete cessation of
dukkha. It had a more simple meaning. It was traditionally used in northern
India in the fifth century b.c. to describe the going out of a fire. You see the
candle burning. When the flame gets put out, we say the flame has «nirvanaed». It’s been put out. The flame burns dependent upon conditions. When its
conditions are in place, a flame burns. When there is wax, and wick and heat,
when these three are there, the flame burns. When we remove the conditions,
one of them at least, heat or wick or wax, then the flame disappears. It’s gone
out. It’s no need for us to ask, «oh, where has the flame gone?» This is not
what we’re doing. There’s no point in asking «where has it gone?». When
we blow out the candle, it doesn’t have to go anywhere. It just ceases to be,
because the conditions that were putting it in place have altered, changed.
\DUrole{pdfpage}{238}  The  conditions  are  impermanent.  The  conditions  on  which  it  stands  have
changed  and  so  the  flame  changes. And  so  it  is  with  this  sense  of  self.  It
keeps being generated when the conditions are in place and when the conditions are no longer there – nirvana. Cessation.

\sphinxAtStartPar
There are two different types of nirvana. There is nirvana here and now
which  is  experienced  when  mind  and  matter  are  still  functioning,  but  the
sense of self ceases. That’s the experience of an enlightened person who still
has a mind and body process that they’re carrying around with them. There
are  people  in  this  country  who  are  enlightened.  Monks  and  nuns  and  laypeople, who have done the training, experienced it, the mind and body still
functioning, still talk. You can have a chat with them. There is nobody there.
No\sphinxhyphen{}one home! The mind and body process still activated but they don’t have
a sense of self. Very interesting to meet these people. If you ever hear about
them, please go and visit them.

\sphinxAtStartPar
Then there is the type of nirvana of the enlightened being who passes
away.  That’s  the  full  nirvana  when  the  mind  and  body  process  no  longer
breathes or is conscious. The normal word we use, «it dies». The person is
dead. Death. When an enlightened being dies – it’s a bit of a problem with
that  sentence  –  because  ‘death’  is  a  word  that  applies  to  a  person  and  if
you are fully enlightened you are no longer a person. So this concept death
doesn’t apply anymore. Death doesn’t take place for an enlightened one. The
elements break apart, and that’s it! Finished.
\sphinxstyleemphasis{«Done is what had to be done.»}
Full final release. Consciousness unencumbered. Unhindered by mind and
body process. Not attached or identifying with any of it. Free. Released!

\sphinxAtStartPar
The  Buddha  also  used  the  term
\sphinxstyleemphasis{suññata},  emptiness  or  voidness  to
describe his most profound and deepest teachings. It means that all things,
all conditioned and unconditioned things are void of self. Suññata. Sometimes we find in the texts, the monks asking the Buddha, «what state does
the blessed one normally dwell in these days?» He says, «at the moment I am
hanging out in suññata.» Voidness, entering the void where there is no self
or things pertaining to self.

\sphinxAtStartPar
So  when  we  put  these  two  noble  truths  together,  the  second  and  the
third, they form two sides of one coin, if you like. Two sides of our experience. We can either be noting and knowing and letting go and experiencing
\DUrole{pdfpage}{239}  the third noble truth, or, we can be appropriating and identifying allowing
craving in and we experience the cause of dukkha and dukkha. Two experiences. It’s up to us in the present moment to decide which one we’re going
to experience. It’s your choice. Dukkha or nirvana. One of them is very easy.
Just lay back and think about yourself all the time. Hmm, just have a nice life
of dukkha. Lots of problems. The other one is a little more tricky to practice
but a lot more worthwhile – the cessation of dukkha. We’ll have to do something. We have been working very well this week striving to train our mind
to be in the present moment so that we can set ourselves up to experience
this phenomena. We need to put all the conditions in place and there’s quite
a few of them. We’ve been explaining them to you this week. Putting these
conditions in place and then to allow the dhamma to unfold.


\section{Dependent origination}
\label{\detokenize{6-b:dependent-origination}}
\sphinxAtStartPar
When we join the second noble truth and the third noble truth together,
it’s referred to as the
\sphinxstyleemphasis{paticca samuppada}, or dependent arising or dependent
origination.
\sphinxstyleemphasis{It’s  the  deepest  and  most  profound  teaching  that  the  Buddha}
\sphinxstyleemphasis{offers  us.}
In  fact,  the  venerable Ananda,  the  Buddha’s  close  attendant  for
25  years,  once  said  to  the  Buddha,
\sphinxstyleemphasis{«ah,  this  dependent  origination  seems
so clear and easy to understand. I just don’t get it, why beings don’t under\sphinxhyphen{}}
\sphinxstyleemphasis{stand.»}
The Buddha admonished him:
\sphinxstyleemphasis{«Oh, do not say so Ananda, do not
say so. This teaching is deep and profound as the ocean is wide. It’s through
not understanding the dependent origination that beings are bound to sam\sphinxhyphen{}}
\sphinxstyleemphasis{sara.» And this is the central teaching we need to understand.}
It’s the key for
unlocking our experience of dhamma.

\sphinxAtStartPar
It teaches us, how the me, or the I arises in the present moment or how
the  me  or  the  I  ceases
\sphinxstyleemphasis{in  the  present  moment}.  It’s  a  very  succinct,  a  very
short but also very illuminating teaching. It has 12 forms, 12 different bases
on  which  the  sense  of  self  arises. And  these  12  bases  or  these  12  conditioned things are linked together by conditioning processes. They are linked
together by what we call the
\sphinxstyleemphasis{idappaccayatā}, or specific conditionality. Each
link conditioning the next.
\sphinxstyleemphasis{«When this is, that comes to be. With the arising
of this, that arises. When this ceases, that ceases. With the cessation of this,}
\sphinxstyleemphasis{that ceases.»}
The things are conditioned. When the conditions are in place
\DUrole{pdfpage}{240}  they arise. When the conditions are no longer in place, they cease to exist.
This is the great teaching that the Buddha uncovered. Conditionality. He saw
how the self is a conditioned phenomena. Arising, when there is ignorance
and  delusion,  and  ceasing,  when  there  is  full  awareness  and  wisdom.  He
explains this in a little road map. You can often see visual representations of
the paticca samuppada. Sometimes it’s in a chain of twelve links. I hope you
had a chance to read the notice board down there. Dependent origination is
explained there, both in the forward process and in the reverse process, in
the dukkha creating process and in the dukkha ceasing process. It’s one coin,
with dukkha on the one side and nirvana on the other. It’s either or. When
you flip it, it doesn’t land on its edge. We’re either dukkhering ourselves or
we’re reaching cessation in the present moment.


\section{Dukkha creating process}
\label{\detokenize{6-b:dukkha-creating-process}}
\sphinxAtStartPar
It starts with ignorance. With ignorance as condition volitional formations  come  to  be.  The  first  four  items  of  the  dependent  origination,  ignorance, conditional formations, consciousness and nama\sphinxhyphen{}rupa. In the forward
direction, if there is ignorance the sankharas or the conditional formations,
the intentional structures condition consciousness and nama\sphinxhyphen{}rupa. They set
that  ignorance  is  conditioning  the  consciousness
\sphinxstyleemphasis{in  the  present  moment}
if
we’re  unaware.  If  we’re  unaware,  the  consciousness  that  arises  at  the  six
sense bases has already been infected by ignorance. Just the fact that we’re
unaware of that moment, means that we are ignorant in that moment. When
ignorance is present, then – consciousness and nama\sphinxhyphen{}rupa, which are stuck
in a vortex conditioning each other arising at the six sense bases – this consciousness  and  nama\sphinxhyphen{}rupa  vortex  becomes  conditioned  by  ignorance,  and
then,  when  it  squeezes  out  through  the  six  sense  bases  somewhere,  the
moment of contact is infected. Contact is there. We can note that contact and
see it happening in real time. We can see the infection taking place if there’s
ignorance.
\sphinxstyleemphasis{This  happens  in  a  snapshot,  this  is  not  a  temporal  movement.}
Visual representations of the dependent origination appear to show it traveling through time from one link to the next link and from the next link to
another one. But it’s more useful to look at these 12 links stacked upon each
other, based upon each other. When we start stacking, they start growing.

\sphinxAtStartPar
\DUrole{pdfpage}{241}  If we’ve been unable to note in the present moment and unawareness
is in that moment, ignorance is in that moment, then the consciousness and
nama\sphinxhyphen{}rupa which arise at the six sense bases become infected. The feeling
that arises joined in this moment becomes a feeling of somebody. A sense of
«I» enters in, a sense of «me» enters in into the scene. «It’s my feeling.» That
feeling starts to condition the mind. It conditions craving.

\sphinxAtStartPar
When  craving  comes  to  be,  clinging  comes  to  be.  With  clinging  as
condition, being. With being as condition, birth. The next four links show
the gradual concentration of the sense of me and mine. The «me» becoming
more and more solid. The craving is just taking things as «me» and «mine»,
happening to me and for me. It does that so rapidly that it builds up a more
concrete  sense  of  subjectivity.  Clinging,  upadana,  we  come  to  attavada  or
the view of self arises. The conceit «I am» starts to arise. It starts to take life,
it believes it’s alive. The subjectivity is arising. A person is manifesting in
the present moment. Clinging then leads to being, the full sense of self and
then a whole history gets mapped out. Birth, that’s the past. Death, that’s the
future. These concepts relate to the person that has just been established in
that moment.

\sphinxAtStartPar
With ignorance as condition, volitional formations. With volitional formations as condition, consciousness. With consciousness as condition, mind
and matter. With mind and matter as condition, the six sense bases. And the
six sense bases are infected, the contact is infected, feeling is infected, and
then craving, clinging and being come to be. Finally it results in birth, the
establishment of an identity and a personality and then sorrow, lamentation,
pain, grief and despair. All those states of suffering come into being – just
because we’ve been unable to note and know in the present moment. That
moment gets infected and becomes a self\sphinxhyphen{}moment adding to the pile of self
that’s  being  created  along  the  way,  adding  to  the  illusion,  reinforcing  the
illusion.


\section{Dukkha ceasing process}
\label{\detokenize{6-b:dukkha-ceasing-process}}
\sphinxAtStartPar
On the other side, if we can be aware of the present moment, what our
Vipassana meditation is all about, if we can note and know rapidly and continuously, if we can increase our noting speed and the number of objects we
\DUrole{pdfpage}{242}  can note, we’ll start to build up a continuous field of awareness and wisdom.
So  any  time  a  door  opens,  we’re  knowing  what’s  there.  We’re  seeing  it
clearly. Ignorance is ceasing in that moment. Unawareness is being removed
and awareness is replacing it. There is awareness of the moment. When this
happens, ignorance is gone. It doesn’t condition the volitional formations.
When those volitional formations are no longer functioning, either as greed,
hatred or delusion or, on the more positive side, as non\sphinxhyphen{}greed, non\sphinxhyphen{}hatred,
non\sphinxhyphen{}delusion, when this conditioning process is not taking place, consciousness  doesn’t  get  conditioned  by  ignorance.  The  ‘consciousness  and  mind
and matter’\sphinxhyphen{}vortex that arises at the six sense doors, has not been infected by
ignorance. That moment is clear, unsubjectified, purified if you like. It’s an
experience of dhamma. The contact point seen very clearly – no self there!
It’s a moment of freedom. We’ve noted the point of contact at the six sense
doors.  The  feeling  that  arises,  the  pleasantness  or  unpleasantness,  is  just
feeling. It doesn’t belong to anyone. It’s not taken as being good or bad. It
ceased to have the opportunity to condition the mind. The pleasantness or
unpleasantness is still there, but it has lost its conditioning ability because
it’s seen clearly for what it is. It’s not going to be appropriated and identified
with. That moment is a moment of knowing, it’s a moment of clear seeing.
It’s  a Vipassana  moment.  Feeling  doesn’t  get  taken,  craving  doesn’t  arise.
Craving ceases, clinging ceasing, being ceases and so does birth, old age,
sickness and death. The whole string of dukkha ceases. Sorrow, lamentation,
pain,  grief,  despair,  depression,  anxiety,  worry,  tension  –  all  these  things
cease in that moment. All these things are concepts that relate to a person
and since there is nobody in that moment, those concepts cease to have any
validity. They don’t exist.

\sphinxAtStartPar
All we have to do, is to extend this from moment to moment. You can
catch it for a second, can we do it for two seconds, three seconds, hold it
back, keep noting and knowing, keep noting and knowing, make it continuous. Don’t allow any gaps to occur. See how long ignorance can cease for
in the moment. See if you can have that experience of the mind and matter
process which is independent of ignorance. Which is not being conditioned
by ignorance, which is, in fact, conditioned by wisdom, by true knowledge.
When we experience this in the present moment, this is the third noble
\DUrole{pdfpage}{243}  truth. Cessation. The mind and body process ceases. And the whole edifice
collapses in that moment. Our job as meditators is to make sure that we can
bring as many of these moments as we can into our experience. The more we
can do it, the more we start to alter our understanding of the mind and body
process. You’ll start to see it more and more clearly. The things that used to
worry us and excite us before, completely loose our attention. Just like when
we used to play with dolls or little trucks in the sand pit, when we were very
small children, we’ve stopped doing that now because it doesn’t have any
interest for us anymore. No longer interesting to play those games. And so it
is with developing a sense of self or developing an identity view. It no longer
has any interest for us. We just see it as a useless activity while others get a
lot of delight from it. We’re just not interested in that anymore. We see the
pointlessness of it, we see the futility of it.

\sphinxAtStartPar
We can break this casual process at the point of contact, if our awareness and wisdom is clear. When consciousness and the mental factors arise
at  the  six  sense  bases,  this  is  where  we  have  to  note.  If  we  can  be  noting
at the point of contact continuously, repeatedly, unremittingly, then we can
have some possibility of reaching the cessation stage. This is our job. This
is what our meditation practice is all about. In fact, this is the role of human
evolution. This  is  what  has  to  be  done.  Eventually,  all  beings  will  escape
samsara through this process of letting go of the sense of self, freeing themselves. All of us will have to go through it at some point. Maybe this week,
this year or this lifetime or some other time. But this is our destiny, this is
our evolution as a species. This is how we move beyond the material realm
and enter into the spiritual realm of the enlightened ones. The realm of the
beings who are free from dukkha.

\sphinxAtStartPar
For now, it’s important for us to recognize that we may break this causal
process and the practicing of satipatthana is the way out of the process that
the Buddha so eloquently described. You just need to examine it closely, not
the books, we need to examine our own mind. We need to see if the map is
reproduced in our own mind and body process. When we see things as they
really are in the present moment, it blocks defilement from arising. It’s like
we’ve neutralized the field of objects. None of them can be used as a sense
of self, as a base for the sense of self. Liking and disliking disappear, judging \DUrole{pdfpage}{244}  and comparing – all gone! The knowledge of arising and passing away,
it’s not the permanent removal of defilement but it is a little bit of a taste of
the deathless state. When we can sit observing the mind, the body and mind
process arising and passing at the six sense doors, uninterruptedly, without
us subjectifying the process, then we have a taste of the deathless state. A
taste of nature. It’s not full nirvana but it’s something getting close to it.

\sphinxAtStartPar
I like to read you now part of the Kaccayanagotta sutta: Kaccayanagotta, just the name of another monk, once visited the Buddha. This text has
a special place in the teaching of the Buddha being used by one of the major
traditions, the Majjhimas. It’s the central text in their teaching of emptiness
and non\sphinxhyphen{}self.

\sphinxAtStartPar
Dwelling  at  Savatthi…  Then  Ven.  Kaccayana  Gotta  approached  the
Blessed One and, on arrival, having bowed down, sat to one side. As he was
sitting there he said to the Blessed One:
\sphinxstyleemphasis{«Lord, ‘Right view, right view,’ it is
said. To what extent is there right view?»}

\sphinxAtStartPar
\sphinxstyleemphasis{«By and large, Kaccayana, this world it depends upon a duality, upon
the notion of being and upon the notion of non\sphinxhyphen{}being. But when one sees the}
\sphinxstylestrong{origin}
\sphinxstyleemphasis{of the world as it actually is with correct wisdom – (the second noble
truth, the arising of dukkha) – there is no notion of non\sphinxhyphen{}being/non\sphinxhyphen{}existence}
\sphinxstyleemphasis{in regard to the world. When one sees the}
\sphinxstylestrong{cessation}
\sphinxstyleemphasis{of the world as it actually is with correct wisdom, there is no notion of being/existence in regard
to this world.}

\sphinxAtStartPar
\sphinxstyleemphasis{By and large, Kaccayana, this world is shackled by attachments, clingings and adherence. But this one {[}with right view{]} does not get involved with
or  cling  to  these  attachments,  clingings,  and  adherence;  he  does  not  take
a stand about ‘my self.’ He has no uncertainty or doubt that what arises is
only suffering arising; what ceases is only suffering ceasing. His knowledge
about this is independent of others. It’s to this way, Kaccayana, that there is
right view.}

\sphinxAtStartPar
\sphinxstyleemphasis{‘Everything exists’: That is one extreme. ‘Everything doesn’t exist’: That
is a second extreme. Avoiding these two extremes, the Tathagata teaches the
dhamma via the middle: With ignorance as condition, volitional formations.
With volitional formations as condition, consciousness. With consciousness
as condition, nama\sphinxhyphen{}rupa. With nama\sphinxhyphen{}rupa as condition, the six sense bases.
\DUrole{pdfpage}{245}  With  the  six  sense  bases  as  condition,  contact.  With  contact  as  condition,
feeling. With feeling as condition, craving. With craving as condition, clinging. With clinging as condition, being. With being as condition, birth. With
birth as condition, aging, death, sorrow, lamentation, pain, distress, despair}
\sphinxstyleemphasis{come to be.}
\sphinxstylestrong{Such is the origination}
\sphinxstyleemphasis{of this entire mass of suffering.}

\sphinxAtStartPar
\sphinxstyleemphasis{But  with  the  remainderless  fading  away  and  cessation  of  ignorance
comes  cessation  of  volitional  formations.  With  the  cessation  of  volitional
formations, cessation of consciousness. With the cessation of consciousness,
cessation of nama\sphinxhyphen{}rupa. With the cessation of nama\sphinxhyphen{}rupa, cessation of the
six sense bases. With the cessation of the six sense bases, cessation of contact. With the cessation of contact, cessation of feeling. With the cessation
of feeling, cessation of craving. With the cessation of craving, cessation of
clinging.  With  the  cessation  of  clinging,  cessation  of  being.  With  the  cessation of being, cessation of birth. With the cessation of birth, cessation of}
\sphinxstyleemphasis{aging,  death,  sorrow,  lamentation,  pain,  distress  and  despair.}
\sphinxstylestrong{Such  is  the}
\sphinxstylestrong{cessation}
\sphinxstyleemphasis{of this entire mass of suffering.»}

\sphinxAtStartPar
So this formula, this dependent origination is a map that shows us very
clearly our options in the present moment. We are either dealing with ignorance, craving, self and dukkha, or we are dealing with wisdom, the cessation of craving, nirvana and freedom. The choice is ours to make.

\sphinxstepscope


\chapter{Day 7, morning}
\label{\detokenize{7-a:day-7-morning}}\label{\detokenize{7-a::doc}}
\LOCALaudiolink{https://www.mixcloud.com/anthonymarkwell/day-7-morning-talk-enlightenment-factors-equanimity/}


\section{Seventh enlightenment factor upekkha}
\label{\detokenize{7-a:seventh-enlightenment-factor-upekkha}}
\sphinxAtStartPar
This morning we’re going to have a look at the seventh enlightenment
factor. It’s known as
\sphinxstyleemphasis{upekkha}
or equanimity. We’ve already had a look at the
first  three  –  mindfulness,  investigation  and  energy  –  spinning  and  turning
on each other, doing the work of Vipassana. Then the next three factors start
to  be  developed  in  sequence  –  the  rapture,  tranquility  and  concentration.
When these have been developed, the seventh factor starts to come into existence or becomes stronger, starts to dominate the mental landscape. These
seven enlightenment factors, when they are developed and start to surround
consciousness, they determine what consciousness knows, and in particular,
how it knows stuff. So these are the workers, the assistances of consciousness, if you like, and they are actually the factors that cause consciousness to
awaken. So it’s the enlightenment factors, the causative factors.

\sphinxAtStartPar
Ajahn Chah, one of the foremost meditation master of Thailand from
the last century, says,
\sphinxstyleemphasis{«whatever arises in the mind, let it go, hold on to noth\sphinxhyphen{}}
\sphinxstyleemphasis{ing»}. Everything has to be let go of. Slowly but surely, we’re reducing and
releasing our bind on our mental and physical phenomena and let go of the
body and feeling, and then we start to let go of our opinions and views. We
start to shake free from our emotions. We recognize the repetitive thought
\DUrole{pdfpage}{247}  patterns and looping we get ourselves in. We start to see that stuff more and
more. And as the practice develops, you start to see it more and more clearly.
And  through  clear  seeing  the  mind  will  step  away  from  the  identification
process.  Stuff  still  arises!  Pleasantness  and  unpleasantness  is  still  arising,
but we have a much better ability to watch it and observe it, not being caught
in the story of our own mind. We’re freeing the mind, we’re setting it free.
We’re releasing the mind – just like a bird, when it’s released from a cage.
We  are  releasing  consciousness  from  mind  and  matter,  from  nama\sphinxhyphen{}rupa,
from the conditional vortex it got itself into. We’re releasing the mind from
craving  to  be  and  all  these  resultant  defilements  and  imperfections  of  the
mind that cause cloudiness and dukkha.

\sphinxAtStartPar
As our practice develops, the quality of our equanimity becomes more
and more natural. Equanimity means calmness or composure. It means evenness, it’s an even mind. Equanimity, it comes from latin. ’equas’ that means
equal, ’animous’ that means mind. Equal mind. It’s the equal mind. It doesn’t
sway to the left or right. It doesn’t get pushed by liking nor disliking. Not
being conditioned by pleasant feeling or unpleasant feeling. It sees through
the  feeling,  it  knows  feeling.  It  understands  it,  so  it’s  not  conditioned  by
it.  And  it  doesn’t  lead  to  reactions  like  disliking  or  identifying,  aversion
or anger. We start to drop that from our experience. Equanimity, when it’s
developed, gives us an evenness of temper. We don’t get so easily triggered.
We can be cool. That’s what it means, in the 1960s definition, if you like.
Yes, being cool with things, that’s what it means. Presence of mind, level
headedness, equilibrium, balance.

\sphinxAtStartPar
Like a strong tree in the forest, an old tree. It doesn’t get pushed around
by  the  wind  or  the  rain. The  branches  may  shake  a  little  bit  but  the  trunk
is  strong.  It  will  become  solid  in  the  sea  of  samsara,  in  the  ebb  and  flow,
through live’s conditions that we never previously even bothered to look at,
which will start to become quite clear to us. We start to see how our mind is
reacting to things and causing us dukkha. Not only do we see it happening,
but we can also start to do something about it. We can start to free ourselves
from the reaction mechanism. It is something that we all should have been
taught when we were children. The educational system, our families should
see that watching of emotional states, the thought patterns, would save a lot
\DUrole{pdfpage}{248}  of trouble for everyone if we understood, that we only have to observe our
mind and stop it from reacting so violently with craving and aversion. And
then the world would be a much more peaceful place. People would be much
happier. They wouldn’t be so tense and be caught up in things.

\sphinxAtStartPar
It’s not that Vipassana does bring these states to us! Vipassana doesn’t
activate  the  five  hindrances  for  example.  The  five  hindrances  are  always
there! It has been our experience for much of our life, but we just haven’t
turned  and  had  a  good  look  at  them  before. We  haven’t  used  a  technique,
where we can observe the mind. Now we are just being aware of them. We
are noting that they are there and we’re stepping back from life’s true conditions.

\sphinxAtStartPar
So this enlightenment factor of equanimity, when it is developed, can
bring our life into a much more balanced experience. We are not high on the
waves, we are not getting sunk down into the troughs. Cruising, on the freeway with the car in the fifth gear, we are cruising. Nothing is really bothering
us too much.

\sphinxAtStartPar
Equanimity allows us to become even minded towards the four foundations on which we are establishing our mindfulness. The physical sensations, the feeling, the emotions and the thoughts, having seen them, having
become aware of them, we practice with them. These are the objects of our
practice. Our practice is to train the mind to become cool with things. We see
things clearly, so we can let them go. This radically transforms our lives. We
stop being a pure reactionary machine, that’s just following our instinct like
an animal. When something arises, we react to it – it’s an immature way of
existing. And then we start to develop our mind, we train our mind so that we
can handle whatever comes along. There will be troubles in life. That’s without doubt. There will be big waves coming, smaller waves as well. But we
need to learn how to ride those waves. We need to be able to swim. We need
to be able to stand up. If we don’t, we’re going to get crushed. We’re being
aware  and  not  reacting  to  the  mental  and  physical  phenomena  with  liking
and disliking, but with equanimity. When we’re not reacting with like and
dislike, when the mind’s cool with it, whatever is arising, we’re cool, «yeah,
that’s fine.» We let it go. Then another thing arises and we let it go. We don’t
become boring by doing this. We slide into a very peaceful and blissful mode
\DUrole{pdfpage}{249}  of existence. It’s like a duck doesn’t get wet. The water just falls off its back.
It can exist with swimming or with rain. It’s fine. No problem. It’s like we
are insulating ourselves from trouble. And we can turn our mind and we can
activate our awareness in the present and observe our mind state. As soon as
we know that we’ve been triggered by something, as soon as we know we
have a little bit of agitation, a little bit of craving or lusting, addiction tendency – as soon as we feel that little trigger, bang, we are on it: we’ve noted
and know it, we have released the mind from it. It’s by not identifying with
these processes that one does not react. If we take it «for me» and «mine»,
then things are happening to «me» and «my self». When we’re not appropriating that stuff, then it is just dependently arisen, conditioned phenomena.
By not reacting one does not create suffering. It’s the reaction process, not
being cool with things, that makes suffering so intolerable. This is the power
of equanimity in the present moment. Very powerful, very liberating, releasing the mind from moment after moment from its own problems.

\sphinxAtStartPar
It’s important to emphasize that all our reactions are driven by the sense
of self. Things that trigger us are not just triggering an impersonal mind and
body process. It’s triggering a mind and body process that believe they are
somebody, the mind and body process that has turned to dukkha. When we
take  things  as  being  «mine»,  we  ever  believe  that  we  own  something  or
could own something. This is a type of craving, it’s a manifestation of craving. When we take things as «I», «this I am», «this is my opinion», «this is
what represents me», there’s a believe that this is what represents me. Our
opinions and views, we like to project them, we like to get them out there.
Some people like to write them down and publish them. Their opinions and
views are so important that everyone should listen to them. This is conceit.
There is a belief that in some way «I am». And thirdly, we come to a view
«this is my self». We believe that this is what represents me. We believe this
is who we are. «This is who I am.» «This is truly what I am.»

\sphinxAtStartPar
So these are just views, and they are all based on reactions of liking
to pleasant feeling and disliking to unpleasant feeling. If we really don’t see
pleasant and unpleasant feeling arising in the present moment, then we are
going to be its slave. Feeling has the power to push our mind, to condition
our mind, to throw us off. We really need to be aware when pleasant feeling
\DUrole{pdfpage}{250}  is arising. Of course, unpleasant feeling, when it’s arising it becomes quite
clear to us. «This is a negative, unfortunate situation. I should be mindful,
note it, know it and let it go.» That is reasonably obvious to us.

\sphinxAtStartPar
What about when it’s a pleasant feeling, the phenomena is associated
with pleasantness? It’s a little bit more difficult to note it, know it and let it
go. There is a strong sense of entitlement to this pleasure. «This pleasure is
mine», «I should be enjoying this», «this is what I have been working hard
for». So we go into that kind of mode. We go into that mode of appropriation of pleasant feeling but it too needs to be let go of. We need to become
equanimous towards it. Positive emotional states, like love for example, love
comes up, it is only associated with pleasant feeling normally. We will have
all experienced love with unpleasant feeling as well, when we often wonder
why we’re doing it to ourselves. And there is going to be pleasantness and
unpleasantness for sure. We need to watch the mind and make sure it doesn’t
get conditioned by this pleasantness. The power, and love, can become very
strong. The pleasant feeling will become very strong. We can become very
attached very quickly.

\sphinxAtStartPar
What does attachment do to the mind? It takes ownership of things. It
takes things as being «mine». Of course, that leads to tears. It’s attachment!
It  is  the  root  cause  of  suffering.  Of  course,  there  is  going  to  be  dukkha  if
you  are  attaching  to  somebody.  Of  course,  it’s  really  something  up  in  the
air. What’s going to happen to it? It’s going to fall. It’s the law of gravity.
The law of attachment, if you’re attached to something, you get dukkha. It’s
a quite simple, linear process: attachment equals dukkha. We don’t want to
often believe that however.

\sphinxAtStartPar
So be aware how your mind gets conditioned by various things. We’re
all  used  to  reacting  in  the  same  way  when  we  get  presented  with  various
situations. We get thrown off. We get turned up side down. All the objects
causing us problems. We are not watching them. I was once meditating with
a  friend  of  mine,  an  Irish  monk  actually. We  used  to  meet  every  morning
before  breakfast.  That’s  the  only  time  when  we  would  see  each  other  for
the  day,  living  in  the  forest,  quite  separated  from  each  other  and  he  came
down one morning and said, he had a breakthrough. (We used to just talk a
little bit about dhamma.) «Really! What’s happened? (That’s kind of things
\DUrole{pdfpage}{251}  that excites monks.) What’s happened? Something happened in your meditation?»  –  «Yes,  I  have  realized  I  am  only  attached  to  three  things  now.»
–  «Wow,  venerable,  that’s  fantastic.  Only  three  things. What  are  they?»  –
And he said: «People, places and things». (big laughter). Ok, that’s a great
breakthrough. Well done!

\sphinxAtStartPar
Ensured, as long as we are reacting to things, as long as we’re pushed
and pulled by our mind’s reaction to things, then there is a person who suffers. If we learn to observe directly and objectively, without identifying with
things, then we reduce suffering. When we have a sense of self and react a
lot, then we have a lot of suffering. When we have less sense of self and we
only react a little, sometimes being triggered, but most of the time being cool
with things, then we experience some suffering and no suffering. If we have
really a little sense of self and we are pretty cool with our practice and we’re
not  reacting  that  much,  then  our  experience  is  very  little  suffering.  Only
major things can come and really upset us. Little things are just whatever.

\sphinxAtStartPar
So you sense the size of your ego and how much you react to the phenomena presented determines how much dukkha you’re going to experience.
Be aware of this! It’s up to you! You have a choice! In every moment! You
have a choice if it’s going to be a moment of release or a moment of dukkha.
So that’s fairly simple to understand but practice, mhh, practice is telling us we have to be patient. We have to know that it is difficult to break
our  old  habits,  old  reaction  patterns.  It’s  not  going  to  happen  over  night.
There can be big breakthroughs though, there can be huge letting\sphinxhyphen{}goes and
breakthroughs, huge openings. We open up and release things which have
long caused us problems and dramas. The power of insight can happen quite
rapidly but mostly the path is a gradual training and a gradual practice and
a gradual progress.

\sphinxAtStartPar
When we understand it’s difficult to stop our reactions to things, once
we’ve been watching our mind for some time, we come to understand we are
a conditioned being. And that helps us appreciate other beings as well. This
helps us in becoming equanimous. When we see that other beings are conditioned and they’re reacting, they’re being triggered and they may not have
undergone  the  training  yet,  the  training  of  observing  the  mind  being  able
to see things clearly. They may just be in the reactionary stage of life blind
\DUrole{pdfpage}{252}  sided by reality just reacting to their thoughts and their emotions, pleasantness  and  unpleasantness.  They  may  be  just  pure  reactors. And  it  gives  us
compassion. We feel empathy toward them and this is a positive mental state
that can arise. The training not only benefits us but also benefits other people
as  well  because  our  interactions  and  our  relationships  also  become  more
balanced. We start to understand how others are trapped. We feel compassion  for  addicts,  for  example.  People  who  are  in  a  looping  state  of  mind.
Sometimes they even see the looping but are struggling to get out of that. We
support that. We support people who are working on themselves.

\sphinxAtStartPar
We are all used to reacting in similar ways. Presented with the same situation, we’ll probably react the same way again and again. Habitual. We’ve
either learned that from our parents or our society – some conditioning is
learned behavior, some conditioning is old karmic stuff as well.

\sphinxAtStartPar
By remaining equanimous in the moment, we allow old karma to bear
their fruit without new karma being produced. Just allow whatever is arising
to bubble up to the surface. We’re cool with it, we note it, we know it, it’s
gone. It’s gone. It’s gone. Whenever you feel yourself triggered or trapped
by an emotion or a thought pattern, note it and know it. Step back from it.
Allow vast layers of defilement to arise to the surface and pass away without
reacting.

\sphinxAtStartPar
On  this  topic  we  can  mention  here  the  precepts.  Following  the  five
precepts, or we have been taking the eight precepts here in the morning. An
extra three. Normally the eight precepts in Buddhist culture are reserved for
the full moon days and the new moon days. – Yes, it’s ironic here on Koh
Phangan, I know. The irony does not escape us – but once a fortnight, two
times a month we keep the eight precepts. Normally we just do five precepts.
Five precepts can, while it may seem a little bit moral or a little bit religious
or  sound  like  someone  saying  something  to  us  from  a  mountain,  we  shall
follow this and that – what the five precepts are actually doing is setting up
the conditions for the old karma that’s going to arise. When we arrange the
conditions  in  our  life,  when  we  maintain  not  killing,  stealing,  sexual  misconduct, lying and use of intoxicants, when we can maintain these precepts,
then what we’re doing is setting up the conditions for which old karma can
come. Old karma can only manifest if there is conditions in place. If we put
\DUrole{pdfpage}{253}  good conditions in place then the karma that can come is going to be good
stuff. If we don’t break the precepts, then what we are doing is effectively
not opening any doors for unwholesome karma to come and manifest in the
present. We’re blocking it with virtue. This is the beauty. This is the beautiful fragrance of virtue. We can manage which karmas start to attack us. Not
completely, not fully but we can definitely organize our lives in a better way.
We  can  limit  the  damage,  if  you  like,  and  promote  the  wholesome  in  our
lives. When  we  live  a  more  virtuous  life,  you’ll  understand  that  suffering
is diminished by this practice. Virtue is a practice. Nobody is just perfectly
virtuous and a wholesome person. We all have to practice these things. We
receive some instructions from our families and society on what is suitable
and  unsuitable,  but  mostly  it’s  about  practicing  ourselves.  Witness  it  for
yourself when you keep the precepts. Make a note of it: «Ah, the day is good
or the day is bad.» Does good karma give its fruit or does bad karma give its
bitter fruit? See how these things are conditioning each other. Stuff can only
arise when its conditions are in place. When this is, that comes to be. With
the arising of this, that arises. It’s impossible for this to arise if the conditions
are not there. When we break the precepts we put conditions in place for all
kinds of stuff to come up. So be aware of that. Particularly if you’re planning
to do some long meditation retreats. Purify your virtue before you do so.

\sphinxAtStartPar
The enlightenment factor of equanimity actually refers to a balancing
of energy. We are balancing stuff. It’s a state of mind which in the center
of  our  experience  is  inclining  to  neither  one  extreme  nor  to  the  other,  not
inclining to attachment or clinging or craving or liking and it’s not inclining
towards aversion or anger or frustration or irritation or disliking. It’s cool
with  whatever  is.  In  our  meditation  various  states  of  mind  are  competing.
You may have noticed that sometimes the wandering mind is taking control,
sometimes it’s mindfulness, other times the energy is weak and we become
lethargic, shoulders lower, back rounds. We get into that kind of ‘waiting for
the bell’\sphinxhyphen{}look. Faith tries to overwhelm our intelligence or our intelligence
thinks it’s much too clever for everything. It starts trying to look for holes
in the Buddha’s teaching listening attentively to try to find some errors or
mistakes  in  the  Buddha’s  teaching.  I  wish  you  good  luck  with  this.  I’ve
been searching for that, it doesn’t exist. The teaching is very complete! Our
\DUrole{pdfpage}{254}  effort competes with concentration. Balance in these pairs of mental states is
essential to maintain our direction and progress in our meditation practice.

\sphinxAtStartPar
We  develop  equanimity,  this  is  not  something  that  just  comes.  It’s  a
mental state which is to be developed and it gets stronger and stronger. It
may start a little bit weak, not fully exercised or trained. Maybe we’ve had
no need to exercise any restraint or control over our reactions before. Maybe
nobody even told us that we should stop reacting in such a way. Equanimity
is the ability to control our reactions. This is the heart of meditation practice.
This is how we reduce our experience of suffering in our daily life. In our
worldly life this is part of the practice that we use to really make big changes
in our life, to transform our experience on a daily level. We can really use
this noting and knowing to let go and free ourselves from emotional states,
to not become frustrated or irritated when we’re having a conversation with
somebody we don’t like or who’s opinion we don’t care for. We’ll be able to
tolerate that person without becoming frustrated or bored with them. Ok, we
can deal with various things.

\sphinxAtStartPar
Most of all, we can deal with ourselves. We’ll be able to deal with our
own internal reactions. The world is fine as it is. Things are arising and passing away. Some things are pleasant, some things are unpleasant. It is as it is.
It’s unfolding naturally according to its dhamma. Natural law. We can create
big problems for ourselves if we’re not willing to follow this natural law. If
you want to fight against gravity, then get ready to start lifting stuff. We’ll be
spending our time lifting things up trying to break gravity. That’s what we
try to do in our meditation practice. We don’t want to be breaking the laws
that are naturally in place. We want to be flowing with them, understanding
what they are, knowing that this path leads to the cessation of suffering and
this reactionary path leads to dukkha, leads to all kinds of self\sphinxhyphen{}opinions and
self\sphinxhyphen{}views. We  can  form  quite  ridiculous  opinions  about  ourselves  just  by
looping  in  a  mind  state.
\sphinxstyleemphasis{«Whatever  a  monk  frequently  thinks  and  ponders}
\sphinxstyleemphasis{upon, that will become the inclination of his mind.»}
We start thinking about
things, we’re reacting to things and we’re unaware that we’re reacting. What
we’re actually doing is conditioning our mind over and over again to react in
a certain way. And if that reaction process results in strong dukkha, then you
should be prepared. If you’re unwilling to work with your reaction processes
\DUrole{pdfpage}{255}  than you just have to crop all the dukkha that comes your way from reacting.
So our practice is to very much develop equanimity. This is useful in
our seated meditation practice on a retreat such as this. The development of
equanimity leads to even deeper states of mind, even deeper states of clarity
and seeing, the more prominent letting go, lots more freedom. In our daily
lives  we  can  pull  ourselves  out  of  little  troubles  that  we  meet  on  a  daily
basis. Little frustrations. People push in in the 7Eleven queue or taxi drivers
try to rip us off. 20 baht or 50 baht. Instead of exploding, we can become
cool  with  those  things.  We  start  to  understand.  It’s  better  just  to  be  cool
with things. «But it’s the principle!» Principles and views cause even more
dukkha when you’re holding on to your view or opinion about something.
Yeah, they are great dukkha\sphinxhyphen{}makers, those ones! Turn on your opinion and
views and see how much dukkha you can attract. We need to let go of this
stuff. We need to let go of this. But because we’ve been reacting so much
throughout our lives, it’s difficult to turn this boat around. But it’s possible.
Gradually and surely, we start to free the mind from our opinions and views.
So  the  characteristic  of  equanimity  is  balancing.  It  balances  the  corresponding mental states.

\sphinxAtStartPar
It’s function is to arrest the mind before it falls into an extreme of liking
or disliking, craving or aversion. Equanimity is right there in the moment if
we  develop  it.  It’s  there  in  its  undeveloped  state  but  if  we  can  develop  it,
it becomes stronger. It starts to overwhelm the other states. It starts to balance the mental states. It starts to bring them together and then one can free
oneself from the dukkha. It fills in where there is some lacking and if there
is some excess it balances that out. It sees where problems are being made,
where reactions are being created, what the real causes of the reactions are.
It sees that all reactions are based on egoic reactions. It’s always the «me»
that’s reacting. The mind and body process don’t react by themselves. They
are pure nature. When there is no «me» then there is no reaction. It’s gone,
no further production or generation of karma.

\sphinxAtStartPar
When  upekkha  is  strong,  there  is  total  balance,  no  inclination  of  the
mind to sway left or right to like or dislike. It’s very happy to sit right in the
middle and be balanced. It seems as if mindfulness is taking care of everything.  When  these  enlightenment  factors,  the  seven  enlightenment  factors
\DUrole{pdfpage}{256}  come together as a team, they become very powerful. The simile in the text
is given of a chariot or a carriage with three pairs of horses in front, so six
horses. Two, two, two. The driver of the carriage is mindfulness, awareness
to the present moment, attention to the present moment. On the right side, we
have the energizing states. Energy, investigation and piti or rapture. These
states  energize  the  mind,  excite  the  mind,  fuel  development.  On  the  other
side, we have the calming or tranquilizing states. The states of tranquility,
concentration and equanimity. These horses on both sides have to balance
each other out. And this is what we do in the mind. Energizing, noting and
knowing, letting go, tranquilizing, calming things down, becoming equanimous – these things are working together. This is samatha and Vipassana.
Samatha  is  calming  states.  Vipassana  is  knowing  and  seeing  states.  They
work together as a team, elevate consciousness out of dukkha.

\sphinxAtStartPar
We bring equanimity about through wise attention. Attending carefully
to the moment, seeing the causes and the motivations behind our actions and
reactions,  if  we  are  continually  mindful  from  moment  to  moment  without
a  break  based  on  the  intention  to  be  equanimous.  This  is  how  we  do  our
practice.

\sphinxAtStartPar
Wake up one morning and say today is a day of equanimity and cruise
through it. See if you can watch your mind. Count your reactions. See how
many go on the aversion side. How many frustrated, irritated little triggers
arise. See how many of them you react to and see if you can count any you
don’t  react  to,  in  which  equanimity  has  arisen. Try  to  manage  your  mind.
Keep a score if you like. And on the craving side of things, notice the tendency  of  the  mind  to  react  to  wanting,  to  craving,  to  desiring.  These  two
functions  of  the  mind,  pushing  and  pulling.  Do  that  for  a  day  and  have  a
look. Be supremely mindful and attentive to the moment and to your mind
and watch what’s happening. You may be surprised. You’ll learn a lot. You’ll
definitely be able to monitor your mind more successfully after you’ve done
this  exercise. You  will  have  seen  your  nature.  It  may  not  be  pretty.  Don’t
chastise  yourself.  Don’t  be  upset  with  yourself,  if  you  notice  that  you  are
a supreme reactor. Taking everything in and getting upset or getting super
excited  about  it.  Be  aware  of  what  is  happening.  Be  happy,  with  the  fact,
well, while there is reaction going on, at least you know it’s going on. At
\DUrole{pdfpage}{257}  least you’re one of the few people who can see reactions happening in yourself. And you’re on it walking the path now. You’re doing the work that has
to be done and we’ve all got different types of work to do.

\sphinxAtStartPar
Equanimity doesn’t arise easily in the mind of beginning meditators.
Especially on the retreat, all of you will have experienced zero equanimity
perhaps  on  the  first  two  days.  Remember  those  reactions.  Remember  how
we were resisting, dukkhering ourselves. There is nothing wrong with the
mind’s reaction. Once we go through a few of those events, we start to see
the nature of the mind reacting. We start to see dukkha. When we experience
dukkha,  we  know  dukkha  and  then  we  can  let  go. The  mind  will  become
balanced  for  a  while,  and  then  it  goes  off  and  reacts  to  something.  Don’t
become upset with this. It’s been doing this for a long time. It takes time to
make changes. Give yourself the space, give yourself the time. You’re not
going to become perfect over the weekend. It takes time, training and effort,
practice.  Once  equanimity  is  activated,  then  you’ll  see  the  difference,  the
quality difference it makes to your life. This enlightenment factor is really
one  which  makes  a  huge  difference.  It  really  cools  us  out.  It  changes  our
life, if we can just put it into practice. Now we’re here learning it. The real
practice comes when you’re going down the hill and we’ll talk about that
this afternoon. Eventually, we practice equanimity on a daily basis like this.
Eventually, it develops and becomes stronger to the point where we can call
it an enlightenment factor. It’s become strong enough, that it qualifies that
we can watch mental and physical phenomena arising and passing away.

\sphinxAtStartPar
We  should  try  to  be  equanimous  towards  all  living  beings.  To  creatures, to animals, we shouldn’t react to people. We become cool with people.
Just know that beings are conditioned. There are many different types, many
different  old  karmas  running,  streaming  along.  Sometimes  people  can  be
awful. Not because they are an awful person, but maybe because they had an
awful day. Maybe yesterday was just unbearable for them so they’ve woken
up feeling a little unpleasant. Don’t react to them. And to the different places
you visit. Things are different in different places. You may have noticed in
your travels. That’s what makes traveling so interesting. Going to another
country and seeing a whole group of people completely conditioned in the
same way. They believe the same things. They do things in the same way
\DUrole{pdfpage}{258}  as  each  other.  Observe  the  conditioning  of  people.  We’re  all  conditioned
beings. We can become equanimous, when we realize that. Not all people
have  undergone  the  same  conditioning  processes.  In  fact,  there  aren’t  any
two people who have undergone the same conditioning processes. We’ve all
been conditioned in different ways. So make space for that fact in your mind
before  you  become  upset  with  somebody  or  before  you’re  quick  to  judge
them, before you start reacting to them. Before you start reacting and judging a place and generalizing an entire community or an entire nationality of
particular  behavior  or  activity.  Develop  some  equanimity  towards  places,
and  the  things  as  well,  animate  objects.  Become  cool  with  stuff.  Don’t
become upset when things get broken or when things get damaged. This is
their nature, things are impermanent. We can’t expect things to last for ever.
If we’re expecting things to last forever, that means we aren’t seeing things
as they really are. Nothing lasts for ever. Nothing lasts even for a second. It
may repeatedly come into our mind however. Thoughts of things, addictions
to things, cravings for things, or repulsion towards things. So these people,
places and things, we need to develop equanimity on a wide and broad scale.
Most in the little things that are happening and in our meditation practice. As
our practice develops equanimity becomes stronger and stronger.

\sphinxAtStartPar
We should point out that equanimity is not insensitivity. It’s not indifference. It’s not apathy. It’s not the kind of mind that says, «oh well, that’s
just  your  bad  karma».  We’re  not  developing  indifference  towards  people.
We’re still caring. We’re not just becoming still, and «oh, so that’s just your
bad karma so you should deal with that, I’m not going to react to it». We’re
not becoming that kind of person. That’s not equanimity, that’s indifference.
That’s insensitivity. It’s a lack of empathy. No! Real equanimity is a decision
not to react to things. It’s simply non\sphinxhyphen{}preferential. It doesn’t make choices of
regarding some things better than others. We don’t push things aside that we
dislike. When something we dislike arises in our experience, in our world,
we  acknowledge  it. We  accept  that  it’s  arisen. We  note  that  it’s  there. We
come to terms with it. We deal with it in our mind. We accept it, «oh, well
look at that». Smile to yourself. Acknowledge that in the past you may have
got angry or upset about this. But now, you don’t. See the development! See
how the mind is changing and becoming trained. The mind rests then in an
\DUrole{pdfpage}{259}  attitude  of  balance  and  acceptance  of  the  way  things  are. We  see  the  way
things are and we accept the way things are. When we accept, we become
content and when we are content, happiness arises. This is the path to happiness. Becoming cool with everything. The hippies were right. Let’s become
cool and all groovy. Fine.

\sphinxAtStartPar
When equanimity is a factor of enlightenment and it’s present strongly,
we abandon both attachment to things and disliking of things and this is the
path to the end of suffering. It doesn’t mean that we don’t enjoy life. Don’t
misunderstand this. There still will be objects associated with pleasantness.
They will arise, this is our karmic legacy. Enjoy them while they’re there.
Taste the sweetness. Enjoy the coolness, whatever it may be. Enjoy the sensual touch. Enjoy! But don’t attach! Don’t use that object as a base for your
personality, for your sense of self. Don’t react and hold and cling. ‘Pleasantness is arising’ is ‘pleasantness is arising’. There it is! Liking is occurring.
There it  is! And it stays for a while. This is its nature. And  then  it passes
away – this is also its nature! We don’t get upset when pleasant things pass
away, and we don’t get upset when unpleasant things arise. It’s all going on
as it should be! It’s all nature unfolding naturally. It’s all dhamma. We can
become cool with everything and we find ourselves living with a mind that’s
being released from dukkha. You can do it! My teacher used to say,
\sphinxstyleemphasis{«you can}
\sphinxstyleemphasis{do it in this very life!»}
He did it in this very life. It’s quite possible. We just
need to train the mind.

\sphinxAtStartPar
Think of all the things we do to train ourselves. Think of all the years
we’ve  spent  in  school  listening  to,  sometimes  dubious  subjects  that  have
little  to  do  with  our  live  or  have  any  interest  to  us. We  spend  time  doing
that! How many exams do you sit? How many semesters did you go to the
lectures and listen, write all that information, remember and study it, pass
the exam. As soon as the exam is finished – into the bin with a lot of it. «Oh,
that  stuff.»  Completely  useless.  Won’t  need  to  do  any  statistical  analysis
ever  again.  It’s  gone.  So,  we  spend  ourselves  training  doing  that  kind  of
stuff. Why not spend some time actually training the mind to give yourself
a truly better life.

\sphinxAtStartPar
Somehow we think having a job will make our life better. Of course,
you have to earn a living in some way preferably with right livelihood. But
\DUrole{pdfpage}{260}  have a look at the effort in our lives to train ourselves and see the results.
Was it truly important that kind of training? And have a look at the training
that should be done, to make our lives not only tolerable, but blissful as well.
There is a lot of bliss when we are in the present moment and the mind is
equanimous.

\sphinxAtStartPar
So equanimity is of tremendous importance, both in the practice and
in everyday life. We generally either get swept away by people, places and
things, reacting, dukkhering ourselves – we get ourselves worked up into a
state of agitation and distress, we allow our mind to get into a loop of craving
and wanting trying to satisfy the unsatisfiable, running after cravings little
wants. Be gentle when you notice this happening in your mind. Be careful
when you notice this conditioning. This is a long\sphinxhyphen{}term project that we are
undertaking, training the mind. The sooner and the more regularly we do the
practice, the less dukkha you’ll have in your life. It’s about training.


\section{Third jhana}
\label{\detokenize{7-a:third-jhana}}
\sphinxAtStartPar
As  our  meditation  practice  unfolds,  we  start  to  develop  deeper  and
deeper  states  of  concentration. Yesterday,  we  started  to  have  a  look  at  the
jhanas. These states of meditation, part of
\sphinxstyleemphasis{samma\sphinxhyphen{}samadhi}, the four jhanas.
Yesterday we looked at the first and the second one seeing how these deep
states of stability result from letting go. We let go of the sensual pleasure.
We let go of unwholesome states like the hindrances. We let go of the past
and the future. We let go of the external world. We stabilize our attention in
the present moment internalized, watching physical and mental phenomena.
We let go of the pleasant or the unpleasant physical phenomena. We let go
of the unpleasant mind states. We let go of thinking! Until we get our mind
into  such  a  state  where  the  momentum  is  constant,  our  awareness  is  well
established, the body and mind are experiencing piti, rapture and happiness.
We’ve entered into the second jhana, which is comfortable and concentrated
and we start to see things as they really are. There is an internal assurance
about our practice.

\sphinxAtStartPar
But as we keep developing, particularly with the enlightenment factor
of equanimity, our meditation practice progresses even further than this.

\sphinxAtStartPar
\sphinxstyleemphasis{«And furthermore, with the fading away of rapture, he remains equanimous, \DUrole{pdfpage}{261}  mindful and alert still sensing pleasure with the body, he enters into
and abides in the third jhana on account of which the noble ones announce,
equanimous and mindful he has a pleasant abiding. He permeates and pervades,  suffuses  and  fills  this  very  body  with  the  pleasure  divested  of  rapture, so that there is nothing of his entire body unpervaded by the pleasure
divested by rapture.»}

\sphinxAtStartPar
So  rapture  at  this  stage,  that  fourth  enlightenment  factor  that  we’ve
developed to get into the first and second jhanas start to be let go of in the
third  jhana.  In  fact,  we  need  to  let  go  of  it.  The  electric  buzz  of  piti,  the
excitement, the thrill, the pleasure of piti starts to be noted and known. We
recognize its function to energize the mind and energize the practice. And
we notice that what it does, is make things very interesting and enthusiastic.
But we also notice, that this too, is a little bit shaky. It’s still not supremely
stable. We take a look at the photo we’ve taken when we’ve been experiencing piti, it’s still a bit unfocused. The piti is still disturbing the image a little
bit. And as we keep practicing, noting and letting it go, the fifth enlightenment factor of tranquility starts to become more and more powerful. It starts
to overcome the piti. It starts to cool the piti down. Piti doesn’t completely
disappear, but it gets overwhelmed by tranquility and the pleasure that arises
from piti, the pleasantness, the vedana of piti gets replaced by the vedana of
tranquility. It’s another type of pleasantness, another type of vedana but it’s
more subtle. Instead of being an excited state of mind, instead of being on
the dance floor, we’ve transitioned to a nice lounge and we start to relax. The
mind goes into a comfortable, calm couch or sofa. The body becomes very
comfortable. The pleasure that arises is deep and cool. The mind stabilizes
and enters into the third jhana. There’s equanimity arising now. The mind
has become equanimous. It’s not perfect though! That pleasure is still a little
bit agitating. Even though it’s cool, even though it’s calm, there is still some
pleasantness there.

\sphinxAtStartPar
The  instruction  of  the  Buddha  is  to  fill  and  suffuse  and  pervade  this
body with the pleasure that arises independent of rapture. That is the pleasure that arises from tranquility. We suffuse, we fill this body with this tranquil  pleasure. The  body  becomes  still,  super  still,  super  calm.  Completely
balanced. You won’t want to move, even an eye lash, you’re so comfortable.
\DUrole{pdfpage}{262}  And  this  is  an  experience  of  the  third  jhana.  The  nobel  ones,  the  sangha,
those members of the community, who have managed to break through to
the dhamma, even they say the third jhana is the most perfect. There’s a great
deal of pleasure. A great deal of spiritual pleasure. A pleasure which goes far
beyond the little bits of excitement we can create with the body or mind in
sensuality. The sensual states don’t even compare.

\sphinxAtStartPar
The Buddha gives us a simile:
\sphinxstyleemphasis{«Just as in a blue, white or red lotus
pond, there may be some blue, white or red lotuses which born and growing
in the water, stay emerged in the water, flourish without standing out of the
water so that they are permeated, pervaded, suffused and filled by the cool
water, there being nothing from their tips down to their roots unpervaded by}
\sphinxstyleemphasis{the cool waters.»}
Even so we take this pleasure and fill the body, permeate,
pervade, suffuse and fill this very body with the pleasure divested of rapture.
Nothing of this entire body remains unpervaded by the pleasure divested of
rapture. So this cool and calm state – piti is left behind in the lower jhanas –
another thing has been let go of and we move to a state of much bliss, much
happiness, a lot of happiness, a lot of calmness, a lot of stillness and stability.
Very pleasant states of mind.


\section{Fourth jhana}
\label{\detokenize{7-a:fourth-jhana}}
\sphinxAtStartPar
But even this is not the end. Even this has to be let go of. You’ll be
attached to it. It will make you cry when you have to let go but do it and go
to the higher state!

\sphinxAtStartPar
\sphinxstyleemphasis{«And  furthermore,  with  the  abandoning  of  pleasure  and  pain,  with
the earlier disappearance of joy and grief, he enters into and abides in the
fourth jhana, which has a purity of mindfulness due to equanimity with neither  pleasure  nor  pain.  He  sits  permeating  the  body  with  the  pure  bright
awareness so that there is nothing of his entire body unpervaded by the pure
bright awareness just as if a man was sitting wrapped from head to foot with
a white cloth so that there would be no part of his body to which the white
cloth did not extend. Even so a monk sits permeating his body with the pure
bright awareness. There is nothing of his entire body unpervaded by the pure
bright awareness.»}

\sphinxAtStartPar
So when we start to let go of not only the unpleasant physical phenomena, \DUrole{pdfpage}{263}  unpleasant mental phenomena, we’re starting to let go of the pleasant
physical phenomena, the piti, and we’ll start to let go of the pleasant mental
phenomena,  sukkha,  happiness,  the  pleasure.  We’re  letting  go  of  almost
everything. We’re really letting go of things, attaching to nothing. And when
this happens our mindfulness becomes perfected. We have
\sphinxstyleemphasis{samma\sphinxhyphen{}sati}. This
is the perfection of mindfulness due to equanimity. It’s the equanimity that
gives  it  its  perfection.  It’s  the  perfect  state  for  observing,  for  noting  and
knowing,  it’s  the  perfect  state  for  breaking  through  to  the  other  side. The
mind is still and calm. The other factors have been performing their duties.
We become still. We become quiet. The mind becomes very, very radiant.
Lightness and brightness accompany it. We can shine it throughout the body.
In fact it starts shining out through the pores in your skin. There’s an experience of the fourth jhana. It’s the perfect state for awakening, for breaking
through.

\sphinxAtStartPar
Now, this is how the development of the seven enlightenment factors
comes to fulfill the noble eightfold path. All the factors of the noble eightfold
path are developed through these enlightenment factors. In fact, the noble
eightfold path manifests when the enlightenment factors are fully topped up,
when they’ve all been trained and developed. When they’ve all been trained
and developed we see very clearly dukkha, we see the cause of dukkha, we
see the end of dukkha, and we see the path that just made this happen. The
four noble truths are exposed to us and we awaken in the moment to how
things really are.


\section{Determined striving}
\label{\detokenize{7-a:determined-striving}}
\sphinxAtStartPar
When we’re doing the training – there is a wonderful text here:
\sphinxstyleemphasis{«Deter\sphinxhyphen{}}
\sphinxstyleemphasis{mined striving»}
– it’s up to you how much you want to strive, how far you
want to take this teaching. We’ll talk about this this afternoon. But it’s up to
you do decide how far you want to take it. How far do you want to progress
in  this  very  brief  birth  experience,  in  this  very  brief  life?  How  far  on  the
path can you make it? This is all we have to do. This is all our evolution.
All the rest is just fluff and non\sphinxhyphen{}sense. Just the stuff of «me» and «mine».
Just stories about «me», about «mine». Identity development and personality growth. And then we die! And then we start it again! With a new body to
\DUrole{pdfpage}{264}  attach to and a new identity and a new nationality in a new physical location.
And  we  start  to  develop  that  one  again. We  develop  our  identity  until  we
have a full load of dukkha and then we die. And then we do it again! New
circumstance, new body, new experience. Craving is still the same. We got
into the habit of dividing things up temporarily, into a time sequence. We
believe  that  there  is  a  past  and  a  future. We  believe  that  there  is  this  life,
and that there is possibly past lives or future lives. Some of us believe that.
Some of us are not quite there yet, haven’t understood how the conditional
structure is operating. Don’t worry about that. Deal with this existence! Try
to make as much progress as you can.

\sphinxAtStartPar
\sphinxstyleemphasis{«And  how  is  exertion  fruitful,  bikkhus?  How  is  striving  fruitful?»
«Here  bikkhus,  a  bikkhu  is  not  overwhelmed  with  suffering  and  does  not
overwhelm oneself with suffering. And he does not give up the pleasure that
is  caused  with  dhamma,  and  yet  he  is  not  infatuated  with  that  pleasure.»}
He does not give up the pleasure that is caused with dhamma, and yet he is
not infatuated with that pleasure. We don’t have to give up all the pleasant
things! We just don’t have to become infatuated with them. Infatuation leads
to  attachment. Attachment  leads  to  dukkha.  So  enjoy  the  nice  things  that
come into your life! And let them pass away. Don’t try and hold on to things.
Empty  your  bag. You  want  to  travel  around  with  a  30  kilo  backpack  or  a
five kilo backpack? Which one is lighter? Which one is more comfortable?
Which one will give you dukkha? The more stuff you got, both physically
and mentally, the more stuff you’re holding on to, the more dukkha you’ve
got. So open up your backpack and start to throw things out. Removing all
those old attachments, the old things we’ve been clinging to. Old modes of
being and old modes of existence. Maybe the use\sphinxhyphen{}date of these old modes of
being is over. They don’t have to fulfill those functions anymore. You’re not
a student at school, you’re not living at home with your parents, you don’t
have to fulfill all those little things anymore. You can start to drop things.
You can empty your load.

\sphinxAtStartPar
How do we strive?
\sphinxstyleemphasis{«He knows thus, when I strive with determination
this particular source of suffering fades away in me because of that deter\sphinxhyphen{}}
\sphinxstyleemphasis{mined  striving.»}
This  is  activating  our  awareness  in  the  present  moment,
noting, knowing and letting go. This is satipatthana.
\sphinxstyleemphasis{«And when I look on
\DUrole{pdfpage}{265}  with equanimity, this particular source of suffering fades away in me while}
\sphinxstyleemphasis{I develop equanimity.»}
This is the enlightenment factor that we are developing of equanimity. So he strives with determination in regard to that particular source of suffering which fades away in him because of that determined
striving through noting and knowing, and he develops equanimity in regard
to that particular source of suffering which fades away in him while he is
developing equanimity.

\sphinxAtStartPar
So the process of eliminating dukkha from our lives is a two\sphinxhyphen{}fold one:
determined striving and looking on with equanimity. «Determined striving»
is doing the practice, keeping your mind in the present, noting and knowing and letting go. That’s our determined striving and then we look on with
whatever has arisen, whatever is there, we observe it with the mind through
equanimity.

\sphinxAtStartPar
\sphinxstyleemphasis{«When he strives with determination, such and such a source of suffering fades away in him because of the determined striving. Thus that suffering
is exhausted. – Finished! – And when he looks on with equanimity such and
such a source of suffering fades away in him while he develops equanimity.
Thus the suffering is exhausted in him. This, monks is how exertion is fruitful}
\sphinxstyleemphasis{and striving is fruitful.»}

\sphinxAtStartPar
In this passage we can see that the two activities of noting and knowing and the activity of equanimity development are very useful in removing
dukkha from our lives. Two ways! We note and know and we become cool
with it. And then something else arises. We note and know, we become cool
with it. Whenever a particular source of suffering arises, perform this dual
activity.  Perform  this  meditative  practice  in  your  mind  and  see  the  suffering fade away! See it disappear! The power of awareness and wisdom and
equanimity is much stronger than the power of defilement. Defilement will
be washed away if we can activate our awareness.

\sphinxAtStartPar
So this is the completion of our discussion of the enlightenment factor
of equanimity. This afternoon we’ll have a look at some practices we can do
in our daily lives when we leave this place. Things we can do to activate the
dhamma in our life after the meditation retreat.

\sphinxstepscope


\chapter{Day 7, afternoon}
\label{\detokenize{7-b:day-7-afternoon}}\label{\detokenize{7-b::doc}}
\LOCALaudiolink{https://www.mixcloud.com/anthonymarkwell/day-7-afternoon-talk-integration-in-daily-life-and-ten-perfections/}

\sphinxAtStartPar
It’s the last day, the last talk of our meditation retreat. All week we’ve
been discussing the practice of satipatthana or the foundations of mindfulness. We’ve been practicing satipatthana putting it into practice in the present  moment,  developing  our  wisdom  in  the  present  moment.  We’ve  been
listening to these talks. You’ve been doing the practice here in a secluded
environment,  in  a  retreat  environment  where  we’ve  been  able  to  note  and
know, start to see the functioning of our mind, even overcome some emotional states. We’ve all had the ability and the chance to overcome any resistance we may have experienced, overcome any emotional state or some kind
of habitual thought pattern.

\sphinxAtStartPar
When we come out of the retreat, sometimes we have big expectations
that somehow life is going to be miraculously transformed, we’re going to
leave this place as a teflon\sphinxhyphen{}surfer and never be touched by anything again.
We think we’ll deal skillfully with whatever comes our way. And we can and
we will! There just may be some gaps in our awareness, when we leave here.
From  my  own  experience,  having  been  in  retreats  for  many  years,
more than ten years, I found transitioning to the West quite difficult. A lot
is thrown at you outside of the retreat environment. Training and practicing
is one thing, having to deal with the stress and pressures of modern life is
quite another. But the training does work. It works very well! It’s difficult
\DUrole{pdfpage}{267}  to implement and it’s difficult to make solid progress when we have a very
busy life style, but it’s not impossible. And it’s very possible to transform
our daily life in a very meaningful way. This teaching can be used in a very
meaningful way to change our expectations of life and to also allow us to not
get caught up so much in the stories of our mind, get caught up in the stories
of our imagination, get caught by craving – that’s what we’re getting caught
by. We  can’t  expect  to  go  on  living  the  way  we’ve  been  living  thinking  a
little bit of dhamma is going to fix everything. We’ll just incorporate some
mindfulness and everything will be fine, all our problems will be solved. It
rarely works like this! There’s always some adjustment to our lifestyle that’s
to be made if we want to make progress on the path. Certain activities to be
let go of, certain people to be let go of. Certain things to be developed, certain practices to be done, association with people – these are all things that
need to be developed.

\sphinxAtStartPar
How far you want to take this is up to you! You can choose how far
and  how  fast  you  want  to  walk  the  path. This  week  we’ve  been  detailing
some instructions, not only on the basic practices but on the full development of the path practicing in a very intense fashion. Noting and knowing,
letting go, developing our awareness continuously, persistently, overcoming
any doubts. Brushing away any boredom or laziness, not reacting to sensual
desire or ill will, trying to calm the restless mind.

\sphinxAtStartPar
We’ve  been  working  hard. You’ve  all  been  working  very  well. Very
pleased with the way you’ve been working this week.

\sphinxAtStartPar
Some of you have expressed some interest in wanting to know how we
can put this teaching here, which we have been following, into our daily life.
I’ll talk about that in a moment. But the path we’ve been explaining here is
not impractical. Yes, I realize you have a busy life. We know that you won’t
be able to meditate for 17 hours a day every day in your normal daily life.
We are aware of that.

\sphinxAtStartPar
So there needs to be some adjustment to the training and to your expectations. We’ve been delivering a very high level meditation technique to you
here this week. It leads us to removing gross levels of dukkha and takes us
all the way to full and final liberation from samsara.
\sphinxstyleemphasis{The entire training is}
\sphinxstyleemphasis{contained in this teaching.}
How far you want to develop that is going to be
\DUrole{pdfpage}{268}  up to you! It’s your choice.

\sphinxAtStartPar
Given some of you are playing football. It’s your choice. If you like
to play football, you can decide «I like to play football in the park with my
friends after work». And that’s a level of football you might be comfortable
with. If you really like chasing balls, well you can go to the local club. Join
the team. Maybe they train once a week and have a game, a social game,
a social match on the weekend. You can train at that level if you want to.
Bring football into your life a little bit more, a more serious level. If you’re
really into it, you can go to a higher league, maybe try out for the state team.
Maybe you want to try to get into the national team or the olympic team.
Then you have to start training more regularly. You have to start taking care
of your diet, you have to start going to the gym for your muscle development,  you  have  to  start  listening  to  training  advises  and  training  videos.
You’ll have to start training! You’ll have to start practicing, practicing with
the senior players, developing your skill. You can take it to this level, the
state level. If you really want to get going, if you’re really into football, then
you can try to get in one of the premier leagues. It depends how much you
like football. How far are you willing to train? That’s for each person to be
decided for themselves.

\sphinxAtStartPar
It’s  up  to  you!  The  information  has  been  delivered.  The  teaching  is
there. You’re all intelligent enough to be able to listen and to focus on the
teaching. How far do you really understand it? Do you know what needs to
be done? Has this sunk in? Or, you’re just looking for something that’s going
to cure some little problems in your life and then go on as normal. That’s
fine  too.  The  Buddha’s  teaching  is  for  people  who  want  to  practice  at  all
different levels. You don’t have to become a nun, shave your head and live
in the forest in a cave. Although you can! I know there’s a few of you thinking about it. Maybe the guys don’t want to become nuns, you can become
monks.

\sphinxAtStartPar
You can decide that for yourself. In fact, while we’re talking about it,
I encourage you. Be careful for what you wish for. A period of training as a
monk or a nun can be very rewarding. It’s not compulsory to become a monk
or a nun for the rest of your life. It’s a period of training in the Buddha’s
teaching. A time we go in for intensive training. We don’t have to do it for
\DUrole{pdfpage}{269}  ever. It’s very beneficial.

\sphinxAtStartPar
So we can choose to live very simply by dharmic principals. We can
maintain  our  level  of  virtue.  Difficult  sometimes  in  our  social  lives.  Our
social activities are leading us sometimes into different directions. We can
also watch our intentions. Be aware of your intentions! Is your mind filled
with intentions of renunciation, loving kindness and compassion or are you
leaning  toward  accumulation,  getting,  achieving,  becoming?  Have  a  look
at that. Know, one leads to happiness and the other leads to dukkha. Have
a look a your intentions. Use wise attention on a daily basis to examine the
motivations, the reasons and the causes for what you want to do, for what
you are doing.

\sphinxAtStartPar
If  you  need  to  make  a  decision,  examine  it. What  are  the  criteria  on
which you are you’re making your decision making? Examine that. Examine your motivations. Try to see things. Well,
\sphinxstyleemphasis{there’s two choices: this one
is beneficial and wholesome to many people and this one is just greedy for}
\sphinxstyleemphasis{myself}. You have a choice! Have a think about it! Choose what is wholesome
not  what  is  unwholesome,  what  is  positive,  what  is  not  negative.  Choose
something suitable and beneficial. If you’re going through a life transition,
if you are turning your life around wondering to make a new start, a new
career, something, a new job – when you’re thinking about that, when you’re
planning that, use wise attention. Have a look what kind of person you want
to  be  in  society.  If  you  need  to  make  a  job  change  or  career  change,  and
everything’s  open  to  you  and  you’re  not  quite  sure,  then  you  can  choose
something that’s beneficial for others.

\sphinxAtStartPar
Yes,  of  course,  you’ll  be  able  to  make  a  livelihood  from  it. You  can
make some money as well. You have to support yourself. Everybody does.
The Buddha certainly agreed with that. That’s why right livelihood is the fifth
factor of the noble eightfold path. We have to earn our livelihood in a honest
and in a way with integrity. Truthfulness. When we need to make decisions
about these things, choose a job that’s going to be beneficial for others, not
just a job that’s going to get you more money. Money’s not the big criteria
here. Money comes and money goes. Sometimes it’s around, sometimes it’s
not.  Don’t  orientate  your  life  towards  the  collection  of  money.  That’s  not
what we’re here for. If you do a good job and you’re a reasonable person,
\DUrole{pdfpage}{270}  money is going to be coming in. That’s fine.

\sphinxAtStartPar
We’ve  been  saying  all  week,  that  the  Buddha’s  teaching  is  about
removing the sense of self from the picture, erasing the ego taking things as
not «mine», not «I», not «myself». We need to understand that selfishness
needs to be reduced if suffering is to be reduced. Happy people are the ones
that have reduced their sense of self to the point where they’re doing things
for others. Not only for money, but they’re doing things because they like
to give.

\sphinxAtStartPar
We also need to understand that any situation that we find ourselves in
is a presentation of the manifestation of our old karma. Everything is arising
through cause and condition. There is nothing that happens by coincidence.
Everything  is  happening  for  a  reason.  It  has  conditions.  If  there  is  certain
conditions in place, this is what we experience. Don’t appropriate that experience. Stop holding on to it as being «me» or «mine». Brush it off, let it go.
Things happen to us. Someone abuses you or is rude to you, don’t hold on
to that by hating them – you cause yourself a lot of dukkha. Laugh it off!
Let it go! The Buddha said, we should be like the earth if someone speaks
rudely  to  us.  The  earth  doesn’t  respond  if  someone  pours  dirty  liquid  on
it or abuses it in some way. Make your mind like the earth! Make it sand!
Whatever liquid you’re going to pour into it, it’s just going to disappear. Let
it run through. It goes in one ear, and out the other. That used to be a term
of abuse. It’s great when it does that! Perfect! We don’t have to hold on to
any  non\sphinxhyphen{}sense  speech.  Somebody  says  something  rude  –  through  it  goes.
Straight through to the keeper.

\sphinxAtStartPar
So, in our daily lives, we can bring in the dhamma in quite a few ways,
many different ways. The most obvious one is to do some sitting meditation.
Organize yourself a place and a time to do some sitting meditation. The time
of the day is sometimes tricky to organize when we’re social, when we have
jobs, when we’re busy, when we’re traveling especially. I suggest you find a
time in the morning some time when the demands of the family and friends
or the boss is not upon you. Early in the morning – great! Get up half an
hour earlier. Set yourself a goal how long you want to sit and do it. Try not to
start too long. When you leave here, you’re going to be doing three hours in
the morning and three hours in the night, hmmm. Don’t start with such high
\DUrole{pdfpage}{271}  goals or priorities. Just start with 20 minutes or half an hour. If it’s a good
sitting you can sit longer. The way I started sitting was by using an incensestick. It takes about 45 minutes. That’s your effort for the morning. Be aware
with this technique, however, you may end up with a little bit of a winker,
always looking at the incense, seeing whether it’s finished or not. Try to use
your nose, that’s also functioning. You all know that joy. We used to have a
clock in this room, I took it out!

\sphinxAtStartPar
So, setting up yourself to have a sitting practice is a good idea. Try to
find a place at home. If you have a special place, if you can devote a whole
room even better! If you can’t find a room, find a corner of a room. Get a
sitting  mat  and  place  it  there.  That’s  the  place  you’re  going  to  meditate.
And  make  a  time  to  do  that.  Have  a  little  corner.  If  you  like  candles  and
incense, by all means do that. And be strict with it. The most difficult thing
of  establishing  a  sitting  meditation  practice  is  just  actually  sitting  down.
Once you start sitting it’s fine. You’ve crawled down to the floor and folded
your legs and sat down. It’s fine. It’s just walking over to the mat and sitting
down  that’s  quite  often  the  difficulty.  Find  a  time.  Evenings  can  be  often
interrupted by social activities. Meal preparation and such things. Once our
schedule gets busted a few times, it’s hard to get it re\sphinxhyphen{}established again. We
go from seven days a week down to three days a week and then occasionally
to a couple of times a month and then to zero. And I’ll see you back here in
a year. Come for a retune. Yes, welcome back, people.

\sphinxAtStartPar
So find a place. Even better in our cities, in our towns these days are
many sitting groups. You can find them online. Find a meditation group to
join. It doesn’t have to be this type of meditation practice. Just as long as
they  meet  together  and  they  sit  in  silence. You  can  do  your  own  practice.
This has many benefits. The group energy that we’ve been experiencing here
this week pulling us all to the hall. You can feel what a good group energy is
like. Everybody’s coming. Ok, everyone else is doing it, so I have to do it. I
better take myself along. It’s a good incentive. I very much doubt, if I said,
ok, meditate so many hours in your room, that you’ll be doing it by yourself.
Come and do it as a group. So join a group. You may meet some new and
interesting people as well. It’s good to meet some dhamma friends. The venerable Ananda once said to the Buddha:
\sphinxstyleemphasis{«Venerable sir, the holy life, associ\sphinxhyphen{}}
\sphinxstyleemphasis{ating with good friends, paleana mitta, is half of the holy life.»}
\DUrole{pdfpage}{272}   The Buddha
said,
\sphinxstyleemphasis{«Oh, Ananda, do not say so. Do not say so. This holy life is one hundred}
\sphinxstyleemphasis{percent  lived  with  good  friends.»}
So  find  yourself  some  good  friends  that
can not only lead you along the path but be your companion upon the path.
Introduce  it  to  your  existing  friends  or  find  some  new  friends. You’ll  find
that sitting groups often end up becoming quite social. It’s a bit of meditation, it’s a lot of coffee drinking, an enormous amount of gossip and chatter
and we’ll see you next week. And you have some new friends on Facebook.
So join a group if you can find one. There are monasteries all over the place.
There are meditation centers. There are Vipassana centers. Different styles
and different techniques. You’ve learned one of the techniques here. It’s the
technique on which all the other techniques are built, the satipatthana. They
all come from the same source. Slightly different approach in some ways but
pretty much the same.

\sphinxAtStartPar
You can also do meditation around the town. Many of us have to take
public transport when we go to work. Ok, maybe chanting is a little bit too
much for the train but you can certainly meditate. Find yourself a place, sit
down – don’t sit down cross legged on the floor of the train or bus, it may be
a little weird – sit on a chair, close your eyes, bring the attention to yourself.
Loving  kindness  practice  is  really  wonderful.  Sit  in  the  train  with  all  the
‘happy’ people at 7.30 and see the ‘happiness’ all over their face if they go to
work. You can smile, radiate. I warn you about the slow walking meditation.
That may be a little bit too much. You may have some authorities coming to
visit you. «Are you okay, sir?» Just answer them really slowly, «yeeaahh».
So we can do that on the train. We can do that on the bus. When you
go to a place with many, many people, a big shopping mall or a big sporting
field, a big train station, radiate loving kindness. Imagine all those people.
Just send them your love.

\sphinxAtStartPar
Mostly our practice at home will be some form of maintenance practice
where we maintain our development. We have some good sittings, some are
a bit boring. But basically we are just maintaining as we go along. Developing, strengthening the practice, taking it to the level that we already have,
becoming fluent with the meditation. We are practicing. And when we want
to go a little bit deeper, when we feel that we have stalled a little bit, then we
\DUrole{pdfpage}{273}  can come and do a retreat. Seven days like this or longer retreats. 20 days,
two months, seven months, three years. As you like. And you can make some
rapid development. You will lose all your friends, but you will be quite wise.
So do some retreats. Bring them into your life. There are lots of them around
the world. Get some dhamma friends. This is very useful for our practice.

\sphinxAtStartPar
Mostly the practice is done, however, not just in half an hour sitting
sessions. The real practice is done on a daily basis not on an hourly basis.
We all face unwanted situations from time to time. We all lose our balance
from  time  to  time.  We  generate  negativity  in  different  ways  when  we  are
triggered. And we always look for an external cause. «That is what did it. It
was him. It was that. That was the problem.» We start to blame others but the
problem is not there. Because the problems is in here, it’s in our own mind
and body process. So be prepared to confront your reaction patterns. This is
what we do with the dhamma in our daily life. Try to bring and activate your
awareness of the present moment into your daily life. 20 minutes or half an
hour sitting in the morning is one type of practice. Bring your awareness of
the present moment even just 20 or 30 seconds, even just a minute an hour.
That’s when we start to get the practice rolling. It is in reminding ourselves
– this is another aspect of the word sati, it means remembrance or bringing
to memory, recollection. We are reminding ourselves to practice. Regularly.
You  should  be  able  to  bring  your  yourself  into  the  present  moment  for  at
least a few moments every hour. Put aside some time. Find those activities
where you find yourself waiting.

\sphinxAtStartPar
A good one to do is when you are having a cup of tea or a cup of coffee.
You made the tea, it’s in the cup, it’s too hot to drink, put it on the table and
just wait a few minutes. Cool down, bring yourself into the present moment
and start noting. See what’s there, see what’s in your mind. Check your attitude. See how things are. Contemplate your reactions. Have you made a mistake? Do you need to apologize? Did you let yourself go some way? Bring
yourself back! Overcome any emotional state that you may have for the day.
Overcome any issues that you may have with a person. Clear it up. Don’t let
it fester in the mind. Festering is what causes a lot of dukkha, especially in
relationships with people, especially in the work environment.

\sphinxAtStartPar
Work is one of those strange things you have to go and spend eight, ten
\DUrole{pdfpage}{274}  or 12 hours with people that we wouldn’t normally hang out with. And yet
we have to be with them very closely throughout the day. At times this can
cause problems and we bring those problems home into our personal life.

\sphinxAtStartPar
Deal  with  those  problems  as  they  come  up.  Don’t  have  a  collection
of  ill  will  or  animosity  to  different  people  here  and  there.  Don’t  burn  any
bridges. It makes sure everything is kept comfortable. Make an effort to do
that. If you’re not a person that usually forgives and apologizes, if you have
an ego that won’t allow you, pride that doesn’t allow you to do that, then start
working with that. Start to see that as a dukkha\sphinxhyphen{}inducer. This causes dukkha
in your life. Free your mind from that. Work with it. Be on the lookout in
your daily life for things that you need to work on. We all have our own little
things.  Some  get  angry,  some  get  lusty,  some  get  irritated,  but  then  calm
down very quickly, some get super angry and then very, very polite, some
people get all cravy, lusty and addicted. Craving comes in nice and strong
really affecting their mind. Have a look at that stuff. Patience needs to be
developed. All  these  things  can  be  developed. That  is  what  the  training  is
about. You will need to be intelligent and skillful to bring the practice into a
busy Western lifestyle. Of course, it’s different if you live here. Things are a
bit easier. You don’t have so much to deal with but the mind still reacts. You
will have to deal with people, places and things. You still have to become
cool towards them under any circumstance or situation.

\sphinxAtStartPar
So  have  a  look  at  your  reactions.  Try  to  work  with  that.  Work  with
your sitting meditation. Make your awareness more stable. Do your walking meditation. It doesn’t have to be the formal, four stage, super slow walk.
You  can  do  it  walking  your  dog  around  the  park.  Walking  on  the  beach,
walking in the forest, walking to work. You can do all those things. You can
even try meditating on the bicycle. I tried it. It’s kind of okay. You can really
pay attention to only one foot though as it doesn’t stop. It just keeps going
around.

\sphinxAtStartPar
So  that’s  a  few  things  that  we  can  do,  a  few  orientations.  When  we
go  back  home,  find  places,  find  a  group,  go  to  a  temple.  In  various  cities
around  the  world,  there  are  ‘conscious\sphinxhyphen{}events’,  where  people  meet  with
other  conscious  people. You  will  find  many  interesting  people  there.  New
fellow travelers on the path. Those beings who are seeking evolution, seeking \DUrole{pdfpage}{275}   transformation,  seeking  to  remove  their  defilements.  Not  those  beings
who are blinded by following simply the path of sensuality, sensual pleasure.
Or simply just chasing money, because that’s all anyone ever told them to
do. «The job is to get money.» Off you go. So there is lots of different groups
out there.


\section{The ten pāramī}
\label{\detokenize{7-b:the-ten-parami}}
\sphinxAtStartPar
In the Buddha’s tradition, there are what is known as the ten \sphinxstyleemphasis{pāramī}
or the 10 perfections. These are also qualities that we can develop in our daily
life. Those of you who have been around Thailand a little bit, you’ll have
seen  that  in  many  of  the  Buddha  halls,  there  is  various  paintings  around
the inside of the hall. Often these paintings are depictions of the previous
10 lives of the Buddha before he became fully enlightened. In each of the
10 lives, before the Buddha was fully enlightened, he perfected a particular
mind state. And these have come down to us and are known as the 10 perfections. The 10 things that need to be perfected before one becomes a fully self
awakened Buddha.

\sphinxAtStartPar
These are qualities of mind. Behaviors that should be developed. In the
northern Buddhist tradition, they have six pāramīta. It’s just a few elements
joined together from 10 to 6.

\sphinxAtStartPar
So we can try to develop these pāramī. Take this list of pāramī and put
them on the fridge. Observe them. Think about what you’re going to practice
today. What am I going to do today?
\begin{enumerate}
\sphinxsetlistlabels{\arabic}{enumi}{enumii}{}{.}%
\item {} 
\sphinxAtStartPar
The  first  perfection  is  known  as
\sphinxstyleemphasis{dāna}.  Dāna  means  generosity  or
giving  or  sharing.  It  means  giving  something  of  oneself.  Letting  go
even. Being generous and helpful both financially and with your time
and with your efforts. Try to perfect this. Try to bring this behavior into
your  life.  Be  generous!  Share! When  you  have  a  meal,  share  it  with
someone. Try to incline your mind towards giving and sharing. This
is a very wholesome quality, which has incredibly wonderful benefits.
The results of that giving. The results of that karma hundred fold some
say. So become a generous person.

\item {} 
\sphinxAtStartPar
The second pāramī is
\sphinxstyleemphasis{sīla}, or virtue or morality. Try to maintain a
level of virtue. Virtuous behavior, virtuous speech. Try to avoid those
\DUrole{pdfpage}{276}
unwholesome things. Killing and stealing, lying. Don’t take anybody
else’s partner or lover. We don’t need to commit adultery. We don’t need
to cause difficulties and troubles for other people. Try to avoid intoxicants that lead to heedlessness, that lead to drunk and stupid foolishness. Try to avoid these things. Be well disciplined and well refined in
your manners. Clean and pure in your dealings with others. Don’t try
to cheat anyone or rip anyone off. Be honest and straightforward. May
your thoughts and actions and speech be purified.

\item {} 
\sphinxAtStartPar
The third pāramī is
\sphinxstyleemphasis{nekkhama}
or renunciation or letting go. We are
developing the state of mind that inclines towards letting go. It’s leaning towards relinquishing. Going forth from the home into the homeless life. Letting go of our addictions. Letting go of our attachments,
our  collections  of  things.  Moving  forward.  Clearing  things  out.  Not
hoarding. Not holding. Not being tied to stuff. Letting go of things. Not
being selfish and possessive of things. Being selfless and disinterested.
These things don’t belong to ourselves. Don’t hold on to them. Holding
them leads to dukkha. If we hold on to them tightly and they change
– dukkha. If they disappear – dukkha. If they get damaged – dukkha.
Recognize that the act of holding leads to dukkha. And the act of letting go leads to freedom and release. You’re not going to be happy and
free, if you try to collect a whole lot of stuff. If you’re trying to walk
into two opposite directions, you won’t get very far!

\item {} 
\sphinxAtStartPar
The fourth pāramī is
\sphinxstyleemphasis{pañña}
or wisdom. Here we are talking about
perfecting our own knowledge about our mind and body process. Becoming wise as to this thing whatever is going on in it. Physical sensations  and  feelings,  perceptions,  thoughts  and  emotions. All  kinds  of
states of knowing. Pañña, it’s light. It’s the light and the truth. Try to
bring more and more wisdom into your life. Activate your awareness
in the present moment and see what’s there. Try to develop your wisdom in some way. You can read, you can discuss dhamma. But most
importantly you should practice dhamma. You need to experience the
teaching for yourself. It’s through experiencing it in an intensive way,
that transformation takes place.

\item {} 
\sphinxAtStartPar
The fifth perfection is
\sphinxstyleemphasis{viriya}
or energy. We have spoken about that \DUrole{pdfpage}{277}
this week. The perfection of energy. Make sure you are vigorous in all
your activities and energetic. Be perseverant. Don’t give up. Don’t give
up on the commitment that you have made. Finish what you start. Do
the job properly. Don’t be slack, don’t be lazy even if it takes extra time.
Put in the extra effort. You will be well rewarded. Effort is always rewarded. When we put forth effort, we feel good for ourselves, we know
that we’ve done the best that we can do, and others around us, that support us, that help us, they also see it. They see you putting forth effort
and energy. So they come to help us as well. There’s many benefits of
putting forth effort. Be fearless in the face of dangers. Courageously
surmount all obstacles.

\item {} 
\sphinxAtStartPar
The  sixth  perfection  is
\sphinxstyleemphasis{kanthi}.  It  means  patience.  Patience  is  the
highest practice. It’s the highest tapas – not the Spanish tapas, but the
Pali tapas. Tapas is a spiritual practice that leads to removing of defilements. In India it’s still practiced by some saddhus and rishis. They try
to inflict damage onto their body in the idea to develop a lot of dukkha
and experiencing a lot of dukkha, they will be able to come to the other
side faster. It is a misguided belief that we have a vast store of dukkha
to  experience  and  if  we  experience  it  quickly  and  completely  in  one
life,  then  we  break  free  of  our  dukkha.  That’s  not  how  the  Buddha
teaches things. Dukkha is conditioned. The Buddha’s teaching is about
removing the conditions. When we remove the conditions for the arising of self, self doesn’t arise. When self doesn’t arise, dukkha doesn’t
arise. It’s the wonderful thing about
\sphinxstyleemphasis{stream\sphinxhyphen{}entry}. When you reach the
first path of enlightenment, when we have entered the stream, it cuts
off so many causes and conditions.
\sphinxstyleemphasis{It cuts off identity view}, we talked
about the other day. Sakaya diti. The path of stream entry cuts off identity view and self no longer arises.
\sphinxstyleemphasis{When self no longer arises, then the
billions of karma that we have done, most of them don’t have a chance}
\sphinxstyleemphasis{to come up.}
We have cut off the pathway. We’ve blocked the pathway
for the arising of old karma. It’s conditioned stuff. We have cut off the
sense of self. The sense of self was the thing that those karmas could
manifest  by. A  maximum  of  seven  life  times. The  Buddha  was  once
sitting  with  the  monks,  put  his  fingernail  into  this  dirt,  picked  it  up
\DUrole{pdfpage}{278} and asked:
\sphinxstyleemphasis{«Monks, what is more. The soil in my fingernail or the soil
of this great earth?» «Oh, venerable sir. The soil of this great earth is
enormous and massive. It can’t be properly measured and quantified.
And  the  soil  in  your  fingernail  is  such  a  trifling,  tiny  little  peace.  It
hardly even compares.» «So too, monks. So too, those beings who have
not yet achieved stream entry, have as much karma as the earth has
soil to go. And those beings who have entered the stream, have only}
\sphinxstyleemphasis{this much dukkha to experience.»}
So I encourage you, to put forth effort and your patience to reach the path of stream entry, the first stage
of  enlightenment.  Very  doable  in  this  life.  There  are  people  around
this  country  and  in  other  countries,  that  have  done  it,  that  have  broken through and become stream\sphinxhyphen{}enterers. We all have that opportunity.
We are all blessed with the supreme opportunity. There is 34 million
tourists coming to Thailand every year. That is two million a month!
Here we just get 50! You are 50 out of 2 million. You come here for a
reason. Beings are related through their karma. We are related through
something.
\sphinxstyleemphasis{«Through an element is monks, that beings come together
and meet together. Beings of virtuous tastes flowing together, meeting
together the beings of virtuous tastes. As it has been in the past, as it}
\sphinxstyleemphasis{will be in the future, as it is now in the present.»}
Beings come together
because they share an element. It draws them together like a magnet.
They stay for as long as the karma is powerful and then they separate.
This is not something that is unusual. This is just the way things are.
This is how things are functioning. So be aware of that in the connections that you make with people. Look out for those connections that
can be very beneficial for your spiritual development. When you meet
someone who seems to be very present, hang out with them for a while.
See what they have to share. What they have to offer. Patience. Be able
to bear or forbear the wrongs of others. When somebody does something nasty, we can handle that. That is what patience is. It’s not just
being patient standing in the queue at the post office or bank, that’s just
waiting. That’s our opportunity to practice a little bit of standing meditation. Become aware of the whole standing posture. The feet touching
the  ground.  Instead  of  pulling  out  your  iPhone  starting  to  swipe  and
\DUrole{pdfpage}{279}
check, seeing what kind of cupcake your best friend had for breakfast,
bring  your  awareness  to  the  present  moment.  Don’t  spend  your  time
on  foolish  nonsense.  Don’t  waste  your  life.  It’s  precious  the  time.  If
you have any time spare, be present. See how it changes your outlook
to  things.  See  how  it  de\sphinxhyphen{}stresses  you.  You  may  become  completely
chilled out and move to Koh Phangan permanently.

\item {} 
\sphinxAtStartPar
The  seventh  perfection  is
\sphinxstyleemphasis{sacca}
or  truth. Truthfulness.  Be  honest
and truthful who you are, what you’re doing. Make sure what you say
is truthful. Honest. Don’t hide the truth just to be polite. Develop your
integrity in your speech. When you say you’re going to do something,
do it. When you say you’re going to do something by a certain time, do
it. Sometimes things get in the way but try to adhere to that principal.
Try to be an honest, upstanding person. A right kind of person in the
community. If you find something, hand it in. Be that person. Don’t be
a selfish person. Don’t swerve from the path of truth.

\item {} 
\sphinxAtStartPar
The eighth perfection is
\sphinxstyleemphasis{adiṭṭhāna}, meaning resolution or determination. Be resolute. Don’t let your resolutions become like your New
Year’s  resolutions.  Finished  by  February,  forgotten  by  March. Try  to
keep up your practices, the things you do. Have a mind which is resolute and firm and yet soft, gentle and pliable. Set your goal and work
towards it. Complete your tasks. Have your principals if you like – just
don’t attach to them.

\item {} 
\sphinxAtStartPar
The  ninth  perfection  is
\sphinxstyleemphasis{mettā}
or  loving  kindness.  We  have  been
practicing  it  all  week. That  special  intention  that  we  radiate  to  other
beings wishing them to be happy. Filling our mind with the strong wish
and  desire  for  other  beings  to  be  happy.  It’s  a  wonderful  way  to  get
around. It’s a wonderful way to live. If you have any spare time, incline
your mind towards loving kindness. Be friendly and compassionate to
others. See what difference it makes in your life. Be friendly and smiling to people that you come across. Show your love, your compassion
for other beings.

\item {} 
\sphinxAtStartPar
The tenth pāramī, we talked about this morning.
\sphinxstyleemphasis{Upekkha}, equanimity.
The development of equanimity. Calmness, peacefulness, balance. The
balanced mind. The mind that is in the center, neither swaying left nor
\DUrole{pdfpage}{280}  right. Neither getting upset nor reacting with attachment or aversion,
with liking or disliking. It’s the mind that can stand still. It’s the big tree
during the cyclone or hurricane. It doesn’t get uprooted or thrown on
the ground like a bag of toothpicks. It stands solid. This is what equanimity can do for us. We can handle any situation that is going on.

\end{enumerate}

\sphinxAtStartPar
Develop  these!  These  aren’t  things  that  just  automatically  happen.
These are things that we can develop every single day. You can choose any
of these perfections and try to develop them. You can choose one a day. You
can choose one per week if you like. Really try to focus on developing that.
Really bring those qualities of mind into your life. You will find that these
qualities are really wonderful for your meditation practice. In fact, here in
Thailand, if you start meditating and your results come fast and quickly, the
monks and nuns will say you have good pāramī. You have good perfections.
These are like little bank accounts that we are trying to fill up. Our practice
develops as our perfections are fulfilled. The more perfections that we are
doing,  the  better  our  meditation  is  going  to  be.  These  are  states  of  mind
which orientate us to letting go, to calmness, to stillness, to seeing things as
they really are. We can develop these pāramī.

\sphinxAtStartPar
So these ten perfections, we try to develop them in our life. We try to
get them going. Develop them as much as you can! So this is a further set
of practices that you can do whilst at home. That will be beneficial for your
dhamma life. Putting in place a strong foundation of dhamma life, is what
we really need for our meditation practice to get deeper and deeper. We can
go only so far with a certain lifestyle. After that, if we want to get deeper
into the state of peace, we have to start to make some changes. Changes that
affect our mind, that affect our outlook on things.

\sphinxAtStartPar
So talking of the 10 perfections, there will be a chance for you to practice the first one tomorrow when you can give a donation. This meditation
center  here  operates  on  a  donation  basis.  When  you’re  giving  something
tomorrow you’re not paying for something, but think of it that you’re making
a gift of dhamma for others to receive. Someone else can experience it just
like you have. Happily and freely we give a gift of dhamma so that others
may experience it. This dāna that we’re doing here is of very profound and
very deep meaning.
\sphinxstyleemphasis{«The gift of dhamma excels all other gifts.»}

\sphinxAtStartPar
\DUrole{pdfpage}{281}  Your opportunity to give a donation of dhamma is an extremely powerful karma. One of this potentiality that you haven’t come across before. The
strength and quality of it, karma, is measured by four things. It’s measured
by the purity of the giver, the purity of the receiver, the quality of the gift
and  the  intention  behind  the  gift.  The  purity  of  the  giver,  that  is  you,  is
great. You just have been meditating for seven days. Your mind has become
purified somewhat. At least the hindrances have been suppressed for some
period this week. We have been taking the eight precepts which is kind of the
maximum for lay people. So we’re developing our sīla. The quality on the
virtue of the giver is very high. The people that will receive your donation
next month, their sīla will also be very high. The gift that you are giving, the
gift of dhamma, well, the Buddha says no other gift excels it. It’s the highest,  most  profound  gift,  that  you  can  give  to  somebody. And  then  all  that
remains is your intention. I want you to spend some time tomorrow, making
an intention.
\sphinxstyleemphasis{«May this gift be for the gift of dhamma. May this gift be for}
\sphinxstyleemphasis{other beings to realize the teaching.»}
Really do intend that. Make a strong
intention. I encourage you to even write it on the envelope to put it in concrete terms. In this way your donation, your gift will have the strongest possible benefit that you can imagine. This type of karma is the type of karma
that brings us back to find the dhamma again. You have given dhamma. So
you’re going to receive the dhamma in return in this life and in the next lives.
You will come back and you’ll meet the Buddha’s teaching. You won’t be
lost in a sea of samsara swirling in the rounds of rebirth. This type of karmic
intention is very powerful. So we hope you take the opportunity to do that.
Everyone  has  their  own  means.  How  much  you  give  is  up  to  you.  Don’t
be  stingy.  Give  with  an  open  heart.  Give  just  enough  that  it  hurts  a  little
bit. Don’t make yourself broke. So have a good think about that tomorrow
morning when the opportunity arises. I can’t emphasize this enough. Really
incline your mind to giving the gift of dhamma so that other beings may can
become enlightened. This is a supreme gift that you can give in your life. It
will be extremely beneficial for you.

\sphinxstepscope


\chapter{Final words}
\label{\detokenize{final:final-words}}\label{\detokenize{final::doc}}
\sphinxAtStartPar
So we have been practicing the noble eightfold path. We have practiced
it in the moment. This is the only place we can practice it. If we don’t practice now, when are we going to practice? This is the only chance we have to
see things as they are. Now! Now is the time we need to activate our awareness in the present moment. We are practicing the path. Even in the action
of hearing. When hearing occurs, there is right view when we haven’t gone
into the story of ‘I am listening to something’. When we’re back and ‘there
is hearing’. Two very different circumstances. Physically it’s the same thing
going on. Sound is vibrating on the ear drum. ‘I am listening’ takes place.
But  if  we  are  aware  and  present  in  the  moment  ‘there  is  hearing’  taking
place. Two sides of a coin. One of them leads to dukkha. One of them leads
to release. One of them is a manifestation of being in the sense perception
process. The other one is a manifestation of freedom in the present moment.
So we practice in this way. We notice what is there in the present moment,
we  know  what’s  there,  we  step  away  from  it.  This  is  our  right  view. And
right view comes first. We understand that this is just nature. «There is …».
That will be the phrase that will be very helpful to you. Whenever you find
yourself being attacked by something, sadness or grief or anger or aversion,
«there is …». Feel it for yourself! Observe it! Note it! Don’t get sucked into
it. Don’t get playing in the content of your mind. Step back and see the structure, \DUrole{pdfpage}{283}  so the content can be let go of. We don’t hold on to it.

\sphinxAtStartPar
We know that objects are just objects. Nature is just nature. We know
that  experiences  are  just  experiences.  They  are  not  happening  to  anyone.
It’s just experience. There is no experiencer. There’s nobody here. We have
the right attitude and the right intention. We are letting go and renouncing
things. Our effort is in the present. Our awareness is in the present. We have
been able to stabilize the mind so that we can note and know continuously.
So our concentration is there. These five factors of the path are developed
in a single moment of noting, knowing and letting go. We have previously
already  established  right  speech,  right  action  and  right  livelihood.
\sphinxstyleemphasis{So  the
noble  eightfold  path  is  practiced  in  the  present.  It  arises  in  the  present.}
Dhamma is ever present. There is dhamma\sphinxhyphen{}talk everywhere. You don’t have
to listen to me to realize or listen to the dhamma. It’s talking to us all around
us. Nature is teaching us but we are often unable to hear. We are not paying
attention to the present. We have been distracted by the past, future or our
iPhones. We  can’t  know  or  see  dhamma  because  of  the  defilements  in  the
mind. If we can think and see nature as it really is, then the mind is free. Free
from the defilements and there is nature everywhere. We are free in the present moment. All we need to do is extend that moment out, make it regular,
make it arise often. Bring yourself into that moment as much as you can. We
are training the mind to be in that moment. To be in that mode of observation. It’s one thing to get into it on a retreat. It’s another practice to keep it
up.  Keep  opening  your  heart.  Keep  opening  your  mind,  your  awareness,
your wisdom in the present and seeing what’s there not being upset. Accepting. We are accepting! We have to accept! We are trying to think in the right
way  and  we’re  trying  to  be  aware.  We’re  not  complaining  about  what  is
happening. It is as it is. Karma is unfolding. Acceptance is very important.
If we don’t have right view and we don’t have right intention, we won’t be
able to accepting whatever is happening in the present moment. When we
don’t accept, we start to react. When we start to react, we start to dukkher
ourselves. So we need to accept things. Acceptance is important. If we can’t
accept then we can’t learn. If we can’t learn, we can’t let go.

\sphinxAtStartPar
We are not trying to change the experience. We’re trying to change our
attention  to  the  experience.  Whatever  is  happening  is  happening.  We  are
\DUrole{pdfpage}{284}  changing  the  way  we’re  experiencing  it  in  the  moment.  Instead  of  blindly
reacting to it, from a place of ignorance, we are pulling our mind into the
present so that we see things as they are. Letting go happens very naturally.
Especially  when  the  mind  is  properly  trained. When  we  have  gone  to  the
efforts  of  training  ourselves,  then  we’ll  find  it  easier  and  easier  when  we
do get confronted to pull ourselves out of it. We’re not trying to change the
process.

\sphinxAtStartPar
We’re not complaining about what is happening. Everything is experience. Whatever is happening is happening through cause and effect. If our
experience is good, we get really happy about it. If our experience is bad, we
get upset about it. This means we don’t see things clearly. We have been a
reactor reacting to good and bad. A judgement! A judgement formed on the
back of a feeling. Pleasantness or unpleasantness. It’s amazing how all the
good things in life are associated with pleasantness and all the bad things in
life are associated with unpleasantness. Why is that? The good is not pleasant. The  pleasant,  if  unseen  properly,  makes  things  good. The  unpleasant,
if not seen properly, makes things bad.
\sphinxstyleemphasis{Good and bad are our reaction to}
\sphinxstyleemphasis{pleasant and unpleasant. That’s all! It’s just a judgment in our mind}. Just
a  judgment.  It’s  not  truth!  There’s  no  reality.  Judgment  just  passed  away.
Where is it gone? It doesn’t exist anymore. That’s how fleeting the mind is,
how rapidly it changes.

\sphinxAtStartPar
So  we  have  to  understand  that  nature  is  nature  and  the  objects  are
objects. And our mind’s reaction is our mind’s reaction. We won’t be able
to undo our old karma, but we can change our reactions to things. We can
develop  awareness  in  the  present  moment  so  that  we  don’t  react  so  negatively to things. And this is what we do in our daily life practice. Just making
sure that we’re not getting triggered. Making sure we’re not triggering others.
Making sure our behavior is acceptable and suitable. If others are being triggered, of course, that’s their problem, but we shouldn’t contribute knowingly
to  another  person’s  dukkha.  We  are  trying  to  appreciate  how  the  mind  is
working. We are just checking our awareness in the present. We’re not trying
to achieve, to get or to become someone. We are just chilling in the present
moment watching things arise and pass away without reaction. It’s very, very
peaceful and becomes very blissful once the technique is mastered.

\sphinxAtStartPar
\DUrole{pdfpage}{285}  So dhamma is there all the time. It’s everywhere. We just need to take
the time to have a look at it.

\appendix

\sphinxstepscope


\chapter{Chanting}
\label{\detokenize{chanting:chanting}}\label{\detokenize{chanting::doc}}

\section{Morning Chanting}
\label{\detokenize{chanting:morning-chanting}}
\LOCALaudiolink{https://www.mixcloud.com/anthonymarkwell/morning-chanting/}

\sphinxAtStartPar
\DUrole{pdfpage}{287}  \sphinxstylestrong{Araham sammā\sphinxhyphen{}sambuddho bhagavā.} \\
The Blessed One is Worthy \& Rightly Self\sphinxhyphen{}awakened.

\sphinxAtStartPar
\sphinxstylestrong{Buddham bhagavantam abhivādemi.} \\
I bow down before the Awakened, Blessed One.

\sphinxAtStartPar
(BOW DOWN)

\sphinxAtStartPar
\sphinxstylestrong{Svākkhāto bhagavatā dhammo.} \\
The Dhamma is well\sphinxhyphen{}expounded by the Blessed One.

\sphinxAtStartPar
\sphinxstylestrong{Dhammam namassāmi.} \\
I pay homage to the Dhamma.

\sphinxAtStartPar
(BOW DOWN)

\sphinxAtStartPar
\sphinxstylestrong{Supatipanno bhagavato sāvaka\sphinxhyphen{}sangho.} \\
The Sangha of the Blessed One’s disciples has practiced well.

\sphinxAtStartPar
\sphinxstylestrong{Sangham namāmi.} \\
I pay respect to the Sangha.

\sphinxAtStartPar
(BOW DOWN)


\subsection{Homage}
\label{\detokenize{chanting:homage}}
\sphinxAtStartPar
\DUrole{pdfpage}{288}

\sphinxAtStartPar
\sphinxstyleemphasis{LEADER: Handa mayam buddhassa bhagavato pubba\sphinxhyphen{}bhāga\sphinxhyphen{}namakāram karoma se:} \\
Now let us chant the preliminary passage in homage to the Awakened One, the Blessed One:

\sphinxAtStartPar
\sphinxstylestrong{Namo tassa bhagavato arahato sammāsambuddhassa. \\
Namo tassa bhagavato arahato sammāsambuddhassa. \\
Namo tassa bhagavato arahato sammāsambuddhassa.}

\sphinxAtStartPar
Homage to the Blessed One, the Worthy One, the Rightly Self\sphinxhyphen{}Awakened One.


\subsection{Going for Refuge}
\label{\detokenize{chanting:going-for-refuge}}
\sphinxAtStartPar
\DUrole{pdfpage}{289} \sphinxstyleemphasis{LEADER: Handa mayam tisarana\sphinxhyphen{}gamana\sphinxhyphen{}pātham bhanāma se:} \\
Now let us chant the passage going for refuge to the Triple Gem:

\sphinxAtStartPar
\sphinxstylestrong{Buddham saranam gacchāmi.} \\
I go to the Buddha for refuge.

\sphinxAtStartPar
\sphinxstylestrong{Dhammam saranam gacchāmi.} \\
I go to the Dhamma for refuge.

\sphinxAtStartPar
\sphinxstylestrong{Sangham saranam gacchāmi.} \\
I go to the Sangha for refuge.

\sphinxAtStartPar
\sphinxstylestrong{Dutiyampi buddham saranam gacchāmi.} \\
A second time, I go to the Buddha for refuge.

\sphinxAtStartPar
\sphinxstylestrong{Dutiyampi dhammam saranam gacchāmi.} \\
A second time, I go to the Dhamma for refuge.

\sphinxAtStartPar
\sphinxstylestrong{Dutiyampi sangham saranam gacchāmi.} \\
A second time, I go to the Sangha for refuge.

\sphinxAtStartPar
\sphinxstylestrong{Tatiyampi buddham saranam gacchāmi.} \\
A third time, I go to the Buddha for refuge.

\sphinxAtStartPar
\sphinxstylestrong{Tatiyampi dhammam saranam gacchāmi.} \\
A third time, I go to the Dhamma for refuge.

\sphinxAtStartPar
\sphinxstylestrong{Tatiyampi sangham saranam gacchāmi.} \\
A third time, I go to the Sangha for refuge.


\subsection{Taking the Eight Precepts}
\label{\detokenize{chanting:taking-the-eight-precepts}}
\sphinxAtStartPar
\DUrole{pdfpage}{290} \sphinxstylestrong{Pānātipātā veramanī sikkhā\sphinxhyphen{}padam samādiyāmi.} \\
I undertake the training rule to refrain from taking life.

\sphinxAtStartPar
\sphinxstylestrong{Adinnādānā veramanī sikkhā\sphinxhyphen{}padam samādiyāmi.} \\
I undertake the training rule to refrain from stealing.

\sphinxAtStartPar
\sphinxstylestrong{Abrahma\sphinxhyphen{}cariyā veramanī sikkhā\sphinxhyphen{}padam samādiyāmi.} \\
I undertake the training rule to refrain from sexual intercourse.

\sphinxAtStartPar
\sphinxstylestrong{Musāvādā veramanī sikkhā\sphinxhyphen{}padam samādiyāmi.} \\
I undertake the training rule to refrain from telling lies.

\sphinxAtStartPar
\sphinxstylestrong{Surā\sphinxhyphen{}meraya\sphinxhyphen{}majja\sphinxhyphen{}pamādatthānā veramanī sikkhā\sphinxhyphen{}padam samādiyāmi.} \\
I undertake the training rule to refrain from intoxicating liquors \& drugs that lead to heedlessness.

\sphinxAtStartPar
\sphinxstylestrong{Vikāla\sphinxhyphen{}bhojanā veramanī sikkhā\sphinxhyphen{}padam samādiyāmi.} \\
I undertake the training rule to refrain from eating after noon \& before dawn.

\sphinxAtStartPar
\sphinxstylestrong{Nacca\sphinxhyphen{}gīta\sphinxhyphen{}vādita\sphinxhyphen{}visūka\sphinxhyphen{}dassanā mālā\sphinxhyphen{}gandha\sphinxhyphen{}vilepanadhārana\sphinxhyphen{}manḍana\sphinxhyphen{}vibhūsanatthānā veramanī sikkhā\sphinxhyphen{}padam samādiyāmi.} \\
I undertake the training rule to refrain from dancing, singing, music, watching shows, wearing garlands, beautifying myself with perfumes \& cosmetics.

\sphinxAtStartPar
\sphinxstylestrong{Uccāsayana\sphinxhyphen{}mahāsayanā veramanī sikkhā\sphinxhyphen{}padam samādiyāmi.} \\
I undertake the training rule to refrain from high \& luxurious seats \& beds.

\sphinxAtStartPar
\sphinxstylestrong{Imāni attha sikkhā\sphinxhyphen{}padāni samādiyāmi.} {[}×3{]} \\
I undertake these eight precepts.

\sphinxAtStartPar
(BOW THREE TIMES)


\section{Evening chanting}
\label{\detokenize{chanting:evening-chanting}}
\LOCALaudiolink{https://www.mixcloud.com/anthonymarkwell/evening-chanting/}


\subsection{Recollection of the Buddha}
\label{\detokenize{chanting:recollection-of-the-buddha}}
\sphinxAtStartPar
\sphinxstyleemphasis{LEADER: Handa mayam Buddhānussati\sphinxhyphen{}nayam karoma se:} \\
Now let us recite the guide to the recollection of the Buddha:

\sphinxAtStartPar
\sphinxstylestrong{Itipi so bhagavā araham sammā\sphinxhyphen{}sambuddho,} \\
He is a Blessed One, a Worthy One, a Rightly Self\sphinxhyphen{}awakened One,

\sphinxAtStartPar
\sphinxstylestrong{Vijjā\sphinxhyphen{}carana\sphinxhyphen{}sampanno sugato lokavidū,} \\
consummate in knowledge \& conduct, one who has gone the good way, knower of the worlds,

\sphinxAtStartPar
\sphinxstylestrong{Anuttaro purisa\sphinxhyphen{}damma\sphinxhyphen{}sārathi satthā deva manussānam buddho bhagavāti.} \\
unexcelled trainer of those who can be taught, teacher of human \& divine beings; awakened; blessed.


\subsection{Recollection of the Dhamma}
\label{\detokenize{chanting:recollection-of-the-dhamma}}
\sphinxAtStartPar
\sphinxstyleemphasis{LEADER: Handa mayam Dhammānussati\sphinxhyphen{}nayam karoma se:}
Now let us recite the guide to the recollection of the Dhamma:

\sphinxAtStartPar
\sphinxstylestrong{Svākkhāto bhagavatā dhammo,} \\
The Dhamma is well\sphinxhyphen{}expounded by the Blessed One,

\sphinxAtStartPar
\sphinxstylestrong{Sanditthiko akāliko ehipassiko,} \\
to be seen here \& now, timeless, inviting all to come \& see,

\sphinxAtStartPar
\sphinxstylestrong{Opanayiko paccattam veditabbo viññūhīti.} \\
pertinent, to be seen by the observant for themselves.


\subsection{Recollection of the Sangha}
\label{\detokenize{chanting:recollection-of-the-sangha}}
\sphinxAtStartPar
\DUrole{pdfpage}{292} \sphinxstyleemphasis{LEADER: Handa mayam Sanghānussati\sphinxhyphen{}nayam karoma se:} \\
Now let us recite the guide to the recollection of the Sangha:

\sphinxAtStartPar
\sphinxstylestrong{Su\sphinxhyphen{}patipanno bhagavato sāvaka\sphinxhyphen{}sangho,} \\
The Sangha of the Blessed One’s disciples who have practiced well,

\sphinxAtStartPar
\sphinxstylestrong{Uju\sphinxhyphen{}patipanno bhagavato sāvaka\sphinxhyphen{}sangho,} \\
the Sangha of the Blessed One’s disciples who have practiced
straightforwardly,

\sphinxAtStartPar
\sphinxstylestrong{Ñāya\sphinxhyphen{}patipanno bhagavato sāvaka\sphinxhyphen{}sangho,} \\
the Sangha of the Blessed One’s disciples who have practiced
methodically,

\sphinxAtStartPar
\sphinxstylestrong{Sāmīci\sphinxhyphen{}patipanno bhagavato sāvaka\sphinxhyphen{}sangho,} \\
the Sangha of the Blessed One’s disciples who have practiced masterfully,

\sphinxAtStartPar
\sphinxstylestrong{Yadidam cattāri purisa\sphinxhyphen{}yugāni attha purisa\sphinxhyphen{}puggalā:} \\
i.e., the four pairs—the eight types—of Noble Ones:

\sphinxAtStartPar
\sphinxstylestrong{Esa bhagavato sāvaka\sphinxhyphen{}sangho—} \\
That is the Sangha of the Blessed One’s disciples—

\sphinxAtStartPar
\sphinxstylestrong{Ahuneyyo pāhuneyyo dakkhineyyo añjali\sphinxhyphen{}karanīyo,} \\
worthy of gifts, worthy of hospitality, worthy of offerings, worthy of respect,

\sphinxAtStartPar
\sphinxstylestrong{Anuttaram puññakkhettam lokassāti.} \\
the incomparable field of merit for the world.


\subsection{The Sublime Attitudes}
\label{\detokenize{chanting:the-sublime-attitudes}}

\subsubsection{Metta – Loving\sphinxhyphen{}Kindness}
\label{\detokenize{chanting:metta-loving-kindness}}
\sphinxAtStartPar
\DUrole{pdfpage}{293} \sphinxstylestrong{Sabbe sattā sukhitā hontu.} \\
May all living beings be happy.

\sphinxAtStartPar
\sphinxstylestrong{Sabbe sattā averā hontu.} \\
May all living beings be free from animosity.

\sphinxAtStartPar
\sphinxstylestrong{Sabbe sattā abyāpajjhā hontu.} \\
May all living beings be free from oppression.

\sphinxAtStartPar
\sphinxstylestrong{Sabbe sattā anīghā hontu.} \\
May all living beings be free from trouble.

\sphinxAtStartPar
\sphinxstylestrong{Sabbe sattā sukhī attānam pariharantu.} \\
May all living beings look after themselves with ease.


\subsubsection{Karuna – Compassion}
\label{\detokenize{chanting:karuna-compassion}}
\sphinxAtStartPar
\sphinxstylestrong{Sabbe sattā sabba\sphinxhyphen{}dukkhā pamuccantu.} \\
May all living beings be freed from all stress \& pain.


\subsubsection{Mudita – Sympathetic Joy}
\label{\detokenize{chanting:mudita-sympathetic-joy}}
\sphinxAtStartPar
\sphinxstylestrong{Sabbe sattā laddha\sphinxhyphen{}sampattito mā vigacchantu.} \\
May all living beings not be deprived of the good fortune they have attained.


\subsubsection{Upekkha – Equanimity}
\label{\detokenize{chanting:upekkha-equanimity}}
\sphinxAtStartPar
\sphinxstylestrong{Sabbe sattā kamma\sphinxhyphen{}ssakā kamma\sphinxhyphen{}dāyādā kamma\sphinxhyphen{}yonī kamma\sphinxhyphen{}bandhū kamma\sphinxhyphen{}patisaranā.} \\
All living beings are the owners of their actions, heir to their actions, born of their actions, related through their actions, and live dependent on their actions.

\sphinxAtStartPar
\sphinxstylestrong{Yam kammam karissanti kalyānam vā pāpakam vā tassa dāyādā bhavissanti.} \\
Whatever they do, for good or for evil, to that will they fall heir.

\sphinxstepscope


\chapter{Appendices}
\label{\detokenize{appendices:appendices}}\label{\detokenize{appendices::doc}}

\section{Four foundations of mindfulness and the five aggregates}
\label{\detokenize{appendices:four-foundations-of-mindfulness-and-the-five-aggregates}}

\begin{savenotes}\sphinxattablestart
\sphinxthistablewithglobalstyle
\centering
\begin{tabulary}{\linewidth}[t]{TTT}
\sphinxtoprule
\sphinxstyletheadfamily 
\sphinxAtStartPar
Four foundations of mindfulness
&\sphinxstartmulticolumn{2}%
\begin{varwidth}[t]{\sphinxcolwidth{2}{3}}
\sphinxstyletheadfamily \sphinxAtStartPar
 Five aggregates (arising and passing together)
\par
\vskip-\baselineskip\vbox{\hbox{\strut}}\end{varwidth}%
\sphinxstopmulticolumn
\\
\sphinxmidrule
\sphinxtableatstartofbodyhook
\sphinxAtStartPar
Body \\
(kaya\sphinxhyphen{}nupassana)
&
\sphinxAtStartPar
Body / physical sensations
&
\sphinxAtStartPar
Body incl. sense spheres (rūpa)
\\
\sphinxhline
\sphinxAtStartPar
Feeling \\
(vedana\sphinxhyphen{}nupassana)
&
\sphinxAtStartPar
Feeling (vedanā)
&\sphinxmultirow{5}{8}{%
\begin{varwidth}[t]{\sphinxcolwidth{1}{3}}
\sphinxAtStartPar
 Mind (nāma)
\par
\vskip-\baselineskip\vbox{\hbox{\strut}}\end{varwidth}%
}%
\\
\sphinxcline{1-2}\sphinxfixclines{3}&
\sphinxAtStartPar
Perception (sañña)
&\sphinxtablestrut{8}\\
\sphinxcline{1-2}\sphinxfixclines{3}
\sphinxAtStartPar
Mind states \\
(citta\sphinxhyphen{}nupassana)
&
\sphinxAtStartPar
Emotions
&\sphinxtablestrut{8}\\
\sphinxcline{1-2}\sphinxfixclines{3}
\sphinxAtStartPar
Mind objects \\
(dhamma\sphinxhyphen{}nupassana)
&
\sphinxAtStartPar
Thoughts / Reactions (saṅkhārā)
&\sphinxtablestrut{8}\\
\sphinxcline{1-2}\sphinxfixclines{3}&
\sphinxAtStartPar
Consciousness
&\sphinxtablestrut{8}\\
\sphinxbottomrule
\end{tabulary}
\sphinxtableafterendhook\par
\sphinxattableend\end{savenotes}


\section{The noble eightfold path, the five faculties, the seven factors of enlightenment}
\label{\detokenize{appendices:the-noble-eightfold-path-the-five-faculties-the-seven-factors-of-enlightenment}}

\begin{savenotes}\sphinxattablestart
\sphinxthistablewithglobalstyle
\centering
\begin{tabulary}{\linewidth}[t]{TTTT}
\sphinxtoprule
\sphinxstartmulticolumn{2}%
\begin{varwidth}[t]{\sphinxcolwidth{2}{4}}
\sphinxstyletheadfamily \sphinxAtStartPar
 Eightfold path
\par
\vskip-\baselineskip\vbox{\hbox{\strut}}\end{varwidth}%
\sphinxstopmulticolumn
&\sphinxstyletheadfamily 
\sphinxAtStartPar
Five faculties
&\sphinxstyletheadfamily 
\sphinxAtStartPar
Seven factors of enlightenment
\\
\sphinxmidrule
\sphinxtableatstartofbodyhook\sphinxmultirow{3}{4}{%
\begin{varwidth}[t]{\sphinxcolwidth{1}{4}}
\sphinxAtStartPar
 Concentration \\
(samādhi)
\par
\vskip-\baselineskip\vbox{\hbox{\strut}}\end{varwidth}%
}%
&
\sphinxAtStartPar
8. right concentration
&
\sphinxAtStartPar
concentration
&
\sphinxAtStartPar
concentration / stability (6)
\\
\sphinxcline{2-4}\sphinxfixclines{4}\sphinxtablestrut{4}&
\sphinxAtStartPar
7. right mindfulness
&
\sphinxAtStartPar
mindfulness
&
\sphinxAtStartPar
mindfulness (1)
\\
\sphinxcline{2-4}\sphinxfixclines{4}\sphinxtablestrut{4}&
\sphinxAtStartPar
6. right energy / effort
&
\sphinxAtStartPar
energy
&
\sphinxAtStartPar
energy (2)
\\
\sphinxhline\sphinxmultirow{3}{14}{%
\begin{varwidth}[t]{\sphinxcolwidth{1}{4}}
\sphinxAtStartPar
 Morality \\
(sīla)
\par
\vskip-\baselineskip\vbox{\hbox{\strut}}\end{varwidth}%
}%
&\sphinxstartmulticolumn{3}%
\begin{varwidth}[t]{\sphinxcolwidth{3}{4}}
\sphinxAtStartPar
5. right livelihood
\par
\vskip-\baselineskip\vbox{\hbox{\strut}}\end{varwidth}%
\sphinxstopmulticolumn
\\
\sphinxcline{2-4}\sphinxfixclines{4}\sphinxtablestrut{14}&\sphinxstartmulticolumn{3}%
\begin{varwidth}[t]{\sphinxcolwidth{3}{4}}
\sphinxAtStartPar
4. right action
\par
\vskip-\baselineskip\vbox{\hbox{\strut}}\end{varwidth}%
\sphinxstopmulticolumn
\\
\sphinxcline{2-4}\sphinxfixclines{4}\sphinxtablestrut{14}&\sphinxstartmulticolumn{3}%
\begin{varwidth}[t]{\sphinxcolwidth{3}{4}}
\sphinxAtStartPar
3. right speech
\par
\vskip-\baselineskip\vbox{\hbox{\strut}}\end{varwidth}%
\sphinxstopmulticolumn
\\
\sphinxhline\sphinxmultirow{2}{18}{%
\begin{varwidth}[t]{\sphinxcolwidth{1}{4}}
\sphinxAtStartPar
 Wisdom \\
(pañña)
\par
\vskip-\baselineskip\vbox{\hbox{\strut}}\end{varwidth}%
}%
&
\sphinxAtStartPar
2. right thought / intention / aim
&\sphinxstartmulticolumn{2}%
\begin{varwidth}[t]{\sphinxcolwidth{2}{4}}
\sphinxAtStartPar
faith
\par
\vskip-\baselineskip\vbox{\hbox{\strut}}\end{varwidth}%
\sphinxstopmulticolumn
\\
\sphinxcline{2-4}\sphinxfixclines{4}\sphinxtablestrut{18}&
\sphinxAtStartPar
1. right view / right understanding
&
\sphinxAtStartPar
wisdom
&
\sphinxAtStartPar
investigation (3)
\\
\sphinxhline\sphinxstartmulticolumn{3}%
\sphinxmultirow{3}{24}{%
\begin{varwidth}[t]{\sphinxcolwidth{3}{4}}
\sphinxAtStartPar
 
\par
\vskip-\baselineskip\vbox{\hbox{\strut}}\end{varwidth}%
}%
\sphinxstopmulticolumn
&
\sphinxAtStartPar
rapture (4)
\\
\sphinxcline{4-4}\sphinxfixclines{4}\multicolumn{3}{l}{\sphinxtablestrut{24}}&
\sphinxAtStartPar
tranquility (5)
\\
\sphinxcline{4-4}\sphinxfixclines{4}\multicolumn{3}{l}{\sphinxtablestrut{24}}&
\sphinxAtStartPar
equanimity (7)
\\
\sphinxbottomrule
\end{tabulary}
\sphinxtableafterendhook\par
\sphinxattableend\end{savenotes}


\section{Right concentration}
\label{\detokenize{appendices:right-concentration}}

\begin{savenotes}\sphinxattablestart
\sphinxthistablewithglobalstyle
\centering
\begin{tabulary}{\linewidth}[t]{TT}
\sphinxtoprule
\sphinxstartmulticolumn{2}%
\begin{varwidth}[t]{\sphinxcolwidth{2}{2}}
\sphinxstyletheadfamily \sphinxAtStartPar
 Right concentration / stability
\par
\vskip-\baselineskip\vbox{\hbox{\strut}}\end{varwidth}%
\sphinxstopmulticolumn
\\
\sphinxmidrule
\sphinxtableatstartofbodyhook
\sphinxAtStartPar
4. jhana
&
\sphinxAtStartPar
pure bright awareness
\\
\sphinxhline
\sphinxAtStartPar
3. jhana
&
\sphinxAtStartPar
equanimity
\\
\sphinxhline
\sphinxAtStartPar
2. jhana
&
\sphinxAtStartPar
rapture, pleasure  born  of  concentration
\\
\sphinxhline
\sphinxAtStartPar
1. jhana
&
\sphinxAtStartPar
rapture, pleasure born of seclusion
\\
\sphinxbottomrule
\end{tabulary}
\sphinxtableafterendhook\par
\sphinxattableend\end{savenotes}


\section{Dependent origination}
\label{\detokenize{appendices:dependent-origination}}
\sphinxAtStartPar
:pdfpage:\sphinxcode{\sphinxupquote{298}} Dependent origination is not looked at as a temporal sequence, rather as conditional links arising in one moment.


\begin{savenotes}\sphinxattablestart
\sphinxthistablewithglobalstyle
\centering
\begin{tabular}[t]{\X{1}{9}\X{4}{9}\X{4}{9}}
\sphinxtoprule
\sphinxtableatstartofbodyhook
\sphinxAtStartPar
1
&
\sphinxAtStartPar
ignorance / delusion / nonknowledge
&
\sphinxAtStartPar
avijjā / moha
\\
\sphinxhline
\sphinxAtStartPar
2
&
\sphinxAtStartPar
conditional formation / volitional formations / intentional structures / karma
&
\sphinxAtStartPar
saṅkhārā
\\
\sphinxhline
\sphinxAtStartPar
3
&
\sphinxAtStartPar
consciousness
&
\sphinxAtStartPar
viññāṇa
\\
\sphinxhline
\sphinxAtStartPar
4
&
\sphinxAtStartPar
nāma\sphinxhyphen{}rūpa
&
\sphinxAtStartPar
nāma\sphinxhyphen{}rūpa
\\
\sphinxhline
\sphinxAtStartPar
5
&
\sphinxAtStartPar
sense\sphinxhyphen{}bases
&
\sphinxAtStartPar
āyatana
\\
\sphinxhline
\sphinxAtStartPar
6
&
\sphinxAtStartPar
\sphinxstylestrong{contact}
&
\sphinxAtStartPar
\sphinxstylestrong{phassa}
\\
\sphinxhline
\sphinxAtStartPar
7
&
\sphinxAtStartPar
feeling
&
\sphinxAtStartPar
vedanā
\\
\sphinxhline
\sphinxAtStartPar
8
&
\sphinxAtStartPar
craving (me) / aversion
&
\sphinxAtStartPar
taṇhā
\\
\sphinxhline
\sphinxAtStartPar
9
&
\sphinxAtStartPar
clinging (mine)
&
\sphinxAtStartPar
upādāna
\\
\sphinxhline
\sphinxAtStartPar
10
&
\sphinxAtStartPar
being (my self)
&
\sphinxAtStartPar
bhava
\\
\sphinxhline
\sphinxAtStartPar
11
&
\sphinxAtStartPar
birth (I am)
&
\sphinxAtStartPar
jāti
\\
\sphinxhline
\sphinxAtStartPar
12
&
\sphinxAtStartPar
aging, death, sorrow, lamentation, pain, grief, despair
&
\sphinxAtStartPar
jarā, maraṇa, soka, parideva, dukkha, domanassa, upāyāsā
\\
\sphinxbottomrule
\end{tabular}
\sphinxtableafterendhook\par
\sphinxattableend\end{savenotes}


\section{Vipassana insights (vipassanā ñāṇas)}
\label{\detokenize{appendices:vipassana-insights-vipassana-nanas}}\begin{enumerate}
\sphinxsetlistlabels{\arabic}{enumi}{enumii}{}{.}%
\item {} 
\sphinxAtStartPar
:pdfpage:\sphinxcode{\sphinxupquote{299}} knowledge of the difference between nāma and rūpa, \sphinxstylestrong{nāma\sphinxhyphen{}rūpapariccheda\sphinxhyphen{}ñāṇa}

\item {} 
\sphinxAtStartPar
discerning conditions for nāma and rūpa, \sphinxstylestrong{paccaya\sphinxhyphen{}pariggaha\sphinxhyphen{}ñāṇa}

\item {} 
\sphinxAtStartPar
comprehension by groups (the three characteristics), \sphinxstylestrong{sammasana\sphinxhyphen{}ñāṇa}

\sphinxAtStartPar
corresponding to the \sphinxstyleemphasis{first vipassana jhana}

\item {} 
\sphinxAtStartPar
knowledge of arising and falling away, \sphinxstylestrong{udayabbaya\sphinxhyphen{}ñāṇa}

\sphinxAtStartPar
corresponding to the \sphinxstyleemphasis{second vipassana jhana}

\sphinxAtStartPar
insight into path and not\sphinxhyphen{}path: corresponding to the \sphinxstyleemphasis{third vipassana jhana}

\item {} 
\sphinxAtStartPar
knowledge of dissolution, \sphinxstylestrong{bhanga\sphinxhyphen{}ñāṇa}

\sphinxAtStartPar
corresponding to the \sphinxstyleemphasis{fourth vipassana jhana}

\item {} 
\sphinxAtStartPar
knowledge of terror, \sphinxstylestrong{bhaya\sphinxhyphen{}ñāṇa}

\item {} 
\sphinxAtStartPar
knowledge of danger, \sphinxstylestrong{ādīnava\sphinxhyphen{}ñāṇa}

\item {} 
\sphinxAtStartPar
knowledge of dispassion, \sphinxstylestrong{nibbidā\sphinxhyphen{}ñāṇa}

\item {} 
\sphinxAtStartPar
knowledge of desire for deliverance, \sphinxstylestrong{mucitukamyatā\sphinxhyphen{}ñāṇa}

\item {} 
\sphinxAtStartPar
knowledge of reflexion, \sphinxstylestrong{paṭisankhā ñāṇa}

\item {} 
\sphinxAtStartPar
knowledge of equanimity about conditioned dhammas, \sphinxstylestrong{saṅkhārupekkhā ñāṇa}

\item {} 
\sphinxAtStartPar
adaptation or conformity knowledge, \sphinxstylestrong{anuloma ñāṇa}

\item {} 
\sphinxAtStartPar
change\sphinxhyphen{}of\sphinxhyphen{}lineage knowledge, \sphinxstylestrong{gotrabhū ñāṇa}

\item {} 
\sphinxAtStartPar
path knowledge, \sphinxstylestrong{magga ñāṇa}

\item {} 
\sphinxAtStartPar
fruition knowledge, \sphinxstylestrong{phala ñāṇa}

\item {} 
\sphinxAtStartPar
reviewing knowledge, \sphinxstylestrong{paccavekkhaṇa ñāṇa}

\end{enumerate}

\sphinxAtStartPar
Source: \sphinxstyleemphasis{In this very Life}, Sayadaw U Pandita, page 270ff. \sphinxstyleemphasis{The Progress of Insight}; Wisdom Wide and Deep, Shaila Catherine, page 431ff.


\section{The four noble individuals (\sphinxstyleemphasis{ariya\sphinxhyphen{}puggala})}
\label{\detokenize{appendices:the-four-noble-individuals-ariya-puggala}}

\begin{savenotes}\sphinxattablestart
\sphinxthistablewithglobalstyle
\centering
\begin{tabulary}{\linewidth}[t]{TTT}
\sphinxtoprule
\sphinxtableatstartofbodyhook\sphinxmultirow{3}{1}{%
\begin{varwidth}[t]{\sphinxcolwidth{1}{3}}
\sphinxAtStartPar
 \sphinxstylestrong{stream\sphinxhyphen{}enterer} \\
(sotāpanna) \\
eradicates 1\sphinxhyphen{}3
\par
\vskip-\baselineskip\vbox{\hbox{\strut}}\end{varwidth}%
}%
&\sphinxmultirow{5}{2}{%
\begin{varwidth}[t]{\sphinxcolwidth{1}{3}}
\sphinxAtStartPar
 \sphinxstylestrong{non\sphinxhyphen{}returner} \\
(anāgāmi) \\
fully eradicates 1–5
\par
\vskip-\baselineskip\vbox{\hbox{\strut}}\end{varwidth}%
}%
&
\sphinxAtStartPar
1. personality\sphinxhyphen{}belief
\\
\sphinxvlinecrossing{1}\sphinxcline{3-3}\sphinxfixclines{3}\sphinxtablestrut{1}&\sphinxtablestrut{2}&
\sphinxAtStartPar
2. sceptical doubt
\\
\sphinxvlinecrossing{1}\sphinxcline{3-3}\sphinxfixclines{3}\sphinxtablestrut{1}&\sphinxtablestrut{2}&
\sphinxAtStartPar
3. believe in rules \& rituals
\\
\sphinxcline{1-1}\sphinxcline{3-3}\sphinxfixclines{3}\sphinxmultirow{2}{6}{%
\begin{varwidth}[t]{\sphinxcolwidth{1}{3}}
\sphinxAtStartPar
 \sphinxstylestrong{once returner} \\
(sakadāgāmi) \\
weakens 3 and 4
\par
\vskip-\baselineskip\vbox{\hbox{\strut}}\end{varwidth}%
}%
&\sphinxtablestrut{2}&
\sphinxAtStartPar
4. sensous craving
\\
\sphinxvlinecrossing{1}\sphinxcline{3-3}\sphinxfixclines{3}\sphinxtablestrut{6}&\sphinxtablestrut{2}&
\sphinxAtStartPar
5. ill\sphinxhyphen{}will
\\
\sphinxhline\sphinxstartmulticolumn{2}%
\sphinxmultirow{5}{9}{%
\begin{varwidth}[t]{\sphinxcolwidth{2}{3}}
\sphinxAtStartPar
  \sphinxstylestrong{fully liberated person} \\
(arahat) \\
becomes free from 6–10
\par
\vskip-\baselineskip\vbox{\hbox{\strut}}\end{varwidth}%
}%
\sphinxstopmulticolumn
&
\sphinxAtStartPar
6.  craving for fine material rebirth
\\
\sphinxcline{3-3}\sphinxfixclines{3}\multicolumn{2}{l}{\sphinxtablestrut{9}}&
\sphinxAtStartPar
7.  craving for immaterial / mind rebirth
\\
\sphinxcline{3-3}\sphinxfixclines{3}\multicolumn{2}{l}{\sphinxtablestrut{9}}&
\sphinxAtStartPar
8.   conceit
\\
\sphinxcline{3-3}\sphinxfixclines{3}\multicolumn{2}{l}{\sphinxtablestrut{9}}&
\sphinxAtStartPar
9. restlessness
\\
\sphinxcline{3-3}\sphinxfixclines{3}\multicolumn{2}{l}{\sphinxtablestrut{9}}&
\sphinxAtStartPar
10.  ignorance
\\
\sphinxbottomrule
\end{tabulary}
\sphinxtableafterendhook\par
\sphinxattableend\end{savenotes}

\sphinxstepscope


\chapter{Anthony Markwell}
\label{\detokenize{author:anthony-markwell}}\label{\detokenize{author::doc}}
\noindent{\hspace*{\fill}\sphinxincludegraphics[width=0.500\linewidth]{{anthony-markwell}.jpg}\hspace*{\fill}}

\sphinxAtStartPar
\DUrole{pdfpage}{301}  Anthony  ordained  as  a  Buddhist  monk  aged  24  at  Wat  Khao  Chong
Lom in 1995 with his first teacher the late Ven. Phra Acharn Leua Pannavaro. He was instructed in satipatthana vipassana at Vivek Asom Meditation
Centre by Ven. Ajhan Charee Jaruvanno and Ven. Ajahn Somsak Sorado, and
practiced intensely at Wat Don Put under the late Ven. Luang Por Bhavanakitti, a rare monk of noble wisdom. He moved to Myanmar to continue his
training in the Mahasi method in 1996, receiving strict personal supervision
during a seven\sphinxhyphen{}month retreat with the late  Ven. Sayadaw U Pandita at Panditarama, Yangon. Those two years revelled the method and mode of insight
practice.

\sphinxAtStartPar
The lure of the forest monk lifestyle and deep concentration led Anthony
to seek out the Ven. Sayadaw U Acinna in 1997 at the Pa Auk Forest Monastery, where he received instruction for a year in anapanasati (mindfulness
with breathing) and four elements meditation. Anthony retreated to the hills
\DUrole{pdfpage}{302}  of the Shan State for a further year of meditation in an isolated monastery.
After four years in robes and constant practice, Anthony studied Buddhist
texts and Pali language at the International Buddhist University in Yangon
consolidating his meditation experiences with scriptural studies. His interest in Pali was furthered through self\sphinxhyphen{}study whilst living at the Shwedagon
Pagoda in 2000.

\sphinxAtStartPar
A two\sphinxhyphen{}year pilgrimage gave the opportunity to wander and dwell in the
forest meditation monasteries of Sri Lanka. He attended the Kalachakra Initiation with His Holiness the Dalai Lama in Bodhgaya 2002, and developed
an  interest  in  choiceless  awareness  meditation  techniques  during  visits  to
monasteries in India, Nepal, China and Tibet.  Anthony spent three extended
rains retreats between 2002 – 2005 practicing satipatthana vipassana at Wat
Pitsoparam, Ubon Ratchathani, with the late Ven. Luang Por Visarnkemakoon. In  2006, returned home, disrobing from the monkhood after 11 vassa.
He experienced life in Melbourne studying and working in the wholesale diamond and gemstone industry for seven years until 2012. He returned
to Thailand in 2013, and at the request of Ven. Ajahn Poh (Suan Mokkh) led
popular meditation retreats at Wat Kow Tahm between 2013\sphinxhyphen{}2016, where he
guided  thousands  of  meditators  through  silent  monthly  7\sphinxhyphen{}day    and  annual
21\sphinxhyphen{}day vipassana retreats.

\sphinxAtStartPar
\sphinxstylestrong{Present:} He  is  currently  working  to  establish  the  infrastructure  for  Indriya
Retreat – Dhamma and Meditation Hall in a secluded fruit garden in Koh
Phangan. He plans to start leading non\sphinxhyphen{}residential vipassana retreats in 2019.
\begin{itemize}
\item {} 
\sphinxAtStartPar
\sphinxurl{https://indriyaretreat.org}

\item {} 
\sphinxAtStartPar
\sphinxurl{https://anthonymarkwell.com}

\end{itemize}



\renewcommand{\indexname}{Index}
\printindex
\end{document}